%% Generated by Sphinx.
\def\sphinxdocclass{report}
\documentclass[letterpaper,10pt,english]{sphinxmanual}
\ifdefined\pdfpxdimen
   \let\sphinxpxdimen\pdfpxdimen\else\newdimen\sphinxpxdimen
\fi \sphinxpxdimen=.75bp\relax

\usepackage[utf8]{inputenc}
\ifdefined\DeclareUnicodeCharacter
 \ifdefined\DeclareUnicodeCharacterAsOptional
  \DeclareUnicodeCharacter{"00A0}{\nobreakspace}
  \DeclareUnicodeCharacter{"2500}{\sphinxunichar{2500}}
  \DeclareUnicodeCharacter{"2502}{\sphinxunichar{2502}}
  \DeclareUnicodeCharacter{"2514}{\sphinxunichar{2514}}
  \DeclareUnicodeCharacter{"251C}{\sphinxunichar{251C}}
  \DeclareUnicodeCharacter{"2572}{\textbackslash}
 \else
  \DeclareUnicodeCharacter{00A0}{\nobreakspace}
  \DeclareUnicodeCharacter{2500}{\sphinxunichar{2500}}
  \DeclareUnicodeCharacter{2502}{\sphinxunichar{2502}}
  \DeclareUnicodeCharacter{2514}{\sphinxunichar{2514}}
  \DeclareUnicodeCharacter{251C}{\sphinxunichar{251C}}
  \DeclareUnicodeCharacter{2572}{\textbackslash}
 \fi
\fi
\usepackage{cmap}
\usepackage[T1]{fontenc}
\usepackage{amsmath,amssymb,amstext}
\usepackage{babel}
\usepackage{times}
\usepackage[Bjarne]{fncychap}
\usepackage[dontkeepoldnames]{sphinx}

\usepackage{geometry}

% Include hyperref last.
\usepackage{hyperref}
% Fix anchor placement for figures with captions.
\usepackage{hypcap}% it must be loaded after hyperref.
% Set up styles of URL: it should be placed after hyperref.
\urlstyle{same}

\addto\captionsenglish{\renewcommand{\figurename}{Fig.}}
\addto\captionsenglish{\renewcommand{\tablename}{Table}}
\addto\captionsenglish{\renewcommand{\literalblockname}{Listing}}

\addto\captionsenglish{\renewcommand{\literalblockcontinuedname}{continued from previous page}}
\addto\captionsenglish{\renewcommand{\literalblockcontinuesname}{continues on next page}}

\addto\extrasenglish{\def\pageautorefname{page}}

\setcounter{tocdepth}{1}



\title{salabim Documentation}
\date{Oct 07, 2018}
\release{2.4.0}
\author{Ruud van der Ham}
\newcommand{\sphinxlogo}{\vbox{}}
\renewcommand{\releasename}{Release}
\makeindex

\begin{document}

\maketitle
\sphinxtableofcontents
\phantomsection\label{\detokenize{index::doc}}



\chapter{Introduction}
\label{\detokenize{Introduction:introduction}}\label{\detokenize{Introduction::doc}}\label{\detokenize{Introduction:documentation-for-salabim}}
Salabim is a package for discrete event simulation in Python.
It follows the methodology of process description as originally demonstrated in \sphinxstyleemphasis{Simula}
and later in \sphinxstyleemphasis{Prosim}, \sphinxstyleemphasis{Must} and \sphinxstyleemphasis{Tomas}. 
The process interaction methods are also quite similar to \sphinxstyleemphasis{SimPy 2}.

The package comprises discrete event simulation, queue handling, resources, statistical sampling and monitoring.
On top of that real time animation is built in.

The package comes with a number of sample models.


\section{Requirements}
\label{\detokenize{Introduction:requirements}}
Salabim runs on
\begin{itemize}
\item {} 
CPython

\item {} 
PyPy platform

\item {} 
Pythonista (iOS)

\end{itemize}

The package runs under Python 2.7 or 3.x.

The following packages are required:


\begin{savenotes}\sphinxattablestart
\centering
\begin{tabular}[t]{|*{5}{\X{1}{5}|}}
\hline
\sphinxstylethead{\sphinxstyletheadfamily 
Platform
\unskip}\relax &\sphinxstylethead{\sphinxstyletheadfamily 
Base functionality
\unskip}\relax &\sphinxstylethead{\sphinxstyletheadfamily 
Animation
\unskip}\relax &\sphinxstylethead{\sphinxstyletheadfamily 
Video (mp4, avi)
\unskip}\relax &\sphinxstylethead{\sphinxstyletheadfamily 
Animated GIF
\unskip}\relax \\
\hline
CPython
&\begin{itemize}
\item {} 
\end{itemize}
&
Pillow, tkinter
&
opencv, numpy
&
Pillow
\\
\hline
PyPy
&\begin{itemize}
\item {} 
\end{itemize}
&
Pillow, tkinter
&
N/A
&
Pillow
\\
\hline
Pythonista
&\begin{itemize}
\item {} 
\end{itemize}
&
Pillow
&
N/A
&
Pillow
\\
\hline
\end{tabular}
\par
\sphinxattableend\end{savenotes}
\begin{description}
\item[{Several CPython packages, like \sphinxstyleemphasis{WinPython} support Pillow out of the box. If not, install with:}] \leavevmode
\sphinxcode{pip install Pillow}

\item[{Under Linux, Pillow can be installed with:}] \leavevmode
\sphinxcode{sudo apt-get purge python3-pil} 
\sphinxcode{sudo apt-get install python3-pil python3-pil.imagetk}

\item[{For, video production, installation of opencv and numpy may be required with}] \leavevmode
\sphinxcode{pip install opencv-python} 
\sphinxcode{pip install numpy}

\end{description}

Running models under PyPy is highly recommended for production runs,
where run time is important. We have found 6 to 7 times faster execution compared to CPython.
However, for development, nothing can beat CPython or Pythonista.


\section{Installation}
\label{\detokenize{Introduction:installation}}\begin{description}
\item[{The preferred way to install salabim is from PyPI with:}] \leavevmode
\sphinxcode{pip install salabim}

\item[{or to upgrade to a new version:}] \leavevmode
\sphinxcode{pip install salabim -{-}upgrade}

\end{description}

You can find the package along with some support files and sample models on www.github.com/salabim/salabim.
From there you can directly download as a zip file and next extract all files.
Alternatively the repository can be cloned. 

For Pythonista, the easiest way to download salabim is:
\begin{itemize}
\item {} 
Tap ‘Open in…’.

\item {} 
Tap ‘Run Pythonista Script’.

\item {} 
Pick this script and tap the run button

\item {} 
Import file

\item {} 
Possibly after short delay, there will be a salabim-master.zip file in the root directory

\item {} 
Tap this zip file and Extract files

\item {} 
All files are now in a directory called salabim-master

\item {} 
Optionally rename this directory to salabim

\end{itemize}

Salabim itself is provided as one Python script, called salabim.py.
You may place that file in any directory where your models reside. 

If you want salabim to be available from other directories, without copying the salabim.py script, either install from PyPI (see above)
or run the supplied install.py file.
In doing so, you will create (or update) a salabim directory in the site-package directory,
which will then contain a copy of the salabim package.


\section{Python}
\label{\detokenize{Introduction:python}}
Python is a widely used high-level programming language for general-purpose programming,
created by Guido van Rossum and first released in 1991. An interpreted language, Python has a design
philosophy that emphasizes code readability (notably using whitespace indentation to
delimit code blocks rather than curly brackets or keywords), and a syntax that allows programmers
to express concepts in fewer lines of code than might be used in languages
such as C++ or Java. The language provides constructs intended to enable writing clear programs
on both a small and large scale.

A good way to start learning about Python is \sphinxurl{https://www.python.org/about/gettingstarted/}


\chapter{Modeling}
\label{\detokenize{Modeling::doc}}\label{\detokenize{Modeling:modeling}}

\section{A simple model}
\label{\detokenize{Modeling:a-simple-model}}
Let’s start with a very simple model, to demonstrate the basic structure,
process interaction, component definition and output:

\begin{sphinxVerbatim}[commandchars=\\\{\},numbers=left,firstnumber=1,stepnumber=1]
\PYG{c+c1}{\PYGZsh{} Example \PYGZhy{} basic.py}
\PYG{k+kn}{import} \PYG{n+nn}{salabim} \PYG{k}{as} \PYG{n+nn}{sim}

\PYG{k}{class} \PYG{n+nc}{Car}\PYG{p}{(}\PYG{n}{sim}\PYG{o}{.}\PYG{n}{Component}\PYG{p}{)}\PYG{p}{:}
    \PYG{k}{def} \PYG{n+nf}{process}\PYG{p}{(}\PYG{n+nb+bp}{self}\PYG{p}{)}\PYG{p}{:}
        \PYG{k}{while} \PYG{k+kc}{True}\PYG{p}{:}
            \PYG{k}{yield} \PYG{n+nb+bp}{self}\PYG{o}{.}\PYG{n}{hold}\PYG{p}{(}\PYG{l+m+mi}{1}\PYG{p}{)}

\PYG{n}{env} \PYG{o}{=} \PYG{n}{sim}\PYG{o}{.}\PYG{n}{Environment}\PYG{p}{(}\PYG{n}{trace}\PYG{o}{=}\PYG{k+kc}{True}\PYG{p}{)}
\PYG{n}{Car}\PYG{p}{(}\PYG{p}{)}
\PYG{n}{env}\PYG{o}{.}\PYG{n}{run}\PYG{p}{(}\PYG{n}{till}\PYG{o}{=}\PYG{l+m+mi}{5}\PYG{p}{)}
\end{sphinxVerbatim}

In basic steps:

We always start by importing salabim

\begin{sphinxVerbatim}[commandchars=\\\{\}]
\PYG{k+kn}{import} \PYG{n+nn}{salabim} \PYG{k}{as} \PYG{n+nn}{sim}
\end{sphinxVerbatim}

Now we can refer to all salabim classes and function with \sphinxcode{sim.}.
For convenience, some functions or classes can be imported with, for instance

\begin{sphinxVerbatim}[commandchars=\\\{\}]
\PYG{k+kn}{from} \PYG{n+nn}{salabim} \PYG{k}{import} \PYG{n}{now}\PYG{p}{,} \PYG{n}{main}\PYG{p}{,} \PYG{n}{Component}
\end{sphinxVerbatim}

It is also possible to import all methods, classes and globals by

\begin{sphinxVerbatim}[commandchars=\\\{\}]
\PYG{k+kn}{from} \PYG{n+nn}{salabim} \PYG{k}{import} \PYG{o}{*}
\end{sphinxVerbatim}

, but we do not recommend that method.

The main body of every salabim model usually starts with

\begin{sphinxVerbatim}[commandchars=\\\{\}]
\PYG{n}{env} \PYG{o}{=} \PYG{n}{sim}\PYG{o}{.}\PYG{n}{Environment}\PYG{p}{(}\PYG{n}{parameters}\PYG{p}{)}
\end{sphinxVerbatim}

For each (active) component we define a class as in

\begin{sphinxVerbatim}[commandchars=\\\{\}]
\PYG{k}{class} \PYG{n+nc}{Car}\PYG{p}{(}\PYG{n}{sim}\PYG{o}{.}\PYG{n}{Component}\PYG{p}{)}\PYG{p}{:}
\end{sphinxVerbatim}

The class inherits from sim.Component.

Although it is possible to define other processes within a class,
the standard way is to define a generator function called \sphinxcode{process} in the class.
A generator is a function with at least one yield statement. These are used in salabim context as
a signal to give control to the sequence mechanism.

In this example,

\begin{sphinxVerbatim}[commandchars=\\\{\}]
\PYG{k}{yield} \PYG{n+nb+bp}{self}\PYG{o}{.}\PYG{n}{hold}\PYG{p}{(}\PYG{l+m+mi}{1}\PYG{p}{)}
\end{sphinxVerbatim}

gives control,to the sequence mechanism and \sphinxstyleemphasis{comes back} after 1 time unit. The self. part means that
it is this component to be held for some time. We will see later other uses of yield like passivate,
request, wait and standby.

In the main body an instance of a car is created by Car(). It automatically gets the name car.0.
As there is a generator function called
process in Car, this process description will be activated (by default at time now, which is 0 here).
It is possible to start a process later, but this is by far the most common way to start a process.

With

\begin{sphinxVerbatim}[commandchars=\\\{\}]
\PYG{n}{env}\PYG{o}{.}\PYG{n}{run}\PYG{p}{(}\PYG{n}{till}\PYG{o}{=}\PYG{l+m+mi}{5}\PYG{p}{)}
\end{sphinxVerbatim}

we start the simulation and get back control after 5 time units. A component called \sphinxstyleemphasis{main} is defined
under the hood to get access to the main process.

When we run this program, we get the following output

\begin{sphinxVerbatim}[commandchars=\\\{\}]
line\PYGZsh{}         time current component    action                               information
\PYGZhy{}\PYGZhy{}\PYGZhy{}\PYGZhy{}\PYGZhy{}   \PYGZhy{}\PYGZhy{}\PYGZhy{}\PYGZhy{}\PYGZhy{}\PYGZhy{}\PYGZhy{}\PYGZhy{}\PYGZhy{}\PYGZhy{} \PYGZhy{}\PYGZhy{}\PYGZhy{}\PYGZhy{}\PYGZhy{}\PYGZhy{}\PYGZhy{}\PYGZhy{}\PYGZhy{}\PYGZhy{}\PYGZhy{}\PYGZhy{}\PYGZhy{}\PYGZhy{}\PYGZhy{}\PYGZhy{}\PYGZhy{}\PYGZhy{}\PYGZhy{}\PYGZhy{} \PYGZhy{}\PYGZhy{}\PYGZhy{}\PYGZhy{}\PYGZhy{}\PYGZhy{}\PYGZhy{}\PYGZhy{}\PYGZhy{}\PYGZhy{}\PYGZhy{}\PYGZhy{}\PYGZhy{}\PYGZhy{}\PYGZhy{}\PYGZhy{}\PYGZhy{}\PYGZhy{}\PYGZhy{}\PYGZhy{}\PYGZhy{}\PYGZhy{}\PYGZhy{}\PYGZhy{}\PYGZhy{}\PYGZhy{}\PYGZhy{}\PYGZhy{}\PYGZhy{}\PYGZhy{}\PYGZhy{}\PYGZhy{}\PYGZhy{}\PYGZhy{}\PYGZhy{}  \PYGZhy{}\PYGZhy{}\PYGZhy{}\PYGZhy{}\PYGZhy{}\PYGZhy{}\PYGZhy{}\PYGZhy{}\PYGZhy{}\PYGZhy{}\PYGZhy{}\PYGZhy{}\PYGZhy{}\PYGZhy{}\PYGZhy{}\PYGZhy{}\PYGZhy{}\PYGZhy{}\PYGZhy{}\PYGZhy{}\PYGZhy{}\PYGZhy{}\PYGZhy{}\PYGZhy{}\PYGZhy{}\PYGZhy{}\PYGZhy{}\PYGZhy{}\PYGZhy{}\PYGZhy{}\PYGZhy{}\PYGZhy{}\PYGZhy{}\PYGZhy{}\PYGZhy{}\PYGZhy{}\PYGZhy{}\PYGZhy{}\PYGZhy{}\PYGZhy{}\PYGZhy{}\PYGZhy{}\PYGZhy{}\PYGZhy{}\PYGZhy{}\PYGZhy{}\PYGZhy{}\PYGZhy{}
                                        line numbers refers to               Example \PYGZhy{} basic.py
   11                                   default environment initialize
   11                                   main create
   11        0.000 main                 current
   12                                   car.0 create
   12                                   car.0 activate                       scheduled for      0.000 @    6  process=process
   13                                   main run                             scheduled for      5.000 @   13+
    6        0.000 car.0                current
    8                                   car.0 hold                           scheduled for      1.000 @    8+
    8+       1.000 car.0                current
    8                                   car.0 hold                           scheduled for      2.000 @    8+
    8+       2.000 car.0                current
    8                                   car.0 hold                           scheduled for      3.000 @    8+
    8+       3.000 car.0                current
    8                                   car.0 hold                           scheduled for      4.000 @    8+
    8+       4.000 car.0                current
    8                                   car.0 hold                           scheduled for      5.000 @    8+
   13+       5.000 main                 current
\end{sphinxVerbatim}


\section{A bank example}
\label{\detokenize{Modeling:a-bank-example}}
Now let’s move to a more realistic model. Here customers are arriving in
a bank, where there is one clerk. This clerk handles the
customers in first in first out (FIFO) order.
We see the following processes:
\begin{itemize}
\item {} 
The customer generator that creates the customers, with an inter arrival time of uniform(5,15)

\item {} 
The customers

\item {} 
The clerk, which serves the customers in a constant time of 30 (overloaded and non steady state system)

\end{itemize}

And we need a queue for the customers to wait for service.

The model code is:

\begin{sphinxVerbatim}[commandchars=\\\{\},numbers=left,firstnumber=1,stepnumber=1]
\PYG{c+c1}{\PYGZsh{} Example \PYGZhy{} bank, 1 clerk.py}
\PYG{k+kn}{import} \PYG{n+nn}{salabim} \PYG{k}{as} \PYG{n+nn}{sim}


\PYG{k}{class} \PYG{n+nc}{CustomerGenerator}\PYG{p}{(}\PYG{n}{sim}\PYG{o}{.}\PYG{n}{Component}\PYG{p}{)}\PYG{p}{:}
    \PYG{k}{def} \PYG{n+nf}{process}\PYG{p}{(}\PYG{n+nb+bp}{self}\PYG{p}{)}\PYG{p}{:}
        \PYG{k}{while} \PYG{k+kc}{True}\PYG{p}{:}
            \PYG{n}{Customer}\PYG{p}{(}\PYG{p}{)}
            \PYG{k}{yield} \PYG{n+nb+bp}{self}\PYG{o}{.}\PYG{n}{hold}\PYG{p}{(}\PYG{n}{sim}\PYG{o}{.}\PYG{n}{Uniform}\PYG{p}{(}\PYG{l+m+mi}{5}\PYG{p}{,} \PYG{l+m+mi}{15}\PYG{p}{)}\PYG{o}{.}\PYG{n}{sample}\PYG{p}{(}\PYG{p}{)}\PYG{p}{)}


\PYG{k}{class} \PYG{n+nc}{Customer}\PYG{p}{(}\PYG{n}{sim}\PYG{o}{.}\PYG{n}{Component}\PYG{p}{)}\PYG{p}{:}
    \PYG{k}{def} \PYG{n+nf}{process}\PYG{p}{(}\PYG{n+nb+bp}{self}\PYG{p}{)}\PYG{p}{:}
        \PYG{n+nb+bp}{self}\PYG{o}{.}\PYG{n}{enter}\PYG{p}{(}\PYG{n}{waitingline}\PYG{p}{)}
        \PYG{k}{if} \PYG{n}{clerk}\PYG{o}{.}\PYG{n}{ispassive}\PYG{p}{(}\PYG{p}{)}\PYG{p}{:}
            \PYG{n}{clerk}\PYG{o}{.}\PYG{n}{activate}\PYG{p}{(}\PYG{p}{)}
        \PYG{k}{yield} \PYG{n+nb+bp}{self}\PYG{o}{.}\PYG{n}{passivate}\PYG{p}{(}\PYG{p}{)}


\PYG{k}{class} \PYG{n+nc}{Clerk}\PYG{p}{(}\PYG{n}{sim}\PYG{o}{.}\PYG{n}{Component}\PYG{p}{)}\PYG{p}{:}
    \PYG{k}{def} \PYG{n+nf}{process}\PYG{p}{(}\PYG{n+nb+bp}{self}\PYG{p}{)}\PYG{p}{:}
        \PYG{k}{while} \PYG{k+kc}{True}\PYG{p}{:}
            \PYG{k}{while} \PYG{n+nb}{len}\PYG{p}{(}\PYG{n}{waitingline}\PYG{p}{)} \PYG{o}{==} \PYG{l+m+mi}{0}\PYG{p}{:}
                \PYG{k}{yield} \PYG{n+nb+bp}{self}\PYG{o}{.}\PYG{n}{passivate}\PYG{p}{(}\PYG{p}{)}
            \PYG{n+nb+bp}{self}\PYG{o}{.}\PYG{n}{customer} \PYG{o}{=} \PYG{n}{waitingline}\PYG{o}{.}\PYG{n}{pop}\PYG{p}{(}\PYG{p}{)}
            \PYG{k}{yield} \PYG{n+nb+bp}{self}\PYG{o}{.}\PYG{n}{hold}\PYG{p}{(}\PYG{l+m+mi}{30}\PYG{p}{)}
            \PYG{n+nb+bp}{self}\PYG{o}{.}\PYG{n}{customer}\PYG{o}{.}\PYG{n}{activate}\PYG{p}{(}\PYG{p}{)}


\PYG{n}{env} \PYG{o}{=} \PYG{n}{sim}\PYG{o}{.}\PYG{n}{Environment}\PYG{p}{(}\PYG{n}{trace}\PYG{o}{=}\PYG{k+kc}{True}\PYG{p}{)}

\PYG{n}{CustomerGenerator}\PYG{p}{(}\PYG{n}{name}\PYG{o}{=}\PYG{l+s+s1}{\PYGZsq{}}\PYG{l+s+s1}{\PYGZsq{}}\PYG{p}{)}  \PYG{c+c1}{\PYGZsh{} using name=\PYGZsq{}\PYGZsq{} prevents the name customergenerator to be serialized}
\PYG{n}{clerk} \PYG{o}{=} \PYG{n}{Clerk}\PYG{p}{(}\PYG{p}{)}
\PYG{n}{waitingline} \PYG{o}{=} \PYG{n}{sim}\PYG{o}{.}\PYG{n}{Queue}\PYG{p}{(}\PYG{l+s+s1}{\PYGZsq{}}\PYG{l+s+s1}{waitingline}\PYG{l+s+s1}{\PYGZsq{}}\PYG{p}{)}

\PYG{n}{env}\PYG{o}{.}\PYG{n}{run}\PYG{p}{(}\PYG{n}{till}\PYG{o}{=}\PYG{l+m+mi}{50}\PYG{p}{)}
\PYG{n+nb}{print}\PYG{p}{(}\PYG{p}{)}
\PYG{n}{waitingline}\PYG{o}{.}\PYG{n}{print\PYGZus{}statistics}\PYG{p}{(}\PYG{p}{)}
\end{sphinxVerbatim}

Let’s look at some details

\begin{sphinxVerbatim}[commandchars=\\\{\}]
\PYG{k}{yield} \PYG{n+nb+bp}{self}\PYG{o}{.}\PYG{n}{hold}\PYG{p}{(}\PYG{n}{sim}\PYG{o}{.}\PYG{n}{Uniform}\PYG{p}{(}\PYG{l+m+mi}{5}\PYG{p}{,} \PYG{l+m+mi}{15}\PYG{p}{)}\PYG{o}{.}\PYG{n}{sample}\PYG{p}{(}\PYG{p}{)}\PYG{p}{)}
\end{sphinxVerbatim}

will do the statistical sampling and wait for that time till the next customer is created.

With

\begin{sphinxVerbatim}[commandchars=\\\{\}]
\PYG{n+nb+bp}{self}\PYG{o}{.}\PYG{n}{enter}\PYG{p}{(}\PYG{n}{waitingline}\PYG{p}{)}
\end{sphinxVerbatim}

the customer places itself at the tail of the waiting line.

Then, the customer checks whether the clerk is idle, and if so, activates him immediately.

\begin{sphinxVerbatim}[commandchars=\\\{\}]
\PYG{k}{if} \PYG{n}{clerk}\PYG{o}{.}\PYG{n}{ispassive}\PYG{p}{(}\PYG{p}{)}\PYG{p}{:}
    \PYG{n}{clerk}\PYG{o}{.}\PYG{n}{activate}\PYG{p}{(}\PYG{p}{)}
\end{sphinxVerbatim}

Once the clerk is active (again), it gets the first customer out of the waitingline with

\begin{sphinxVerbatim}[commandchars=\\\{\}]
\PYG{n+nb+bp}{self}\PYG{o}{.}\PYG{n}{customer} \PYG{o}{=} \PYG{n}{waitingline}\PYG{o}{.}\PYG{n}{pop}\PYG{p}{(}\PYG{p}{)}
\end{sphinxVerbatim}

and holds for 30 time units with

\begin{sphinxVerbatim}[commandchars=\\\{\}]
\PYG{k}{yield} \PYG{n+nb+bp}{self}\PYG{o}{.}\PYG{n}{hold}\PYG{p}{(}\PYG{l+m+mi}{30}\PYG{p}{)}
\end{sphinxVerbatim}

After that hold the customer is activated and will terminate

\begin{sphinxVerbatim}[commandchars=\\\{\}]
\PYG{n+nb+bp}{self}\PYG{o}{.}\PYG{n}{customer}\PYG{o}{.}\PYG{n}{activate}\PYG{p}{(}\PYG{p}{)}
\end{sphinxVerbatim}

In the main section of the program, we create the CustomerGenerator, the Clerk and a queue called waitingline.
After the simulation is finished, the statistics of the queue are presented with

\begin{sphinxVerbatim}[commandchars=\\\{\}]
\PYG{n}{waitingline}\PYG{o}{.}\PYG{n}{print\PYGZus{}statistics}\PYG{p}{(}\PYG{p}{)}
\end{sphinxVerbatim}

The output looks like

\begin{sphinxVerbatim}[commandchars=\\\{\}]
line\PYGZsh{}         time current component    action                               information
\PYGZhy{}\PYGZhy{}\PYGZhy{}\PYGZhy{}\PYGZhy{}   \PYGZhy{}\PYGZhy{}\PYGZhy{}\PYGZhy{}\PYGZhy{}\PYGZhy{}\PYGZhy{}\PYGZhy{}\PYGZhy{}\PYGZhy{} \PYGZhy{}\PYGZhy{}\PYGZhy{}\PYGZhy{}\PYGZhy{}\PYGZhy{}\PYGZhy{}\PYGZhy{}\PYGZhy{}\PYGZhy{}\PYGZhy{}\PYGZhy{}\PYGZhy{}\PYGZhy{}\PYGZhy{}\PYGZhy{}\PYGZhy{}\PYGZhy{}\PYGZhy{}\PYGZhy{} \PYGZhy{}\PYGZhy{}\PYGZhy{}\PYGZhy{}\PYGZhy{}\PYGZhy{}\PYGZhy{}\PYGZhy{}\PYGZhy{}\PYGZhy{}\PYGZhy{}\PYGZhy{}\PYGZhy{}\PYGZhy{}\PYGZhy{}\PYGZhy{}\PYGZhy{}\PYGZhy{}\PYGZhy{}\PYGZhy{}\PYGZhy{}\PYGZhy{}\PYGZhy{}\PYGZhy{}\PYGZhy{}\PYGZhy{}\PYGZhy{}\PYGZhy{}\PYGZhy{}\PYGZhy{}\PYGZhy{}\PYGZhy{}\PYGZhy{}\PYGZhy{}\PYGZhy{}  \PYGZhy{}\PYGZhy{}\PYGZhy{}\PYGZhy{}\PYGZhy{}\PYGZhy{}\PYGZhy{}\PYGZhy{}\PYGZhy{}\PYGZhy{}\PYGZhy{}\PYGZhy{}\PYGZhy{}\PYGZhy{}\PYGZhy{}\PYGZhy{}\PYGZhy{}\PYGZhy{}\PYGZhy{}\PYGZhy{}\PYGZhy{}\PYGZhy{}\PYGZhy{}\PYGZhy{}\PYGZhy{}\PYGZhy{}\PYGZhy{}\PYGZhy{}\PYGZhy{}\PYGZhy{}\PYGZhy{}\PYGZhy{}\PYGZhy{}\PYGZhy{}\PYGZhy{}\PYGZhy{}\PYGZhy{}\PYGZhy{}\PYGZhy{}\PYGZhy{}\PYGZhy{}\PYGZhy{}\PYGZhy{}\PYGZhy{}\PYGZhy{}\PYGZhy{}\PYGZhy{}\PYGZhy{}
                                        line numbers refers to               Example \PYGZhy{} bank, 1 clerk.py
   30                                   default environment initialize
   30                                   main create
   30        0.000 main                 current
   32                                   customergenerator create
   32                                   customergenerator activate           scheduled for      0.000 @    6  process=process
   33                                   clerk.0 create
   33                                   clerk.0 activate                     scheduled for      0.000 @   21  process=process
   34                                   waitingline create
   36                                   main run                             scheduled for     50.000 @   36+
    6        0.000 customergenerator    current
    8                                   customer.0 create
    8                                   customer.0 activate                  scheduled for      0.000 @   13  process=process
    9                                   customergenerator hold               scheduled for     14.631 @    9+
   21        0.000 clerk.0              current
   24                                   clerk.0 passivate
   13        0.000 customer.0           current
   14                                   customer.0                           enter waitingline
   16                                   clerk.0 activate                     scheduled for      0.000 @   24+
   17                                   customer.0 passivate
   24+       0.000 clerk.0              current
   25                                   customer.0                           leave waitingline
   26                                   clerk.0 hold                         scheduled for     30.000 @   26+
    9+      14.631 customergenerator    current
    8                                   customer.1 create
    8                                   customer.1 activate                  scheduled for     14.631 @   13  process=process
    9                                   customergenerator hold               scheduled for     21.989 @    9+
   13       14.631 customer.1           current
   14                                   customer.1                           enter waitingline
   17                                   customer.1 passivate
    9+      21.989 customergenerator    current
    8                                   customer.2 create
    8                                   customer.2 activate                  scheduled for     21.989 @   13  process=process
    9                                   customergenerator hold               scheduled for     32.804 @    9+
   13       21.989 customer.2           current
   14                                   customer.2                           enter waitingline
   17                                   customer.2 passivate
   26+      30.000 clerk.0              current
   27                                   customer.0 activate                  scheduled for     30.000 @   17+
   25                                   customer.1                           leave waitingline
   26                                   clerk.0 hold                         scheduled for     60.000 @   26+
   17+      30.000 customer.0           current
                                        customer.0 ended
    9+      32.804 customergenerator    current
    8                                   customer.3 create
    8                                   customer.3 activate                  scheduled for     32.804 @   13  process=process
    9                                   customergenerator hold               scheduled for     40.071 @    9+
   13       32.804 customer.3           current
   14                                   customer.3                           enter waitingline
   17                                   customer.3 passivate
    9+      40.071 customergenerator    current
    8                                   customer.4 create
    8                                   customer.4 activate                  scheduled for     40.071 @   13  process=process
    9                                   customergenerator hold               scheduled for     54.737 @    9+
   13       40.071 customer.4           current
   14                                   customer.4                           enter waitingline
   17                                   customer.4 passivate
   36+      50.000 main                 current

Statistics of waitingline at        50
                                                                     all    excl.zero         zero
\PYGZhy{}\PYGZhy{}\PYGZhy{}\PYGZhy{}\PYGZhy{}\PYGZhy{}\PYGZhy{}\PYGZhy{}\PYGZhy{}\PYGZhy{}\PYGZhy{}\PYGZhy{}\PYGZhy{}\PYGZhy{}\PYGZhy{}\PYGZhy{}\PYGZhy{}\PYGZhy{}\PYGZhy{}\PYGZhy{}\PYGZhy{}\PYGZhy{}\PYGZhy{}\PYGZhy{}\PYGZhy{}\PYGZhy{}\PYGZhy{}\PYGZhy{}\PYGZhy{}\PYGZhy{}\PYGZhy{}\PYGZhy{}\PYGZhy{}\PYGZhy{}\PYGZhy{}\PYGZhy{}\PYGZhy{}\PYGZhy{}\PYGZhy{}\PYGZhy{}\PYGZhy{}\PYGZhy{}\PYGZhy{}\PYGZhy{} \PYGZhy{}\PYGZhy{}\PYGZhy{}\PYGZhy{}\PYGZhy{}\PYGZhy{}\PYGZhy{}\PYGZhy{}\PYGZhy{}\PYGZhy{}\PYGZhy{}\PYGZhy{}\PYGZhy{}\PYGZhy{} \PYGZhy{}\PYGZhy{}\PYGZhy{}\PYGZhy{}\PYGZhy{}\PYGZhy{}\PYGZhy{}\PYGZhy{}\PYGZhy{}\PYGZhy{}\PYGZhy{}\PYGZhy{} \PYGZhy{}\PYGZhy{}\PYGZhy{}\PYGZhy{}\PYGZhy{}\PYGZhy{}\PYGZhy{}\PYGZhy{}\PYGZhy{}\PYGZhy{}\PYGZhy{}\PYGZhy{} \PYGZhy{}\PYGZhy{}\PYGZhy{}\PYGZhy{}\PYGZhy{}\PYGZhy{}\PYGZhy{}\PYGZhy{}\PYGZhy{}\PYGZhy{}\PYGZhy{}\PYGZhy{}
Length of waitingline                        duration             50           35.369       14.631
                                             mean                  1.410        1.993
                                             std.deviation         1.107        0.754

                                             minimum               0            1
                                             median                2            2
                                             90\PYGZpc{} percentile        3            3
                                             95\PYGZpc{} percentile        3            3
                                             maximum               3            3

Length of stay in waitingline                entries               2            1            1
                                             mean                  7.684       15.369
                                             std.deviation         7.684        0

                                             minimum               0           15.369
                                             median               15.369       15.369
                                             90\PYGZpc{} percentile       15.369       15.369
                                             95\PYGZpc{} percentile       15.369       15.369
                                             maximum              15.369       15.369
\end{sphinxVerbatim}

Now, let’s add more clerks. Here we have chosen to put the three clerks in a list

\begin{sphinxVerbatim}[commandchars=\\\{\}]
\PYG{n}{clerks} \PYG{o}{=} \PYG{p}{[}\PYG{n}{Clerk}\PYG{p}{(}\PYG{p}{)} \PYG{k}{for} \PYG{n}{\PYGZus{}} \PYG{o+ow}{in} \PYG{n+nb}{range}\PYG{p}{(}\PYG{l+m+mi}{3}\PYG{p}{)}\PYG{p}{]}
\end{sphinxVerbatim}

although in this case we could have also put them in a salabim queue, like

\begin{sphinxVerbatim}[commandchars=\\\{\}]
\PYG{n}{clerks} \PYG{o}{=} \PYG{n}{sim}\PYG{o}{.}\PYG{n}{Queue}\PYG{p}{(}\PYG{l+s+s1}{\PYGZsq{}}\PYG{l+s+s1}{clerks}\PYG{l+s+s1}{\PYGZsq{}}\PYG{p}{)}
\PYG{k}{for} \PYG{n}{\PYGZus{}} \PYG{o+ow}{in} \PYG{n+nb}{range}\PYG{p}{(}\PYG{l+m+mi}{3}\PYG{p}{)}\PYG{p}{:}
    \PYG{n}{Clerk}\PYG{p}{(}\PYG{p}{)}\PYG{o}{.}\PYG{n}{enter}\PYG{p}{(}\PYG{n}{clerks}\PYG{p}{)}
\end{sphinxVerbatim}

And, to restart a clerk

\begin{sphinxVerbatim}[commandchars=\\\{\}]
\PYG{k}{for} \PYG{n}{clerk} \PYG{o+ow}{in} \PYG{n}{clerks}\PYG{p}{:}
    \PYG{k}{if} \PYG{n}{clerk}\PYG{o}{.}\PYG{n}{ispassive}\PYG{p}{(}\PYG{p}{)}\PYG{p}{:}
       \PYG{n}{clerk}\PYG{o}{.}\PYG{n}{activate}\PYG{p}{(}\PYG{p}{)}
       \PYG{k}{break}  \PYG{c+c1}{\PYGZsh{} reactivate only one clerk}
\end{sphinxVerbatim}

The complete source of a three clerk post office:

\begin{sphinxVerbatim}[commandchars=\\\{\}]
\PYG{c+c1}{\PYGZsh{} Example \PYGZhy{} bank, 3 clerks.py}
\PYG{k+kn}{import} \PYG{n+nn}{salabim} \PYG{k}{as} \PYG{n+nn}{sim}


\PYG{k}{class} \PYG{n+nc}{CustomerGenerator}\PYG{p}{(}\PYG{n}{sim}\PYG{o}{.}\PYG{n}{Component}\PYG{p}{)}\PYG{p}{:}
    \PYG{k}{def} \PYG{n+nf}{process}\PYG{p}{(}\PYG{n+nb+bp}{self}\PYG{p}{)}\PYG{p}{:}
        \PYG{k}{while} \PYG{k+kc}{True}\PYG{p}{:}
            \PYG{n}{Customer}\PYG{p}{(}\PYG{p}{)}
            \PYG{k}{yield} \PYG{n+nb+bp}{self}\PYG{o}{.}\PYG{n}{hold}\PYG{p}{(}\PYG{n}{sim}\PYG{o}{.}\PYG{n}{Uniform}\PYG{p}{(}\PYG{l+m+mi}{5}\PYG{p}{,} \PYG{l+m+mi}{15}\PYG{p}{)}\PYG{o}{.}\PYG{n}{sample}\PYG{p}{(}\PYG{p}{)}\PYG{p}{)}


\PYG{k}{class} \PYG{n+nc}{Customer}\PYG{p}{(}\PYG{n}{sim}\PYG{o}{.}\PYG{n}{Component}\PYG{p}{)}\PYG{p}{:}
    \PYG{k}{def} \PYG{n+nf}{process}\PYG{p}{(}\PYG{n+nb+bp}{self}\PYG{p}{)}\PYG{p}{:}
        \PYG{n+nb+bp}{self}\PYG{o}{.}\PYG{n}{enter}\PYG{p}{(}\PYG{n}{waitingline}\PYG{p}{)}
        \PYG{k}{for} \PYG{n}{clerk} \PYG{o+ow}{in} \PYG{n}{clerks}\PYG{p}{:}
            \PYG{k}{if} \PYG{n}{clerk}\PYG{o}{.}\PYG{n}{ispassive}\PYG{p}{(}\PYG{p}{)}\PYG{p}{:}
                \PYG{n}{clerk}\PYG{o}{.}\PYG{n}{activate}\PYG{p}{(}\PYG{p}{)}
                \PYG{k}{break}  \PYG{c+c1}{\PYGZsh{} activate only one clerk}
        \PYG{k}{yield} \PYG{n+nb+bp}{self}\PYG{o}{.}\PYG{n}{passivate}\PYG{p}{(}\PYG{p}{)}


\PYG{k}{class} \PYG{n+nc}{Clerk}\PYG{p}{(}\PYG{n}{sim}\PYG{o}{.}\PYG{n}{Component}\PYG{p}{)}\PYG{p}{:}
    \PYG{k}{def} \PYG{n+nf}{process}\PYG{p}{(}\PYG{n+nb+bp}{self}\PYG{p}{)}\PYG{p}{:}
        \PYG{k}{while} \PYG{k+kc}{True}\PYG{p}{:}
            \PYG{k}{while} \PYG{n+nb}{len}\PYG{p}{(}\PYG{n}{waitingline}\PYG{p}{)} \PYG{o}{==} \PYG{l+m+mi}{0}\PYG{p}{:}
                \PYG{k}{yield} \PYG{n+nb+bp}{self}\PYG{o}{.}\PYG{n}{passivate}\PYG{p}{(}\PYG{p}{)}
            \PYG{n+nb+bp}{self}\PYG{o}{.}\PYG{n}{customer} \PYG{o}{=} \PYG{n}{waitingline}\PYG{o}{.}\PYG{n}{pop}\PYG{p}{(}\PYG{p}{)}
            \PYG{k}{yield} \PYG{n+nb+bp}{self}\PYG{o}{.}\PYG{n}{hold}\PYG{p}{(}\PYG{l+m+mi}{30}\PYG{p}{)}
            \PYG{n+nb+bp}{self}\PYG{o}{.}\PYG{n}{customer}\PYG{o}{.}\PYG{n}{activate}\PYG{p}{(}\PYG{p}{)}


\PYG{n}{env} \PYG{o}{=} \PYG{n}{sim}\PYG{o}{.}\PYG{n}{Environment}\PYG{p}{(}\PYG{n}{trace}\PYG{o}{=}\PYG{k+kc}{False}\PYG{p}{)}
\PYG{n}{CustomerGenerator}\PYG{p}{(}\PYG{n}{name}\PYG{o}{=}\PYG{l+s+s1}{\PYGZsq{}}\PYG{l+s+s1}{\PYGZsq{}}\PYG{p}{)}
\PYG{n}{clerks} \PYG{o}{=} \PYG{p}{[}\PYG{n}{Clerk}\PYG{p}{(}\PYG{p}{)} \PYG{k}{for} \PYG{n}{\PYGZus{}} \PYG{o+ow}{in} \PYG{n+nb}{range}\PYG{p}{(}\PYG{l+m+mi}{3}\PYG{p}{)}\PYG{p}{]}

\PYG{n}{waitingline} \PYG{o}{=} \PYG{n}{sim}\PYG{o}{.}\PYG{n}{Queue}\PYG{p}{(}\PYG{l+s+s1}{\PYGZsq{}}\PYG{l+s+s1}{waitingline}\PYG{l+s+s1}{\PYGZsq{}}\PYG{p}{)}

\PYG{n}{env}\PYG{o}{.}\PYG{n}{run}\PYG{p}{(}\PYG{n}{till}\PYG{o}{=}\PYG{l+m+mi}{50000}\PYG{p}{)}
\PYG{n}{waitingline}\PYG{o}{.}\PYG{n}{print\PYGZus{}histograms}\PYG{p}{(}\PYG{p}{)}

\PYG{n}{waitingline}\PYG{o}{.}\PYG{n}{print\PYGZus{}info}\PYG{p}{(}\PYG{p}{)}
\end{sphinxVerbatim}


\section{A bank office example with resources}
\label{\detokenize{Modeling:a-bank-office-example-with-resources}}
The salabim package contains another useful concept for modelling: resources.
Resources have a limited capacity and can be claimed by components and released later.

In the model of the bank with the same functionality as the above example, the
clerks are defined as a resource with capacity 3.

The model code is:

\begin{sphinxVerbatim}[commandchars=\\\{\}]
\PYG{c+c1}{\PYGZsh{} Example \PYGZhy{} bank, 3 clerks (resources).py}
\PYG{k+kn}{import} \PYG{n+nn}{salabim} \PYG{k}{as} \PYG{n+nn}{sim}


\PYG{k}{class} \PYG{n+nc}{CustomerGenerator}\PYG{p}{(}\PYG{n}{sim}\PYG{o}{.}\PYG{n}{Component}\PYG{p}{)}\PYG{p}{:}
    \PYG{k}{def} \PYG{n+nf}{process}\PYG{p}{(}\PYG{n+nb+bp}{self}\PYG{p}{)}\PYG{p}{:}
        \PYG{k}{while} \PYG{k+kc}{True}\PYG{p}{:}
            \PYG{n}{Customer}\PYG{p}{(}\PYG{p}{)}
            \PYG{k}{yield} \PYG{n+nb+bp}{self}\PYG{o}{.}\PYG{n}{hold}\PYG{p}{(}\PYG{n}{sim}\PYG{o}{.}\PYG{n}{Uniform}\PYG{p}{(}\PYG{l+m+mi}{5}\PYG{p}{,} \PYG{l+m+mi}{15}\PYG{p}{)}\PYG{o}{.}\PYG{n}{sample}\PYG{p}{(}\PYG{p}{)}\PYG{p}{)}


\PYG{k}{class} \PYG{n+nc}{Customer}\PYG{p}{(}\PYG{n}{sim}\PYG{o}{.}\PYG{n}{Component}\PYG{p}{)}\PYG{p}{:}
    \PYG{k}{def} \PYG{n+nf}{process}\PYG{p}{(}\PYG{n+nb+bp}{self}\PYG{p}{)}\PYG{p}{:}
        \PYG{k}{yield} \PYG{n+nb+bp}{self}\PYG{o}{.}\PYG{n}{request}\PYG{p}{(}\PYG{n}{clerks}\PYG{p}{)}
        \PYG{k}{yield} \PYG{n+nb+bp}{self}\PYG{o}{.}\PYG{n}{hold}\PYG{p}{(}\PYG{l+m+mi}{30}\PYG{p}{)}
        \PYG{n+nb+bp}{self}\PYG{o}{.}\PYG{n}{release}\PYG{p}{(}\PYG{p}{)}


\PYG{n}{env} \PYG{o}{=} \PYG{n}{sim}\PYG{o}{.}\PYG{n}{Environment}\PYG{p}{(}\PYG{n}{trace}\PYG{o}{=}\PYG{k+kc}{False}\PYG{p}{)}
\PYG{n}{CustomerGenerator}\PYG{p}{(}\PYG{n}{name}\PYG{o}{=}\PYG{l+s+s1}{\PYGZsq{}}\PYG{l+s+s1}{customergenerator}\PYG{l+s+s1}{\PYGZsq{}}\PYG{p}{)}
\PYG{n}{clerks} \PYG{o}{=} \PYG{n}{sim}\PYG{o}{.}\PYG{n}{Resource}\PYG{p}{(}\PYG{l+s+s1}{\PYGZsq{}}\PYG{l+s+s1}{clerks}\PYG{l+s+s1}{\PYGZsq{}}\PYG{p}{,} \PYG{n}{capacity}\PYG{o}{=}\PYG{l+m+mi}{3}\PYG{p}{)}

\PYG{n}{env}\PYG{o}{.}\PYG{n}{run}\PYG{p}{(}\PYG{n}{till}\PYG{o}{=}\PYG{l+m+mi}{50000}\PYG{p}{)}

\PYG{n}{clerks}\PYG{o}{.}\PYG{n}{print\PYGZus{}histograms}\PYG{p}{(}\PYG{p}{)}
\PYG{n}{clerks}\PYG{o}{.}\PYG{n}{print\PYGZus{}info}\PYG{p}{(}\PYG{p}{)}
\end{sphinxVerbatim}

Let’s look at some details.

\begin{sphinxVerbatim}[commandchars=\\\{\}]
\PYG{n}{clerks} \PYG{o}{=} \PYG{n}{sim}\PYG{o}{.}\PYG{n}{Resource}\PYG{p}{(}\PYG{l+s+s1}{\PYGZsq{}}\PYG{l+s+s1}{clerks}\PYG{l+s+s1}{\PYGZsq{}}\PYG{p}{,} \PYG{n}{capacity}\PYG{o}{=}\PYG{l+m+mi}{3}\PYG{p}{)}
\end{sphinxVerbatim}

This defines a resource with a capacity of 3.

And then, a customer, just tries to claim one unit (=clerk) from the resource with

\begin{sphinxVerbatim}[commandchars=\\\{\}]
\PYG{k}{yield} \PYG{n+nb+bp}{self}\PYG{o}{.}\PYG{n}{request}\PYG{p}{(}\PYG{n}{clerks}\PYG{p}{)}
\end{sphinxVerbatim}

Here, we use the default of 1 unit. If the resource is not available, the customer just
waits for it to become available (in order of arrival).

In contrast with the previous example, the customer now holds itself for 10 time units.

And after these 10 time units, the customer releases the resource with

\begin{sphinxVerbatim}[commandchars=\\\{\}]
\PYG{n+nb+bp}{self}\PYG{o}{.}\PYG{n}{release}\PYG{p}{(}\PYG{p}{)}
\end{sphinxVerbatim}

The effect is that salabim then tries to honor the next pending request, if any.

The statistics are maintained in a system queue, called clerk.requesters().

The output is very similar to the earlier example. The statistics are exactly the same.


\section{The bank office example with balking and reneging}
\label{\detokenize{Modeling:the-bank-office-example-with-balking-and-reneging}}
Now, we assume that clients are not going to the queue when there are more than 5 clients
waiting (balking). On top of that, if a client is waiting longer than 50, he/she will
leave as well (reneging).

The model code is:

\begin{sphinxVerbatim}[commandchars=\\\{\}]
\PYG{c+c1}{\PYGZsh{} Example \PYGZhy{} bank, 3 clerks, reneging.py}
\PYG{k+kn}{import} \PYG{n+nn}{salabim} \PYG{k}{as} \PYG{n+nn}{sim}


\PYG{k}{class} \PYG{n+nc}{CustomerGenerator}\PYG{p}{(}\PYG{n}{sim}\PYG{o}{.}\PYG{n}{Component}\PYG{p}{)}\PYG{p}{:}
    \PYG{k}{def} \PYG{n+nf}{process}\PYG{p}{(}\PYG{n+nb+bp}{self}\PYG{p}{)}\PYG{p}{:}
        \PYG{k}{while} \PYG{k+kc}{True}\PYG{p}{:}
            \PYG{n}{Customer}\PYG{p}{(}\PYG{p}{)}
            \PYG{k}{yield} \PYG{n+nb+bp}{self}\PYG{o}{.}\PYG{n}{hold}\PYG{p}{(}\PYG{n}{sim}\PYG{o}{.}\PYG{n}{Uniform}\PYG{p}{(}\PYG{l+m+mi}{5}\PYG{p}{,} \PYG{l+m+mi}{15}\PYG{p}{)}\PYG{o}{.}\PYG{n}{sample}\PYG{p}{(}\PYG{p}{)}\PYG{p}{)}


\PYG{k}{class} \PYG{n+nc}{Customer}\PYG{p}{(}\PYG{n}{sim}\PYG{o}{.}\PYG{n}{Component}\PYG{p}{)}\PYG{p}{:}
    \PYG{k}{def} \PYG{n+nf}{process}\PYG{p}{(}\PYG{n+nb+bp}{self}\PYG{p}{)}\PYG{p}{:}
        \PYG{k}{if} \PYG{n+nb}{len}\PYG{p}{(}\PYG{n}{waitingline}\PYG{p}{)} \PYG{o}{\PYGZgt{}}\PYG{o}{=} \PYG{l+m+mi}{5}\PYG{p}{:}
            \PYG{n}{env}\PYG{o}{.}\PYG{n}{number\PYGZus{}balked} \PYG{o}{+}\PYG{o}{=} \PYG{l+m+mi}{1}
            \PYG{n}{env}\PYG{o}{.}\PYG{n}{print\PYGZus{}trace}\PYG{p}{(}\PYG{l+s+s1}{\PYGZsq{}}\PYG{l+s+s1}{\PYGZsq{}}\PYG{p}{,} \PYG{l+s+s1}{\PYGZsq{}}\PYG{l+s+s1}{\PYGZsq{}}\PYG{p}{,} \PYG{l+s+s1}{\PYGZsq{}}\PYG{l+s+s1}{balked}\PYG{l+s+s1}{\PYGZsq{}}\PYG{p}{)}
            \PYG{k}{yield} \PYG{n+nb+bp}{self}\PYG{o}{.}\PYG{n}{cancel}\PYG{p}{(}\PYG{p}{)}
        \PYG{n+nb+bp}{self}\PYG{o}{.}\PYG{n}{enter}\PYG{p}{(}\PYG{n}{waitingline}\PYG{p}{)}
        \PYG{k}{for} \PYG{n}{clerk} \PYG{o+ow}{in} \PYG{n}{clerks}\PYG{p}{:}
            \PYG{k}{if} \PYG{n}{clerk}\PYG{o}{.}\PYG{n}{ispassive}\PYG{p}{(}\PYG{p}{)}\PYG{p}{:}
                \PYG{n}{clerk}\PYG{o}{.}\PYG{n}{activate}\PYG{p}{(}\PYG{p}{)}
                \PYG{k}{break}  \PYG{c+c1}{\PYGZsh{} activate only one clerk}
        \PYG{k}{yield} \PYG{n+nb+bp}{self}\PYG{o}{.}\PYG{n}{hold}\PYG{p}{(}\PYG{l+m+mi}{50}\PYG{p}{)}  \PYG{c+c1}{\PYGZsh{} if not serviced within this time, renege}
        \PYG{k}{if} \PYG{n+nb+bp}{self} \PYG{o+ow}{in} \PYG{n}{waitingline}\PYG{p}{:}
            \PYG{n+nb+bp}{self}\PYG{o}{.}\PYG{n}{leave}\PYG{p}{(}\PYG{n}{waitingline}\PYG{p}{)}
            \PYG{n}{env}\PYG{o}{.}\PYG{n}{number\PYGZus{}reneged} \PYG{o}{+}\PYG{o}{=} \PYG{l+m+mi}{1}
            \PYG{n}{env}\PYG{o}{.}\PYG{n}{print\PYGZus{}trace}\PYG{p}{(}\PYG{l+s+s1}{\PYGZsq{}}\PYG{l+s+s1}{\PYGZsq{}}\PYG{p}{,} \PYG{l+s+s1}{\PYGZsq{}}\PYG{l+s+s1}{\PYGZsq{}}\PYG{p}{,} \PYG{l+s+s1}{\PYGZsq{}}\PYG{l+s+s1}{reneged}\PYG{l+s+s1}{\PYGZsq{}}\PYG{p}{)}
        \PYG{k}{else}\PYG{p}{:}
            \PYG{k}{yield} \PYG{n+nb+bp}{self}\PYG{o}{.}\PYG{n}{passivate}\PYG{p}{(}\PYG{p}{)}  \PYG{c+c1}{\PYGZsh{} wait for service to be completed}


\PYG{k}{class} \PYG{n+nc}{Clerk}\PYG{p}{(}\PYG{n}{sim}\PYG{o}{.}\PYG{n}{Component}\PYG{p}{)}\PYG{p}{:}
    \PYG{k}{def} \PYG{n+nf}{process}\PYG{p}{(}\PYG{n+nb+bp}{self}\PYG{p}{)}\PYG{p}{:}
        \PYG{k}{while} \PYG{k+kc}{True}\PYG{p}{:}
            \PYG{k}{while} \PYG{n+nb}{len}\PYG{p}{(}\PYG{n}{waitingline}\PYG{p}{)} \PYG{o}{==} \PYG{l+m+mi}{0}\PYG{p}{:}
                \PYG{k}{yield} \PYG{n+nb+bp}{self}\PYG{o}{.}\PYG{n}{passivate}\PYG{p}{(}\PYG{p}{)}
            \PYG{n+nb+bp}{self}\PYG{o}{.}\PYG{n}{customer} \PYG{o}{=} \PYG{n}{waitingline}\PYG{o}{.}\PYG{n}{pop}\PYG{p}{(}\PYG{p}{)}
            \PYG{n+nb+bp}{self}\PYG{o}{.}\PYG{n}{customer}\PYG{o}{.}\PYG{n}{activate}\PYG{p}{(}\PYG{p}{)}  \PYG{c+c1}{\PYGZsh{} get the customer out of it\PYGZsq{}s hold(50)}
            \PYG{k}{yield} \PYG{n+nb+bp}{self}\PYG{o}{.}\PYG{n}{hold}\PYG{p}{(}\PYG{l+m+mi}{30}\PYG{p}{)}
            \PYG{n+nb+bp}{self}\PYG{o}{.}\PYG{n}{customer}\PYG{o}{.}\PYG{n}{activate}\PYG{p}{(}\PYG{p}{)}  \PYG{c+c1}{\PYGZsh{} signal the customer that\PYGZsq{}s all\PYGZsq{}s done}


\PYG{n}{env} \PYG{o}{=} \PYG{n}{sim}\PYG{o}{.}\PYG{n}{Environment}\PYG{p}{(}\PYG{n}{trace}\PYG{o}{=}\PYG{k+kc}{False}\PYG{p}{)}
\PYG{n}{CustomerGenerator}\PYG{p}{(}\PYG{n}{name}\PYG{o}{=}\PYG{l+s+s1}{\PYGZsq{}}\PYG{l+s+s1}{customergenerator}\PYG{l+s+s1}{\PYGZsq{}}\PYG{p}{)}
\PYG{n}{env}\PYG{o}{.}\PYG{n}{number\PYGZus{}balked} \PYG{o}{=} \PYG{l+m+mi}{0}
\PYG{n}{env}\PYG{o}{.}\PYG{n}{number\PYGZus{}reneged} \PYG{o}{=} \PYG{l+m+mi}{0}
\PYG{n}{clerks} \PYG{o}{=} \PYG{p}{[}\PYG{n}{Clerk}\PYG{p}{(}\PYG{p}{)} \PYG{k}{for} \PYG{n}{\PYGZus{}} \PYG{o+ow}{in} \PYG{n+nb}{range}\PYG{p}{(}\PYG{l+m+mi}{3}\PYG{p}{)}\PYG{p}{]}

\PYG{n}{waitingline} \PYG{o}{=} \PYG{n}{sim}\PYG{o}{.}\PYG{n}{Queue}\PYG{p}{(}\PYG{l+s+s1}{\PYGZsq{}}\PYG{l+s+s1}{waitingline}\PYG{l+s+s1}{\PYGZsq{}}\PYG{p}{)}
\PYG{n}{waitingline}\PYG{o}{.}\PYG{n}{length}\PYG{o}{.}\PYG{n}{monitor}\PYG{p}{(}\PYG{k+kc}{False}\PYG{p}{)}
\PYG{n}{env}\PYG{o}{.}\PYG{n}{run}\PYG{p}{(}\PYG{n}{duration}\PYG{o}{=}\PYG{l+m+mi}{1500}\PYG{p}{)}  \PYG{c+c1}{\PYGZsh{} first do a prerun of 1500 time units without collecting data}
\PYG{n}{waitingline}\PYG{o}{.}\PYG{n}{length}\PYG{o}{.}\PYG{n}{monitor}\PYG{p}{(}\PYG{k+kc}{True}\PYG{p}{)}
\PYG{n}{env}\PYG{o}{.}\PYG{n}{run}\PYG{p}{(}\PYG{n}{duration}\PYG{o}{=}\PYG{l+m+mi}{1500}\PYG{p}{)}  \PYG{c+c1}{\PYGZsh{} now do the actual data collection for 1500 time units}
\PYG{n}{waitingline}\PYG{o}{.}\PYG{n}{length}\PYG{o}{.}\PYG{n}{print\PYGZus{}histogram}\PYG{p}{(}\PYG{l+m+mi}{30}\PYG{p}{,} \PYG{l+m+mi}{0}\PYG{p}{,} \PYG{l+m+mi}{1}\PYG{p}{)}
\PYG{n+nb}{print}\PYG{p}{(}\PYG{p}{)}
\PYG{n}{waitingline}\PYG{o}{.}\PYG{n}{length\PYGZus{}of\PYGZus{}stay}\PYG{o}{.}\PYG{n}{print\PYGZus{}histogram}\PYG{p}{(}\PYG{l+m+mi}{30}\PYG{p}{,} \PYG{l+m+mi}{0}\PYG{p}{,} \PYG{l+m+mi}{10}\PYG{p}{)}
\PYG{n+nb}{print}\PYG{p}{(}\PYG{l+s+s1}{\PYGZsq{}}\PYG{l+s+s1}{number reneged}\PYG{l+s+s1}{\PYGZsq{}}\PYG{p}{,} \PYG{n}{env}\PYG{o}{.}\PYG{n}{number\PYGZus{}reneged}\PYG{p}{)}
\PYG{n+nb}{print}\PYG{p}{(}\PYG{l+s+s1}{\PYGZsq{}}\PYG{l+s+s1}{number balked}\PYG{l+s+s1}{\PYGZsq{}}\PYG{p}{,} \PYG{n}{env}\PYG{o}{.}\PYG{n}{number\PYGZus{}balked}\PYG{p}{)}

\end{sphinxVerbatim}

Let’s look at some details.

\begin{sphinxVerbatim}[commandchars=\\\{\}]
\PYG{k}{yield} \PYG{n+nb+bp}{self}\PYG{o}{.}\PYG{n}{cancel}\PYG{p}{(}\PYG{p}{)}
\end{sphinxVerbatim}

This makes the current component (a customer) a data component (and be subject to
garbage collection), if the queue length is 5 or more.

The reneging is implemented by a hold of 50. If a clerk can service a customer, it will take
the customer out of the waitingline and will activate it at that moment. The customer just has to check
whether he/she is still in the waiting line. If so, he/she has been serviced in time and thus will renege.

\begin{sphinxVerbatim}[commandchars=\\\{\}]
\PYG{k}{yield} \PYG{n+nb+bp}{self}\PYG{o}{.}\PYG{n}{hold}\PYG{p}{(}\PYG{l+m+mi}{50}\PYG{p}{)}
\PYG{k}{if} \PYG{n+nb+bp}{self} \PYG{o+ow}{in} \PYG{n}{waitingline}\PYG{p}{:}
    \PYG{n+nb+bp}{self}\PYG{o}{.}\PYG{n}{leave}\PYG{p}{(}\PYG{n}{waitingline}\PYG{p}{)}
    \PYG{n}{env}\PYG{o}{.}\PYG{n}{number\PYGZus{}reneged} \PYG{o}{+}\PYG{o}{=} \PYG{l+m+mi}{1}
\PYG{k}{else}\PYG{p}{:}
     \PYG{n+nb+bp}{self}\PYG{o}{.}\PYG{n}{passivate}\PYG{p}{(}\PYG{p}{)}
\end{sphinxVerbatim}

All the clerk has to do when starting servicing a client is to get the next customer in line
out of the queue (as before) and activate this customer (at time now). The effect is that the hold
of the customer will end.

\begin{sphinxVerbatim}[commandchars=\\\{\}]
\PYG{n+nb+bp}{self}\PYG{o}{.}\PYG{n}{customer} \PYG{o}{=} \PYG{n}{waitingline}\PYG{o}{.}\PYG{n}{pop}\PYG{p}{(}\PYG{p}{)}
\PYG{n+nb+bp}{self}\PYG{o}{.}\PYG{n}{customer}\PYG{o}{.}\PYG{n}{activate}\PYG{p}{(}\PYG{p}{)}
\end{sphinxVerbatim}


\section{The bank office example with balking and reneging (resources)}
\label{\detokenize{Modeling:the-bank-office-example-with-balking-and-reneging-resources}}
Now we show how the balking and reneging is implemented with resources.

The model code is:

\begin{sphinxVerbatim}[commandchars=\\\{\}]
\PYG{c+c1}{\PYGZsh{} Example \PYGZhy{} bank, 3 clerks, reneging (resources).py}
\PYG{k+kn}{import} \PYG{n+nn}{salabim} \PYG{k}{as} \PYG{n+nn}{sim}


\PYG{k}{class} \PYG{n+nc}{CustomerGenerator}\PYG{p}{(}\PYG{n}{sim}\PYG{o}{.}\PYG{n}{Component}\PYG{p}{)}\PYG{p}{:}
    \PYG{k}{def} \PYG{n+nf}{process}\PYG{p}{(}\PYG{n+nb+bp}{self}\PYG{p}{)}\PYG{p}{:}
        \PYG{k}{while} \PYG{k+kc}{True}\PYG{p}{:}
            \PYG{n}{Customer}\PYG{p}{(}\PYG{p}{)}
            \PYG{k}{yield} \PYG{n+nb+bp}{self}\PYG{o}{.}\PYG{n}{hold}\PYG{p}{(}\PYG{n}{sim}\PYG{o}{.}\PYG{n}{Uniform}\PYG{p}{(}\PYG{l+m+mi}{5}\PYG{p}{,} \PYG{l+m+mi}{15}\PYG{p}{)}\PYG{o}{.}\PYG{n}{sample}\PYG{p}{(}\PYG{p}{)}\PYG{p}{)}


\PYG{k}{class} \PYG{n+nc}{Customer}\PYG{p}{(}\PYG{n}{sim}\PYG{o}{.}\PYG{n}{Component}\PYG{p}{)}\PYG{p}{:}
    \PYG{k}{def} \PYG{n+nf}{process}\PYG{p}{(}\PYG{n+nb+bp}{self}\PYG{p}{)}\PYG{p}{:}
        \PYG{k}{if} \PYG{n+nb}{len}\PYG{p}{(}\PYG{n}{clerks}\PYG{o}{.}\PYG{n}{requesters}\PYG{p}{(}\PYG{p}{)}\PYG{p}{)} \PYG{o}{\PYGZgt{}}\PYG{o}{=} \PYG{l+m+mi}{5}\PYG{p}{:}
            \PYG{n}{env}\PYG{o}{.}\PYG{n}{number\PYGZus{}balked} \PYG{o}{+}\PYG{o}{=} \PYG{l+m+mi}{1}
            \PYG{n}{env}\PYG{o}{.}\PYG{n}{print\PYGZus{}trace}\PYG{p}{(}\PYG{l+s+s1}{\PYGZsq{}}\PYG{l+s+s1}{\PYGZsq{}}\PYG{p}{,} \PYG{l+s+s1}{\PYGZsq{}}\PYG{l+s+s1}{\PYGZsq{}}\PYG{p}{,} \PYG{l+s+s1}{\PYGZsq{}}\PYG{l+s+s1}{balked}\PYG{l+s+s1}{\PYGZsq{}}\PYG{p}{)}
            \PYG{k}{yield} \PYG{n+nb+bp}{self}\PYG{o}{.}\PYG{n}{cancel}\PYG{p}{(}\PYG{p}{)}
        \PYG{k}{yield} \PYG{n+nb+bp}{self}\PYG{o}{.}\PYG{n}{request}\PYG{p}{(}\PYG{n}{clerks}\PYG{p}{,} \PYG{n}{fail\PYGZus{}delay}\PYG{o}{=}\PYG{l+m+mi}{50}\PYG{p}{)}
        \PYG{k}{if} \PYG{n+nb+bp}{self}\PYG{o}{.}\PYG{n}{failed}\PYG{p}{(}\PYG{p}{)}\PYG{p}{:}
            \PYG{n}{env}\PYG{o}{.}\PYG{n}{number\PYGZus{}reneged} \PYG{o}{+}\PYG{o}{=} \PYG{l+m+mi}{1}
            \PYG{n}{env}\PYG{o}{.}\PYG{n}{print\PYGZus{}trace}\PYG{p}{(}\PYG{l+s+s1}{\PYGZsq{}}\PYG{l+s+s1}{\PYGZsq{}}\PYG{p}{,} \PYG{l+s+s1}{\PYGZsq{}}\PYG{l+s+s1}{\PYGZsq{}}\PYG{p}{,} \PYG{l+s+s1}{\PYGZsq{}}\PYG{l+s+s1}{reneged}\PYG{l+s+s1}{\PYGZsq{}}\PYG{p}{)}
        \PYG{k}{else}\PYG{p}{:}
            \PYG{k}{yield} \PYG{n+nb+bp}{self}\PYG{o}{.}\PYG{n}{hold}\PYG{p}{(}\PYG{l+m+mi}{30}\PYG{p}{)}
            \PYG{n+nb+bp}{self}\PYG{o}{.}\PYG{n}{release}\PYG{p}{(}\PYG{p}{)}


\PYG{n}{env} \PYG{o}{=} \PYG{n}{sim}\PYG{o}{.}\PYG{n}{Environment}\PYG{p}{(}\PYG{n}{trace}\PYG{o}{=}\PYG{k+kc}{False}\PYG{p}{)}
\PYG{n}{CustomerGenerator}\PYG{p}{(}\PYG{n}{name}\PYG{o}{=}\PYG{l+s+s1}{\PYGZsq{}}\PYG{l+s+s1}{customergenerator}\PYG{l+s+s1}{\PYGZsq{}}\PYG{p}{)}
\PYG{n}{env}\PYG{o}{.}\PYG{n}{number\PYGZus{}balked} \PYG{o}{=} \PYG{l+m+mi}{0}
\PYG{n}{env}\PYG{o}{.}\PYG{n}{number\PYGZus{}reneged} \PYG{o}{=} \PYG{l+m+mi}{0}
\PYG{n}{clerks} \PYG{o}{=} \PYG{n}{sim}\PYG{o}{.}\PYG{n}{Resource}\PYG{p}{(}\PYG{l+s+s1}{\PYGZsq{}}\PYG{l+s+s1}{clerk}\PYG{l+s+s1}{\PYGZsq{}}\PYG{p}{,} \PYG{l+m+mi}{3}\PYG{p}{)}

\PYG{n}{env}\PYG{o}{.}\PYG{n}{run}\PYG{p}{(}\PYG{n}{till}\PYG{o}{=}\PYG{l+m+mi}{50000}\PYG{p}{)}

\PYG{n}{clerks}\PYG{o}{.}\PYG{n}{requesters}\PYG{p}{(}\PYG{p}{)}\PYG{o}{.}\PYG{n}{length}\PYG{o}{.}\PYG{n}{print\PYGZus{}histogram}\PYG{p}{(}\PYG{l+m+mi}{30}\PYG{p}{,} \PYG{l+m+mi}{0}\PYG{p}{,} \PYG{l+m+mi}{1}\PYG{p}{)}
\PYG{n+nb}{print}\PYG{p}{(}\PYG{p}{)}
\PYG{n}{clerks}\PYG{o}{.}\PYG{n}{requesters}\PYG{p}{(}\PYG{p}{)}\PYG{o}{.}\PYG{n}{length\PYGZus{}of\PYGZus{}stay}\PYG{o}{.}\PYG{n}{print\PYGZus{}histogram}\PYG{p}{(}\PYG{l+m+mi}{30}\PYG{p}{,} \PYG{l+m+mi}{0}\PYG{p}{,} \PYG{l+m+mi}{10}\PYG{p}{)}
\PYG{n+nb}{print}\PYG{p}{(}\PYG{l+s+s1}{\PYGZsq{}}\PYG{l+s+s1}{number reneged}\PYG{l+s+s1}{\PYGZsq{}}\PYG{p}{,} \PYG{n}{env}\PYG{o}{.}\PYG{n}{number\PYGZus{}reneged}\PYG{p}{)}
\PYG{n+nb}{print}\PYG{p}{(}\PYG{l+s+s1}{\PYGZsq{}}\PYG{l+s+s1}{number balked}\PYG{l+s+s1}{\PYGZsq{}}\PYG{p}{,} \PYG{n}{env}\PYG{o}{.}\PYG{n}{number\PYGZus{}balked}\PYG{p}{)}
\end{sphinxVerbatim}

As you can see, the balking part is exactly the same as in the example without resources.

For the renenging, all we have to do is add a fail\_delay

\begin{sphinxVerbatim}[commandchars=\\\{\}]
\PYG{k}{yield} \PYG{n+nb+bp}{self}\PYG{o}{.}\PYG{n}{request}\PYG{p}{(}\PYG{n}{clerks}\PYG{p}{,} \PYG{n}{fail\PYGZus{}delay}\PYG{o}{=}\PYG{l+m+mi}{50}\PYG{p}{)}
\end{sphinxVerbatim}

If the request is not honored within 50 time units, the process continues after that request statement.
And then, we just check whether the request has failed

\begin{sphinxVerbatim}[commandchars=\\\{\}]
\PYG{k}{if} \PYG{n+nb+bp}{self}\PYG{o}{.}\PYG{n}{failed}\PYG{p}{(}\PYG{p}{)}\PYG{p}{:}
    \PYG{n}{env}\PYG{o}{.}\PYG{n}{number\PYGZus{}reneged} \PYG{o}{+}\PYG{o}{=} \PYG{l+m+mi}{1}
\end{sphinxVerbatim}

This example shows clearly the advantage of the resource solution over the passivate/activate method, in this example.


\section{The bank office example with states}
\label{\detokenize{Modeling:the-bank-office-example-with-states}}
The salabim package contains yet another useful concept for modelling: states.
In this case, we define a state called worktodo.

The model code is:

\begin{sphinxVerbatim}[commandchars=\\\{\}]
\PYG{c+c1}{\PYGZsh{} Example \PYGZhy{} bank, 3 clerks (state).py}
\PYG{k+kn}{import} \PYG{n+nn}{salabim} \PYG{k}{as} \PYG{n+nn}{sim}


\PYG{k}{class} \PYG{n+nc}{CustomerGenerator}\PYG{p}{(}\PYG{n}{sim}\PYG{o}{.}\PYG{n}{Component}\PYG{p}{)}\PYG{p}{:}
    \PYG{k}{def} \PYG{n+nf}{process}\PYG{p}{(}\PYG{n+nb+bp}{self}\PYG{p}{)}\PYG{p}{:}
        \PYG{k}{while} \PYG{k+kc}{True}\PYG{p}{:}
            \PYG{n}{Customer}\PYG{p}{(}\PYG{p}{)}
            \PYG{k}{yield} \PYG{n+nb+bp}{self}\PYG{o}{.}\PYG{n}{hold}\PYG{p}{(}\PYG{n}{sim}\PYG{o}{.}\PYG{n}{Uniform}\PYG{p}{(}\PYG{l+m+mi}{5}\PYG{p}{,} \PYG{l+m+mi}{15}\PYG{p}{)}\PYG{o}{.}\PYG{n}{sample}\PYG{p}{(}\PYG{p}{)}\PYG{p}{)}


\PYG{k}{class} \PYG{n+nc}{Customer}\PYG{p}{(}\PYG{n}{sim}\PYG{o}{.}\PYG{n}{Component}\PYG{p}{)}\PYG{p}{:}
    \PYG{k}{def} \PYG{n+nf}{process}\PYG{p}{(}\PYG{n+nb+bp}{self}\PYG{p}{)}\PYG{p}{:}
        \PYG{n+nb+bp}{self}\PYG{o}{.}\PYG{n}{enter}\PYG{p}{(}\PYG{n}{waitingline}\PYG{p}{)}
        \PYG{n}{worktodo}\PYG{o}{.}\PYG{n}{trigger}\PYG{p}{(}\PYG{n+nb}{max}\PYG{o}{=}\PYG{l+m+mi}{1}\PYG{p}{)}
        \PYG{k}{yield} \PYG{n+nb+bp}{self}\PYG{o}{.}\PYG{n}{passivate}\PYG{p}{(}\PYG{p}{)}


\PYG{k}{class} \PYG{n+nc}{Clerk}\PYG{p}{(}\PYG{n}{sim}\PYG{o}{.}\PYG{n}{Component}\PYG{p}{)}\PYG{p}{:}
    \PYG{k}{def} \PYG{n+nf}{process}\PYG{p}{(}\PYG{n+nb+bp}{self}\PYG{p}{)}\PYG{p}{:}
        \PYG{k}{while} \PYG{k+kc}{True}\PYG{p}{:}
            \PYG{k}{if} \PYG{n+nb}{len}\PYG{p}{(}\PYG{n}{waitingline}\PYG{p}{)} \PYG{o}{==} \PYG{l+m+mi}{0}\PYG{p}{:}
                \PYG{k}{yield} \PYG{n+nb+bp}{self}\PYG{o}{.}\PYG{n}{wait}\PYG{p}{(}\PYG{n}{worktodo}\PYG{p}{)}
            \PYG{n+nb+bp}{self}\PYG{o}{.}\PYG{n}{customer} \PYG{o}{=} \PYG{n}{waitingline}\PYG{o}{.}\PYG{n}{pop}\PYG{p}{(}\PYG{p}{)}
            \PYG{k}{yield} \PYG{n+nb+bp}{self}\PYG{o}{.}\PYG{n}{hold}\PYG{p}{(}\PYG{l+m+mi}{30}\PYG{p}{)}
            \PYG{n+nb+bp}{self}\PYG{o}{.}\PYG{n}{customer}\PYG{o}{.}\PYG{n}{activate}\PYG{p}{(}\PYG{p}{)}


\PYG{n}{env} \PYG{o}{=} \PYG{n}{sim}\PYG{o}{.}\PYG{n}{Environment}\PYG{p}{(}\PYG{n}{trace}\PYG{o}{=}\PYG{k+kc}{False}\PYG{p}{)}
\PYG{n}{CustomerGenerator}\PYG{p}{(}\PYG{n}{name}\PYG{o}{=}\PYG{l+s+s1}{\PYGZsq{}}\PYG{l+s+s1}{customergenerator}\PYG{l+s+s1}{\PYGZsq{}}\PYG{p}{)}
\PYG{k}{for} \PYG{n}{i} \PYG{o+ow}{in} \PYG{n+nb}{range}\PYG{p}{(}\PYG{l+m+mi}{3}\PYG{p}{)}\PYG{p}{:}
    \PYG{n}{Clerk}\PYG{p}{(}\PYG{p}{)}
\PYG{n}{waitingline} \PYG{o}{=} \PYG{n}{sim}\PYG{o}{.}\PYG{n}{Queue}\PYG{p}{(}\PYG{l+s+s1}{\PYGZsq{}}\PYG{l+s+s1}{waitingline}\PYG{l+s+s1}{\PYGZsq{}}\PYG{p}{)}
\PYG{n}{worktodo} \PYG{o}{=} \PYG{n}{sim}\PYG{o}{.}\PYG{n}{State}\PYG{p}{(}\PYG{l+s+s1}{\PYGZsq{}}\PYG{l+s+s1}{worktodo}\PYG{l+s+s1}{\PYGZsq{}}\PYG{p}{)}

\PYG{n}{env}\PYG{o}{.}\PYG{n}{run}\PYG{p}{(}\PYG{n}{till}\PYG{o}{=}\PYG{l+m+mi}{50000}\PYG{p}{)}
\PYG{n}{waitingline}\PYG{o}{.}\PYG{n}{print\PYGZus{}histograms}\PYG{p}{(}\PYG{p}{)}
\PYG{n}{worktodo}\PYG{o}{.}\PYG{n}{print\PYGZus{}histograms}\PYG{p}{(}\PYG{p}{)}
\end{sphinxVerbatim}

Let’s look at some details.

\begin{sphinxVerbatim}[commandchars=\\\{\}]
\PYG{n}{worktodo} \PYG{o}{=} \PYG{n}{sim}\PYG{o}{.}\PYG{n}{State}\PYG{p}{(}\PYG{l+s+s1}{\PYGZsq{}}\PYG{l+s+s1}{worktodo}\PYG{l+s+s1}{\PYGZsq{}}\PYG{p}{)}
\end{sphinxVerbatim}

This defines a state with an initial value False.

In the code of the customer, the customer tries to trigger one clerk with

\begin{sphinxVerbatim}[commandchars=\\\{\}]
\PYG{n}{worktodo}\PYG{o}{.}\PYG{n}{trigger}\PYG{p}{(}\PYG{n+nb}{max}\PYG{o}{=}\PYG{l+m+mi}{1}\PYG{p}{)}
\end{sphinxVerbatim}

The effect is that if there are clerks waiting for worktodo, the first clerk’s wait is honored and
that clerk continues its process after

\begin{sphinxVerbatim}[commandchars=\\\{\}]
\PYG{k}{yield} \PYG{n+nb+bp}{self}\PYG{o}{.}\PYG{n}{wait}\PYG{p}{(}\PYG{n}{worktodo}\PYG{p}{)}
\end{sphinxVerbatim}

Note that the clerk is only going to wait for worktodo after completion of a job if there
are no customers waiting.


\section{The bank office example with standby}
\label{\detokenize{Modeling:the-bank-office-example-with-standby}}
The salabim package contains yet another powerful process mechanism, called standby. When a component
is in standby mode, it will become current after \sphinxstyleemphasis{each} event. Normally, the standby will be
used in a while loop where at every event one or more conditions are checked.

The model with standby is

\begin{sphinxVerbatim}[commandchars=\\\{\}]
\PYG{o}{.}\PYG{o}{.} \PYG{n}{literalinclude}\PYG{p}{:}\PYG{p}{:} \PYG{o}{.}\PYG{o}{.}\PYGZbs{}\PYG{o}{.}\PYG{o}{.}\PYGZbs{}\PYG{n}{Example} \PYG{o}{\PYGZhy{}} \PYG{n}{bank}\PYG{p}{,} \PYG{l+m+mi}{3} \PYG{n}{clerks} \PYG{p}{(}\PYG{n}{standby}\PYG{o}{.}\PYG{n}{py}\PYG{p}{)}
\end{sphinxVerbatim}

In this case, the condition is checked frequently with

\begin{sphinxVerbatim}[commandchars=\\\{\}]
\PYG{k}{while} \PYG{n+nb}{len}\PYG{p}{(}\PYG{n}{waitingline}\PYG{p}{)} \PYG{o}{==} \PYG{l+m+mi}{0}\PYG{p}{:}
    \PYG{k}{yield} \PYG{n+nb+bp}{self}\PYG{o}{.}\PYG{n}{standby}\PYG{p}{(}\PYG{p}{)}
\end{sphinxVerbatim}

The rest of the code is very similar to the version with states.

\begin{sphinxadmonition}{warning}{Warning:}
It is very important to realize that this mechanism can have significant impact on the performance,
as after EACH event, the component becomes current and has to be checked.
In general it is recommended to try and use states or a more straightforward passivate/activate
construction.
\end{sphinxadmonition}


\chapter{Component}
\label{\detokenize{Component:component}}\label{\detokenize{Component::doc}}
Components are the key elements of salabim simulations.

Components can be either data or active. An active component has one or more process descriptions and is activated
at some point of time. You can make a data component active with activate. And an active component can become
data either with a cancel or by reaching the end of its process method.

It is easy to create a data component by:

\begin{sphinxVerbatim}[commandchars=\\\{\}]
\PYG{n}{data\PYGZus{}component} \PYG{o}{=} \PYG{n}{sim}\PYG{o}{.}\PYG{n}{Component}\PYG{p}{(}\PYG{p}{)}
\end{sphinxVerbatim}

Data components may be placed in a queue. You can’t activate this component as such as there is no associated process method.

In order to make an active component it is necessary to first define a class:

\begin{sphinxVerbatim}[commandchars=\\\{\}]
\PYG{k}{class} \PYG{n+nc}{Ship}\PYG{p}{(}\PYG{n}{sim}\PYG{o}{.}\PYG{n}{Component}\PYG{p}{)}\PYG{p}{:}
\end{sphinxVerbatim}

And then there has to be at least one generator method, normally called process:

\begin{sphinxVerbatim}[commandchars=\\\{\}]
\PYG{k}{class} \PYG{n+nc}{Ship}\PYG{p}{(}\PYG{n}{sim}\PYG{o}{.}\PYG{n}{Component}\PYG{p}{)}\PYG{p}{:}
    \PYG{k}{def} \PYG{n+nf}{process}\PYG{p}{(}\PYG{n+nb+bp}{self}\PYG{p}{)}\PYG{p}{:}
        \PYG{o}{.}\PYG{o}{.}\PYG{o}{.}\PYG{o}{.}
        \PYG{k}{yield} \PYG{o}{.}\PYG{o}{.}\PYG{o}{.}
        \PYG{o}{.}\PYG{o}{.}\PYG{o}{.}\PYG{o}{.}
\end{sphinxVerbatim}

The process has to have at least one yield (or yield from) statement!

Creation and activation can be combined by making a new instance of the class:

\begin{sphinxVerbatim}[commandchars=\\\{\}]
\PYG{n}{ship1} \PYG{o}{=} \PYG{n}{Ship}\PYG{p}{(}\PYG{p}{)}
\PYG{n}{ship2} \PYG{o}{=} \PYG{n}{Ship}\PYG{p}{(}\PYG{p}{)}
\PYG{n}{ship3} \PYG{o}{=} \PYG{n}{Ship}\PYG{p}{(}\PYG{p}{)}
\end{sphinxVerbatim}

This causes three Ships to be created and to start them at Sim.process().
The ships will automatically get the name \sphinxcode{ship.0}, etc., unless a name
is given explicitly.

If no process method is found for Ship, the ship will be a data component.
In that case, it becomes active by means of an activate statement:

\begin{sphinxVerbatim}[commandchars=\\\{\}]
\PYG{k}{class} \PYG{n+nc}{Crane}\PYG{p}{(}\PYG{n}{sim}\PYG{o}{.}\PYG{n}{Component}\PYG{p}{)}\PYG{p}{:}
    \PYG{k}{def} \PYG{n+nf}{unload}\PYG{p}{(}\PYG{n+nb+bp}{self}\PYG{p}{)}\PYG{p}{:}
        \PYG{o}{.}\PYG{o}{.}\PYG{o}{.}\PYG{o}{.}
        \PYG{k}{yield} \PYG{o}{.}\PYG{o}{.}\PYG{o}{.}
        \PYG{o}{.}\PYG{o}{.}\PYG{o}{.}\PYG{o}{.}

\PYG{n}{crane1} \PYG{o}{=} \PYG{n}{Crane}\PYG{p}{(}\PYG{p}{)}
\PYG{n}{crane1}\PYG{o}{.}\PYG{n}{activate}\PYG{p}{(}\PYG{n}{process}\PYG{o}{=}\PYG{l+s+s1}{\PYGZsq{}}\PYG{l+s+s1}{unload}\PYG{l+s+s1}{\PYGZsq{}}\PYG{p}{)}

\PYG{n}{crane2} \PYG{o}{=} \PYG{n}{Crane}\PYG{p}{(}\PYG{n}{process}\PYG{o}{=}\PYG{l+s+s1}{\PYGZsq{}}\PYG{l+s+s1}{unload}\PYG{l+s+s1}{\PYGZsq{}}\PYG{p}{)}
\end{sphinxVerbatim}

Effectively, creation and start of crane1 and crane2 is the same.

Although not very common, it is possible to activate a component at a certain time or with a
specified delay:

\begin{sphinxVerbatim}[commandchars=\\\{\}]
\PYG{n}{ship1}\PYG{o}{.}\PYG{n}{activate}\PYG{p}{(}\PYG{n}{at}\PYG{o}{=}\PYG{l+m+mi}{100}\PYG{p}{)}
\PYG{n}{ship2}\PYG{o}{.}\PYG{n}{activate}\PYG{p}{(}\PYG{n}{delay}\PYG{o}{=}\PYG{l+m+mi}{50}\PYG{p}{)}
\end{sphinxVerbatim}

At time of creation it is sometimes useful to be able to set attributes, prepare for actions, etc.
This is possible in salabim by defining an \_\_init\_\_ and/or a setup method:

If the \_\_init\_\_ method is used, it is required to call the Component.\_\_init\_\_ method from within the
overridden method:

\begin{sphinxVerbatim}[commandchars=\\\{\}]
\PYG{k}{class} \PYG{n+nc}{Ship}\PYG{p}{(}\PYG{n}{sim}\PYG{o}{.}\PYG{n}{Component}\PYG{p}{)}\PYG{p}{:}
    \PYG{k}{def} \PYG{n+nf+fm}{\PYGZus{}\PYGZus{}init\PYGZus{}\PYGZus{}}\PYG{p}{(}\PYG{n+nb+bp}{self}\PYG{p}{,} \PYG{n}{length}\PYG{p}{,} \PYG{o}{*}\PYG{n}{args}\PYG{p}{,} \PYG{o}{*}\PYG{o}{*}\PYG{n}{kwargs}\PYG{p}{)}\PYG{p}{:}
        \PYG{n}{sim}\PYG{o}{.}\PYG{n}{Component}\PYG{o}{.}\PYG{n+nf+fm}{\PYGZus{}\PYGZus{}init\PYGZus{}\PYGZus{}}\PYG{p}{(}\PYG{n+nb+bp}{self}\PYG{p}{,} \PYG{o}{*}\PYG{n}{args}\PYG{p}{,} \PYG{o}{*}\PYG{o}{*}\PYG{n}{kwargs}\PYG{p}{)}
        \PYG{n+nb+bp}{self}\PYG{o}{.}\PYG{n}{length} \PYG{o}{=} \PYG{n}{length}

\PYG{n}{ship} \PYG{o}{=} \PYG{n}{Ship}\PYG{p}{(}\PYG{n}{length}\PYG{o}{=}\PYG{l+m+mi}{250}\PYG{p}{)}
\end{sphinxVerbatim}

This sets ship.length to 250.

In most cases, the setup method is preferred, however. This method is called after ALL initialization code
of Component is executed.

\begin{sphinxVerbatim}[commandchars=\\\{\}]
\PYG{k}{class} \PYG{n+nc}{Ship}\PYG{p}{(}\PYG{n}{sim}\PYG{o}{.}\PYG{n}{Component}\PYG{p}{)}\PYG{p}{:}
    \PYG{k}{def} \PYG{n+nf}{setup}\PYG{p}{(}\PYG{n+nb+bp}{self}\PYG{p}{,} \PYG{n}{length}\PYG{p}{)}\PYG{p}{:}
        \PYG{n+nb+bp}{self}\PYG{o}{.}\PYG{n}{length} \PYG{o}{=} \PYG{n}{length}

\PYG{n}{ship} \PYG{o}{=} \PYG{n}{Ship}\PYG{p}{(}\PYG{n}{length}\PYG{o}{=}\PYG{l+m+mi}{250}\PYG{p}{)}
\end{sphinxVerbatim}

Now, ship.length will be 250.

Note that setup gets all arguments and keyword arguments, that are not ‘consumed’  by \_\_init\_\_ and/or
the process call.

Only in very specific cases, \_\_init\_\_ will be necessary.

Note that the setup code can be used for data components as well.


\section{Process interaction}
\label{\detokenize{Component:process-interaction}}
A component may be in one of the following states:
\begin{itemize}
\item {} 
data

\item {} 
current

\item {} 
scheduled

\item {} 
passive

\item {} 
requesting

\item {} 
waiting

\item {} 
standby

\item {} 
interrupted

\end{itemize}

The scheme below shows how components can go from state to state.


\begin{savenotes}\sphinxattablestart
\centering
\begin{tabulary}{\linewidth}[t]{|T|T|T|T|T|T|T|T|T|}
\hline
\sphinxstylethead{\sphinxstyletheadfamily 
from/to
\unskip}\relax &\sphinxstylethead{\sphinxstyletheadfamily 
data
\unskip}\relax &\sphinxstylethead{\sphinxstyletheadfamily 
current
\unskip}\relax &\sphinxstylethead{\sphinxstyletheadfamily 
scheduled
\unskip}\relax &\sphinxstylethead{\sphinxstyletheadfamily 
passive
\unskip}\relax &\sphinxstylethead{\sphinxstyletheadfamily 
requesting
\unskip}\relax &\sphinxstylethead{\sphinxstyletheadfamily 
waiting
\unskip}\relax &\sphinxstylethead{\sphinxstyletheadfamily 
standby
\unskip}\relax &\sphinxstylethead{\sphinxstyletheadfamily 
interrupted
\unskip}\relax \\
\hline
data
&&
activate{[}1{]}
&
activate
&&&&&\\
\hline
current
&
process end
&&
yield hold
&
yield passivate
&
yield request
&
yield wait
&
yield standby
&\\
\hline
.
&
yield cancel
&&
yield activate
&&&&&\\
\hline
scheduled
&
cancel
&
next event
&
hold
&
passivate
&
request
&
wait
&
standby
&
interrupt
\\
\hline
.
&&&
activate
&&&&&\\
\hline
passive
&
cancel
&
activate{[}1{]}
&
activate
&&
request
&
wait
&
standby
&
interrupt
\\
\hline
.
&&&
hold{[}2{]}
&&&&&\\
\hline
requesting
&
cancel
&
claim honor
&
activate{[}3{]}
&
passivate
&
request
&
wait
&
standby
&
interrupt
\\
\hline
.
&&
time out
&&&
activate{[}4{]}
&&&\\
\hline
waiting
&
cancel
&
wait honor
&
activate{[}5{]}
&
passivate
&
wait
&
wait
&
standby
&
interrupt
\\
\hline
.
&&
timeout
&&&&
activate{[}6{]}
&&\\
\hline
standby
&
cancel
&
next event
&
activate
&
passivate
&
request
&
wait
&&
interrupt
\\
\hline
interrupted
&
cancel
&&
resume{[}7{]}
&
resume{[}7{]}
&
resume{[}7{]}
&
resume{[}7{]}
&
resume{[}7{]}
&
interrupt{[}8{]}
\\
\hline
.
&&&
activate
&
passivate
&
request
&
wait
&
standby
&\\
\hline
\end{tabulary}
\par
\sphinxattableend\end{savenotes}

{[}1{]} via scheduled 
{[}2{]} not recommended 
{[}3{]} with keep\_request=False (default) 
{[}4{]} with keep\_request=True. This allows to set a new time out 
{[}5{]} with keep\_wait=False (default) 
{[}6{]} with keep\_wait=True. This allows to set a new time out 
{[}7{]} state at time of interrupt 
{[}8{]} increases the interrupt\_level 


\subsection{Creation of a component}
\label{\detokenize{Component:creation-of-a-component}}
Although it is possible to create a component directly with \sphinxtitleref{x=sim.Component()}, this
makes it very hard to make that component into an active component,
because there’s no process method. So, nearly always we define a class based on
sim.Component

\begin{sphinxVerbatim}[commandchars=\\\{\}]
\PYG{k}{def} \PYG{n+nf}{Car}\PYG{p}{(}\PYG{n}{sim}\PYG{o}{.}\PYG{n}{Component}\PYG{p}{)}\PYG{p}{:}
    \PYG{k}{def} \PYG{n+nf}{process}\PYG{p}{(}\PYG{n+nb+bp}{self}\PYG{p}{)}\PYG{p}{:}
        \PYG{o}{.}\PYG{o}{.}\PYG{o}{.}
\end{sphinxVerbatim}

If we then say \sphinxtitleref{car=Car()}, a component is created and it activated from process. This
process has to be a generator function, so needs to contain at least one yield (or yield from) statement.

The result is that car is put on the future event list (for time now) and when it’s its
turn, the component becomes current.

It is also possible to set a time at which the component (car) becomes active, like \sphinxtitleref{car=Car(at=10)}.

And instead of starting at process, the component may be initialized to start at another generation function,
like \sphinxtitleref{car=Car(process=’wash’)}.

And, finally, if there is a process method, you can disable the automatic activation (i.e.
make it a data component) , by specifying \sphinxtitleref{process=’‘}.

If there is no process method, and process= is not given, the component becomes a data component.


\subsection{activate}
\label{\detokenize{Component:activate}}
Activate is the way to turn a data component into a live component. If you do not specify a process,
the generator function process is assumed. So you can say

\begin{sphinxVerbatim}[commandchars=\\\{\}]
\PYG{n}{car0} \PYG{o}{=} \PYG{n}{Car}\PYG{p}{(}\PYG{n}{process}\PYG{o}{=}\PYG{l+s+s1}{\PYGZsq{}}\PYG{l+s+s1}{\PYGZsq{}}\PYG{p}{)}  \PYG{c+c1}{\PYGZsh{} data component}
\PYG{n}{car0}\PYG{o}{.}\PYG{n}{activate}\PYG{p}{(}\PYG{p}{)}  \PYG{c+c1}{\PYGZsh{} activate @ process if exists, otherwise error}
\PYG{n}{car1} \PYG{o}{=} \PYG{n}{Car}\PYG{p}{(}\PYG{n}{process}\PYG{o}{=}\PYG{l+s+s1}{\PYGZsq{}}\PYG{l+s+s1}{\PYGZsq{}}\PYG{p}{)}  \PYG{c+c1}{\PYGZsh{} data component}
\PYG{n}{car1}\PYG{o}{.}\PYG{n}{activate}\PYG{p}{(}\PYG{n}{process}\PYG{o}{=}\PYG{l+s+s1}{\PYGZsq{}}\PYG{l+s+s1}{wash}\PYG{l+s+s1}{\PYGZsq{}}\PYG{p}{)}  \PYG{c+c1}{\PYGZsh{} activate @ wash}
\end{sphinxVerbatim}
\begin{itemize}
\item {} 
If the component to be activated is current, always use yield self.activate. The effect is that the
component becomes scheduled, thus this is essentially equivalent to the preferred hold method.

\item {} 
If the component to be activated is passive, the component will be activated at the specified time.

\item {} 
If the component to be activated is scheduled, the component will get a new scheduled time.

\item {} 
If the component to be activated is requesting, the request will be
terminated, the attribute failed set and the component will become scheduled. If keep\_request=True
is specified, only the fail\_at will be updated and the component will stay requesting.

\item {} 
If the component to be activated is waiting, the wait will be
terminated, the attribute failed set and the component will become scheduled. If keep\_wait=True
is specified, only the fail\_at will be updated and the component will stay waiting.

\item {} 
If the component to be activated is standby, the component will get a new scheduled time and become
scheduled.

\item {} 
If the component is interrupted, the component will be activated at the specified time.

\end{itemize}


\subsection{hold}
\label{\detokenize{Component:hold}}
Hold is the way to make a, usually current, component scheduled.
\begin{itemize}
\item {} 
If the component to be held is current, the component becomes scheduled for the specified time. Always
use yield self.hold() is this case.

\item {} 
If the component to be held is passive, the component becomes scheduled for the specified time.

\item {} 
If the component to be held is scheduled, the component will be rescheduled for the specified time, thus
essentially the same as activate.

\item {} 
If the component to be held is standby, the component becomes scheduled for the specified time.

\item {} 
If the component to be activated is requesting, the request will be terminated, the attribute failed
set and the component will become scheduled. It is recommended to use the more versatile activate method.

\item {} 
If the component to be activated is waiting, the wait will be
terminated, the attribute failed set and the component will become scheduled. It is recommended to
use the more versatile activate method.

\item {} 
If the component is interrupted, the component will be activated at the specified time.

\end{itemize}


\subsection{passivate}
\label{\detokenize{Component:passivate}}
Passivate is the way to make a, usually current, component passive. This is actually the
same as scheduling for time=inf.
\begin{itemize}
\item {} 
If the component to be passivated is current, the component becomes passive. Always
use yield seld.passivate() is this case.

\item {} 
If the component to be passivated is passive, the component remains passive.

\item {} 
If the component to be passivated is scheduled, the component becomes passive.

\item {} 
If the component to be held is standby, the component becomes passive.

\item {} 
If the component to be activated is requesting, the request will be terminated, the attribute failed
set and the component becomes passive. It is recommended to use the more versatile activate method.

\item {} 
If the component to be activated is waiting, the wait will be
terminated, the attribute failed set and the component becomes passive. It is recommended to
use the more versatile activate method.

\item {} 
If the component is interrupted, the component becomes passive.

\end{itemize}


\subsection{cancel}
\label{\detokenize{Component:cancel}}
Cancel has the effect that the component becomes a data component.
\begin{itemize}
\item {} 
If the component to be cancelled is current, always use yield self.cancel().

\item {} 
If the component to be cancelled is passive, scheduled, interrupted  or standby, the component becomes a data component.

\item {} 
If the component to be cancelled is requesting, the request will be terminated, the attribute failed
set and the component becomes a data component.

\item {} 
If the component to be cancelled is waiting, the wait will be terminated, the attribute failed
set and the component becomes a data component.

\end{itemize}


\subsection{standby}
\label{\detokenize{Component:standby}}
Standby has the effect that the component will be triggered on the next simulation event.
\begin{itemize}
\item {} 
If the component is current, use always yield self.standby()

\item {} 
Although theoretically possible it is not recommended to use standby for non current components.

\end{itemize}


\subsection{request}
\label{\detokenize{Component:request}}
Request has the effect that the component will check whether the requested quantity from a resource is available. It is
possible to check for multiple availability of a certain quantity from several resources.
By default, there is no limit on the time to wait for the resource(s) to become available. But, it is possible to set
a time with fail\_at at which the condition has to be met. If that failed, the component becomes current at the given point of time.
The code should then check whether the request had failed. That can be checked with the Component.failed() method.

If the component is canceled, activated, passivated, interrupted or held the failed flag will be set as well.
\begin{itemize}
\item {} 
If the component is current, use always yield self.request()

\item {} 
Although theoretically possible it is not recommended to use request for non current components.

\end{itemize}


\subsection{wait}
\label{\detokenize{Component:wait}}
Wait has the effect that the component will check whether the value of a state meets a given condition.
available. It is
possible to check for multiple states.
By default, there is no limit on the time to wait for the condition(s) to be met. But, it is possible to set
a time with fail\_at at which the condition has to be met. If that failed, the component becomes current at the given point of time.
The code should then check whether the wait had failed. That can be checked with the Component.failed() method.

If the component is canceled, activated, passivated, interrupted or held the failed flag will be set as well.
\begin{itemize}
\item {} 
If the component is current, use always yield self.wait()

\item {} 
Although theoretically possible it is not recommended to use wait for non current components.

\end{itemize}


\subsection{interrupt}
\label{\detokenize{Component:interrupt}}
With interrupt components that are not current or data can be temporarily be interrupted. Once a resume is called for
the component, the component will continue (for scheduled with the remaining time, for waiting or requesting possibly with
the remaining fail\_at duration.


\section{Usage of process interaction methods within a function or method}
\label{\detokenize{Component:usage-of-process-interaction-methods-within-a-function-or-method}}
There is a way to put process interaction statement in another function or method.
This requires a slightly different way than just calling the method.

As an example, let’s assume that we want a method that holds a component for a number of minutes and that the time unit is actually seconds.
So we need a method to wait 60 times the given parameter

We start with a not so nice solution:

\begin{sphinxVerbatim}[commandchars=\\\{\}]
\PYG{k}{class} \PYG{n+nc}{X}\PYG{p}{(}\PYG{n}{sim}\PYG{o}{.}\PYG{n}{Component}\PYG{p}{)}\PYG{p}{:}
    \PYG{k}{def} \PYG{n+nf}{process}\PYG{p}{(}\PYG{n+nb+bp}{self}\PYG{p}{)}\PYG{p}{:}
        \PYG{k}{yield} \PYG{n+nb+bp}{self}\PYG{o}{.}\PYG{n}{hold}\PYG{p}{(}\PYG{l+m+mi}{60} \PYG{o}{*} \PYG{l+m+mi}{2}\PYG{p}{)}
        \PYG{k}{yield} \PYG{n+nb+bp}{self}\PYG{o}{.}\PYG{n}{hold}\PYG{p}{(}\PYG{l+m+mi}{60} \PYG{o}{*} \PYG{l+m+mi}{5}\PYG{p}{)}
\end{sphinxVerbatim}

Now we just addd a method hold\_minutes:

\begin{sphinxVerbatim}[commandchars=\\\{\}]
\PYG{k}{def} \PYG{n+nf}{hold\PYGZus{}minutes}\PYG{p}{(}\PYG{n+nb+bp}{self}\PYG{p}{,} \PYG{n}{minutes}\PYG{p}{)}\PYG{p}{:}
    \PYG{k}{yield} \PYG{n+nb+bp}{self}\PYG{o}{.}\PYG{n}{hold}\PYG{p}{(}\PYG{l+m+mi}{60} \PYG{o}{*} \PYG{n}{minutes}\PYG{p}{)}
\end{sphinxVerbatim}

Direct calling hold\_minutes is not possible. Instead we have to say:

\begin{sphinxVerbatim}[commandchars=\\\{\}]
\PYG{k}{class} \PYG{n+nc}{X}\PYG{p}{(}\PYG{n}{sim}\PYG{o}{.}\PYG{n}{Component}\PYG{p}{)}\PYG{p}{:}
   \PYG{k}{def} \PYG{n+nf}{hold\PYGZus{}minutes}\PYG{p}{(}\PYG{n+nb+bp}{self}\PYG{p}{,} \PYG{n}{minutes}\PYG{p}{)}\PYG{p}{:}
        \PYG{k}{yield} \PYG{n+nb+bp}{self}\PYG{o}{.}\PYG{n}{hold}\PYG{p}{(}\PYG{l+m+mi}{60} \PYG{o}{*} \PYG{n}{minutes}\PYG{p}{)}

   \PYG{k}{def} \PYG{n+nf}{process}\PYG{p}{(}\PYG{n+nb+bp}{self}\PYG{p}{)}\PYG{p}{:}
        \PYG{k}{yield from} \PYG{n+nb+bp}{self}\PYG{o}{.}\PYG{n}{hold\PYGZus{}minutes}\PYG{p}{(}\PYG{l+m+mi}{2}\PYG{p}{)}
        \PYG{k}{yield from} \PYG{n+nb+bp}{self}\PYG{o}{.}\PYG{n}{hold\PYGZus{}minutes}\PYG{p}{(}\PYG{l+m+mi}{5}\PYG{p}{)}
\end{sphinxVerbatim}

All process interaction statements including passivate, request and wait can be used that way!

So remember if the method contains a yield statement (technically speaking that’s a generator function), it should be called with \sphinxcode{yield from}.


\chapter{Queue}
\label{\detokenize{Queue::doc}}\label{\detokenize{Queue:queue}}
Salabim has a class Queue for queue handling of components. The advantage over the standard list and deque are:
\begin{itemize}
\item {} 
double linked, resulting in easy and efficient insertion and deletion at any place

\item {} 
builtin data collection and statistics

\item {} 
priority sorting

\end{itemize}

Salabim uses queues internally for resources and states as well.

Definition of a queue is simple:

\begin{sphinxVerbatim}[commandchars=\\\{\}]
\PYG{n}{waitingline}\PYG{o}{=}\PYG{n}{sim}\PYG{o}{.}\PYG{n}{Queue}\PYG{p}{(}\PYG{l+s+s1}{\PYGZsq{}}\PYG{l+s+s1}{waitingline}\PYG{l+s+s1}{\PYGZsq{}}\PYG{p}{)}
\end{sphinxVerbatim}

The name of a queue can retrieved with \sphinxcode{q.name()}.

There is a set of methods for components to enter and leave a queue and retrieval:


\begin{savenotes}\sphinxattablestart
\centering
\begin{tabulary}{\linewidth}[t]{|T|T|T|}
\hline
\sphinxstylethead{\sphinxstyletheadfamily 
Component
\unskip}\relax &\sphinxstylethead{\sphinxstyletheadfamily 
Queue
\unskip}\relax &\sphinxstylethead{\sphinxstyletheadfamily 
Description
\unskip}\relax \\
\hline
c.enter(q)
&
q.add(c) or q.append(c)
&
c enters q at the tail
\\
\hline
c.enter\_to\_head(q)
&
q.add\_at\_head(c)
&
c enters q at the head
\\
\hline
c.enter\_in\_front(q, c1)
&
q.add\_in\_front\_of(c, c1)
&
c enters q in front of c1
\\
\hline
c.enter\_behind(q, c1)
&
q.add\_behind(c, c1){}`
&
c enters q behind c1
\\
\hline
c.enter\_sorted(q, p)
&
q.add\_sorted(c, p)
&
c enters q according to priority p
\\
\hline
c.leave(q)
&
q.remove(c)
q.insert(c,i)
q.pop()
q.pop(i)
q.head() or q{[}0{]}
q.tail() or q{[}-1{]}
q.index(c)
q.component\_with\_name(n)
&
c leaves q
insert c just before the i-th component in q
removes head of q and returns it
removes i-th component in q and returns it
returns head of q
returns tail of q
returns the position of c in q
returns the component with name n in q
\\
\hline
c.successor(q)
&
q.successor(c)
&
successor of c in q
\\
\hline
c.predecessor(q)
&
q.predecessor(c)
&
predecessor of c in q
\\
\hline
c.count(q)
&
q.count(c)
&
returns 1 if c in q, 0 otherwise
\\
\hline
c.queues()
&&
returns a set with all queues where c is in
\\
\hline
c.count()
&\sphinxstartmulticolumn{2}%
\begin{varwidth}[t]{\sphinxcolwidth{2}{3}}
returns number of queues c is in
\par
\vskip-\baselineskip\strut\end{varwidth}%
\sphinxstopmulticolumn
\\
\hline
\end{tabulary}
\par
\sphinxattableend\end{savenotes}

Queue is a standard ABC class, which means that the following methods are supported:
\begin{itemize}
\item {} 
\sphinxcode{len(q)} to retrieve the length of a queue, alternatively via the timestamped monitor with \sphinxcode{q.length()}

\item {} 
\sphinxcode{c in q} to check whether a component is in a queue

\item {} 
\sphinxcode{for c in q:} to traverse a queue (Note that it is even possible to remove and add components in the for body).

\item {} 
\sphinxcode{reversed(q)} for the components in the queue in reverse order

\item {} 
slicing is supported, so it is possible to get the 2nd, 3rd and 4th component in a queue with \sphinxcode{q{[}1:4{]}} or \sphinxcode{q{[}::-1{]}}
for all elements in reverse order.

\item {} 
\sphinxcode{del q{[}i{]}} removes the i’th component. Also slicing is supported, so e.g. to delete the last three elements from queue,
\sphinxcode{del q{[}-1:-4:-1{]}}

\item {} 
\sphinxcode{q.append(c)} is equivalent to \sphinxcode{q.add(c)}

\end{itemize}

It is possible to do a number of operations that work on the queues:
\begin{itemize}
\item {} 
\sphinxcode{q.intersection(q1)} or \sphinxcode{q \& q1} returns a new queue with components that are both in q and q1

\item {} 
\sphinxcode{q.difference(q1)} or \sphinxcode{q - q1{}`} returns a new queue with components that are in q1 but not in q2

\item {} 
\sphinxcode{q.union(q1)} or \sphinxcode{q \textbar{} q1} returns a new queue with components that are in q or q1

\item {} 
\sphinxcode{q.symmetric\_difference(q)} or \sphinxcode{q \textasciicircum{} q1} returns a queue with components that are in q or q1, but not both

\item {} 
\sphinxcode{q.clear()} empties a queue

\item {} 
\sphinxcode{q.copy()} copies all components in q to a new queue. The queue q is untouched.

\item {} 
\sphinxcode{q.move()} copies all components in q to a new queue. The queue q is emptied.

\item {} 
\sphinxcode{q.extend(q1)} extends the q with elements in q1, that are not yet in q

\end{itemize}

Salabim keeps track of the enter time in a queue: \sphinxcode{c.enter\_time(q)}

Unless disabled explicitly, the length of the queue and length of stay of components are monitored in
\sphinxcode{q.length} and \sphinxcode{q.length\_of\_stay}. It is possible to obtain a number of statistics on these monitors (cf. Monitor and MonitorTimestamp).

With \sphinxcode{q.print\_statistics()} the key statistics of these two monitors are printed.

E.g.:

\begin{sphinxVerbatim}[commandchars=\\\{\}]
\PYGZhy{}\PYGZhy{}\PYGZhy{}\PYGZhy{}\PYGZhy{}\PYGZhy{}\PYGZhy{}\PYGZhy{}\PYGZhy{}\PYGZhy{}\PYGZhy{}\PYGZhy{}\PYGZhy{}\PYGZhy{}\PYGZhy{}\PYGZhy{}\PYGZhy{}\PYGZhy{}\PYGZhy{}\PYGZhy{}\PYGZhy{}\PYGZhy{}\PYGZhy{}\PYGZhy{}\PYGZhy{}\PYGZhy{}\PYGZhy{}\PYGZhy{}\PYGZhy{}\PYGZhy{}\PYGZhy{}\PYGZhy{}\PYGZhy{}\PYGZhy{}\PYGZhy{}\PYGZhy{}\PYGZhy{}\PYGZhy{}\PYGZhy{}\PYGZhy{}\PYGZhy{}\PYGZhy{}\PYGZhy{}\PYGZhy{} \PYGZhy{}\PYGZhy{}\PYGZhy{}\PYGZhy{}\PYGZhy{}\PYGZhy{}\PYGZhy{}\PYGZhy{}\PYGZhy{}\PYGZhy{}\PYGZhy{}\PYGZhy{}\PYGZhy{}\PYGZhy{} \PYGZhy{}\PYGZhy{}\PYGZhy{}\PYGZhy{}\PYGZhy{}\PYGZhy{}\PYGZhy{}\PYGZhy{}\PYGZhy{}\PYGZhy{}\PYGZhy{}\PYGZhy{} \PYGZhy{}\PYGZhy{}\PYGZhy{}\PYGZhy{}\PYGZhy{}\PYGZhy{}\PYGZhy{}\PYGZhy{}\PYGZhy{}\PYGZhy{}\PYGZhy{}\PYGZhy{} \PYGZhy{}\PYGZhy{}\PYGZhy{}\PYGZhy{}\PYGZhy{}\PYGZhy{}\PYGZhy{}\PYGZhy{}\PYGZhy{}\PYGZhy{}\PYGZhy{}\PYGZhy{}
Length of waitingline                        duration          50000        48499.381     1500.619
                                             mean                  8.427        8.687
                                             std.deviation         4.852        4.691

                                             minimum               0            1
                                             median                9           10
                                             90\PYGZpc{} percentile       14           14
                                             95\PYGZpc{} percentile       16           16
                                             maximum              21           21

Length of stay in waitingline                entries            4995         4933           62
                                             mean                 84.345       85.405
                                             std.deviation        48.309       47.672

                                             minimum               0            0.006
                                             median               94.843       95.411
                                             90\PYGZpc{} percentile      142.751      142.975
                                             95\PYGZpc{} percentile      157.467      157.611
                                             maximum             202.153      202.153
\end{sphinxVerbatim}

With \sphinxcode{q.print\_info()} a summary of the contents of a queue can be printed.

E.g.

\begin{sphinxVerbatim}[commandchars=\\\{\}]
Queue 0x20e116153c8
  name=waitingline
  component(s):
    customer.4995        enter\PYGZus{}time 49978.472 priority=0
    customer.4996        enter\PYGZus{}time 49991.298 priority=0
\end{sphinxVerbatim}


\chapter{Resource}
\label{\detokenize{Resource::doc}}\label{\detokenize{Resource:resource}}
Resources are a powerful way of process interaction.

A resource has always a capacity (which can be zero and even negative). This capacity will be specified at time of creation, but
may change over time.
There are two of types resources:
\begin{itemize}
\item {} 
standard resources, where each claim is associated with a component (the claimer). It is not necessary that the claimed quantities are integer.

\item {} 
anonymous resources, where only the claimed quantity is registered. This is most useful for dealing with levels, lengths, etc.

\end{itemize}

Resources are defined like

\begin{sphinxVerbatim}[commandchars=\\\{\}]
\PYG{n}{clerks} \PYG{o}{=} \PYG{n}{Resource}\PYG{p}{(}\PYG{l+s+s1}{\PYGZsq{}}\PYG{l+s+s1}{clerks}\PYG{l+s+s1}{\PYGZsq{}}\PYG{p}{,} \PYG{n}{capacity}\PYG{o}{=}\PYG{l+m+mi}{3}\PYG{p}{)}
\end{sphinxVerbatim}

And then a component can request a clerk

\begin{sphinxVerbatim}[commandchars=\\\{\}]
\PYG{k}{yield} \PYG{n+nb+bp}{self}\PYG{o}{.}\PYG{n}{request}\PYG{p}{(}\PYG{n}{clerks}\PYG{p}{)}  \PYG{c+c1}{\PYGZsh{} request 1 from clerks}
\end{sphinxVerbatim}

It is also possible to request for more resources at once

\begin{sphinxVerbatim}[commandchars=\\\{\}]
\PYG{k}{yield} \PYG{n+nb+bp}{self}\PYG{o}{.}\PYG{n}{request}\PYG{p}{(}\PYG{n}{clerks}\PYG{p}{,}\PYG{p}{(}\PYG{n}{assistance}\PYG{p}{,}\PYG{l+m+mi}{2}\PYG{p}{)}\PYG{p}{)}  \PYG{c+c1}{\PYGZsh{} request 1 from clerks AND 2 from assistance}
\end{sphinxVerbatim}

Resources have a queue \sphinxcode{requesters} containing all components trying to claim from the resource.
And a queue \sphinxcode{claimers} containing all components claiming from the resource
(not for anonymous resources).

It is possible to release a quantity from a resource with c.release(), e.g.

\begin{sphinxVerbatim}[commandchars=\\\{\}]
\PYG{n+nb+bp}{self}\PYG{o}{.}\PYG{n}{release}\PYG{p}{(}\PYG{n}{r}\PYG{p}{)}  \PYG{c+c1}{\PYGZsh{} releases all claimed quantity from r}
\PYG{n+nb+bp}{self}\PYG{o}{.}\PYG{n}{release}\PYG{p}{(}\PYG{p}{(}\PYG{n}{r}\PYG{p}{,}\PYG{l+m+mi}{2}\PYG{p}{)}\PYG{p}{)}  \PYG{c+c1}{\PYGZsh{} release quantity 2 from r}
\end{sphinxVerbatim}

Alternatively, it is possible to release from a resource directly, e.g.

\begin{sphinxVerbatim}[commandchars=\\\{\}]
\PYG{n}{r}\PYG{o}{.}\PYG{n}{release}\PYG{p}{(}\PYG{p}{)}  \PYG{c+c1}{\PYGZsh{} releases the total quantity from all claiming components}
\PYG{n}{r}\PYG{o}{.}\PYG{n}{release}\PYG{p}{(}\PYG{l+m+mi}{10}\PYG{p}{)}  \PYG{c+c1}{\PYGZsh{} releases 10 from the resource; only valid for anonymous resources}
\end{sphinxVerbatim}

After a release, all requesting components will be checked whether their claim can be honored.

Resources have a number of monitors and timestamped monitors:
\begin{itemize}
\item {} 
claimers().length

\item {} 
claimers().length\_of\_stay

\item {} 
requesters().length

\item {} 
requesters().length\_of\_stay

\item {} 
claimed\_quantity

\item {} 
available\_quantity

\item {} 
capacity

\item {} 
occupancy

\end{itemize}

By default, all monitors are enabled.

With \sphinxcode{r.print\_statistics()} the key statistics of these all monitors are printed.

E.g.:

\begin{sphinxVerbatim}[commandchars=\\\{\}]
Statistics of clerk at     50000.000
                                                                     all    excl.zero         zero
\PYGZhy{}\PYGZhy{}\PYGZhy{}\PYGZhy{}\PYGZhy{}\PYGZhy{}\PYGZhy{}\PYGZhy{}\PYGZhy{}\PYGZhy{}\PYGZhy{}\PYGZhy{}\PYGZhy{}\PYGZhy{}\PYGZhy{}\PYGZhy{}\PYGZhy{}\PYGZhy{}\PYGZhy{}\PYGZhy{}\PYGZhy{}\PYGZhy{}\PYGZhy{}\PYGZhy{}\PYGZhy{}\PYGZhy{}\PYGZhy{}\PYGZhy{}\PYGZhy{}\PYGZhy{}\PYGZhy{}\PYGZhy{}\PYGZhy{}\PYGZhy{}\PYGZhy{}\PYGZhy{}\PYGZhy{}\PYGZhy{}\PYGZhy{}\PYGZhy{}\PYGZhy{}\PYGZhy{}\PYGZhy{}\PYGZhy{} \PYGZhy{}\PYGZhy{}\PYGZhy{}\PYGZhy{}\PYGZhy{}\PYGZhy{}\PYGZhy{}\PYGZhy{}\PYGZhy{}\PYGZhy{}\PYGZhy{}\PYGZhy{}\PYGZhy{}\PYGZhy{} \PYGZhy{}\PYGZhy{}\PYGZhy{}\PYGZhy{}\PYGZhy{}\PYGZhy{}\PYGZhy{}\PYGZhy{}\PYGZhy{}\PYGZhy{}\PYGZhy{}\PYGZhy{} \PYGZhy{}\PYGZhy{}\PYGZhy{}\PYGZhy{}\PYGZhy{}\PYGZhy{}\PYGZhy{}\PYGZhy{}\PYGZhy{}\PYGZhy{}\PYGZhy{}\PYGZhy{} \PYGZhy{}\PYGZhy{}\PYGZhy{}\PYGZhy{}\PYGZhy{}\PYGZhy{}\PYGZhy{}\PYGZhy{}\PYGZhy{}\PYGZhy{}\PYGZhy{}\PYGZhy{}
Length of requesters of clerk                duration          50000        48499.381     1500.619
                                             mean                  8.427        8.687
                                             std.deviation         4.852        4.691

                                             minimum               0            1
                                             median                9           10
                                             90\PYGZpc{} percentile       14           14
                                             95\PYGZpc{} percentile       16           16
                                             maximum              21           21

Length of stay in requesters of clerk        entries            4995         4933           62
                                             mean                 84.345       85.405
                                             std.deviation        48.309       47.672

                                             minimum               0            0.006
                                             median               94.843       95.411
                                             90\PYGZpc{} percentile      142.751      142.975
                                             95\PYGZpc{} percentile      157.467      157.611
                                             maximum             202.153      202.153

Length of claimers of clerk                  duration          50000        50000            0
                                             mean                  2.996        2.996
                                             std.deviation         0.068        0.068

                                             minimum               1            1
                                             median                3            3
                                             90\PYGZpc{} percentile        3            3
                                             95\PYGZpc{} percentile        3            3
                                             maximum               3            3

Length of stay in claimers of clerk          entries            4992         4992            0
                                             mean                 30           30
                                             std.deviation         0.000        0.000

                                             minimum              30.000       30.000
                                             median               30           30
                                             90\PYGZpc{} percentile       30           30
                                             95\PYGZpc{} percentile       30           30
                                             maximum              30.000       30.000

Capacity of clerk                            duration          50000        50000            0
                                             mean                  3            3
                                             std.deviation         0            0

                                             minimum               3            3
                                             median                3            3
                                             90\PYGZpc{} percentile        3            3
                                             95\PYGZpc{} percentile        3            3
                                             maximum               3            3

Available quantity of clerk                  duration          50000          187.145    49812.855
                                             mean                  0.004        1.078
                                             std.deviation         0.068        0.268

                                             minimum               0            1
                                             median                0            1
                                             90\PYGZpc{} percentile        0            1
                                             95\PYGZpc{} percentile        0            2
                                             maximum               2            2

Claimed quantity of clerk                    duration          50000        50000            0
                                             mean                  2.996        2.996
                                             std.deviation         0.068        0.068

                                             minimum               1            1
                                             median                3            3
                                             90\PYGZpc{} percentile        3            3
                                             95\PYGZpc{} percentile        3            3
                                             maximum               3            3
\end{sphinxVerbatim}

With \sphinxcode{r.print\_info()} a summary of the contents of the queues can be printed.

E.g.

\begin{sphinxVerbatim}[commandchars=\\\{\}]
\PYG{n}{Resource} \PYG{l+m+mh}{0x112e8f0b8}
  \PYG{n}{name}\PYG{o}{=}\PYG{n}{clerk}
  \PYG{n}{capacity}\PYG{o}{=}\PYG{l+m+mi}{3}
  \PYG{n}{requesting} \PYG{n}{component}\PYG{p}{(}\PYG{n}{s}\PYG{p}{)}\PYG{p}{:}
    \PYG{n}{customer}\PYG{o}{.}\PYG{l+m+mi}{4995}        \PYG{n}{quantity}\PYG{o}{=}\PYG{l+m+mi}{1}
    \PYG{n}{customer}\PYG{o}{.}\PYG{l+m+mi}{4996}        \PYG{n}{quantity}\PYG{o}{=}\PYG{l+m+mi}{1}
  \PYG{n}{claimed\PYGZus{}quantity}\PYG{o}{=}\PYG{l+m+mi}{3}
  \PYG{n}{claimed} \PYG{n}{by}\PYG{p}{:}
    \PYG{n}{customer}\PYG{o}{.}\PYG{l+m+mi}{4992}        \PYG{n}{quantity}\PYG{o}{=}\PYG{l+m+mi}{1}
    \PYG{n}{customer}\PYG{o}{.}\PYG{l+m+mi}{4993}        \PYG{n}{quantity}\PYG{o}{=}\PYG{l+m+mi}{1}
    \PYG{n}{customer}\PYG{o}{.}\PYG{l+m+mi}{4994}        \PYG{n}{quantity}\PYG{o}{=}\PYG{l+m+mi}{1}
\end{sphinxVerbatim}

The capacity may be changed with \sphinxcode{r.set\_capacity(x)}. Note that this may lead to requesting
components to be honored.

Querying of the capacity, claimed quantity and available quantity can be done via the timestamped monitors:
\sphinxcode{r.capacity()}, \sphinxcode{r.claimed\_quantity()} and \sphinxcode{r.available\_quantity()}

It is possible to calculate the occupancy of a resource with

\begin{sphinxVerbatim}[commandchars=\\\{\}]
\PYG{n}{occupancy} \PYG{o}{=} \PYG{n}{r}\PYG{o}{.}\PYG{n}{claimed\PYGZus{}quantity}\PYG{p}{(}\PYG{p}{)}\PYG{o}{.}\PYG{n}{mean} \PYG{o}{/} \PYG{n}{r}\PYG{o}{.}\PYG{n}{capacity}\PYG{p}{(}\PYG{p}{)}\PYG{o}{.}\PYG{n}{mean}
\end{sphinxVerbatim}

or, by tallying at each change:
\begin{quote}

r.occupancy.mean()
\end{quote}

Note that these two methods do not return the same result.


\chapter{State}
\label{\detokenize{State::doc}}\label{\detokenize{State:state}}
States together with the Component.wait() method provide a powerful way of process interaction.

A state will have a certain value at a given time. In its simplest form a component can then wait for
a specific value of a state. Once that value is reached, the component will be resumed.

Definition is simple, like \sphinxcode{dooropen=sim.State('dooropen')}. The default initial value is False, meaning
the door is closed.

Now we can say

\begin{sphinxVerbatim}[commandchars=\\\{\}]
\PYG{n}{dooropen}\PYG{o}{.}\PYG{n}{set}\PYG{p}{(}\PYG{p}{)}
\end{sphinxVerbatim}

to open the door.

If we want a person to wait for an open door, we could say

\begin{sphinxVerbatim}[commandchars=\\\{\}]
\PYG{k}{yield} \PYG{n+nb+bp}{self}\PYG{o}{.}\PYG{n}{wait}\PYG{p}{(}\PYG{n}{dooropen}\PYG{p}{)}
\end{sphinxVerbatim}

If we just want at most one person to enter, we say \sphinxcode{dooropen.trigger(max=1)}.

We can obtain the current value by just calling the state, like in

\begin{sphinxVerbatim}[commandchars=\\\{\}]
\PYG{n+nb}{print}\PYG{p}{(}\PYG{l+s+s1}{\PYGZsq{}}\PYG{l+s+s1}{door is }\PYG{l+s+s1}{\PYGZsq{}}\PYG{p}{,}\PYG{p}{(}\PYG{l+s+s1}{\PYGZsq{}}\PYG{l+s+s1}{open}\PYG{l+s+s1}{\PYGZsq{}} \PYG{k}{if} \PYG{n}{dooropen}\PYG{p}{(}\PYG{p}{)} \PYG{k}{else} \PYG{l+s+s1}{\PYGZsq{}}\PYG{l+s+s1}{closed}\PYG{l+s+s1}{\PYGZsq{}}\PYG{p}{)}\PYG{p}{)}
\end{sphinxVerbatim}

Alternatively, we can get the current value with the get method

\begin{sphinxVerbatim}[commandchars=\\\{\}]
\PYG{n+nb}{print}\PYG{p}{(}\PYG{l+s+s1}{\PYGZsq{}}\PYG{l+s+s1}{door is }\PYG{l+s+s1}{\PYGZsq{}}\PYG{p}{,}\PYG{p}{(}\PYG{l+s+s1}{\PYGZsq{}}\PYG{l+s+s1}{open}\PYG{l+s+s1}{\PYGZsq{}} \PYG{k}{if} \PYG{n}{dooropen}\PYG{o}{.}\PYG{n}{get}\PYG{p}{(}\PYG{p}{)} \PYG{k}{else} \PYG{l+s+s1}{\PYGZsq{}}\PYG{l+s+s1}{closed}\PYG{l+s+s1}{\PYGZsq{}}\PYG{p}{)}\PYG{p}{)}
\end{sphinxVerbatim}

The value of a state is automatically monitored in the state.value timestamped monitor.

All components waiting for a state are in a salabim queue, called waiters().

States can be used also for non values other than bool type. E.g.

\begin{sphinxVerbatim}[commandchars=\\\{\}]
\PYG{n}{light}\PYG{o}{=}\PYG{n}{sim}\PYG{o}{.}\PYG{n}{State}\PYG{p}{(}\PYG{l+s+s1}{\PYGZsq{}}\PYG{l+s+s1}{light}\PYG{l+s+s1}{\PYGZsq{}}\PYG{p}{,} \PYG{n}{value}\PYG{o}{=}\PYG{l+s+s1}{\PYGZsq{}}\PYG{l+s+s1}{red}\PYG{l+s+s1}{\PYGZsq{}}\PYG{p}{)}
\PYG{o}{.}\PYG{o}{.}\PYG{o}{.}
\PYG{n}{light}\PYG{o}{.}\PYG{n}{state}\PYG{o}{.}\PYG{n}{set}\PYG{p}{(}\PYG{l+s+s1}{\PYGZsq{}}\PYG{l+s+s1}{green}\PYG{l+s+s1}{\PYGZsq{}}\PYG{p}{)}
\end{sphinxVerbatim}

Or define a int/float state

\begin{sphinxVerbatim}[commandchars=\\\{\}]
\PYG{n}{level}\PYG{o}{=}\PYG{n}{sim}\PYG{o}{.}\PYG{n}{State}\PYG{p}{(}\PYG{l+s+s1}{\PYGZsq{}}\PYG{l+s+s1}{level}\PYG{l+s+s1}{\PYGZsq{}}\PYG{p}{,} \PYG{n}{value}\PYG{o}{=}\PYG{l+m+mi}{0}\PYG{p}{)}
\PYG{o}{.}\PYG{o}{.}\PYG{o}{.}
\PYG{n}{level}\PYG{o}{.}\PYG{n}{set}\PYG{p}{(}\PYG{n}{level}\PYG{p}{(}\PYG{p}{)}\PYG{o}{+}\PYG{l+m+mi}{10}\PYG{p}{)}
\end{sphinxVerbatim}

States have a number if monitors and timestamped monitors:
\begin{itemize}
\item {} 
value, where all the values ae collected over time

\item {} 
waiters().length

\item {} 
waiters().length\_of\_stay

\end{itemize}


\section{Process interaction with wait()}
\label{\detokenize{State:process-interaction-with-wait}}
A component can wait for a state to get a certain value. In its most simple form

\begin{sphinxVerbatim}[commandchars=\\\{\}]
\PYG{k}{yield} \PYG{n+nb+bp}{self}\PYG{o}{.}\PYG{n}{wait}\PYG{p}{(}\PYG{n}{dooropen}\PYG{p}{)}
\end{sphinxVerbatim}

Once the dooropen state is True, the component will continue.

As with request() it is possible to set a timeout with fail\_at or fail\_delay

\begin{sphinxVerbatim}[commandchars=\\\{\}]
\PYG{k}{yield} \PYG{n+nb+bp}{self}\PYG{o}{.}\PYG{n}{wait}\PYG{p}{(}\PYG{n}{dooropen}\PYG{p}{,} \PYG{n}{fail\PYGZus{}delay}\PYG{o}{=}\PYG{l+m+mi}{10}\PYG{p}{)}
\PYG{k}{if} \PYG{n+nb+bp}{self}\PYG{o}{.}\PYG{n}{failed}\PYG{p}{:}
    \PYG{n+nb}{print}\PYG{p}{(}\PYG{l+s+s1}{\PYGZsq{}}\PYG{l+s+s1}{impatient ...}\PYG{l+s+s1}{\PYGZsq{}}\PYG{p}{)}
\end{sphinxVerbatim}

In the above example we tested for a state to be True.

There are three ways to test for a value:


\subsection{Scalar testing}
\label{\detokenize{State:scalar-testing}}
It is possible to test for a certain value

\begin{sphinxVerbatim}[commandchars=\\\{\}]
\PYG{k}{yield} \PYG{n+nb+bp}{self}\PYG{o}{.}\PYG{n}{wait}\PYG{p}{(}\PYG{p}{(}\PYG{n}{light}\PYG{p}{,} \PYG{l+s+s1}{\PYGZsq{}}\PYG{l+s+s1}{green}\PYG{l+s+s1}{\PYGZsq{}}\PYG{p}{)}\PYG{p}{)}
\end{sphinxVerbatim}

Or more states at once

\begin{sphinxVerbatim}[commandchars=\\\{\}]
\PYG{k}{yield} \PYG{n+nb+bp}{self}\PYG{o}{.}\PYG{n}{wait}\PYG{p}{(}\PYG{p}{(}\PYG{n}{light}\PYG{p}{,} \PYG{l+s+s1}{\PYGZsq{}}\PYG{l+s+s1}{green}\PYG{l+s+s1}{\PYGZsq{}}\PYG{p}{)}\PYG{p}{,} \PYG{n}{night}\PYG{p}{)}  \PYG{c+c1}{\PYGZsh{} honored as soon as light is green OR it\PYGZsq{}s night}
\PYG{k}{yield} \PYG{n+nb+bp}{self}\PYG{o}{.}\PYG{n}{wait}\PYG{p}{(}\PYG{p}{(}\PYG{n}{light}\PYG{p}{,} \PYG{l+s+s1}{\PYGZsq{}}\PYG{l+s+s1}{green}\PYG{l+s+s1}{\PYGZsq{}}\PYG{p}{)}\PYG{p}{,} \PYG{p}{(}\PYG{n}{light}\PYG{p}{,} \PYG{l+s+s1}{\PYGZsq{}}\PYG{l+s+s1}{yellow}\PYG{l+s+s1}{\PYGZsq{}}\PYG{p}{)}\PYG{p}{)}  \PYG{c+c1}{\PYGZsh{} honored as soon is light is green OR yellow}
\end{sphinxVerbatim}

It is also possible to wait for all conditions to be satisfied, by adding \sphinxcode{all=True}:

\begin{sphinxVerbatim}[commandchars=\\\{\}]
\PYG{k}{yield} \PYG{n+nb+bp}{self}\PYG{o}{.}\PYG{n}{wait}\PYG{p}{(}\PYG{p}{(}\PYG{n}{light}\PYG{p}{,}\PYG{l+s+s1}{\PYGZsq{}}\PYG{l+s+s1}{green}\PYG{l+s+s1}{\PYGZsq{}}\PYG{p}{)}\PYG{p}{,} \PYG{n}{enginerunning}\PYG{p}{,} \PYG{n+nb}{all}\PYG{o}{=}\PYG{k+kc}{True}\PYG{p}{)}  \PYG{c+c1}{\PYGZsh{} honored as soon as light is green AND engine is running}
\end{sphinxVerbatim}


\subsection{Evaluation testing}
\label{\detokenize{State:evaluation-testing}}
Here, we use a string containing an expression that can evaluate to True or False. This is
done by specifying at least one \sphinxcode{\$} in the test-string. This \sphinxcode{\$} will be replaced at run time by
\sphinxcode{state.value()}, where state is the state under test. Here are some examples

\begin{sphinxVerbatim}[commandchars=\\\{\}]
\PYG{k}{yield} \PYG{n+nb+bp}{self}\PYG{o}{.}\PYG{n}{wait}\PYG{p}{(}\PYG{p}{(}\PYG{n}{light}\PYG{p}{,} \PYG{l+s+s1}{\PYGZsq{}}\PYG{l+s+s1}{\PYGZdl{} in (}\PYG{l+s+s1}{\PYGZdq{}}\PYG{l+s+s1}{green}\PYG{l+s+s1}{\PYGZdq{}}\PYG{l+s+s1}{,}\PYG{l+s+s1}{\PYGZdq{}}\PYG{l+s+s1}{yellow}\PYG{l+s+s1}{\PYGZdq{}}\PYG{l+s+s1}{)}\PYG{l+s+s1}{\PYGZsq{}}\PYG{p}{)}\PYG{p}{)}
    \PYG{c+c1}{\PYGZsh{} if at run time light.value() is \PYGZsq{}green\PYGZsq{}, test for eval(state.value() in (\PYGZdq{}green,\PYGZdq{}yellow\PYGZdq{})) ==\PYGZgt{} True}
\PYG{k}{yield} \PYG{n+nb+bp}{self}\PYG{o}{.}\PYG{n}{wait}\PYG{p}{(}\PYG{p}{(}\PYG{n}{level}\PYG{p}{,} \PYG{l+s+s1}{\PYGZsq{}}\PYG{l+s+s1}{\PYGZdl{} \PYGZlt{} 30}\PYG{l+s+s1}{\PYGZsq{}}\PYG{p}{)}\PYG{p}{)}
    \PYG{c+c1}{\PYGZsh{} if at run time level.value() is 50, test for eval(state.value() \PYGZlt{} 30) ==\PYGZgt{} False}
\end{sphinxVerbatim}

During the evaluation, \sphinxcode{self} refers to the component under test and \sphinxcode{state} to the state under test.
E.g.

\begin{sphinxVerbatim}[commandchars=\\\{\}]
\PYG{n+nb+bp}{self}\PYG{o}{.}\PYG{n}{limit} \PYG{o}{=} \PYG{l+m+mi}{30}
\PYG{k}{yield} \PYG{n+nb+bp}{self}\PYG{o}{.}\PYG{n}{wait}\PYG{p}{(}\PYG{p}{(}\PYG{n}{level}\PYG{p}{,} \PYG{l+s+s1}{\PYGZsq{}}\PYG{l+s+s1}{self.limit \PYGZgt{}= \PYGZdl{}}\PYG{l+s+s1}{\PYGZsq{}}\PYG{p}{)}\PYG{p}{)}
    \PYG{c+c1}{\PYGZsh{} if at run time level.value() is 10, test for eval(self.limit \PYGZgt{}= state.get()) ==\PYGZgt{} True, so honored}
\end{sphinxVerbatim}


\subsection{Function testing}
\label{\detokenize{State:function-testing}}
This is a more complicated but also more versatile way of specifying the honor-condition.
In that case, a function is required to specify the condition. The function needs to accept three
arguments:
\begin{itemize}
\item {} 
x = state.get()

\item {} 
component component under test

\item {} 
state under test

\end{itemize}

E.g.:

\begin{sphinxVerbatim}[commandchars=\\\{\}]
\PYG{k}{yield} \PYG{n+nb+bp}{self}\PYG{o}{.}\PYG{n}{wait}\PYG{p}{(}\PYG{p}{(}\PYG{n}{light}\PYG{p}{,} \PYG{k}{lambda} \PYG{n}{x}\PYG{p}{,} \PYG{n}{\PYGZus{}}\PYG{p}{,} \PYG{n}{\PYGZus{}}\PYG{p}{:} \PYG{n}{x} \PYG{o+ow}{in} \PYG{p}{(}\PYG{l+s+s1}{\PYGZsq{}}\PYG{l+s+s1}{green}\PYG{l+s+s1}{\PYGZsq{}}\PYG{p}{,} \PYG{l+s+s1}{\PYGZsq{}}\PYG{l+s+s1}{yellow}\PYG{l+s+s1}{\PYGZsq{}}\PYG{p}{)}\PYG{p}{)}
    \PYG{c+c1}{\PYGZsh{} x is light.get()}
\PYG{k}{yield} \PYG{n+nb+bp}{self}\PYG{o}{.}\PYG{n}{wait}\PYG{p}{(}\PYG{p}{(}\PYG{n}{level}\PYG{p}{,} \PYG{k}{lambda} \PYG{n}{x}\PYG{p}{,} \PYG{n}{\PYGZus{}}\PYG{p}{,} \PYG{n}{\PYGZus{}}\PYG{p}{:} \PYG{n}{x} \PYG{o}{\PYGZgt{}}\PYG{o}{=} \PYG{l+m+mi}{30}\PYG{p}{)}\PYG{p}{)}
    \PYG{c+c1}{\PYGZsh{} x is level.get()}
\end{sphinxVerbatim}

And, of course, it is possible to define a function

\begin{sphinxVerbatim}[commandchars=\\\{\}]
\PYG{k}{def} \PYG{n+nf}{levelreached}\PYG{p}{(}\PYG{n}{x}\PYG{p}{)}\PYG{p}{:}
    \PYG{n}{value}\PYG{p}{,} \PYG{n}{component}\PYG{p}{,} \PYG{n}{\PYGZus{}} \PYG{o}{=} \PYG{n}{x}
    \PYG{k}{return} \PYG{n}{value} \PYG{o}{\PYGZlt{}} \PYG{n}{component}\PYG{o}{.}\PYG{n}{limit}

\PYG{o}{.}\PYG{o}{.}\PYG{o}{.}

\PYG{n+nb+bp}{self}\PYG{o}{.}\PYG{n}{limit} \PYG{o}{=} \PYG{l+m+mi}{30}
\PYG{k}{yield} \PYG{n+nb+bp}{self}\PYG{o}{.}\PYG{n}{wait}\PYG{p}{(}\PYG{p}{(}\PYG{n}{level}\PYG{p}{,} \PYG{n}{levelreached}\PYG{p}{)}\PYG{p}{)}
\end{sphinxVerbatim}


\subsection{Combination of testing methods}
\label{\detokenize{State:combination-of-testing-methods}}
It is possible to mix scalar, evaluation and function testing. And it’s also possible to specify all=True
in any case.


\chapter{Monitor}
\label{\detokenize{Monitor:monitor}}\label{\detokenize{Monitor::doc}}
Monitors are a way to collect data from the simulation. They are automatically collected
for resources, queues and states. On top of that the user can define its own monitors.

Monitors can be used to get statistics and as a feed for graphical tools, like matplotlib.

There are two types of monitors:
\begin{itemize}
\item {} 
level monitors
Level monitors are useful to collect data about a variable that keeps its value over a certain length
of time, like the length of a queue, the colour of a traffic light, etc.

\item {} 
non level monitors
Non level monitors are useful to collect data about a values that occur just once. Examples, are the length of stay in a queue and
the number of processing steps of a part.

\end{itemize}

For both types, the time is always collected, along with the value.

Non level monitors can be weighted, if required.


\section{Non level monitor}
\label{\detokenize{Monitor:non-level-monitor}}
Non level monitors collects values which do notrefelect a level, e.g. the processing time of a part.

We define the monitor with \sphinxcode{processingtime=sim.Monitor('processingtime')} and then
collect values by \sphinxcode{processingtime.tally(this\_duration)}

By default, the collected values are stored in a list. Alternatively, it is possible to store
the values in an array of one of the following types:


\begin{savenotes}\sphinxattablestart
\centering
\begin{tabulary}{\linewidth}[t]{|T|T|T|T|T|}
\hline
\sphinxstylethead{\sphinxstyletheadfamily 
type
\unskip}\relax &\sphinxstylethead{\sphinxstyletheadfamily 
stored as
\unskip}\relax &\sphinxstylethead{\sphinxstyletheadfamily 
lowerbound
\unskip}\relax &\sphinxstylethead{\sphinxstyletheadfamily 
upperbound
\unskip}\relax &\sphinxstylethead{\sphinxstyletheadfamily 
number of bytes
\unskip}\relax \\
\hline
‘any’
&
list
&
N/A
&
N/A
&
depends on data
\\
\hline
‘bool’
&
integer
&
False
&
True
&
1
\\
\hline
‘int8’
&
integer
&
-128
&
127
&
1
\\
\hline
‘uint8’
&
integer
&
0
&
255
&
1
\\
\hline
‘int16’
&
integer
&
-32768
&
32767
&
2
\\
\hline
‘uint16’
&
integer
&
0
&
65535
&
2
\\
\hline
‘int32’
&
integer
&
2147483648
&
2147483647
&
4
\\
\hline
‘uint32’
&
integer
&
0
&
4294967295
&
4
\\
\hline
‘int64’
&
integer
&
-9223372036854775808
&
9223372036854775807
&
8
\\
\hline
‘uint64’
&
integer
&
0
&
18446744073709551615
&
8
\\
\hline
‘float’
&
float
&
-inf
&
inf
&
8
\\
\hline
\end{tabulary}
\par
\sphinxattableend\end{savenotes}

Monitoring with arrays takes up less space. Particularly when tallying a large
number of values, this is strongly advised.

Note that if non numeric values are stored (only possible with the default setting (‘any’)),
a tallied value is converted, if required, to a numeric value if possible, or 0 if not.

It is possible to use monitors with weighted data. In that case, just add a second parameter to tally, which defaults to 1.
All statistics will take the weights into account.

There is set of statistical data available:
\begin{itemize}
\item {} 
number\_of\_entries

\item {} 
number\_of\_entries\_zero

\item {} 
weight

\item {} 
weight\_zero

\item {} 
mean

\item {} 
std

\item {} 
minimum

\item {} 
median

\item {} 
maximum

\item {} 
percentile

\item {} 
bin\_number\_of\_entries (number of entries between two given values)

\item {} 
bin\_weight (total weight of entries between two given values)

\item {} 
value\_number\_of\_entries (number of entries equal to a given value or set of values)

\item {} 
value\_weight (total weight of entries equal to a given value or set of values)

\end{itemize}

For all these statistics, it is possible to exclude zero entries,
e.g. \sphinxcode{m.mean(ex0=True)} returns the mean, excluding zero entries.

Besides, it is possible to get all collected values as an array with x(). In that case of ‘any’ monitors,
the values might be converted. By specifying \sphinxcode{force\_numeric=False} the collected values will be returned as stored.

With the monitor method, the monitor can be enbled or disabled. Note that a tally is just ignored when
the monitor is disabled.

Also, the current status (enabled/disabled) can be retrieved.

\begin{sphinxVerbatim}[commandchars=\\\{\}]
\PYG{n}{proctime}\PYG{o}{.}\PYG{n}{monitor}\PYG{p}{(}\PYG{n+nb+bp}{False}\PYG{p}{)}  \PYG{c+c1}{\PYGZsh{} disable monitoring}
\PYG{n}{proctime}\PYG{o}{.}\PYG{n}{monitor}\PYG{p}{(}\PYG{n+nb+bp}{True}\PYG{p}{)}  \PYG{c+c1}{\PYGZsh{} enable monitoring}
\PYG{k}{if} \PYG{n}{proctime}\PYG{o}{.}\PYG{n}{monitor}\PYG{p}{(}\PYG{p}{)}\PYG{p}{:}
    \PYG{k}{print}\PYG{p}{(}\PYG{l+s+s1}{\PYGZsq{}}\PYG{l+s+s1}{proctime is enabled}\PYG{l+s+s1}{\PYGZsq{}}\PYG{p}{)}
\end{sphinxVerbatim}

Calling m.reset() will clear all tallied values.

The statistics of a monitor can be printed with \sphinxcode{print\_statistics()}.
E.g: \sphinxcode{waitingline.length\_of\_stay.print\_statistics()}:

\begin{sphinxVerbatim}[commandchars=\\\{\}]
Statistics of Length of stay in waitingline at     50000
                        all    excl.zero         zero
\PYGZhy{}\PYGZhy{}\PYGZhy{}\PYGZhy{}\PYGZhy{}\PYGZhy{}\PYGZhy{}\PYGZhy{}\PYGZhy{}\PYGZhy{}\PYGZhy{}\PYGZhy{}\PYGZhy{}\PYGZhy{} \PYGZhy{}\PYGZhy{}\PYGZhy{}\PYGZhy{}\PYGZhy{}\PYGZhy{}\PYGZhy{}\PYGZhy{}\PYGZhy{}\PYGZhy{}\PYGZhy{}\PYGZhy{} \PYGZhy{}\PYGZhy{}\PYGZhy{}\PYGZhy{}\PYGZhy{}\PYGZhy{}\PYGZhy{}\PYGZhy{}\PYGZhy{}\PYGZhy{}\PYGZhy{}\PYGZhy{} \PYGZhy{}\PYGZhy{}\PYGZhy{}\PYGZhy{}\PYGZhy{}\PYGZhy{}\PYGZhy{}\PYGZhy{}\PYGZhy{}\PYGZhy{}\PYGZhy{}\PYGZhy{}
entries            4995         4933           62
mean                 84.345       85.405
std.deviation        48.309       47.672

minimum               0            0.006
median               94.843       95.411
90\PYGZpc{} percentile      142.751      142.975
95\PYGZpc{} percentile      157.467      157.611
maximum             202.153      202.153
\end{sphinxVerbatim}

And, a histogram can be printed with \sphinxcode{print\_histogram()}. E.g.
\sphinxcode{waitingline.length\_of\_stay.print\_histogram(30, 0, 10)}:

\begin{sphinxVerbatim}[commandchars=\\\{\}]
Histogram of Length of stay in waitingline
                        all    excl.zero         zero
\PYGZhy{}\PYGZhy{}\PYGZhy{}\PYGZhy{}\PYGZhy{}\PYGZhy{}\PYGZhy{}\PYGZhy{}\PYGZhy{}\PYGZhy{}\PYGZhy{}\PYGZhy{}\PYGZhy{}\PYGZhy{} \PYGZhy{}\PYGZhy{}\PYGZhy{}\PYGZhy{}\PYGZhy{}\PYGZhy{}\PYGZhy{}\PYGZhy{}\PYGZhy{}\PYGZhy{}\PYGZhy{}\PYGZhy{} \PYGZhy{}\PYGZhy{}\PYGZhy{}\PYGZhy{}\PYGZhy{}\PYGZhy{}\PYGZhy{}\PYGZhy{}\PYGZhy{}\PYGZhy{}\PYGZhy{}\PYGZhy{} \PYGZhy{}\PYGZhy{}\PYGZhy{}\PYGZhy{}\PYGZhy{}\PYGZhy{}\PYGZhy{}\PYGZhy{}\PYGZhy{}\PYGZhy{}\PYGZhy{}\PYGZhy{}
entries            4995         4933           62
mean                 84.345       85.405
std.deviation        48.309       47.672

minimum               0            0.006
median               94.843       95.411
90\PYGZpc{} percentile      142.751      142.975
95\PYGZpc{} percentile      157.467      157.611
maximum             202.153      202.153

           \PYGZlt{}=       entries     \PYGZpc{}  cum\PYGZpc{}
        0            62       1.2   1.2 \textbar{}
       10           169       3.4   4.6 ** \textbar{}
       20           284       5.7  10.3 ****    \textbar{}
       30           424       8.5  18.8 ******         \textbar{}
       40           372       7.4  26.2 *****               \textbar{}
       50           296       5.9  32.2 ****                     \textbar{}
       60           231       4.6  36.8 ***                          \textbar{}
       70           192       3.8  40.6 ***                             \textbar{}
       80           188       3.8  44.4 ***                                \textbar{}
       90           136       2.7  47.1 **                                   \textbar{}
      100           352       7.0  54.2 *****                                      \textbar{}
      110           491       9.8  64.0 *******                                            \textbar{}
      120           414       8.3  72.3 ******                                                   \textbar{}
      130           467       9.3  81.6 *******                                                          \textbar{}
      140           351       7.0  88.7 *****                                                                 \textbar{}
      150           224       4.5  93.2 ***                                                                       \textbar{}
      160           127       2.5  95.7 **                                                                          \textbar{}
      170            67       1.3  97.0 *                                                                            \textbar{}
      180            59       1.2  98.2                                                                               \textbar{}
      190            61       1.2  99.4                                                                                \textbar{}
      200            24       0.5  99.9                                                                                \textbar{}
      210             4       0.1 100                                                                                   \textbar{}
      220             0       0   100                                                                                   \textbar{}
      230             0       0   100                                                                                   \textbar{}
      240             0       0   100                                                                                   \textbar{}
      250             0       0   100                                                                                   \textbar{}
      260             0       0   100                                                                                   \textbar{}
      270             0       0   100                                                                                   \textbar{}
      280             0       0   100                                                                                   \textbar{}
      290             0       0   100                                                                                   \textbar{}
      300             0       0   100                                                                                   \textbar{}
          inf         0       0   100
\end{sphinxVerbatim}

If neither number\_of\_bins, nor lowerbound nor bin\_width are specified, the histogram will be autoscaled.

Histograms can be printed with their values, instead of bins. This is particularly useful for non
numeric tallied values, like names:

\begin{sphinxVerbatim}[commandchars=\\\{\}]
\PYG{k+kn}{import} \PYG{n+nn}{salabim} \PYG{k}{as} \PYG{n+nn}{sim}

\PYG{n}{env} \PYG{o}{=} \PYG{n}{sim}\PYG{o}{.}\PYG{n}{Environment}\PYG{p}{(}\PYG{p}{)}

\PYG{n}{monitor\PYGZus{}names}\PYG{o}{=} \PYG{n}{sim}\PYG{o}{.}\PYG{n}{Monitor}\PYG{p}{(}\PYG{n}{name}\PYG{o}{=}\PYG{l+s+s1}{\PYGZsq{}}\PYG{l+s+s1}{names}\PYG{l+s+s1}{\PYGZsq{}}\PYG{p}{)}
\PYG{k}{for} \PYG{n}{\PYGZus{}} \PYG{o+ow}{in} \PYG{n+nb}{range}\PYG{p}{(}\PYG{l+m+mi}{10000}\PYG{p}{)}\PYG{p}{:}
    \PYG{n}{name} \PYG{o}{=} \PYG{n}{sim}\PYG{o}{.}\PYG{n}{Pdf}\PYG{p}{(}\PYG{p}{(}\PYG{l+s+s1}{\PYGZsq{}}\PYG{l+s+s1}{John}\PYG{l+s+s1}{\PYGZsq{}}\PYG{p}{,} \PYG{l+m+mi}{30}\PYG{p}{,} \PYG{l+s+s1}{\PYGZsq{}}\PYG{l+s+s1}{Peter}\PYG{l+s+s1}{\PYGZsq{}}\PYG{p}{,} \PYG{l+m+mi}{20}\PYG{p}{,} \PYG{l+s+s1}{\PYGZsq{}}\PYG{l+s+s1}{Mike}\PYG{l+s+s1}{\PYGZsq{}}\PYG{p}{,} \PYG{l+m+mi}{20}\PYG{p}{,} \PYG{l+s+s1}{\PYGZsq{}}\PYG{l+s+s1}{Andrew}\PYG{l+s+s1}{\PYGZsq{}}\PYG{p}{,} \PYG{l+m+mi}{20}\PYG{p}{,} \PYG{l+s+s1}{\PYGZsq{}}\PYG{l+s+s1}{Ruud}\PYG{l+s+s1}{\PYGZsq{}}\PYG{p}{,} \PYG{l+m+mi}{5}\PYG{p}{,} \PYG{l+s+s1}{\PYGZsq{}}\PYG{l+s+s1}{Jan}\PYG{l+s+s1}{\PYGZsq{}}\PYG{p}{,} \PYG{l+m+mi}{5}\PYG{p}{)}\PYG{p}{)}\PYG{o}{.}\PYG{n}{sample}\PYG{p}{(}\PYG{p}{)}
    \PYG{n}{monitor\PYGZus{}names}\PYG{o}{.}\PYG{n}{tally}\PYG{p}{(}\PYG{n}{name}\PYG{p}{)}

\PYG{n}{monitor\PYGZus{}names}\PYG{o}{.}\PYG{n}{print\PYGZus{}histograms}\PYG{p}{(}\PYG{n}{values}\PYG{o}{=}\PYG{k+kc}{True}\PYG{p}{)}
\end{sphinxVerbatim}

The ouput of this:

\begin{sphinxVerbatim}[commandchars=\\\{\}]
Histogram of names
entries          10000

value               entries
Andrew                 2031( 20.3\PYGZpc{}) ****************
Jan                     495(  5.0\PYGZpc{}) ***
John                   2961( 29.6\PYGZpc{}) ***********************
Mike                   1989( 19.9\PYGZpc{}) ***************
Peter                  2048( 20.5\PYGZpc{}) ****************
Ruud                    476(  4.8\PYGZpc{}) ***
\end{sphinxVerbatim}


\section{Level monitor}
\label{\detokenize{Monitor:level-monitor}}
Level monitors tally levels along with the current (simulation) time.
e.g. the number of parts a machine is working on.

A level monitor can be defined by specifying \sphinxcode{level=True} in the initialization of Monitor, e.g.

\begin{sphinxVerbatim}[commandchars=\\\{\}]
\PYG{n}{working\PYGZus{}on\PYGZus{}parts} \PYG{o}{=} \PYG{n}{sim}\PYG{o}{.}\PYG{n}{Monitor}\PYG{p}{(}\PYG{n}{name}\PYG{o}{=}\PYG{l+s+s1}{\PYGZsq{}}\PYG{l+s+s1}{working\PYGZus{}on\PYGZus{}parts}\PYG{l+s+s1}{\PYGZsq{}}\PYG{p}{,} \PYG{n}{level}\PYG{o}{=}\PYG{k+kc}{True}\PYG{p}{,} \PYG{n}{initial\PYGZus{}tally}\PYG{o}{=}\PYG{l+m+mi}{0}\PYG{p}{)}
\end{sphinxVerbatim}

By default, the collected x-values are stored in a list. Alternatively, it is possible to store
the x-values in an array of one of the following types:


\begin{savenotes}\sphinxattablestart
\centering
\begin{tabulary}{\linewidth}[t]{|T|T|T|T|T|T|}
\hline
\sphinxstylethead{\sphinxstyletheadfamily 
type
\unskip}\relax &\sphinxstylethead{\sphinxstyletheadfamily 
stored as
\unskip}\relax &\sphinxstylethead{\sphinxstyletheadfamily 
lowerbound
\unskip}\relax &\sphinxstylethead{\sphinxstyletheadfamily 
upperbound
\unskip}\relax &\sphinxstylethead{\sphinxstyletheadfamily 
number of bytes
\unskip}\relax &\sphinxstylethead{\sphinxstyletheadfamily 
do not tally (=off)
\unskip}\relax \\
\hline
‘any’
&
list
&
N/A
&
N/A
&
depends on data
&
N/A{}`
\\
\hline
‘bool’
&
integer
&
False
&
True
&
1
&
255
\\
\hline
‘int8’
&
integer
&
-127
&
127
&
1
&
-128
\\
\hline
‘uint8’
&
integer
&
0
&
254
&
1
&
255
\\
\hline
‘int16’
&
integer
&
-32767
&
32767
&
2
&
-32768
\\
\hline
‘uint16’
&
integer
&
0
&
65534
&
2
&
65535
\\
\hline
‘int32’
&
integer
&
2147483647
&
2147483647
&
4
&
2147483648
\\
\hline
‘uint32’
&
integer
&
0
&
4294967294
&
4
&
4294967295
\\
\hline
‘int64’
&
integer
&
-9223372036854775807
&
9223372036854775807
&
8
&
-9223372036854775808
\\
\hline
‘uint64’
&
integer
&
0
&
18446744073709551614
&
8
&
18446744073709551615
\\
\hline
‘float’
&
float
&
-inf
&
inf
&
8
&
-inf
\\
\hline
\end{tabulary}
\par
\sphinxattableend\end{savenotes}

Monitoring with arrays takes up less space. Particularly when tallying a large
number of values, this is strongly advised.

Note that if non numeric x-values are stored (only possible with the default setting (‘any’)),
a tallied values is converted, if required, to a numeric value if possible, or 0 if not.

During the simulation run, it is possible to retrieve the last tallied value (which represents the ‘current’ value)
by calling Monitor.get(). 
It’s also possible to directly call the level monitor to get the current value, e.g.

\begin{sphinxVerbatim}[commandchars=\\\{\}]
\PYG{n}{mylevel} \PYG{o}{=} \PYG{n}{sim}\PYG{o}{.}\PYG{n}{Monitor}\PYG{p}{(}\PYG{l+s+s1}{\PYGZsq{}}\PYG{l+s+s1}{level}\PYG{l+s+s1}{\PYGZsq{}}\PYG{p}{,} \PYG{n}{level}\PYG{o}{=}\PYG{k+kc}{True}\PYG{p}{,} \PYG{n}{initial\PYGZus{}tally}\PYG{o}{=}\PYG{l+m+mi}{0}\PYG{p}{)}
\PYG{o}{.}\PYG{o}{.}\PYG{o}{.}
\PYG{n}{mylevel}\PYG{o}{.}\PYG{n}{tally}\PYG{p}{(}\PYG{l+m+mi}{10}\PYG{p}{)}
\PYG{o}{.}\PYG{o}{.}\PYG{o}{.}
\PYG{n+nb}{print} \PYG{p}{(}\PYG{n}{mylevel}\PYG{p}{(}\PYG{p}{)}\PYG{p}{)}  \PYG{c+c1}{\PYGZsh{} will print 10}
\end{sphinxVerbatim}

For the same reason, the standard length monitor of a queue can be used to get the current length of a queue: \sphinxcode{q.length()} although
the more Pythonic \sphinxcode{len(q)} is prefered.

When Monitor.get() is called with a time parameter or a direct call with a time parameter, the value at that time will be returned

\begin{sphinxVerbatim}[commandchars=\\\{\}]
\PYG{n+nb}{print} \PYG{p}{(}\PYG{n}{mylevel}\PYG{o}{.}\PYG{n}{get}\PYG{p}{(}\PYG{l+m+mi}{4}\PYG{p}{)}\PYG{p}{)}  \PYG{c+c1}{\PYGZsh{} will print the value at time 4}
\PYG{n+nb}{print} \PYG{p}{(}\PYG{n}{mylevel}\PYG{p}{(}\PYG{l+m+mi}{4}\PYG{p}{)}\PYG{p}{)}  \PYG{c+c1}{\PYGZsh{} will print the value at time 4}
\end{sphinxVerbatim}

There is whole set of statistical data available, which are all weighted with their duration:
\begin{itemize}
\item {} 
duration

\item {} 
duration\_zero (time that the value was zero)

\item {} 
mean

\item {} 
std

\item {} 
minimum

\item {} 
median

\item {} 
maximum

\item {} 
percentile

\item {} 
bin\_number\_of\_entries (number of entries between two given values)

\item {} 
bin\_duration (total duration of entries between two given values)

\item {} 
value\_number\_of\_entries (number of entries equal to a given value or set of values)

\item {} 
value\_duration (total duration of entries equal to a given value or set of values)

\end{itemize}

For all these statistics, it is possible to exclude zero entries, e.g. \sphinxcode{m.mean(ex0=True)} returns the mean, excluding zero entries.

The individual x-values and their duration can be retrieved xduration(). By default, the x-values will be returned as an array, even if
the type is ‘any’. In case the type is ‘any’ (stored as a list), the tallied x-values will be converted to a numeric value or 0 if
that’s not possible. By specifying \sphinxcode{force\_numeric=False} the collected x-values will be returned as stored.

The individual x-values and the associated timestamps can be retrieved with xt() or tx(). By default, the x-values will be returned as an array, even if
the type is ‘any’. In case the type is ‘any’ (stored as a list), the tallied x-values will be converted to a numeric value or 0 if
that’s not possible. By specifying \sphinxcode{force\_numeric=False} the collected x-values will be returned as stored.

When monitoring is disabled, an off value (see table above) will be tallied. All statistics will ignore the periods from this
off to a non-off value. This also holds for the xduration() method, but NOT for xt() and tx(). Thus,
the x-arrays of xduration() are not necessarily the same as the x-arrays in xt() and tx(). This is
the reason why there’s no x() or t() method. 
It is easy to get just the x-array with xduration(){[}0{]} or xt(){[}0{]}.

With the monitor method, a level monitor can be enbled or disabled.

Also, the current status (enabled/disabled) can be retrieved.

\begin{sphinxVerbatim}[commandchars=\\\{\}]
\PYG{n}{mylevel}\PYG{o}{.}\PYG{n}{monitor}\PYG{p}{(}\PYG{n+nb+bp}{False}\PYG{p}{)}  \PYG{c+c1}{\PYGZsh{} disable monitoring}
\PYG{n}{mylevel}\PYG{o}{.}\PYG{n}{monitor}\PYG{p}{(}\PYG{n+nb+bp}{True}\PYG{p}{)}  \PYG{c+c1}{\PYGZsh{} enable monitoring}
\PYG{k}{if} \PYG{n}{mylevel}\PYG{o}{.}\PYG{n}{monitor}\PYG{p}{(}\PYG{p}{)}\PYG{p}{:}
    \PYG{k}{print}\PYG{p}{(}\PYG{l+s+s1}{\PYGZsq{}}\PYG{l+s+s1}{level is enabled}\PYG{l+s+s1}{\PYGZsq{}}\PYG{p}{)}
\end{sphinxVerbatim}

It is strongly advised to keep tallying even when monitoring is off, in order to be able to access the current value at any time. The values tallied when monitoring is off
are not stored.

Calling m.reset() will clear all tallied values and timestamps.

The statistics of a level monitor can be printed with \sphinxcode{print\_statistics()}.
E.g: \sphinxcode{waitingline.length.print\_statistics()}:

\begin{sphinxVerbatim}[commandchars=\\\{\}]
Statistics of Length of waitingline at     50000
                        all    excl.zero         zero
\PYGZhy{}\PYGZhy{}\PYGZhy{}\PYGZhy{}\PYGZhy{}\PYGZhy{}\PYGZhy{}\PYGZhy{}\PYGZhy{}\PYGZhy{}\PYGZhy{}\PYGZhy{}\PYGZhy{}\PYGZhy{} \PYGZhy{}\PYGZhy{}\PYGZhy{}\PYGZhy{}\PYGZhy{}\PYGZhy{}\PYGZhy{}\PYGZhy{}\PYGZhy{}\PYGZhy{}\PYGZhy{}\PYGZhy{} \PYGZhy{}\PYGZhy{}\PYGZhy{}\PYGZhy{}\PYGZhy{}\PYGZhy{}\PYGZhy{}\PYGZhy{}\PYGZhy{}\PYGZhy{}\PYGZhy{}\PYGZhy{} \PYGZhy{}\PYGZhy{}\PYGZhy{}\PYGZhy{}\PYGZhy{}\PYGZhy{}\PYGZhy{}\PYGZhy{}\PYGZhy{}\PYGZhy{}\PYGZhy{}\PYGZhy{}
duration          50000        48499.381     1500.619
mean                  8.427        8.687
std.deviation         4.852        4.691

minimum               0            1
median                9           10
90\PYGZpc{} percentile       14           14
95\PYGZpc{} percentile       16           16
maximum              21           21
\end{sphinxVerbatim}

And, a histogram can be printed with \sphinxcode{print\_histogram()}. E.g.

\begin{sphinxVerbatim}[commandchars=\\\{\}]
\PYG{n}{waitingline}\PYG{o}{.}\PYG{n}{length}\PYG{o}{.}\PYG{n}{print\PYGZus{}histogram}\PYG{p}{(}\PYG{l+m+mi}{30}\PYG{p}{,} \PYG{l+m+mi}{0}\PYG{p}{,} \PYG{l+m+mi}{1}\PYG{p}{)}
\end{sphinxVerbatim}

\begin{sphinxVerbatim}[commandchars=\\\{\}]
Histogram of Length of waitingline
                        all    excl.zero         zero
\PYGZhy{}\PYGZhy{}\PYGZhy{}\PYGZhy{}\PYGZhy{}\PYGZhy{}\PYGZhy{}\PYGZhy{}\PYGZhy{}\PYGZhy{}\PYGZhy{}\PYGZhy{}\PYGZhy{}\PYGZhy{} \PYGZhy{}\PYGZhy{}\PYGZhy{}\PYGZhy{}\PYGZhy{}\PYGZhy{}\PYGZhy{}\PYGZhy{}\PYGZhy{}\PYGZhy{}\PYGZhy{}\PYGZhy{} \PYGZhy{}\PYGZhy{}\PYGZhy{}\PYGZhy{}\PYGZhy{}\PYGZhy{}\PYGZhy{}\PYGZhy{}\PYGZhy{}\PYGZhy{}\PYGZhy{}\PYGZhy{} \PYGZhy{}\PYGZhy{}\PYGZhy{}\PYGZhy{}\PYGZhy{}\PYGZhy{}\PYGZhy{}\PYGZhy{}\PYGZhy{}\PYGZhy{}\PYGZhy{}\PYGZhy{}
duration          50000        48499.381     1500.619
mean                  8.427        8.687
std.deviation         4.852        4.691

minimum               0            1
median                9           10
90\PYGZpc{} percentile       14           14
95\PYGZpc{} percentile       16           16
maximum              21           21

           \PYGZlt{}=      duration     \PYGZpc{}  cum\PYGZpc{}
        0          1500.619   3.0   3.0 **\textbar{}
        1          2111.284   4.2   7.2 ***  \textbar{}
        2          3528.851   7.1  14.3 *****      \textbar{}
        3          4319.406   8.6  22.9 ******            \textbar{}
        4          3354.732   6.7  29.6 *****                  \textbar{}
        5          2445.603   4.9  34.5 ***                        \textbar{}
        6          2090.759   4.2  38.7 ***                           \textbar{}
        7          2046.126   4.1  42.8 ***                               \textbar{}
        8          1486.956   3.0  45.8 **                                  \textbar{}
        9          2328.863   4.7  50.4 ***                                     \textbar{}
       10          4337.502   8.7  59.1 ******                                         \textbar{}
       11          4546.145   9.1  68.2 *******                                               \textbar{}
       12          4484.405   9.0  77.2 *******                                                      \textbar{}
       13          4134.094   8.3  85.4 ******                                                              \textbar{}
       14          2813.860   5.6  91.1 ****                                                                    \textbar{}
       15          1714.894   3.4  94.5 **                                                                         \textbar{}
       16           992.690   2.0  96.5 *                                                                            \textbar{}
       17           541.546   1.1  97.6                                                                               \textbar{}
       18           625.048   1.3  98.8 *                                                                              \textbar{}
       19           502.291   1.0  99.8                                                                                \textbar{}
       20            86.168   0.2 100.0                                                                                \textbar{}
       21             8.162   0.0 100                                                                                   \textbar{}
       22             0       0   100                                                                                   \textbar{}
       23             0       0   100                                                                                   \textbar{}
       24             0       0   100                                                                                   \textbar{}
       25             0       0   100                                                                                   \textbar{}
       26             0       0   100                                                                                   \textbar{}
       27             0       0   100                                                                                   \textbar{}
       28             0       0   100                                                                                   \textbar{}
       29             0       0   100                                                                                   \textbar{}
       30             0       0   100                                                                                   \textbar{}
          inf         0       0   100
\end{sphinxVerbatim}

If neither number\_of\_bins, nor lowerbound nor bin\_width are specified, the histogram will be autoscaled.

Histograms can be printed with their values, instead of bins. This is particularly useful for non
numeric tallied values, like names.


\section{Merging of monitors}
\label{\detokenize{Monitor:merging-of-monitors}}
Monitors can be merged, to create a new monitor, nearly always to collect aggregated data.

The method Monitor.mertge() is used for that, like

\begin{sphinxVerbatim}[commandchars=\\\{\}]
\PYG{n}{mc} \PYG{o}{=} \PYG{n}{m0}\PYG{o}{.}\PYG{n}{merge}\PYG{p}{(}\PYG{n}{m1}\PYG{p}{,} \PYG{n}{m2}\PYG{p}{)}
\end{sphinxVerbatim}

Then we can just get the mean of the monitors m0, m1 and m2 combined.

For non level monitors, all of the tallied x-values are copied from the to be merged monitors.
For level monitors, the x-values are summed, for all the periods where all the monitors were on.
Periods where one or more monitors were off, are excluded.
Note that the merge only takes place at creation of the (timestamped) monitor and not dynamically later.

Sample usage:
Suppose we have three types of products, that each have a queue for processing, so
a.processing, b.processing, c.processing.
If we want to print the histogram of the combined (=summed) length of these queues

\begin{sphinxVerbatim}[commandchars=\\\{\}]
\PYG{n}{a}\PYG{o}{.}\PYG{n}{processing}\PYG{o}{.}\PYG{n}{length}\PYG{o}{.}\PYG{n}{merge}\PYG{p}{(}\PYG{n}{b}\PYG{o}{.}\PYG{n}{processing}\PYG{o}{.}\PYG{n}{length}\PYG{p}{,} \PYG{n}{c}\PYG{o}{.}\PYG{n}{processing}\PYG{o}{.}\PYG{n}{length}\PYG{p}{,} \PYG{n}{name}\PYG{o}{=}\PYG{l+s+s1}{\PYGZsq{}}\PYG{l+s+s1}{combined processing length}\PYG{l+s+s1}{\PYGZsq{}}\PYG{p}{)}\PYG{p}{)}\PYG{o}{.}\PYG{n}{print\PYGZus{}histogram}\PYG{p}{(}\PYG{p}{)}
\end{sphinxVerbatim}

and to print the histogram of the length\_of\_stay for all entries

\begin{sphinxVerbatim}[commandchars=\\\{\}]
\PYG{n}{a}\PYG{o}{.}\PYG{n}{processing}\PYG{o}{.}\PYG{n}{length}\PYG{o}{.}\PYG{n}{merge}\PYG{p}{(}\PYG{n}{b}\PYG{o}{.}\PYG{n}{processing}\PYG{o}{.}\PYG{n}{length}\PYG{p}{,} \PYG{n}{c}\PYG{o}{.}\PYG{n}{processing}\PYG{o}{.}\PYG{n}{length}\PYG{p}{,} \PYG{n}{name}\PYG{o}{=}\PYG{l+s+s1}{\PYGZsq{}}\PYG{l+s+s1}{combined processing length}\PYG{l+s+s1}{\PYGZsq{}}\PYG{p}{)}\PYG{p}{)}\PYG{o}{.}\PYG{n}{print\PYGZus{}histogram}\PYG{p}{(}\PYG{p}{)}

\PYG{n}{Monitor}\PYG{p}{(}\PYG{n}{name}\PYG{o}{=}\PYG{l+s+s1}{\PYGZsq{}}\PYG{l+s+s1}{combined processing length of stay}\PYG{l+s+s1}{\PYGZsq{}}\PYG{p}{,}
    \PYG{n}{merge}\PYG{o}{=}\PYG{p}{(}\PYG{n}{a}\PYG{o}{.}\PYG{n}{processing}\PYG{o}{.}\PYG{n}{length\PYGZus{}of\PYGZus{}stay}\PYG{p}{,} \PYG{n}{b}\PYG{o}{.}\PYG{n}{processing}\PYG{o}{.}\PYG{n}{length\PYGZus{}of\PYGZus{}stay}\PYG{p}{,} \PYG{n}{c}\PYG{o}{.}\PYG{n}{processing}\PYG{o}{.}\PYG{n}{length\PYGZus{}of\PYGZus{}stay}\PYG{p}{)}\PYG{p}{)}\PYG{o}{.}\PYG{n}{print\PYGZus{}histogram}\PYG{p}{(}\PYG{p}{)}
\end{sphinxVerbatim}


\section{Slicing of monitors}
\label{\detokenize{Monitor:slicing-of-monitors}}
It is possible to slice a monitor with Monitor.slice(), which has two applications:
\begin{itemize}
\item {} 
to get statistics on a monitor with respect to a given time period, most likely a subrun

\item {} 
to get statistics on a monitor with respect to a recurring time period, like hour 0-1, hour 0-2, etc.

\end{itemize}

Examples

\begin{sphinxVerbatim}[commandchars=\\\{\}]
\PYG{k}{for} \PYG{n}{i} \PYG{o+ow}{in} \PYG{n+nb}{range}\PYG{p}{(}\PYG{l+m+mi}{10}\PYG{p}{)}\PYG{p}{:}
    \PYG{n}{start} \PYG{o}{=} \PYG{n}{i} \PYG{o}{*} \PYG{l+m+mi}{1000}
    \PYG{n}{stop} \PYG{o}{=} \PYG{p}{(}\PYG{n}{i}\PYG{o}{+}\PYG{l+m+mi}{1}\PYG{p}{)} \PYG{o}{*} \PYG{l+m+mi}{1000}
    \PYG{n+nb}{print}\PYG{p}{(}\PYG{n}{f}\PYG{l+s+s1}{\PYGZsq{}}\PYG{l+s+s1}{mean length of q in [}\PYG{l+s+si}{\PYGZob{}start\PYGZcb{}}\PYG{l+s+s1}{,}\PYG{l+s+si}{\PYGZob{}stop\PYGZcb{}}\PYG{l+s+s1}{)=}\PYG{l+s+s1}{\PYGZob{}}\PYG{l+s+s1}{q.length.slice(start,stop).mean()\PYGZcb{}}\PYG{l+s+s1}{\PYGZsq{}}
    \PYG{n+nb}{print}\PYG{p}{(}\PYG{n}{f}\PYG{l+s+s1}{\PYGZsq{}}\PYG{l+s+s1}{mean length of stay in [}\PYG{l+s+si}{\PYGZob{}start\PYGZcb{}}\PYG{l+s+s1}{,}\PYG{l+s+si}{\PYGZob{}stop\PYGZcb{}}\PYG{l+s+s1}{)=}\PYG{l+s+s1}{\PYGZob{}}\PYG{l+s+s1}{q.length\PYGZus{}of\PYGZus{}stay.slice(start,stop).mean()\PYGZcb{}}\PYG{l+s+s1}{\PYGZsq{}}

\PYG{k}{for} \PYG{n}{i} \PYG{o+ow}{in} \PYG{n+nb}{range}\PYG{p}{(}\PYG{l+m+mi}{24}\PYG{p}{)}\PYG{p}{:}
    \PYG{n+nb}{print}\PYG{p}{(}\PYG{n}{f}\PYG{l+s+s1}{\PYGZsq{}}\PYG{l+s+s1}{mean length of q in hour }\PYG{l+s+si}{\PYGZob{}i\PYGZcb{}}\PYG{l+s+s1}{=}\PYG{l+s+s1}{\PYGZob{}}\PYG{l+s+s1}{q.length.slice(i, i+1, 24).mean()\PYGZcb{}}\PYG{l+s+s1}{\PYGZsq{}}
    \PYG{n+nb}{print}\PYG{p}{(}\PYG{n}{f}\PYG{l+s+s1}{\PYGZsq{}}\PYG{l+s+s1}{mean length of stay of q in hour }\PYG{l+s+si}{\PYGZob{}i\PYGZcb{}}\PYG{l+s+s1}{=}\PYG{l+s+s1}{\PYGZob{}}\PYG{l+s+s1}{q.length\PYGZus{}of\PYGZus{}stay.slice(i, i+1, 24).mean()\PYGZcb{}}\PYG{l+s+s1}{\PYGZsq{}}
\end{sphinxVerbatim}

Instead of slice(), a monitor can be sliced as well with the standard slice operator {[}{]}, like

\begin{sphinxVerbatim}[commandchars=\\\{\}]
\PYG{n}{q}\PYG{o}{.}\PYG{n}{length}\PYG{p}{[}\PYG{l+m+mi}{1000}\PYG{p}{:}\PYG{l+m+mi}{2000}\PYG{p}{]}\PYG{o}{.}\PYG{n}{print\PYGZus{}histogram}\PYG{p}{(}\PYG{p}{)}
\PYG{n}{q}\PYG{o}{.}\PYG{n}{length}\PYG{p}{[}\PYG{l+m+mi}{2}\PYG{p}{:}\PYG{l+m+mi}{3}\PYG{p}{:}\PYG{l+m+mi}{24}\PYG{p}{]}\PYG{o}{.}\PYG{n}{print\PYGZus{}histogram}\PYG{p}{(}\PYG{p}{)}
\end{sphinxVerbatim}


\chapter{Distributions}
\label{\detokenize{Distributions:distributions}}\label{\detokenize{Distributions::doc}}
Salabim can be used with the standard random module, but it is easier to use the salabim distributions.

Internally, salabim uses the random module. There is always a seed associated with each distribution, which
is normally random.random.

When a new environment is created, the random seed 1234567 will be set by default. However, it is possible to
override this behaviour with the random\_seed parameter:
\begin{itemize}
\item {} 
any hashable value, to set another seed

\item {} 
null string (‘’): no reseeding

\item {} 
None: true random, non reproducible (based on current time)

\end{itemize}

As a distribution is an instance of a class, it can be used in assignment, parameters, etc. E.g.

\begin{sphinxVerbatim}[commandchars=\\\{\}]
\PYG{n}{inter\PYGZus{}arrival\PYGZus{}time} \PYG{o}{=} \PYG{n}{sim}\PYG{o}{.}\PYG{n}{Uniform}\PYG{p}{(}\PYG{l+m+mi}{10}\PYG{p}{,}\PYG{l+m+mi}{15}\PYG{p}{)}
\end{sphinxVerbatim}

And then, to wait for a time sampled from this distribution

\begin{sphinxVerbatim}[commandchars=\\\{\}]
\PYG{k}{yield} \PYG{n+nb+bp}{self}\PYG{o}{.}\PYG{n}{hold}\PYG{p}{(}\PYG{n}{inter\PYGZus{}arrival\PYGZus{}time}\PYG{o}{.}\PYG{n}{sample}\PYG{p}{(}\PYG{p}{)}\PYG{p}{)}
\end{sphinxVerbatim}
\begin{description}
\item[{or ::}] \leavevmode
yield self.hold(inter\_arrival\_time())

\end{description}

or

\begin{sphinxVerbatim}[commandchars=\\\{\}]
\PYG{k}{yield} \PYG{n+nb+bp}{self}\PYG{o}{.}\PYG{n}{hold}\PYG{p}{(}\PYG{n}{sim}\PYG{o}{.}\PYG{n}{Uniform}\PYG{p}{(}\PYG{l+m+mi}{10}\PYG{p}{,}\PYG{l+m+mi}{15}\PYG{p}{)}\PYG{o}{.}\PYG{n}{sample}\PYG{p}{(}\PYG{p}{)}\PYG{p}{)}
\end{sphinxVerbatim}

or

\begin{sphinxVerbatim}[commandchars=\\\{\}]
\PYG{k}{yield} \PYG{n+nb+bp}{self}\PYG{o}{.}\PYG{n}{hold}\PYG{p}{(}\PYG{n}{sim}\PYG{o}{.}\PYG{n}{Uniform}\PYG{p}{(}\PYG{l+m+mi}{10}\PYG{p}{,}\PYG{l+m+mi}{15}\PYG{p}{)}\PYG{p}{(}\PYG{p}{)}\PYG{p}{)}
\end{sphinxVerbatim}

All distributions are a subclass of \_Distribution which supports the following methods:
\begin{itemize}
\item {} 
mean()

\item {} 
sample()

\item {} 
direct calling as an alternative to sample, like Uniform(12,15)()

\item {} 
bounded\_sample()  \# see below

\end{itemize}

For each distribution it is possible to limit the sampled values between a lowerbound and an upperbound, by using
the method bounded\_sample(). This is particularly useful when a duration is required, which has to be positive, but the distribution
it self does not guarantee a positive value, e.g.

\begin{sphinxVerbatim}[commandchars=\\\{\}]
\PYG{n}{duration} \PYG{o}{=} \PYG{n}{sim}\PYG{o}{.}\PYG{n}{Normal}\PYG{p}{(}\PYG{l+m+mi}{5}\PYG{p}{,}\PYG{l+m+mi}{5}\PYG{p}{)}\PYG{o}{.}\PYG{n}{bounded\PYGZus{}sample}\PYG{p}{(}\PYG{n}{lowerbound}\PYG{o}{=}\PYG{l+m+mi}{0}\PYG{p}{)}
\end{sphinxVerbatim}

Salabim provides the following distribution classes:


\section{Beta}
\label{\detokenize{Distributions:beta}}
Beta distribution with a given
\begin{itemize}
\item {} 
alpha (shape)

\item {} 
beta (shape)

\end{itemize}

E.g.

\begin{sphinxVerbatim}[commandchars=\\\{\}]
\PYG{n}{processing\PYGZus{}time} \PYG{o}{=} \PYG{n}{sim}\PYG{o}{.}\PYG{n}{Beta}\PYG{p}{(}\PYG{l+m+mi}{2}\PYG{p}{,}\PYG{l+m+mi}{4}\PYG{p}{)}  \PYG{c+c1}{\PYGZsh{} Beta with alpha=2, beta=4{}`}
\end{sphinxVerbatim}


\section{Constant}
\label{\detokenize{Distributions:constant}}
No sampling is required for this distribution, as it always returns the same value. E.g.

\begin{sphinxVerbatim}[commandchars=\\\{\}]
\PYG{n}{processing\PYGZus{}time} \PYG{o}{=} \PYG{n}{sim}\PYG{o}{.}\PYG{n}{Constant}\PYG{p}{(}\PYG{l+m+mi}{10}\PYG{p}{)}
\end{sphinxVerbatim}


\section{Erlang}
\label{\detokenize{Distributions:erlang}}
Erlang distribution with a givenl
\begin{itemize}
\item {} 
shape (k)

\item {} 
rate (lambda) or scale (mu)

\end{itemize}

E.g.

\begin{sphinxVerbatim}[commandchars=\\\{\}]
\PYG{n}{inter\PYGZus{}arrival\PYGZus{}time} \PYG{o}{=} \PYG{n}{sim}\PYG{o}{.}\PYG{n}{Erlang}\PYG{p}{(}\PYG{l+m+mi}{2}\PYG{p}{,} \PYG{n}{rate}\PYG{o}{=}\PYG{l+m+mi}{2}\PYG{p}{)}  \PYG{c+c1}{\PYGZsh{} Erlang\PYGZhy{}2, with lambda = 2}
\end{sphinxVerbatim}


\section{Gamma}
\label{\detokenize{Distributions:gamma}}
Gamma distribution with given
\begin{itemize}
\item {} 
shape (k)

\item {} 
scale (teta) or rate (beta)

\end{itemize}

E.g.

\begin{sphinxVerbatim}[commandchars=\\\{\}]
\PYG{n}{processing\PYGZus{}time} \PYG{o}{=} \PYG{n}{sim}\PYG{o}{.}\PYG{n}{Gamma}\PYG{p}{(}\PYG{l+m+mi}{2}\PYG{p}{,}\PYG{l+m+mi}{3}\PYG{p}{)}  \PYG{c+c1}{\PYGZsh{} Gamma with k=2, teta=3}
\end{sphinxVerbatim}


\section{Exponential}
\label{\detokenize{Distributions:exponential}}
Exponential distribution with a given
\begin{itemize}
\item {} 
mean or rate (lambda)

\end{itemize}

E.g.

\begin{sphinxVerbatim}[commandchars=\\\{\}]
\PYG{n}{inter\PYGZus{}arrival\PYGZus{}time} \PYG{o}{=} \PYG{n}{sim}\PYG{o}{.}\PYG{n}{Exponential}\PYG{p}{(}\PYG{l+m+mi}{10}\PYG{p}{)}  \PYG{c+c1}{\PYGZsh{} on an average every 10 time units}
\end{sphinxVerbatim}


\section{IntUniform}
\label{\detokenize{Distributions:intuniform}}
Integer uniform distribution between a given
\begin{itemize}
\item {} 
lowerbound

\item {} 
upperbound (inclusive)

\end{itemize}

E.g.

\begin{sphinxVerbatim}[commandchars=\\\{\}]
\PYG{n}{die} \PYG{o}{=} \PYG{n}{sim}\PYG{o}{.}\PYG{n}{IntUniform}\PYG{p}{(}\PYG{l+m+mi}{1}\PYG{p}{,} \PYG{l+m+mi}{6}\PYG{p}{)}
\end{sphinxVerbatim}


\section{Normal}
\label{\detokenize{Distributions:normal}}
Normal distribution with a given
\begin{itemize}
\item {} 
mean

\item {} 
standard deviation

\end{itemize}

E.g.

\begin{sphinxVerbatim}[commandchars=\\\{\}]
\PYG{n}{processing\PYGZus{}time} \PYG{o}{=} \PYG{n}{sim}\PYG{o}{.}\PYG{n}{Normal}\PYG{p}{(}\PYG{l+m+mi}{10}\PYG{p}{,} \PYG{l+m+mi}{2}\PYG{p}{)}  \PYG{c+c1}{\PYGZsh{} Normal with mean=10, standard deviation=2}
\end{sphinxVerbatim}

Note that this might result in negative values, which might not correct if it is a duration. In that case,
use bounded\_sample like

\begin{sphinxVerbatim}[commandchars=\\\{\}]
\PYG{k}{yield} \PYG{n+nb+bp}{self}\PYG{o}{.}\PYG{n}{hold}\PYG{p}{(}\PYG{n}{processing\PYGZus{}time}\PYG{o}{.}\PYG{n}{bounded\PYGZus{}sample}\PYG{p}{(}\PYG{p}{)}\PYG{p}{)}
\end{sphinxVerbatim}

Normally, sampling is done with the random.normalvariate method. Alternatively, the random.gauss method can be used.


\section{Poisson}
\label{\detokenize{Distributions:poisson}}
Poisson distribution with a lambda

E.g.

\begin{sphinxVerbatim}[commandchars=\\\{\}]
\PYG{n}{occurences\PYGZus{}in\PYGZus{}one\PYGZus{}hour} \PYG{o}{=} \PYG{n}{sim}\PYG{o}{.}\PYG{n}{Poisson}\PYG{p}{(}\PYG{l+m+mi}{10}\PYG{p}{)}  \PYG{c+c1}{\PYGZsh{} Poisson distribution with lambda (and thus mean) = 10}
\end{sphinxVerbatim}


\section{Triangular}
\label{\detokenize{Distributions:triangular}}
Triangular distribution with a given
\begin{itemize}
\item {} 
lowerbound

\item {} 
upperbound

\item {} 
median

\end{itemize}

E.g.

\begin{sphinxVerbatim}[commandchars=\\\{\}]
\PYG{n}{processing\PYGZus{}time} \PYG{o}{=} \PYG{n}{sim}\PYG{o}{.}\PYG{n}{Triangular}\PYG{p}{(}\PYG{l+m+mi}{5}\PYG{p}{,} \PYG{l+m+mi}{15}\PYG{p}{,} \PYG{l+m+mi}{8}\PYG{p}{)}
\end{sphinxVerbatim}


\section{Uniform}
\label{\detokenize{Distributions:uniform}}
Uniform distribution between a given
\begin{itemize}
\item {} 
lowerbound

\item {} 
upperbound

\end{itemize}

E.g.

\begin{sphinxVerbatim}[commandchars=\\\{\}]
\PYG{n}{processing\PYGZus{}time} \PYG{o}{=} \PYG{n}{sim}\PYG{o}{.}\PYG{n}{Uniform}\PYG{p}{(}\PYG{l+m+mi}{5}\PYG{p}{,} \PYG{l+m+mi}{15}\PYG{p}{)}
\end{sphinxVerbatim}


\section{Weibull}
\label{\detokenize{Distributions:weibull}}
Weibull distribution with given
\begin{itemize}
\item {} 
scale (alpha or k)

\item {} 
shape (beta or lambda)

\end{itemize}

E.g.

\begin{sphinxVerbatim}[commandchars=\\\{\}]
\PYG{n}{time\PYGZus{}between\PYGZus{}failure} \PYG{o}{=} \PYG{n}{sim}\PYG{o}{.}\PYG{n}{Weibull}\PYG{p}{(}\PYG{l+m+mi}{2}\PYG{p}{,} \PYG{l+m+mi}{5}\PYG{p}{)}  \PYG{c+c1}{\PYGZsh{} Weibull with k=2. lambda=5}
\end{sphinxVerbatim}


\section{Cdf}
\label{\detokenize{Distributions:cdf}}
Cumulative distribution function, specified as a list or tuple with x{[}i{]},p{[}i{]} values, where p{[}i{]} is the cumulative probability
that xn\textless{}=pn. E.g.

\begin{sphinxVerbatim}[commandchars=\\\{\}]
\PYG{n}{processingtime} \PYG{o}{=} \PYG{n}{sim}\PYG{o}{.}\PYG{n}{Cdf}\PYG{p}{(}\PYG{p}{(}\PYG{l+m+mi}{5}\PYG{p}{,} \PYG{l+m+mi}{0}\PYG{p}{,} \PYG{l+m+mi}{10}\PYG{p}{,} \PYG{l+m+mi}{50}\PYG{p}{,} \PYG{l+m+mi}{15}\PYG{p}{,} \PYG{l+m+mi}{90}\PYG{p}{,} \PYG{l+m+mi}{30}\PYG{p}{,} \PYG{l+m+mi}{95}\PYG{p}{,} \PYG{l+m+mi}{60}\PYG{p}{,} \PYG{l+m+mi}{100}\PYG{p}{)}\PYG{p}{)}
\end{sphinxVerbatim}

This means that 0\% is \textless{}5, 50\% is \textless{} 10, 90\% is \textless{} 15, 95\% is \textless{} 30 and 100\% is \textless{}60.

\begin{sphinxadmonition}{note}{Note:}
It is required that p{[}0{]} is 0 and that p{[}i{]}\textless{}=p{[}i+1{]} and that x{[}i{]}\textless{}=x{[}i+1{]}.
\end{sphinxadmonition}

It is not required that the last p{[}{]} is 100, as all p{[}{]}’s are automatically scaled. This means that the two distributions below are
identical to the first example

\begin{sphinxVerbatim}[commandchars=\\\{\}]
\PYG{n}{processingtime} \PYG{o}{=} \PYG{n}{sim}\PYG{o}{.}\PYG{n}{Cdf}\PYG{p}{(}\PYG{p}{(}\PYG{l+m+mi}{5}\PYG{p}{,} \PYG{l+m+mf}{0.00}\PYG{p}{,} \PYG{l+m+mi}{10}\PYG{p}{,} \PYG{l+m+mf}{0.50}\PYG{p}{,} \PYG{l+m+mi}{15}\PYG{p}{,} \PYG{l+m+mf}{0.90}\PYG{p}{,} \PYG{l+m+mi}{30}\PYG{p}{,} \PYG{l+m+mf}{0.95}\PYG{p}{,} \PYG{l+m+mi}{60}\PYG{p}{,} \PYG{l+m+mf}{1.00}\PYG{p}{)}\PYG{p}{)}
\PYG{n}{processingtime} \PYG{o}{=} \PYG{n}{sim}\PYG{o}{.}\PYG{n}{Cdf}\PYG{p}{(}\PYG{p}{(}\PYG{l+m+mi}{5}\PYG{p}{,}    \PYG{l+m+mi}{0}\PYG{p}{,} \PYG{l+m+mi}{10}\PYG{p}{,}   \PYG{l+m+mi}{10}\PYG{p}{,} \PYG{l+m+mi}{15}\PYG{p}{,}   \PYG{l+m+mi}{18}\PYG{p}{,} \PYG{l+m+mi}{30}\PYG{p}{,}   \PYG{l+m+mi}{19}\PYG{p}{,} \PYG{l+m+mi}{60}\PYG{p}{,}   \PYG{l+m+mi}{20}\PYG{p}{)}\PYG{p}{)}
\end{sphinxVerbatim}


\section{Pdf}
\label{\detokenize{Distributions:pdf}}
Probability density function, specified as:
\begin{enumerate}
\item {} 
list or tuple of x{[}i{]}, p{[}i{]} where p{[}i{]} is the probability (density)

\item {} 
list or tuple of x{[}i{]} followed by a list or tuple p{[}i{]}

\item {} 
list or tuple of x{[}i{]} followed by a scalar (value not important)

\end{enumerate}

\begin{sphinxadmonition}{note}{Note:}
It is required that the sum of p{[}i{]}’s is \sphinxstylestrong{greater than} 0.
\end{sphinxadmonition}

E.g.

\begin{sphinxVerbatim}[commandchars=\\\{\}]
\PYG{n}{processingtime} \PYG{o}{=} \PYG{n}{sim}\PYG{o}{.}\PYG{n}{Pdf}\PYG{p}{(}\PYG{p}{(}\PYG{l+m+mi}{5}\PYG{p}{,} \PYG{l+m+mi}{10}\PYG{p}{,} \PYG{l+m+mi}{10}\PYG{p}{,} \PYG{l+m+mi}{50}\PYG{p}{,} \PYG{l+m+mi}{15}\PYG{p}{,} \PYG{l+m+mi}{40}\PYG{p}{)}\PYG{p}{)}
\end{sphinxVerbatim}

This means that 10\% is 5, 50\% is 10 and 40\% is 15.

It is not required that the sum of the p{[}i{]}’s is 100, as all p{[}{]}’s are automatically scaled. This means that the two distributions below are
identical to the first example

\begin{sphinxVerbatim}[commandchars=\\\{\}]
\PYG{n}{processingtime} \PYG{o}{=} \PYG{n}{sim}\PYG{o}{.}\PYG{n}{Pdf}\PYG{p}{(}\PYG{p}{(}\PYG{l+m+mi}{5}\PYG{p}{,} \PYG{l+m+mf}{0.10}\PYG{p}{,} \PYG{l+m+mi}{10}\PYG{p}{,} \PYG{l+m+mf}{0.50}\PYG{p}{,} \PYG{l+m+mi}{15}\PYG{p}{,} \PYG{l+m+mf}{0.40}\PYG{p}{)}\PYG{p}{)}
\PYG{n}{processingtime} \PYG{o}{=} \PYG{n}{sim}\PYG{o}{.}\PYG{n}{Pdf}\PYG{p}{(}\PYG{p}{(}\PYG{l+m+mi}{5}\PYG{p}{,}    \PYG{l+m+mi}{2}\PYG{p}{,} \PYG{l+m+mi}{10}\PYG{p}{,}   \PYG{l+m+mi}{10}\PYG{p}{,} \PYG{l+m+mi}{15}\PYG{p}{,}    \PYG{l+m+mi}{8}\PYG{p}{)}\PYG{p}{)}
\end{sphinxVerbatim}

And the same with the second form

\begin{sphinxVerbatim}[commandchars=\\\{\}]
\PYG{n}{processingtime} \PYG{o}{=} \PYG{n}{sim}\PYG{o}{.}\PYG{n}{Pdf}\PYG{p}{(}\PYG{p}{(}\PYG{l+m+mi}{5}\PYG{p}{,} \PYG{l+m+mi}{10}\PYG{p}{,} \PYG{l+m+mi}{15}\PYG{p}{)}\PYG{p}{,} \PYG{p}{(}\PYG{l+m+mi}{10}\PYG{p}{,} \PYG{l+m+mi}{50}\PYG{p}{,} \PYG{l+m+mi}{40}\PYG{p}{)}\PYG{p}{)}
\end{sphinxVerbatim}

If all x{[}i{]}’s have the same probability, the third form is very useful

\begin{sphinxVerbatim}[commandchars=\\\{\}]
\PYG{n}{dice} \PYG{o}{=} \PYG{n}{sim}\PYG{o}{.}\PYG{n}{Pdf}\PYG{p}{(}\PYG{p}{(}\PYG{l+m+mi}{1}\PYG{p}{,}\PYG{l+m+mi}{2}\PYG{p}{,}\PYG{l+m+mi}{3}\PYG{p}{,}\PYG{l+m+mi}{4}\PYG{p}{,}\PYG{l+m+mi}{5}\PYG{p}{,}\PYG{l+m+mi}{6}\PYG{p}{)}\PYG{p}{,}\PYG{l+m+mi}{1}\PYG{p}{)}  \PYG{c+c1}{\PYGZsh{} the distribution IntUniform(1,6) does the job as well}
\PYG{n}{dice} \PYG{o}{=} \PYG{n}{sim}\PYG{o}{.}\PYG{n}{Pdf}\PYG{p}{(}\PYG{n+nb}{range}\PYG{p}{(}\PYG{l+m+mi}{1}\PYG{p}{,}\PYG{l+m+mi}{7}\PYG{p}{)}\PYG{p}{,}\PYG{l+m+mi}{1}\PYG{p}{)}  \PYG{c+c1}{\PYGZsh{} same as above}
\end{sphinxVerbatim}

x{[}i{]} may be of any type, so it possible to use

\begin{sphinxVerbatim}[commandchars=\\\{\}]
\PYG{n}{color} \PYG{o}{=} \PYG{n}{sim}\PYG{o}{.}\PYG{n}{Pdf}\PYG{p}{(}\PYG{p}{(}\PYG{l+s+s1}{\PYGZsq{}}\PYG{l+s+s1}{Green}\PYG{l+s+s1}{\PYGZsq{}}\PYG{p}{,} \PYG{l+m+mi}{45}\PYG{p}{,} \PYG{l+s+s1}{\PYGZsq{}}\PYG{l+s+s1}{Yellow}\PYG{l+s+s1}{\PYGZsq{}}\PYG{p}{,} \PYG{l+m+mi}{10}\PYG{p}{,} \PYG{l+s+s1}{\PYGZsq{}}\PYG{l+s+s1}{Red}\PYG{l+s+s1}{\PYGZsq{}}\PYG{p}{,} \PYG{l+m+mi}{45}\PYG{p}{)}\PYG{p}{)}
\PYG{n}{cartype} \PYG{o}{=} \PYG{n}{sim}\PYG{o}{.}\PYG{n}{Pdf}\PYG{p}{(}\PYG{n}{ordertypes}\PYG{p}{,}\PYG{l+m+mi}{1}\PYG{p}{)}
\end{sphinxVerbatim}

If the x-value is a salabim distribution, not the distribution but a sample of that distribution is returned when sampling

\begin{sphinxVerbatim}[commandchars=\\\{\}]
\PYG{n}{processingtime} \PYG{o}{=} \PYG{n}{sim}\PYG{o}{.}\PYG{n}{Pdf}\PYG{p}{(}\PYG{p}{(}\PYG{n}{sim}\PYG{o}{.}\PYG{n}{Uniform}\PYG{p}{(}\PYG{l+m+mi}{5}\PYG{p}{,} \PYG{l+m+mi}{10}\PYG{p}{)}\PYG{p}{,} \PYG{l+m+mi}{50}\PYG{p}{,} \PYG{n}{sim}\PYG{o}{.}\PYG{n}{Uniform}\PYG{p}{(}\PYG{l+m+mi}{10}\PYG{p}{,} \PYG{l+m+mi}{15}\PYG{p}{)}\PYG{p}{,} \PYG{l+m+mi}{40}\PYG{p}{,} \PYG{n}{sim}\PYG{o}{.}\PYG{n}{Uniform}\PYG{p}{(}\PYG{l+m+mi}{15}\PYG{p}{,} \PYG{l+m+mi}{20}\PYG{p}{)}\PYG{p}{,} \PYG{l+m+mi}{10}\PYG{p}{)}\PYG{p}{)}
\PYG{n}{proctime}\PYG{o}{=}\PYG{n}{processingtime}\PYG{o}{.}\PYG{n}{sample}\PYG{p}{(}\PYG{p}{)}
\end{sphinxVerbatim}

Here proctime will have a probability of 50\% being between 5 and 10, 40\% between 10 and 15 and 10\% between 15 and 20.


\section{CumPdf}
\label{\detokenize{Distributions:cumpdf}}
Probability density function, specified as:
\begin{enumerate}
\item {} 
list or tuple of x{[}i{]}, p{[}i{]} where p{[}i{]} is the cumulative probability (density)

\item {} 
list or tuple of x{[}i{]} followed by a list or tuple p{[}i{]}

\end{enumerate}

\begin{sphinxadmonition}{note}{Note:}
It is required that p{[}i{]}\textless{}=p{[}i+1{]}.
\end{sphinxadmonition}

E.g.

\begin{sphinxVerbatim}[commandchars=\\\{\}]
\PYG{n}{processingtime} \PYG{o}{=} \PYG{n}{sim}\PYG{o}{.}\PYG{n}{CumPdf}\PYG{p}{(}\PYG{p}{(}\PYG{l+m+mi}{5}\PYG{p}{,} \PYG{l+m+mi}{10}\PYG{p}{,} \PYG{l+m+mi}{10}\PYG{p}{,} \PYG{l+m+mi}{60}\PYG{p}{,} \PYG{l+m+mi}{15}\PYG{p}{,} \PYG{l+m+mi}{100}\PYG{p}{)}\PYG{p}{)}
\end{sphinxVerbatim}

This means that 10\% is 5, 50\% is 10 and 40\% is 15.

It is not required that the sum of the p{[}i{]}’s is 100, as all p{[}{]}’s are automatically scaled. This means that the two distributions below are
identical to the first example

\begin{sphinxVerbatim}[commandchars=\\\{\}]
\PYG{n}{processingtime} \PYG{o}{=} \PYG{n}{sim}\PYG{o}{.}\PYG{n}{CumPdf}\PYG{p}{(}\PYG{p}{(}\PYG{l+m+mi}{5}\PYG{p}{,} \PYG{l+m+mf}{0.10}\PYG{p}{,} \PYG{l+m+mi}{10}\PYG{p}{,} \PYG{l+m+mf}{0.60}\PYG{p}{,} \PYG{l+m+mi}{15}\PYG{p}{,} \PYG{l+m+mf}{1.00}\PYG{p}{)}\PYG{p}{)}
\PYG{n}{processingtime} \PYG{o}{=} \PYG{n}{sim}\PYG{o}{.}\PYG{n}{CumPdf}\PYG{p}{(}\PYG{p}{(}\PYG{l+m+mi}{5}\PYG{p}{,}    \PYG{l+m+mi}{2}\PYG{p}{,} \PYG{l+m+mi}{10}\PYG{p}{,}   \PYG{l+m+mi}{12}\PYG{p}{,} \PYG{l+m+mi}{15}\PYG{p}{,}   \PYG{l+m+mi}{20}\PYG{p}{)}\PYG{p}{)}
\end{sphinxVerbatim}

And the same with the second form

\begin{sphinxVerbatim}[commandchars=\\\{\}]
\PYG{n}{processingtime} \PYG{o}{=} \PYG{n}{sim}\PYG{o}{.}\PYG{n}{CumPdf}\PYG{p}{(}\PYG{p}{(}\PYG{l+m+mi}{5}\PYG{p}{,} \PYG{l+m+mi}{10}\PYG{p}{,} \PYG{l+m+mi}{15}\PYG{p}{)}\PYG{p}{,} \PYG{p}{(}\PYG{l+m+mi}{10}\PYG{p}{,} \PYG{l+m+mi}{60}\PYG{p}{,} \PYG{l+m+mi}{100}\PYG{p}{)}\PYG{p}{)}
\end{sphinxVerbatim}

x{[}i{]} may be of any type, so it possible to use

\begin{sphinxVerbatim}[commandchars=\\\{\}]
\PYG{n}{color} \PYG{o}{=} \PYG{n}{sim}\PYG{o}{.}\PYG{n}{CumPdf}\PYG{p}{(}\PYG{p}{(}\PYG{l+s+s1}{\PYGZsq{}}\PYG{l+s+s1}{Green}\PYG{l+s+s1}{\PYGZsq{}}\PYG{p}{,} \PYG{l+m+mi}{45}\PYG{p}{,} \PYG{l+s+s1}{\PYGZsq{}}\PYG{l+s+s1}{Red}\PYG{l+s+s1}{\PYGZsq{}}\PYG{p}{,} \PYG{l+m+mi}{100}\PYG{p}{)}\PYG{p}{)}
\end{sphinxVerbatim}

If the x-value is a salabim distribution, not the distribution but a sample of that distribution is returned when sampling

\begin{sphinxVerbatim}[commandchars=\\\{\}]
\PYG{n}{processingtime} \PYG{o}{=} \PYG{n}{sim}\PYG{o}{.}\PYG{n}{CumPdf}\PYG{p}{(}\PYG{p}{(}\PYG{n}{sim}\PYG{o}{.}\PYG{n}{Uniform}\PYG{p}{(}\PYG{l+m+mi}{5}\PYG{p}{,} \PYG{l+m+mi}{10}\PYG{p}{)}\PYG{p}{,} \PYG{l+m+mi}{50}\PYG{p}{,} \PYG{n}{sim}\PYG{o}{.}\PYG{n}{Uniform}\PYG{p}{(}\PYG{l+m+mi}{10}\PYG{p}{,} \PYG{l+m+mi}{15}\PYG{p}{)}\PYG{p}{,} \PYG{l+m+mi}{90}\PYG{p}{,} \PYG{n}{sim}\PYG{o}{.}\PYG{n}{Uniform}\PYG{p}{(}\PYG{l+m+mi}{15}\PYG{p}{,} \PYG{l+m+mi}{20}\PYG{p}{)}\PYG{p}{,} \PYG{l+m+mi}{100}\PYG{p}{)}\PYG{p}{)}
\PYG{n}{proctime}\PYG{o}{=}\PYG{n}{processingtime}\PYG{o}{.}\PYG{n}{sample}\PYG{p}{(}\PYG{p}{)}
\end{sphinxVerbatim}

Here proctime will have a probability of 50\% being between 5 and 10, 40\% between 10 and 15 and 10\% between 15 and 20.


\section{Distribution}
\label{\detokenize{Distributions:distribution}}
A special distribution is the Distribution class. Here, a string will contain the specification of the distribution.
This is particularly useful when the distributions are specified in an external file. E.g.

\begin{sphinxVerbatim}[commandchars=\\\{\}]
\PYG{k}{with} \PYG{n+nb}{open}\PYG{p}{(}\PYG{l+s+s1}{\PYGZsq{}}\PYG{l+s+s1}{experiment1.txt}\PYG{l+s+s1}{\PYGZsq{}}\PYG{p}{,} \PYG{l+s+s1}{\PYGZsq{}}\PYG{l+s+s1}{r}\PYG{l+s+s1}{\PYGZsq{}}\PYG{p}{)} \PYG{k}{as} \PYG{n}{f}\PYG{p}{:}
    \PYG{n}{interarrivaltime} \PYG{o}{=} \PYG{n}{sim}\PYG{o}{.}\PYG{n}{Distribution}\PYG{p}{(}\PYG{n}{read}\PYG{p}{(}\PYG{n}{f}\PYG{p}{)}\PYG{p}{)}
    \PYG{n}{processingtime} \PYG{o}{=} \PYG{n}{sim}\PYG{o}{.}\PYG{n}{Distribution}\PYG{p}{(}\PYG{n}{read}\PYG{p}{(}\PYG{n}{f}\PYG{p}{)}\PYG{p}{)}
    \PYG{n}{numberofparcels} \PYG{o}{=} \PYG{n}{sim}\PYG{o}{.}\PYG{n}{Distribution}\PYG{p}{(}\PYG{n}{read}\PYG{p}{(}\PYG{n}{f}\PYG{p}{)}\PYG{p}{)}
\end{sphinxVerbatim}

With a file experiment.txt

\begin{sphinxVerbatim}[commandchars=\\\{\}]
\PYG{n}{Uniform}\PYG{p}{(}\PYG{l+m+mi}{10}\PYG{p}{,}\PYG{l+m+mi}{15}\PYG{p}{)}
\PYG{n}{Triangular}\PYG{p}{(}\PYG{l+m+mi}{1}\PYG{p}{,}\PYG{l+m+mi}{5}\PYG{p}{,}\PYG{l+m+mi}{2}\PYG{p}{)}
\PYG{n}{IntUniform}\PYG{p}{(}\PYG{l+m+mi}{10}\PYG{p}{,}\PYG{l+m+mi}{20}\PYG{p}{)}
\end{sphinxVerbatim}

or with abbreviation

\begin{sphinxVerbatim}[commandchars=\\\{\}]
\PYG{n}{Uni}\PYG{p}{(}\PYG{l+m+mi}{10}\PYG{p}{,}\PYG{l+m+mi}{15}\PYG{p}{)}
\PYG{n}{Tri}\PYG{p}{(}\PYG{l+m+mi}{1}\PYG{p}{,}\PYG{l+m+mi}{5}\PYG{p}{,}\PYG{l+m+mi}{2}\PYG{p}{)}
\PYG{n}{Int}\PYG{p}{(}\PYG{l+m+mi}{10}\PYG{p}{,}\PYG{l+m+mi}{20}\PYG{p}{)}
\end{sphinxVerbatim}

or even

\begin{sphinxVerbatim}[commandchars=\\\{\}]
\PYG{n}{U}\PYG{p}{(}\PYG{l+m+mi}{10}\PYG{p}{,}\PYG{l+m+mi}{15}\PYG{p}{)}
\PYG{n}{T}\PYG{p}{(}\PYG{l+m+mi}{1}\PYG{p}{,}\PYG{l+m+mi}{5}\PYG{p}{,}\PYG{l+m+mi}{2}\PYG{p}{)}
\PYG{n}{I}\PYG{p}{(}\PYG{l+m+mi}{10}\PYG{p}{,}\PYG{l+m+mi}{20}\PYG{p}{)}
\end{sphinxVerbatim}


\chapter{Miscellaneous}
\label{\detokenize{Miscellaneous::doc}}\label{\detokenize{Miscellaneous:miscellaneous}}

\section{Time units}
\label{\detokenize{Miscellaneous:time-units}}
By default, salabim time does not have a specific dimension, which means that is up to
the modeller what time unit is used, be it seconds, hours, days or whatever.

It can be useful to work in specific time unit, as this opens the possibility to specify times and durations in another unit.

In order to use time unit, the environment has to be initialized with a time\_unit parameter, like

\begin{sphinxVerbatim}[commandchars=\\\{\}]
\PYG{n}{env} \PYG{o}{=} \PYG{n}{sim}\PYG{o}{.}\PYG{n}{Environment}\PYG{p}{(}\PYG{n}{time\PYGZus{}unit}\PYG{o}{=}\PYG{l+s+s1}{\PYGZsq{}}\PYG{l+s+s1}{hours}\PYG{l+s+s1}{\PYGZsq{}}\PYG{p}{)}
\end{sphinxVerbatim}

From then on, the simulation runs in hours. Standard output is in then in hours and for instance

\begin{sphinxVerbatim}[commandchars=\\\{\}]
\PYG{n+nb+bp}{self}\PYG{o}{.}\PYG{n}{enter}\PYG{p}{(}\PYG{n}{q}\PYG{p}{)}
\PYG{k}{yield} \PYG{n+nb+bp}{self}\PYG{o}{.}\PYG{n}{hold}\PYG{p}{(}\PYG{l+m+mi}{48}\PYG{p}{)}
\PYG{n+nb}{print}\PYG{p}{(}\PYG{n}{env}\PYG{o}{.}\PYG{n}{now}\PYG{p}{(}\PYG{p}{)} \PYG{o}{\PYGZhy{}} \PYG{n+nb+bp}{self}\PYG{o}{.}\PYG{n}{queuetime}\PYG{p}{(}\PYG{p}{)}\PYG{p}{)}
\end{sphinxVerbatim}

means hold for 48 (hours) and 48 will be printed.

But, now we also specify a time in another time unit and get times in a specific time unit

\begin{sphinxVerbatim}[commandchars=\\\{\}]
\PYG{n+nb+bp}{self}\PYG{o}{.}\PYG{n}{enter}\PYG{p}{(}\PYG{n}{q}\PYG{p}{)}
\PYG{k}{yield} \PYG{n+nb+bp}{self}\PYG{o}{.}\PYG{n}{hold}\PYG{p}{(}\PYG{n}{env}\PYG{o}{.}\PYG{n}{days}\PYG{p}{(}\PYG{l+m+mi}{2}\PYG{p}{)}\PYG{p}{)}
\PYG{n+nb}{print}\PYG{p}{(}\PYG{n}{env}\PYG{o}{.}\PYG{n}{to\PYGZus{}minutes}\PYG{p}{(}\PYG{n}{env}\PYG{o}{.}\PYG{n}{now}\PYG{p}{(}\PYG{p}{)} \PYG{o}{\PYGZhy{}} \PYG{n+nb+bp}{self}\PYG{o}{.}\PYG{n}{queuetime}\PYG{p}{(}\PYG{p}{)}\PYG{p}{)}\PYG{p}{)}
\end{sphinxVerbatim}

means hold for 2 days = 48 hours and 2880 (48 * 60) will be printed.

The following time units are available:
\begin{itemize}
\item {} 
‘years’

\item {} 
‘weeks’

\item {} 
‘days’

\item {} 
‘hours’

\item {} 
‘minutes’

\item {} 
‘seconds’

\item {} 
‘milliseconds’

\item {} 
‘microseconds’

\item {} 
‘n/a’ which means nothing is assigned and conversions are not supported

\end{itemize}

For conversion from a given time unit to the simulation time unit, the following calls are available:
\begin{itemize}
\item {} 
years()

\item {} 
weeks()

\item {} 
days()

\item {} 
hours()

\item {} 
minutes()

\item {} 
seconds()

\item {} 
milliseconds()

\item {} 
microseconds()

\end{itemize}

For conversion from the simulation time unit to a given time unit, the following calls are available:
\begin{itemize}
\item {} 
to\_years()

\item {} 
to\_weeks()

\item {} 
to\_days()

\item {} 
to\_hours()

\item {} 
to\_minutes()

\item {} 
to\_seconds()

\item {} 
to\_milliseconds()

\item {} 
to\_microseconds()

\end{itemize}

Distributions (apart from IntUniform, Poisson and Beta) can also specify the time unit, like

\begin{sphinxVerbatim}[commandchars=\\\{\}]
\PYG{n}{env} \PYG{o}{=} \PYG{n}{sim}\PYG{o}{.}\PYG{n}{Environment}\PYG{p}{(}\PYG{n}{time\PYGZus{}unit}\PYG{o}{=}\PYG{l+s+s1}{\PYGZsq{}}\PYG{l+s+s1}{seconds}\PYG{l+s+s1}{\PYGZsq{}}\PYG{p}{)}
\PYG{n}{processingtime\PYGZus{}dis} \PYG{o}{=} \PYG{n}{sim}\PYG{o}{.}\PYG{n}{Uniform}\PYG{p}{(}\PYG{l+m+mi}{10}\PYG{p}{,} \PYG{l+m+mi}{20}\PYG{p}{,} \PYG{l+s+s1}{\PYGZsq{}}\PYG{l+s+s1}{minutes}\PYG{l+s+s1}{\PYGZsq{}}\PYG{p}{)}
\PYG{n}{dryingtime\PYGZus{}dis} \PYG{o}{=} \PYG{n}{sim}\PYG{o}{.}\PYG{n}{Normal}\PYG{p}{(}\PYG{l+m+mi}{2}\PYG{p}{,} \PYG{l+m+mf}{0.1}\PYG{p}{,} \PYG{l+s+s1}{\PYGZsq{}}\PYG{l+s+s1}{hours}\PYG{l+s+s1}{\PYGZsq{}}\PYG{p}{)}
\end{sphinxVerbatim}

Note that the conversion to the current time unit is made immediately and that all related output
is therefore in the current simulation time unit, so

\begin{sphinxVerbatim}[commandchars=\\\{\}]
\PYG{n}{processingtime\PYGZus{}dis}\PYG{o}{.}\PYG{n}{print\PYGZus{}info}\PYG{p}{(}\PYG{p}{)}
\PYG{n}{dryingtime\PYGZus{}dis}\PYG{o}{.}\PYG{n}{print\PYGZus{}info}\PYG{p}{(}\PYG{p}{)}
\end{sphinxVerbatim}

will print

\begin{sphinxVerbatim}[commandchars=\\\{\}]
\PYG{n}{Uniform} \PYG{n}{distribution} \PYG{l+m+mh}{0x25783c11358}
  \PYG{n}{lowerbound}\PYG{o}{=}\PYG{l+m+mf}{600.0}
  \PYG{n}{upperbound}\PYG{o}{=}\PYG{l+m+mf}{1200.0}
  \PYG{n}{randomstream}\PYG{o}{=}\PYG{l+m+mh}{0x25783b89818}
\PYG{n}{Normal} \PYG{n}{distribution} \PYG{l+m+mh}{0x25783bff8d0}
  \PYG{n}{mean}\PYG{o}{=}\PYG{l+m+mf}{7200.0}
  \PYG{n}{standard\PYGZus{}deviation}\PYG{o}{=}\PYG{l+m+mf}{360.0}
  \PYG{n}{coefficient\PYGZus{}of\PYGZus{}variation}\PYG{o}{=}\PYG{l+m+mf}{0.05}
  \PYG{n}{randomstream}\PYG{o}{=}\PYG{l+m+mh}{0x25783b89818}
\end{sphinxVerbatim}


\chapter{Animation}
\label{\detokenize{Animation:animation}}\label{\detokenize{Animation::doc}}
Animation is a powerful tool to debug, test and demonstrate simulations.

It is possible to show a number of shapes (lines, rectangles, circles, etc), texts as well (images) in a window. These objects can be dynamically updated.
(Timestamped) monitors may be animated by showing the current value against the time or index.
Furthermore the components in a queue may be shown in a highly customizable way.
As text animation may be dynamically updated, it is even possible to show the current state, (monitor) statistics, etc. in the animation windows.

Salabim’s animation engine also allows some user input.

It is important to realize that animation calls can be still given when animation is actually off. In that case, there is hardly any impact on the performance.

Salabim animations can be
\begin{itemize}
\item {} 
synchronized with the simulation clock and run in real time (synchronized)

\item {} 
advance per simulation event (non synchronized)

\end{itemize}

In synchronized mode, one time unit in the simulation can correspond to any period in real time, e.g.
\begin{itemize}
\item {} 
1 time unit in simulation time \textendash{}\textgreater{} 1 second real time (speed = 1) (default)

\item {} 
1 time unit in simulation time \textendash{}\textgreater{} 4 seconds real time (speed = 0.25)

\item {} 
4 time units in simulation time \textendash{}\textgreater{} 1 second real time (speed = 4)

\end{itemize}

The most common way to start an animation is by calling
{}`{}` env.animate(True){}`{}` or with a call to \sphinxcode{animation\_parameters}.

Animations can be started en stopped during execution (i.e. run). When main is active, the animation
is always stopped.

The animation uses a coordinate system that -by default- is in screen pixels. The lower left corner is (0,0).
But, the user can change both the coordinate of the lower left corner (translation) as well as set the x-coordinate
of the lower right hand corner (scaling). Note that x- and y-scaling are always the same. 
Furthermore, it is possible to specify the colour of the background with \sphinxcode{animation\_parameters}.

Prior to version 2.3.0 there was actually just one animation object class: Animate. This
interface is described later as the new animation classes are easier to use and even offer some
additional functionality.

New style animation classes can be used to put texts, rectangles, polygon, lines, series of points, circles
or images on the screen. All types can be connected to an optional text.

Here is a sample program to show of all the new style animation classes

\begin{sphinxVerbatim}[commandchars=\\\{\}]
    \PYG{n}{env}\PYG{o}{=}\PYG{n}{sim}\PYG{o}{.}\PYG{n}{Environment}\PYG{p}{(}\PYG{n}{trace}\PYG{o}{=}\PYG{k+kc}{False}\PYG{p}{)}
    \PYG{n}{env}\PYG{o}{.}\PYG{n}{animate}\PYG{p}{(}\PYG{k+kc}{True}\PYG{p}{)}
    \PYG{n}{env}\PYG{o}{.}\PYG{n}{background\PYGZus{}color}\PYG{p}{(}\PYG{l+s+s1}{\PYGZsq{}}\PYG{l+s+s1}{20}\PYG{l+s+si}{\PYGZpc{}g}\PYG{l+s+s1}{ray}\PYG{l+s+s1}{\PYGZsq{}}\PYG{p}{)}

    \PYG{n}{sim}\PYG{o}{.}\PYG{n}{AnimatePolygon}\PYG{p}{(}\PYG{n}{spec}\PYG{o}{=}\PYG{p}{(}\PYG{l+m+mi}{100}\PYG{p}{,} \PYG{l+m+mi}{100}\PYG{p}{,} \PYG{l+m+mi}{300}\PYG{p}{,} \PYG{l+m+mi}{100}\PYG{p}{,} \PYG{l+m+mi}{200}\PYG{p}{,}\PYG{l+m+mi}{190}\PYG{p}{)}\PYG{p}{,} \PYG{n}{text}\PYG{o}{=}\PYG{l+s+s1}{\PYGZsq{}}\PYG{l+s+s1}{This is}\PYG{l+s+se}{\PYGZbs{}n}\PYG{l+s+s1}{a polygon}\PYG{l+s+s1}{\PYGZsq{}}\PYG{p}{)}
    \PYG{n}{sim}\PYG{o}{.}\PYG{n}{AnimateLine}\PYG{p}{(}\PYG{n}{spec}\PYG{o}{=}\PYG{p}{(}\PYG{l+m+mi}{100}\PYG{p}{,} \PYG{l+m+mi}{200}\PYG{p}{,} \PYG{l+m+mi}{300}\PYG{p}{,} \PYG{l+m+mi}{300}\PYG{p}{)}\PYG{p}{,} \PYG{n}{text}\PYG{o}{=}\PYG{l+s+s1}{\PYGZsq{}}\PYG{l+s+s1}{This is a line}\PYG{l+s+s1}{\PYGZsq{}}\PYG{p}{)}
    \PYG{n}{sim}\PYG{o}{.}\PYG{n}{AnimateRectangle}\PYG{p}{(}\PYG{n}{spec}\PYG{o}{=}\PYG{p}{(}\PYG{l+m+mi}{100}\PYG{p}{,} \PYG{l+m+mi}{10}\PYG{p}{,} \PYG{l+m+mi}{300}\PYG{p}{,} \PYG{l+m+mi}{30}\PYG{p}{)}\PYG{p}{,} \PYG{n}{text}\PYG{o}{=}\PYG{l+s+s1}{\PYGZsq{}}\PYG{l+s+s1}{This is a rectangle}\PYG{l+s+s1}{\PYGZsq{}}\PYG{p}{)}
    \PYG{n}{sim}\PYG{o}{.}\PYG{n}{AnimateCircle}\PYG{p}{(}\PYG{n}{radius}\PYG{o}{=}\PYG{l+m+mi}{60}\PYG{p}{,} \PYG{n}{x}\PYG{o}{=}\PYG{l+m+mi}{100}\PYG{p}{,} \PYG{n}{y}\PYG{o}{=}\PYG{l+m+mi}{400}\PYG{p}{,} \PYG{n}{text}\PYG{o}{=}\PYG{l+s+s1}{\PYGZsq{}}\PYG{l+s+s1}{This is a cicle}\PYG{l+s+s1}{\PYGZsq{}}\PYG{p}{)}
    \PYG{n}{sim}\PYG{o}{.}\PYG{n}{AnimateCircle}\PYG{p}{(}\PYG{n}{radius}\PYG{o}{=}\PYG{l+m+mi}{60}\PYG{p}{,} \PYG{n}{radius1}\PYG{o}{=}\PYG{l+m+mi}{30}\PYG{p}{,} \PYG{n}{x}\PYG{o}{=}\PYG{l+m+mi}{300}\PYG{p}{,} \PYG{n}{y}\PYG{o}{=}\PYG{l+m+mi}{400}\PYG{p}{,} \PYG{n}{text}\PYG{o}{=}\PYG{l+s+s1}{\PYGZsq{}}\PYG{l+s+s1}{This is an ellipse}\PYG{l+s+s1}{\PYGZsq{}}\PYG{p}{)}
    \PYG{n}{sim}\PYG{o}{.}\PYG{n}{AnimatePoints}\PYG{p}{(}\PYG{n}{spec}\PYG{o}{=}\PYG{p}{(}\PYG{l+m+mi}{100}\PYG{p}{,}\PYG{l+m+mi}{500}\PYG{p}{,} \PYG{l+m+mi}{150}\PYG{p}{,} \PYG{l+m+mi}{550}\PYG{p}{,} \PYG{l+m+mi}{180}\PYG{p}{,} \PYG{l+m+mi}{570}\PYG{p}{,} \PYG{l+m+mi}{250}\PYG{p}{,} \PYG{l+m+mi}{500}\PYG{p}{,} \PYG{l+m+mi}{300}\PYG{p}{,} \PYG{l+m+mi}{500}\PYG{p}{)}\PYG{p}{,} \PYG{n}{text}\PYG{o}{=}\PYG{l+s+s1}{\PYGZsq{}}\PYG{l+s+s1}{These are points}\PYG{l+s+s1}{\PYGZsq{}}\PYG{p}{)}
    \PYG{n}{sim}\PYG{o}{.}\PYG{n}{AnimateText}\PYG{p}{(}\PYG{n}{text}\PYG{o}{=}\PYG{l+s+s1}{\PYGZsq{}}\PYG{l+s+s1}{This is a one\PYGZhy{}line text}\PYG{l+s+s1}{\PYGZsq{}}\PYG{p}{,} \PYG{n}{x}\PYG{o}{=}\PYG{l+m+mi}{100}\PYG{p}{,} \PYG{n}{y}\PYG{o}{=}\PYG{l+m+mi}{600}\PYG{p}{)}
    \PYG{n}{sim}\PYG{o}{.}\PYG{n}{AnimateText}\PYG{p}{(}\PYG{n}{text}\PYG{o}{=}\PYG{l+s+s1}{\PYGZsq{}\PYGZsq{}\PYGZsq{}}\PYG{l+s+se}{\PYGZbs{}}
\PYG{l+s+s1}{Multi line text}
\PYG{l+s+s1}{\PYGZhy{}\PYGZhy{}\PYGZhy{}\PYGZhy{}\PYGZhy{}\PYGZhy{}\PYGZhy{}\PYGZhy{}\PYGZhy{}\PYGZhy{}\PYGZhy{}\PYGZhy{}\PYGZhy{}\PYGZhy{}\PYGZhy{}\PYGZhy{}\PYGZhy{}}
\PYG{l+s+s1}{Lorem ipsum dolor sit amet, consectetur}
\PYG{l+s+s1}{adipiscing elit, sed do eiusmod tempor}
\PYG{l+s+s1}{incididunt ut labore et dolore magna aliqua.}
\PYG{l+s+s1}{Ut enim ad minim veniam, quis nostrud}
\PYG{l+s+s1}{exercitation ullamco laboris nisi ut}
\PYG{l+s+s1}{aliquip ex ea commodo consequat. Duis aute}
\PYG{l+s+s1}{irure dolor in reprehenderit in voluptate}
\PYG{l+s+s1}{velit esse cillum dolore eu fugiat nulla}
\PYG{l+s+s1}{pariatur.}

\PYG{l+s+s1}{Excepteur sint occaecat cupidatat non}
\PYG{l+s+s1}{proident, sunt in culpa qui officia}
\PYG{l+s+s1}{deserunt mollit anim id est laborum.}
\PYG{l+s+s1}{\PYGZsq{}\PYGZsq{}\PYGZsq{}}\PYG{p}{,} \PYG{n}{x}\PYG{o}{=}\PYG{l+m+mi}{500}\PYG{p}{,} \PYG{n}{y}\PYG{o}{=}\PYG{l+m+mi}{100}\PYG{p}{)}

    \PYG{n}{sim}\PYG{o}{.}\PYG{n}{AnimateImage}\PYG{p}{(}\PYG{l+s+s1}{\PYGZsq{}}\PYG{l+s+s1}{Pas un pipe.jpg}\PYG{l+s+s1}{\PYGZsq{}}\PYG{p}{,} \PYG{n}{x}\PYG{o}{=}\PYG{l+m+mi}{500}\PYG{p}{,} \PYG{n}{y}\PYG{o}{=}\PYG{l+m+mi}{400}\PYG{p}{)}
    \PYG{n}{env}\PYG{o}{.}\PYG{n}{run}\PYG{p}{(}\PYG{l+m+mi}{100}\PYG{p}{)}
\end{sphinxVerbatim}

Resulting in:

\noindent\sphinxincludegraphics{{Pic1}.png}

Animation of the components of a queue is accomplished with \sphinxcode{AnimateQueue()}.
It is possible to use the standard shape of components, which is a rectangle with the sequence number or define
your own shape(s). The queue can be build up in west, east, north or south directions.
It is possible to limit the number of component shown.

Monitors and timestamped monitors can be visualized dynamically with \sphinxcode{AnimateMonitor()}.

These features are demonstrated in \sphinxstyleemphasis{Demo queue animation.py}

\begin{sphinxVerbatim}[commandchars=\\\{\}]
\PYG{k+kn}{import} \PYG{n+nn}{salabim} \PYG{k}{as} \PYG{n+nn}{sim}

\PYG{l+s+sd}{\PYGZsq{}\PYGZsq{}\PYGZsq{}}
\PYG{l+s+sd}{This us a demonstration of several ways to show queues dynamically and the corresponding statistics}
\PYG{l+s+sd}{The model simply generates components that enter a queue and leave after a certain time.}

\PYG{l+s+sd}{Note that the actual model code (in the process description of X does not contain any reference}
\PYG{l+s+sd}{to the animation!}
\PYG{l+s+sd}{\PYGZsq{}\PYGZsq{}\PYGZsq{}}


\PYG{k}{class} \PYG{n+nc}{X}\PYG{p}{(}\PYG{n}{sim}\PYG{o}{.}\PYG{n}{Component}\PYG{p}{)}\PYG{p}{:}
    \PYG{k}{def} \PYG{n+nf}{setup}\PYG{p}{(}\PYG{n+nb+bp}{self}\PYG{p}{,} \PYG{n}{i}\PYG{p}{)}\PYG{p}{:}
        \PYG{n+nb+bp}{self}\PYG{o}{.}\PYG{n}{i} \PYG{o}{=} \PYG{n}{i}

    \PYG{k}{def} \PYG{n+nf}{animation\PYGZus{}objects}\PYG{p}{(}\PYG{n+nb+bp}{self}\PYG{p}{,} \PYG{n+nb}{id}\PYG{p}{)}\PYG{p}{:}
        \PYG{l+s+sd}{\PYGZsq{}\PYGZsq{}\PYGZsq{}}
\PYG{l+s+sd}{        the way the component is determined by the id, specified in AnimateQueue}
\PYG{l+s+sd}{        \PYGZsq{}text\PYGZsq{} means just the name}
\PYG{l+s+sd}{        any other value represents the colour}
\PYG{l+s+sd}{        \PYGZsq{}\PYGZsq{}\PYGZsq{}}
        \PYG{k}{if} \PYG{n+nb}{id} \PYG{o}{==} \PYG{l+s+s1}{\PYGZsq{}}\PYG{l+s+s1}{text}\PYG{l+s+s1}{\PYGZsq{}}\PYG{p}{:}
            \PYG{n}{ao0} \PYG{o}{=} \PYG{n}{sim}\PYG{o}{.}\PYG{n}{AnimateText}\PYG{p}{(}\PYG{n}{text}\PYG{o}{=}\PYG{n+nb+bp}{self}\PYG{o}{.}\PYG{n}{name}\PYG{p}{(}\PYG{p}{)}\PYG{p}{,} \PYG{n}{textcolor}\PYG{o}{=}\PYG{l+s+s1}{\PYGZsq{}}\PYG{l+s+s1}{fg}\PYG{l+s+s1}{\PYGZsq{}}\PYG{p}{,} \PYG{n}{text\PYGZus{}anchor}\PYG{o}{=}\PYG{l+s+s1}{\PYGZsq{}}\PYG{l+s+s1}{nw}\PYG{l+s+s1}{\PYGZsq{}}\PYG{p}{)}
            \PYG{k}{return} \PYG{l+m+mi}{0}\PYG{p}{,} \PYG{l+m+mi}{16}\PYG{p}{,} \PYG{n}{ao0}
        \PYG{k}{else}\PYG{p}{:}
            \PYG{n}{ao0} \PYG{o}{=} \PYG{n}{sim}\PYG{o}{.}\PYG{n}{AnimateRectangle}\PYG{p}{(}\PYG{p}{(}\PYG{o}{\PYGZhy{}}\PYG{l+m+mi}{20}\PYG{p}{,} \PYG{l+m+mi}{0}\PYG{p}{,} \PYG{l+m+mi}{20}\PYG{p}{,} \PYG{l+m+mi}{20}\PYG{p}{)}\PYG{p}{,}
                \PYG{n}{text}\PYG{o}{=}\PYG{n+nb+bp}{self}\PYG{o}{.}\PYG{n}{name}\PYG{p}{(}\PYG{p}{)}\PYG{p}{,} \PYG{n}{fillcolor}\PYG{o}{=}\PYG{n+nb}{id}\PYG{p}{,} \PYG{n}{textcolor}\PYG{o}{=}\PYG{l+s+s1}{\PYGZsq{}}\PYG{l+s+s1}{white}\PYG{l+s+s1}{\PYGZsq{}}\PYG{p}{,} \PYG{n}{arg}\PYG{o}{=}\PYG{n+nb+bp}{self}\PYG{p}{)}
            \PYG{k}{return} \PYG{l+m+mi}{45}\PYG{p}{,} \PYG{l+m+mi}{0}\PYG{p}{,} \PYG{n}{ao0}

    \PYG{k}{def} \PYG{n+nf}{process}\PYG{p}{(}\PYG{n+nb+bp}{self}\PYG{p}{)}\PYG{p}{:}
        \PYG{k}{while} \PYG{k+kc}{True}\PYG{p}{:}
            \PYG{k}{yield} \PYG{n+nb+bp}{self}\PYG{o}{.}\PYG{n}{hold}\PYG{p}{(}\PYG{n}{sim}\PYG{o}{.}\PYG{n}{Uniform}\PYG{p}{(}\PYG{l+m+mi}{0}\PYG{p}{,} \PYG{l+m+mi}{20}\PYG{p}{)}\PYG{p}{(}\PYG{p}{)}\PYG{p}{)}
            \PYG{n+nb+bp}{self}\PYG{o}{.}\PYG{n}{enter}\PYG{p}{(}\PYG{n}{q}\PYG{p}{)}
            \PYG{k}{yield} \PYG{n+nb+bp}{self}\PYG{o}{.}\PYG{n}{hold}\PYG{p}{(}\PYG{n}{sim}\PYG{o}{.}\PYG{n}{Uniform}\PYG{p}{(}\PYG{l+m+mi}{0}\PYG{p}{,} \PYG{l+m+mi}{20}\PYG{p}{)}\PYG{p}{(}\PYG{p}{)}\PYG{p}{)}
            \PYG{n+nb+bp}{self}\PYG{o}{.}\PYG{n}{leave}\PYG{p}{(}\PYG{p}{)}


\PYG{n}{env} \PYG{o}{=} \PYG{n}{sim}\PYG{o}{.}\PYG{n}{Environment}\PYG{p}{(}\PYG{n}{trace}\PYG{o}{=}\PYG{k+kc}{False}\PYG{p}{)}
\PYG{n}{env}\PYG{o}{.}\PYG{n}{background\PYGZus{}color}\PYG{p}{(}\PYG{l+s+s1}{\PYGZsq{}}\PYG{l+s+s1}{20}\PYG{l+s+si}{\PYGZpc{}g}\PYG{l+s+s1}{ray}\PYG{l+s+s1}{\PYGZsq{}}\PYG{p}{)}

\PYG{n}{q} \PYG{o}{=} \PYG{n}{sim}\PYG{o}{.}\PYG{n}{Queue}\PYG{p}{(}\PYG{l+s+s1}{\PYGZsq{}}\PYG{l+s+s1}{queue}\PYG{l+s+s1}{\PYGZsq{}}\PYG{p}{)}

\PYG{n}{qa0} \PYG{o}{=} \PYG{n}{sim}\PYG{o}{.}\PYG{n}{AnimateQueue}\PYG{p}{(}\PYG{n}{q}\PYG{p}{,} \PYG{n}{x}\PYG{o}{=}\PYG{l+m+mi}{100}\PYG{p}{,} \PYG{n}{y}\PYG{o}{=}\PYG{l+m+mi}{50}\PYG{p}{,} \PYG{n}{title}\PYG{o}{=}\PYG{l+s+s1}{\PYGZsq{}}\PYG{l+s+s1}{queue, normal}\PYG{l+s+s1}{\PYGZsq{}}\PYG{p}{,} \PYG{n}{direction}\PYG{o}{=}\PYG{l+s+s1}{\PYGZsq{}}\PYG{l+s+s1}{e}\PYG{l+s+s1}{\PYGZsq{}}\PYG{p}{,} \PYG{n+nb}{id}\PYG{o}{=}\PYG{l+s+s1}{\PYGZsq{}}\PYG{l+s+s1}{blue}\PYG{l+s+s1}{\PYGZsq{}}\PYG{p}{)}
\PYG{n}{qa1} \PYG{o}{=} \PYG{n}{sim}\PYG{o}{.}\PYG{n}{AnimateQueue}\PYG{p}{(}\PYG{n}{q}\PYG{p}{,} \PYG{n}{x}\PYG{o}{=}\PYG{l+m+mi}{100}\PYG{p}{,} \PYG{n}{y}\PYG{o}{=}\PYG{l+m+mi}{250}\PYG{p}{,} \PYG{n}{title}\PYG{o}{=}\PYG{l+s+s1}{\PYGZsq{}}\PYG{l+s+s1}{queue, maximum 6 components}\PYG{l+s+s1}{\PYGZsq{}}\PYG{p}{,} \PYG{n}{direction}\PYG{o}{=}\PYG{l+s+s1}{\PYGZsq{}}\PYG{l+s+s1}{e}\PYG{l+s+s1}{\PYGZsq{}}\PYG{p}{,} \PYG{n}{max\PYGZus{}length}\PYG{o}{=}\PYG{l+m+mi}{6}\PYG{p}{,} \PYG{n+nb}{id}\PYG{o}{=}\PYG{l+s+s1}{\PYGZsq{}}\PYG{l+s+s1}{red}\PYG{l+s+s1}{\PYGZsq{}}\PYG{p}{)}
\PYG{n}{qa2} \PYG{o}{=} \PYG{n}{sim}\PYG{o}{.}\PYG{n}{AnimateQueue}\PYG{p}{(}\PYG{n}{q}\PYG{p}{,} \PYG{n}{x}\PYG{o}{=}\PYG{l+m+mi}{100}\PYG{p}{,} \PYG{n}{y}\PYG{o}{=}\PYG{l+m+mi}{150}\PYG{p}{,} \PYG{n}{title}\PYG{o}{=}\PYG{l+s+s1}{\PYGZsq{}}\PYG{l+s+s1}{queue, reversed}\PYG{l+s+s1}{\PYGZsq{}}\PYG{p}{,} \PYG{n}{direction}\PYG{o}{=}\PYG{l+s+s1}{\PYGZsq{}}\PYG{l+s+s1}{e}\PYG{l+s+s1}{\PYGZsq{}}\PYG{p}{,} \PYG{n}{reverse}\PYG{o}{=}\PYG{k+kc}{True}\PYG{p}{,} \PYG{n+nb}{id}\PYG{o}{=}\PYG{l+s+s1}{\PYGZsq{}}\PYG{l+s+s1}{green}\PYG{l+s+s1}{\PYGZsq{}}\PYG{p}{)}
\PYG{n}{qa3} \PYG{o}{=} \PYG{n}{sim}\PYG{o}{.}\PYG{n}{AnimateQueue}\PYG{p}{(}\PYG{n}{q}\PYG{p}{,} \PYG{n}{x}\PYG{o}{=}\PYG{l+m+mi}{100}\PYG{p}{,} \PYG{n}{y}\PYG{o}{=}\PYG{l+m+mi}{440}\PYG{p}{,} \PYG{n}{title}\PYG{o}{=}\PYG{l+s+s1}{\PYGZsq{}}\PYG{l+s+s1}{queue, text only}\PYG{l+s+s1}{\PYGZsq{}}\PYG{p}{,} \PYG{n}{direction}\PYG{o}{=}\PYG{l+s+s1}{\PYGZsq{}}\PYG{l+s+s1}{s}\PYG{l+s+s1}{\PYGZsq{}}\PYG{p}{,} \PYG{n+nb}{id}\PYG{o}{=}\PYG{l+s+s1}{\PYGZsq{}}\PYG{l+s+s1}{text}\PYG{l+s+s1}{\PYGZsq{}}\PYG{p}{)}

\PYG{n}{sim}\PYG{o}{.}\PYG{n}{AnimateMonitor}\PYG{p}{(}\PYG{n}{q}\PYG{o}{.}\PYG{n}{length}\PYG{p}{,} \PYG{n}{x}\PYG{o}{=}\PYG{l+m+mi}{10}\PYG{p}{,} \PYG{n}{y}\PYG{o}{=}\PYG{l+m+mi}{450}\PYG{p}{,} \PYG{n}{width}\PYG{o}{=}\PYG{l+m+mi}{480}\PYG{p}{,} \PYG{n}{height}\PYG{o}{=}\PYG{l+m+mi}{100}\PYG{p}{,} \PYG{n}{horizontal\PYGZus{}scale}\PYG{o}{=}\PYG{l+m+mi}{5}\PYG{p}{,} \PYG{n}{vertical\PYGZus{}scale}\PYG{o}{=}\PYG{l+m+mi}{5}\PYG{p}{)}

\PYG{n}{sim}\PYG{o}{.}\PYG{n}{AnimateMonitor}\PYG{p}{(}\PYG{n}{q}\PYG{o}{.}\PYG{n}{length\PYGZus{}of\PYGZus{}stay}\PYG{p}{,} \PYG{n}{x}\PYG{o}{=}\PYG{l+m+mi}{10}\PYG{p}{,} \PYG{n}{y}\PYG{o}{=}\PYG{l+m+mi}{570}\PYG{p}{,} \PYG{n}{width}\PYG{o}{=}\PYG{l+m+mi}{480}\PYG{p}{,} \PYG{n}{height}\PYG{o}{=}\PYG{l+m+mi}{100}\PYG{p}{,} \PYG{n}{horizontal\PYGZus{}scale}\PYG{o}{=}\PYG{l+m+mi}{5}\PYG{p}{,} \PYG{n}{vertical\PYGZus{}scale}\PYG{o}{=}\PYG{l+m+mi}{5}\PYG{p}{)}

\PYG{n}{sim}\PYG{o}{.}\PYG{n}{AnimateText}\PYG{p}{(}\PYG{n}{text}\PYG{o}{=}\PYG{k}{lambda}\PYG{p}{:} \PYG{n}{q}\PYG{o}{.}\PYG{n}{length}\PYG{o}{.}\PYG{n}{print\PYGZus{}histogram}\PYG{p}{(}\PYG{n}{as\PYGZus{}str}\PYG{o}{=}\PYG{k+kc}{True}\PYG{p}{)}\PYG{p}{,} \PYG{n}{x}\PYG{o}{=}\PYG{l+m+mi}{500}\PYG{p}{,} \PYG{n}{y}\PYG{o}{=}\PYG{l+m+mi}{700}\PYG{p}{,}
    \PYG{n}{text\PYGZus{}anchor}\PYG{o}{=}\PYG{l+s+s1}{\PYGZsq{}}\PYG{l+s+s1}{nw}\PYG{l+s+s1}{\PYGZsq{}}\PYG{p}{,} \PYG{n}{font}\PYG{o}{=}\PYG{l+s+s1}{\PYGZsq{}}\PYG{l+s+s1}{narrow}\PYG{l+s+s1}{\PYGZsq{}}\PYG{p}{,} \PYG{n}{fontsize}\PYG{o}{=}\PYG{l+m+mi}{10}\PYG{p}{)}

\PYG{n}{sim}\PYG{o}{.}\PYG{n}{AnimateText}\PYG{p}{(}\PYG{n}{text}\PYG{o}{=}\PYG{k}{lambda}\PYG{p}{:} \PYG{n}{q}\PYG{o}{.}\PYG{n}{print\PYGZus{}info}\PYG{p}{(}\PYG{n}{as\PYGZus{}str}\PYG{o}{=}\PYG{k+kc}{True}\PYG{p}{)}\PYG{p}{,} \PYG{n}{x}\PYG{o}{=}\PYG{l+m+mi}{500}\PYG{p}{,} \PYG{n}{y}\PYG{o}{=}\PYG{l+m+mi}{340}\PYG{p}{,}
    \PYG{n}{text\PYGZus{}anchor}\PYG{o}{=}\PYG{l+s+s1}{\PYGZsq{}}\PYG{l+s+s1}{nw}\PYG{l+s+s1}{\PYGZsq{}}\PYG{p}{,} \PYG{n}{font}\PYG{o}{=}\PYG{l+s+s1}{\PYGZsq{}}\PYG{l+s+s1}{narrow}\PYG{l+s+s1}{\PYGZsq{}}\PYG{p}{,} \PYG{n}{fontsize}\PYG{o}{=}\PYG{l+m+mi}{10}\PYG{p}{)}

\PYG{p}{[}\PYG{n}{X}\PYG{p}{(}\PYG{n}{i}\PYG{o}{=}\PYG{n}{i}\PYG{p}{)} \PYG{k}{for} \PYG{n}{i} \PYG{o+ow}{in} \PYG{n+nb}{range}\PYG{p}{(}\PYG{l+m+mi}{15}\PYG{p}{)}\PYG{p}{]}
\PYG{n}{env}\PYG{o}{.}\PYG{n}{animate}\PYG{p}{(}\PYG{k+kc}{True}\PYG{p}{)}
\PYG{n}{env}\PYG{o}{.}\PYG{n}{modelname}\PYG{p}{(}\PYG{l+s+s1}{\PYGZsq{}}\PYG{l+s+s1}{Demo queue animation}\PYG{l+s+s1}{\PYGZsq{}}\PYG{p}{)}
\PYG{n}{env}\PYG{o}{.}\PYG{n}{run}\PYG{p}{(}\PYG{p}{)}
\end{sphinxVerbatim}

Here is snapshot of this powerful, dynamics (including the histogram!):

\noindent\sphinxincludegraphics{{Pic2}.png}


\section{Advanced}
\label{\detokenize{Animation:advanced}}
The various classes have a lot of parameters, like color, line width, font, etc.

These parameters can be given just as a scalar, like:

\sphinxcode{sim.AnimateText(text='Hello world', x=200, y=300, textcolor='red')}

But each of these parameters may also be a:
\begin{itemize}
\item {} 
function with zero arguments

\item {} 
function with one argument being the time t

\item {} 
function with two arguments being ‘arg’ and the time t

\item {} 
a method with instance ‘arg’ and the time t

\end{itemize}

The function or method is called at each animation frame update (maximum of 30 frames per second).

This makes it for instance possible to show dynamically the mean of monitor m, like in

\sphinxcode{sim.AnimateRectangle(spec=(10, 10, 200, 30), text=lambda: str(m.mean())}


\section{Class Animate}
\label{\detokenize{Animation:class-animate}}
This class can be used to show:
\begin{itemize}
\item {} 
line (if line0 is specified)

\item {} 
rectangle (if rectangle0 is specified)

\item {} 
polygon (if polygon0 is specified)

\item {} 
circle (if circle0 is specified)

\item {} 
text (if text is specified)

\item {} 
image (if image is specicified)

\end{itemize}

Note that only one type is allowed per instance of Animate.

Nearly all attributes of an Animate object are interpolated between time t0 and t1.
If t0 is not specified, now() is assumed.
If t1 is not specified inf is assumed, which means that the attribute will be the ‘0’ attribute.

E.g.:

\sphinxcode{Animate(x0=100,y0=100,rectangle0==(-10,-10,10,10))} will show a square around (100,100) for ever 
\sphinxcode{Animate(x0=100,y0=100,x1=200,y1=0,rectangle0=(-10,-10,10,10))} will still show the same square around (100,100) as t1 is not specified 
\sphinxcode{Animate(t1=env.now()+10,x0=100,y0=100,x1=200,y1=0,rectangle0=(-10,-10,10,10))} will show a square moving from (100,100) to (200,0) in 10 units of time. 

It also possible to let the rectangle change shape over time:

\sphinxcode{Animate(t1=env.now(),x0=100,y0=100,x1=200,y1=0,rectangle0=(-10,-10,10,10),rectangle1=(-20,-20,20,20))} will show a moving and growing rectangle. 

By default, the animation object will not change anymore after t1, but will remain visible. Alternatively, if
keep=False is specified, the object will disappear at time t1.

Also, colors, fontsizes, angles can be changed in a linear way over time.

E.g.:

\sphinxcode{Animate(t1=env.now()+10,text='Test',textcolor0='red',textcolor1='blue',angle0=0,angle1=360)} will show a rotating text changing from red to blue in 10 units of time.

The animation object can be updated with the update method. Here, once again, all the attributes can be specified to change over time. Note that the defaults for the ‘0’ values are the actual values at t=now().

Thus,

\sphinxcode{an=Animate(t0=0,t1=10,x0=0,x1=100,y0=0,circle0=(10,),circle1=(20,))}
will show a horizontally moving, growing circle.

Now, at time t=5, we issue
\sphinxcode{an.update(t1=10,y1=50,circle1=(10,))}
Then x0 will be set 50 (halfway 0 an 100) and cicle0 to (15,) (halfway 10 and 20).
Thus the circle will shrink to its original size and move vertically from (50,0) to (50,50).
This concept is very useful for moving objects whose position and orientation are controlled by the simulation.

Here we explain how an attribute changes during time. We use x as an example.
Normally, x=x0 at t=t0 and x=x1 at t\textgreater{}=t1. between t=t0 and t=t1, x is linearly interpolated.
An application can however override the x method. The prefered way is to subclass the Animate class:

\begin{sphinxVerbatim}[commandchars=\\\{\}]
\PYG{c+c1}{\PYGZsh{} Demo animate 1}
\PYG{k+kn}{import} \PYG{n+nn}{salabim} \PYG{k}{as} \PYG{n+nn}{sim}


\PYG{k}{class} \PYG{n+nc}{AnimateMovingText}\PYG{p}{(}\PYG{n}{sim}\PYG{o}{.}\PYG{n}{Animate}\PYG{p}{)}\PYG{p}{:}
    \PYG{k}{def} \PYG{n+nf}{\PYGZus{}\PYGZus{}init\PYGZus{}\PYGZus{}}\PYG{p}{(}\PYG{n+nb+bp}{self}\PYG{p}{)}\PYG{p}{:}
        \PYG{n}{sim}\PYG{o}{.}\PYG{n}{Animate}\PYG{o}{.}\PYG{n+nf+fm}{\PYGZus{}\PYGZus{}init\PYGZus{}\PYGZus{}}\PYG{p}{(}\PYG{n+nb+bp}{self}\PYG{p}{,} \PYG{n}{text}\PYG{o}{=}\PYG{l+s+s1}{\PYGZsq{}}\PYG{l+s+s1}{\PYGZsq{}}\PYG{p}{,} \PYG{n}{x0}\PYG{o}{=}\PYG{l+m+mi}{100}\PYG{p}{,} \PYG{n}{x1}\PYG{o}{=}\PYG{l+m+mi}{1000}\PYG{p}{,} \PYG{n}{y0}\PYG{o}{=}\PYG{l+m+mi}{100}\PYG{p}{,} \PYG{n}{t1}\PYG{o}{=}\PYG{n}{env}\PYG{o}{.}\PYG{n}{now}\PYG{p}{(}\PYG{p}{)} \PYG{o}{+} \PYG{l+m+mi}{10}\PYG{p}{)}

    \PYG{k}{def} \PYG{n+nf}{x}\PYG{p}{(}\PYG{n+nb+bp}{self}\PYG{p}{,} \PYG{n}{t}\PYG{p}{)}\PYG{p}{:}
        \PYG{k}{return} \PYG{n}{sim}\PYG{o}{.}\PYG{n}{interpolate}\PYG{p}{(}\PYG{n}{sim}\PYG{o}{.}\PYG{n}{interpolate}\PYG{p}{(}\PYG{n}{t}\PYG{p}{,} \PYG{n+nb+bp}{self}\PYG{o}{.}\PYG{n}{t0}\PYG{p}{,} \PYG{n+nb+bp}{self}\PYG{o}{.}\PYG{n}{t1}\PYG{p}{,} \PYG{l+m+mi}{0}\PYG{p}{,} \PYG{l+m+mi}{1}\PYG{p}{)}\PYG{o}{*}\PYG{o}{*}\PYG{l+m+mi}{2}\PYG{p}{,} \PYG{l+m+mi}{0}\PYG{p}{,} \PYG{l+m+mi}{1}\PYG{p}{,} \PYG{n+nb+bp}{self}\PYG{o}{.}\PYG{n}{x0}\PYG{p}{,} \PYG{n+nb+bp}{self}\PYG{o}{.}\PYG{n}{x1}\PYG{p}{)}

    \PYG{k}{def} \PYG{n+nf}{y}\PYG{p}{(}\PYG{n+nb+bp}{self}\PYG{p}{,} \PYG{n}{t}\PYG{p}{)}\PYG{p}{:}
        \PYG{k}{return} \PYG{n+nb}{int}\PYG{p}{(}\PYG{n}{t}\PYG{p}{)} \PYG{o}{*} \PYG{l+m+mi}{50}

    \PYG{k}{def} \PYG{n+nf}{text}\PYG{p}{(}\PYG{n+nb+bp}{self}\PYG{p}{,} \PYG{n}{t}\PYG{p}{)}\PYG{p}{:}
        \PYG{k}{return} \PYG{l+s+s1}{\PYGZsq{}}\PYG{l+s+si}{\PYGZob{}:0.1f\PYGZcb{}}\PYG{l+s+s1}{\PYGZsq{}}\PYG{o}{.}\PYG{n}{format}\PYG{p}{(}\PYG{n}{t}\PYG{p}{)}


\PYG{n}{env} \PYG{o}{=} \PYG{n}{sim}\PYG{o}{.}\PYG{n}{Environment}\PYG{p}{(}\PYG{p}{)}

\PYG{n}{env}\PYG{o}{.}\PYG{n}{animation\PYGZus{}parameters}\PYG{p}{(}\PYG{p}{)}

\PYG{n}{AnimateMovingText}\PYG{p}{(}\PYG{p}{)}

\PYG{n}{env}\PYG{o}{.}\PYG{n}{run}\PYG{p}{(}\PYG{p}{)}


\end{sphinxVerbatim}

This code will show the current simulation time moving from left to right, uniformly accelerated.
And the text will be shown a bit higher up, every second. It is not necessary to use t0, t1, x0, x1, but
is a convenient way of setting attributes.

The following methods may be overridden:


\begin{savenotes}\sphinxattablestart
\centering
\begin{tabular}[t]{|*{7}{\X{1}{7}|}}
\hline
\sphinxstylethead{\sphinxstyletheadfamily 
method
\unskip}\relax &\sphinxstylethead{\sphinxstyletheadfamily 
circle
\unskip}\relax &\sphinxstylethead{\sphinxstyletheadfamily 
image
\unskip}\relax &\sphinxstylethead{\sphinxstyletheadfamily 
line
\unskip}\relax &\sphinxstylethead{\sphinxstyletheadfamily 
polygon
\unskip}\relax &\sphinxstylethead{\sphinxstyletheadfamily 
rectangle
\unskip}\relax &\sphinxstylethead{\sphinxstyletheadfamily 
text
\unskip}\relax \\
\hline
anchor
&&\begin{itemize}
\item {} 
\end{itemize}
&&&&\\
\hline
angle
&\begin{itemize}
\item {} 
\end{itemize}
&\begin{itemize}
\item {} 
\end{itemize}
&\begin{itemize}
\item {} 
\end{itemize}
&\begin{itemize}
\item {} 
\end{itemize}
&\begin{itemize}
\item {} 
\end{itemize}
&\begin{itemize}
\item {} 
\end{itemize}
\\
\hline
circle
&\begin{itemize}
\item {} 
\end{itemize}
&&&&&\\
\hline
fillcolor
&\begin{itemize}
\item {} 
\end{itemize}
&&&\begin{itemize}
\item {} 
\end{itemize}
&\begin{itemize}
\item {} 
\end{itemize}
&\\
\hline
fontsize
&&&&&&\begin{itemize}
\item {} 
\end{itemize}
\\
\hline
image
&&\begin{itemize}
\item {} 
\end{itemize}
&&&&\\
\hline
layer
&\begin{itemize}
\item {} 
\end{itemize}
&\begin{itemize}
\item {} 
\end{itemize}
&\begin{itemize}
\item {} 
\end{itemize}
&\begin{itemize}
\item {} 
\end{itemize}
&\begin{itemize}
\item {} 
\end{itemize}
&\begin{itemize}
\item {} 
\end{itemize}
\\
\hline
line
&&&\begin{itemize}
\item {} 
\end{itemize}
&&&\\
\hline
linecolor
&\begin{itemize}
\item {} 
\end{itemize}
&&\begin{itemize}
\item {} 
\end{itemize}
&\begin{itemize}
\item {} 
\end{itemize}
&\begin{itemize}
\item {} 
\end{itemize}
&\\
\hline
linewidth
&\begin{itemize}
\item {} 
\end{itemize}
&&\begin{itemize}
\item {} 
\end{itemize}
&\begin{itemize}
\item {} 
\end{itemize}
&\begin{itemize}
\item {} 
\end{itemize}
&\\
\hline
max\_lines
&&&&&&\begin{itemize}
\item {} 
\end{itemize}
\\
\hline
offsetx
&\begin{itemize}
\item {} 
\end{itemize}
&\begin{itemize}
\item {} 
\end{itemize}
&\begin{itemize}
\item {} 
\end{itemize}
&\begin{itemize}
\item {} 
\end{itemize}
&\begin{itemize}
\item {} 
\end{itemize}
&\begin{itemize}
\item {} 
\end{itemize}
\\
\hline
offsety
&\begin{itemize}
\item {} 
\end{itemize}
&\begin{itemize}
\item {} 
\end{itemize}
&\begin{itemize}
\item {} 
\end{itemize}
&\begin{itemize}
\item {} 
\end{itemize}
&\begin{itemize}
\item {} 
\end{itemize}
&\begin{itemize}
\item {} 
\end{itemize}
\\
\hline
polygon
&&&&\begin{itemize}
\item {} 
\end{itemize}
&&\\
\hline
rectangle
&&&&&\begin{itemize}
\item {} 
\end{itemize}
&\\
\hline
text
&&&&&&\begin{itemize}
\item {} 
\end{itemize}
\\
\hline
text\_anchor
&&&&&&\begin{itemize}
\item {} 
\end{itemize}
\\
\hline
textcolor
&&&&&&\begin{itemize}
\item {} 
\end{itemize}
\\
\hline
visible
&\begin{itemize}
\item {} 
\end{itemize}
&\begin{itemize}
\item {} 
\end{itemize}
&\begin{itemize}
\item {} 
\end{itemize}
&\begin{itemize}
\item {} 
\end{itemize}
&\begin{itemize}
\item {} 
\end{itemize}
&\begin{itemize}
\item {} 
\end{itemize}
\\
\hline
width
&&\begin{itemize}
\item {} 
\end{itemize}
&&&&\\
\hline
x
&\begin{itemize}
\item {} 
\end{itemize}
&\begin{itemize}
\item {} 
\end{itemize}
&\begin{itemize}
\item {} 
\end{itemize}
&\begin{itemize}
\item {} 
\end{itemize}
&\begin{itemize}
\item {} 
\end{itemize}
&\begin{itemize}
\item {} 
\end{itemize}
\\
\hline
xy\_anchor
&\begin{itemize}
\item {} 
\end{itemize}
&\begin{itemize}
\item {} 
\end{itemize}
&\begin{itemize}
\item {} 
\end{itemize}
&\begin{itemize}
\item {} 
\end{itemize}
&\begin{itemize}
\item {} 
\end{itemize}
&\begin{itemize}
\item {} 
\end{itemize}
\\
\hline
y
&\begin{itemize}
\item {} 
\end{itemize}
&\begin{itemize}
\item {} 
\end{itemize}
&\begin{itemize}
\item {} 
\end{itemize}
&\begin{itemize}
\item {} 
\end{itemize}
&\begin{itemize}
\item {} 
\end{itemize}
&\begin{itemize}
\item {} 
\end{itemize}
\\
\hline
\end{tabular}
\par
\sphinxattableend\end{savenotes}

\sphinxstyleemphasis{Dashboard animation}

Here we present an example model where the simulation code is completely separated from the animation code.
This makes communication and debugging and switching off animation much easier.

The example below generates 15 persons starting at time 0, 1, … . These persons enter a queue called q and
stay there 15 time units.

The animation dashboard shows the first 10 persons in the queue q, along with the length of that q.

\begin{sphinxVerbatim}[commandchars=\\\{\}]
\PYG{c+c1}{\PYGZsh{} Demo animate 2.py}
\PYG{k+kn}{import} \PYG{n+nn}{salabim} \PYG{k}{as} \PYG{n+nn}{sim}


\PYG{k}{class} \PYG{n+nc}{AnimateWaitSquare}\PYG{p}{(}\PYG{n}{sim}\PYG{o}{.}\PYG{n}{Animate}\PYG{p}{)}\PYG{p}{:}
    \PYG{k}{def} \PYG{n+nf}{\PYGZus{}\PYGZus{}init\PYGZus{}\PYGZus{}}\PYG{p}{(}\PYG{n+nb+bp}{self}\PYG{p}{,} \PYG{n}{i}\PYG{p}{)}\PYG{p}{:}
        \PYG{n+nb+bp}{self}\PYG{o}{.}\PYG{n}{i} \PYG{o}{=} \PYG{n}{i}
        \PYG{n}{sim}\PYG{o}{.}\PYG{n}{Animate}\PYG{o}{.}\PYG{n+nf+fm}{\PYGZus{}\PYGZus{}init\PYGZus{}\PYGZus{}}\PYG{p}{(}\PYG{n+nb+bp}{self}\PYG{p}{,}
            \PYG{n}{rectangle0}\PYG{o}{=}\PYG{p}{(}\PYG{o}{\PYGZhy{}}\PYG{l+m+mi}{10}\PYG{p}{,} \PYG{o}{\PYGZhy{}}\PYG{l+m+mi}{10}\PYG{p}{,} \PYG{l+m+mi}{10}\PYG{p}{,} \PYG{l+m+mi}{10}\PYG{p}{)}\PYG{p}{,} \PYG{n}{x0}\PYG{o}{=}\PYG{l+m+mi}{300} \PYG{o}{\PYGZhy{}} \PYG{l+m+mi}{30} \PYG{o}{*} \PYG{n}{i}\PYG{p}{,} \PYG{n}{y0}\PYG{o}{=}\PYG{l+m+mi}{100}\PYG{p}{,} \PYG{n}{fillcolor0}\PYG{o}{=}\PYG{l+s+s1}{\PYGZsq{}}\PYG{l+s+s1}{red}\PYG{l+s+s1}{\PYGZsq{}}\PYG{p}{,} \PYG{n}{linewidth0}\PYG{o}{=}\PYG{l+m+mi}{0}\PYG{p}{)}

    \PYG{k}{def} \PYG{n+nf}{visible}\PYG{p}{(}\PYG{n+nb+bp}{self}\PYG{p}{,} \PYG{n}{t}\PYG{p}{)}\PYG{p}{:}
        \PYG{k}{return} \PYG{n}{q}\PYG{p}{[}\PYG{n+nb+bp}{self}\PYG{o}{.}\PYG{n}{i}\PYG{p}{]} \PYG{o+ow}{is} \PYG{o+ow}{not} \PYG{k+kc}{None}


\PYG{k}{class} \PYG{n+nc}{AnimateWaitText}\PYG{p}{(}\PYG{n}{sim}\PYG{o}{.}\PYG{n}{Animate}\PYG{p}{)}\PYG{p}{:}
    \PYG{k}{def} \PYG{n+nf}{\PYGZus{}\PYGZus{}init\PYGZus{}\PYGZus{}}\PYG{p}{(}\PYG{n+nb+bp}{self}\PYG{p}{,} \PYG{n}{i}\PYG{p}{)}\PYG{p}{:}
        \PYG{n+nb+bp}{self}\PYG{o}{.}\PYG{n}{i} \PYG{o}{=} \PYG{n}{i}
        \PYG{n}{sim}\PYG{o}{.}\PYG{n}{Animate}\PYG{o}{.}\PYG{n+nf+fm}{\PYGZus{}\PYGZus{}init\PYGZus{}\PYGZus{}}\PYG{p}{(}\PYG{n+nb+bp}{self}\PYG{p}{,} \PYG{n}{text}\PYG{o}{=}\PYG{l+s+s1}{\PYGZsq{}}\PYG{l+s+s1}{\PYGZsq{}}\PYG{p}{,} \PYG{n}{x0}\PYG{o}{=}\PYG{l+m+mi}{300} \PYG{o}{\PYGZhy{}} \PYG{l+m+mi}{30} \PYG{o}{*} \PYG{n}{i}\PYG{p}{,} \PYG{n}{y0}\PYG{o}{=}\PYG{l+m+mi}{100}\PYG{p}{,} \PYG{n}{textcolor0}\PYG{o}{=}\PYG{l+s+s1}{\PYGZsq{}}\PYG{l+s+s1}{white}\PYG{l+s+s1}{\PYGZsq{}}\PYG{p}{)}

    \PYG{k}{def} \PYG{n+nf}{text}\PYG{p}{(}\PYG{n+nb+bp}{self}\PYG{p}{,} \PYG{n}{t}\PYG{p}{)}\PYG{p}{:}
        \PYG{n}{component\PYGZus{}i} \PYG{o}{=} \PYG{n}{q}\PYG{p}{[}\PYG{n+nb+bp}{self}\PYG{o}{.}\PYG{n}{i}\PYG{p}{]}

        \PYG{k}{if} \PYG{n}{component\PYGZus{}i} \PYG{o+ow}{is} \PYG{k+kc}{None}\PYG{p}{:}
            \PYG{k}{return} \PYG{l+s+s1}{\PYGZsq{}}\PYG{l+s+s1}{\PYGZsq{}}
        \PYG{k}{else}\PYG{p}{:}
            \PYG{k}{return} \PYG{n}{component\PYGZus{}i}\PYG{o}{.}\PYG{n}{name}\PYG{p}{(}\PYG{p}{)}


\PYG{k}{def} \PYG{n+nf}{do\PYGZus{}animation}\PYG{p}{(}\PYG{p}{)}\PYG{p}{:}
    \PYG{n}{env}\PYG{o}{.}\PYG{n}{animation\PYGZus{}parameters}\PYG{p}{(}\PYG{p}{)}
    \PYG{k}{for} \PYG{n}{i} \PYG{o+ow}{in} \PYG{n+nb}{range}\PYG{p}{(}\PYG{l+m+mi}{10}\PYG{p}{)}\PYG{p}{:}
        \PYG{n}{AnimateWaitSquare}\PYG{p}{(}\PYG{n}{i}\PYG{p}{)}
        \PYG{n}{AnimateWaitText}\PYG{p}{(}\PYG{n}{i}\PYG{p}{)}
    \PYG{n}{show\PYGZus{}length} \PYG{o}{=} \PYG{n}{sim}\PYG{o}{.}\PYG{n}{Animate}\PYG{p}{(}\PYG{n}{text}\PYG{o}{=}\PYG{l+s+s1}{\PYGZsq{}}\PYG{l+s+s1}{\PYGZsq{}}\PYG{p}{,} \PYG{n}{x0}\PYG{o}{=}\PYG{l+m+mi}{330}\PYG{p}{,} \PYG{n}{y0}\PYG{o}{=}\PYG{l+m+mi}{100}\PYG{p}{,} \PYG{n}{textcolor0}\PYG{o}{=}\PYG{l+s+s1}{\PYGZsq{}}\PYG{l+s+s1}{black}\PYG{l+s+s1}{\PYGZsq{}}\PYG{p}{,} \PYG{n}{anchor}\PYG{o}{=}\PYG{l+s+s1}{\PYGZsq{}}\PYG{l+s+s1}{w}\PYG{l+s+s1}{\PYGZsq{}}\PYG{p}{)}
    \PYG{n}{show\PYGZus{}length}\PYG{o}{.}\PYG{n}{text} \PYG{o}{=} \PYG{k}{lambda} \PYG{n}{t}\PYG{p}{:} \PYG{l+s+s1}{\PYGZsq{}}\PYG{l+s+s1}{Length= }\PYG{l+s+s1}{\PYGZsq{}} \PYG{o}{+} \PYG{n+nb}{str}\PYG{p}{(}\PYG{n+nb}{len}\PYG{p}{(}\PYG{n}{q}\PYG{p}{)}\PYG{p}{)}


\PYG{k}{class} \PYG{n+nc}{Person}\PYG{p}{(}\PYG{n}{sim}\PYG{o}{.}\PYG{n}{Component}\PYG{p}{)}\PYG{p}{:}
    \PYG{k}{def} \PYG{n+nf}{process}\PYG{p}{(}\PYG{n+nb+bp}{self}\PYG{p}{)}\PYG{p}{:}
        \PYG{n+nb+bp}{self}\PYG{o}{.}\PYG{n}{enter}\PYG{p}{(}\PYG{n}{q}\PYG{p}{)}
        \PYG{k}{yield} \PYG{n+nb+bp}{self}\PYG{o}{.}\PYG{n}{hold}\PYG{p}{(}\PYG{l+m+mi}{15}\PYG{p}{)}
        \PYG{n+nb+bp}{self}\PYG{o}{.}\PYG{n}{leave}\PYG{p}{(}\PYG{n}{q}\PYG{p}{)}


\PYG{n}{env} \PYG{o}{=} \PYG{n}{sim}\PYG{o}{.}\PYG{n}{Environment}\PYG{p}{(}\PYG{n}{trace}\PYG{o}{=}\PYG{k+kc}{True}\PYG{p}{)}

\PYG{n}{q} \PYG{o}{=} \PYG{n}{sim}\PYG{o}{.}\PYG{n}{Queue}\PYG{p}{(}\PYG{l+s+s1}{\PYGZsq{}}\PYG{l+s+s1}{q}\PYG{l+s+s1}{\PYGZsq{}}\PYG{p}{)}
\PYG{k}{for} \PYG{n}{i} \PYG{o+ow}{in} \PYG{n+nb}{range}\PYG{p}{(}\PYG{l+m+mi}{15}\PYG{p}{)}\PYG{p}{:}
    \PYG{n}{Person}\PYG{p}{(}\PYG{n}{name}\PYG{o}{=}\PYG{l+s+s1}{\PYGZsq{}}\PYG{l+s+si}{\PYGZob{}:02d\PYGZcb{}}\PYG{l+s+s1}{\PYGZsq{}}\PYG{o}{.}\PYG{n}{format}\PYG{p}{(}\PYG{n}{i}\PYG{p}{)}\PYG{p}{,} \PYG{n}{at}\PYG{o}{=}\PYG{n}{i}\PYG{p}{)}

\PYG{n}{do\PYGZus{}animation}\PYG{p}{(}\PYG{p}{)}

\PYG{n}{env}\PYG{o}{.}\PYG{n}{run}\PYG{p}{(}\PYG{p}{)}
\end{sphinxVerbatim}

All animation initialization is in \sphinxcode{do\_animation}, where first 10 rectangle and text Animate
objects are created. These are classes that are inherited from sim.Animate.

The AnimateWaitSquare defines a red rectangle at a specific position in the \sphinxcode{sim.Animate.\_\_init\_\_()} call.
Note that normally these squares should be displayed. But, here we have overridden the visible method.
If there is no i-th component in the q, the square will be made invisible. Otherwise, it is visible.

The AnimateWaitText is more or less defined in a similar way. It defines a text in white at a specific position.
Only the text method is overridden and will return the name of the i-th component in the queue, if any. Otherwise
the null string will be returned.

The length of the queue q could be defined also by subclassing sim.Animate, but here we just make a direct instance
of Animate with the null string as the text to be displayed. And then we immediately override the text method with
a lambda function. Note that in this case, self is not available!


\section{Video production and snapshots}
\label{\detokenize{Animation:video-production-and-snapshots}}
An animation can be recorded as an .mp4 video by sprecifying \sphinxcode{video=filename} in the call to animation\_parameters.
The effect is that 30 time per second (scaled animation time) a frame is written. In this case, the animation does not
run synchronized with the wall clock anymore. Depending on the complexity of the animation, the simulation might run
faster of slower than real time. Other than with an ordinary animation, frames are never skipped.

Once control is given back to main, the .mp4 file is closed.

Salabim also suppports taking a snapshot of an animated screen with \sphinxcode{Environment.snapshot()}.


\chapter{Reading items from a file}
\label{\detokenize{Reading items from a file:reading-items-from-a-file}}\label{\detokenize{Reading items from a file::doc}}
Salabim models often need to read input values from a file.

As these data are quite often quite unstructured, using the standard read facilities of text files can be rather tedious.

Therefore, salabim offers the possibility to read a file item by item.

Example usage

\begin{sphinxVerbatim}[commandchars=\\\{\}]
\PYG{k}{with} \PYG{n}{sim}\PYG{o}{.}\PYG{n}{ItemFile}\PYG{p}{(}\PYG{n}{filename}\PYG{p}{)} \PYG{k}{as} \PYG{n}{f}\PYG{p}{:}
    \PYG{n}{run\PYGZus{}length} \PYG{o}{=} \PYG{n}{f}\PYG{o}{.}\PYG{n}{read\PYGZus{}item\PYGZus{}float}\PYG{p}{(}\PYG{p}{)}
    \PYG{n}{run\PYGZus{}name} \PYG{o}{=} \PYG{n}{f}\PYG{o}{.}\PYG{n}{read\PYGZus{}item}\PYG{p}{(}\PYG{p}{)}
\end{sphinxVerbatim}

Or (not recommended)

\begin{sphinxVerbatim}[commandchars=\\\{\}]
\PYG{n}{f} \PYG{o}{=} \PYG{n}{sim}\PYG{o}{.}\PYG{n}{InputFile}\PYG{p}{(}\PYG{n}{filename}\PYG{p}{)}
\PYG{n}{run\PYGZus{}length} \PYG{o}{=} \PYG{n}{f}\PYG{o}{.}\PYG{n}{read\PYGZus{}item\PYGZus{}float}\PYG{p}{(}\PYG{p}{)}
\PYG{n}{run\PYGZus{}name} \PYG{o}{=} \PYG{n}{f}\PYG{o}{.}\PYG{n}{read\PYGZus{}item}\PYG{p}{(}\PYG{p}{)}
\PYG{n}{f}\PYG{o}{.}\PYG{n}{close}\PYG{p}{(}\PYG{p}{)}
\end{sphinxVerbatim}

The input file is read per item, where blanks, linefeeds, tabs are treated as separators. 
Any text on a line after a \# character is ignored. 
Any text within curly brackets ( \{\} ) is ignored (and treated as an item separator). 
Note that this strictly on a per line basis. 
If a blank or tab is to be included in a string, use single or double quotes.  
The recommended way to end a list of values is //

So, a typical input file is

\begin{sphinxVerbatim}[commandchars=\\\{\}]
\PYG{c+c1}{\PYGZsh{} Typical experiment file for a salabim model}
\PYG{l+m+mi}{1000}              \PYG{c+c1}{\PYGZsh{} run length}
\PYG{l+s+s1}{\PYGZsq{}}\PYG{l+s+s1}{Experiment 2.0}\PYG{l+s+s1}{\PYGZsq{}}  \PYG{c+c1}{\PYGZsh{} run name}

 \PYG{c+c1}{\PYGZsh{}Model          speed color}
 \PYG{c+c1}{\PYGZsh{}\PYGZhy{}\PYGZhy{}\PYGZhy{}\PYGZhy{}\PYGZhy{}\PYGZhy{}\PYGZhy{}\PYGZhy{}\PYGZhy{}\PYGZhy{}\PYGZhy{}\PYGZhy{}\PYGZhy{}\PYGZhy{} \PYGZhy{}\PYGZhy{}\PYGZhy{}\PYGZhy{}\PYGZhy{} \PYGZhy{}\PYGZhy{}\PYGZhy{}\PYGZhy{}\PYGZhy{}\PYGZhy{}}

 \PYG{l+s+s1}{\PYGZsq{}}\PYG{l+s+s1}{Peugeot 208}\PYG{l+s+s1}{\PYGZsq{}}       \PYG{l+m+mi}{150} \PYG{n}{red}
 \PYG{l+s+s1}{\PYGZsq{}}\PYG{l+s+s1}{Peugeot 3008}\PYG{l+s+s1}{\PYGZsq{}}      \PYG{l+m+mi}{175} \PYG{n}{orange}
 \PYG{l+s+s1}{\PYGZsq{}}\PYG{l+s+s1}{Citroen C5}\PYG{l+s+s1}{\PYGZsq{}}        \PYG{l+m+mi}{160} \PYG{n}{blue}
 \PYG{l+s+s1}{\PYGZsq{}}\PYG{l+s+s1}{Renault }\PYG{l+s+s1}{\PYGZdq{}}\PYG{l+s+s1}{Twingo}\PYG{l+s+s1}{\PYGZdq{}}\PYG{l+s+s1}{\PYGZsq{}}  \PYG{l+m+mi}{165} \PYG{n}{green}
 \PYG{o}{/}\PYG{o}{/}

 \PYG{n}{France} \PYG{p}{\PYGZob{}}\PYG{n}{country}\PYG{p}{\PYGZcb{}} \PYG{n}{Europe} \PYG{p}{\PYGZob{}}\PYG{n}{continent}\PYG{p}{\PYGZcb{}}

 \PYG{c+c1}{\PYGZsh{}end of file}
\end{sphinxVerbatim}

Instead of the filename as a parameter to ItemFile, also a string with the content can be given. In that
case, at least one linefeed has to be in the content string. Usually, the content string will be triple
quoted. This can be very useful during testing as the input is part of the source file and not external, e.g.

\begin{sphinxVerbatim}[commandchars=\\\{\}]
\PYG{n}{test\PYGZus{}input} \PYG{o}{=} \PYG{l+s+s1}{\PYGZsq{}\PYGZsq{}\PYGZsq{}}
\PYG{l+s+s1}{one two}
\PYG{l+s+s1}{three four}
\PYG{l+s+s1}{five}
\PYG{l+s+s1}{\PYGZsq{}\PYGZsq{}\PYGZsq{}}
\PYG{k}{with} \PYG{n}{sim}\PYG{o}{.}\PYG{n}{ItemFile}\PYG{p}{(}\PYG{n}{test\PYGZus{}input}\PYG{p}{)} \PYG{k}{as} \PYG{n}{f}\PYG{p}{:}
    \PYG{k}{while} \PYG{k+kc}{True}\PYG{p}{:}
        \PYG{k}{try}\PYG{p}{:}
            \PYG{n+nb}{print}\PYG{p}{(}\PYG{n}{f}\PYG{o}{.}\PYG{n}{read\PYGZus{}item}\PYG{p}{(}\PYG{p}{)}\PYG{p}{)}
        \PYG{k}{except} \PYG{n+ne}{EOFError}\PYG{p}{:}
            \PYG{k}{break}
\end{sphinxVerbatim}


\chapter{Reference}
\label{\detokenize{Reference:reference}}\label{\detokenize{Reference::doc}}

\section{Animation}
\label{\detokenize{Reference:animation}}\index{Animate (class in salabim)}

\begin{fulllineitems}
\phantomsection\label{\detokenize{Reference:salabim.Animate}}\pysiglinewithargsret{\sphinxbfcode{class }\sphinxcode{salabim.}\sphinxbfcode{Animate}}{\emph{parent=None}, \emph{layer=0}, \emph{keep=True}, \emph{visible=True}, \emph{screen\_coordinates=False}, \emph{t0=None}, \emph{x0=0}, \emph{y0=0}, \emph{offsetx0=0}, \emph{offsety0=0}, \emph{circle0=None}, \emph{line0=None}, \emph{polygon0=None}, \emph{rectangle0=None}, \emph{points0=None}, \emph{image=None}, \emph{text=None}, \emph{font=''}, \emph{anchor='c'}, \emph{as\_points=False}, \emph{max\_lines=0}, \emph{text\_anchor=None}, \emph{linewidth0=None}, \emph{fillcolor0=None}, \emph{linecolor0='fg'}, \emph{textcolor0='fg'}, \emph{angle0=0}, \emph{fontsize0=20}, \emph{width0=None}, \emph{t1=None}, \emph{x1=None}, \emph{y1=None}, \emph{offsetx1=None}, \emph{offsety1=None}, \emph{circle1=None}, \emph{line1=None}, \emph{polygon1=None}, \emph{rectangle1=None}, \emph{points1=None}, \emph{linewidth1=None}, \emph{fillcolor1=None}, \emph{linecolor1=None}, \emph{textcolor1=None}, \emph{angle1=None}, \emph{fontsize1=None}, \emph{width1=None}, \emph{xy\_anchor=''}, \emph{env=None}}{}
defines an animation object
\begin{quote}\begin{description}
\item[{Parameters}] \leavevmode\begin{itemize}
\item {} 
\sphinxstyleliteralstrong{parent} ({\hyperref[\detokenize{Reference:salabim.Component}]{\sphinxcrossref{\sphinxstyleliteralemphasis{Component}}}}) \textendash{} component where this animation object belongs to (default None) 
if given, the animation object will be removed
automatically upon termination of the parent

\item {} 
\sphinxstyleliteralstrong{layer} (\sphinxstyleliteralemphasis{int}) \textendash{} layer value 
lower layer values are on top of higher layer values (default 0)

\item {} 
\sphinxstyleliteralstrong{keep} (\sphinxstyleliteralemphasis{bool}) \textendash{} keep 
if False, animation object is hidden after t1, shown otherwise
(default True)

\item {} 
\sphinxstyleliteralstrong{visible} (\sphinxstyleliteralemphasis{bool}) \textendash{} visible 
if False, animation object is not shown, shown otherwise
(default True)

\item {} 
\sphinxstyleliteralstrong{screen\_coordinates} (\sphinxstyleliteralemphasis{bool}) \textendash{} use screen\_coordinates 
normally, the scale parameters are use for positioning and scaling
objects. 
if True, screen\_coordinates will be used instead.

\item {} 
\sphinxstyleliteralstrong{xy\_anchor} (\sphinxstyleliteralemphasis{str}) \textendash{} specifies where x and y (i.e. x0, y0, x1 and y1) are relative to 
possible values are (default: sw) : 
\sphinxcode{nw    n    ne} 
\sphinxcode{w     c     e} 
\sphinxcode{sw    s    se} 
If ‘’, the given coordimates are used untranslated

\item {} 
\sphinxstyleliteralstrong{t0} (\sphinxstyleliteralemphasis{float}) \textendash{} time of start of the animation (default: now)

\item {} 
\sphinxstyleliteralstrong{x0} (\sphinxstyleliteralemphasis{float}) \textendash{} x-coordinate of the origin at time t0 (default 0)

\item {} 
\sphinxstyleliteralstrong{y0} (\sphinxstyleliteralemphasis{float}) \textendash{} y-coordinate of the origin at time t0 (default 0)

\item {} 
\sphinxstyleliteralstrong{offsetx0} (\sphinxstyleliteralemphasis{float}) \textendash{} offsets the x-coordinate of the object at time t0 (default 0)

\item {} 
\sphinxstyleliteralstrong{offsety0} (\sphinxstyleliteralemphasis{float}) \textendash{} offsets the y-coordinate of the object at time t0 (default 0)

\item {} 
\sphinxstyleliteralstrong{circle0} (\sphinxstyleliteralemphasis{float}\sphinxstyleliteralemphasis{ or }\sphinxstyleliteralemphasis{tuple/list}) \textendash{} the circle spec of the circle at time t0 
- radius 
- one item tuple/list containing the radius 
- five items tuple/list cntaining radius, radius1, arc\_angle0, arc\_angle1 and draw\_arc
(see class AnimateCircle for details)

\item {} 
\sphinxstyleliteralstrong{line0} (\sphinxstyleliteralemphasis{tuple}) \textendash{} the line(s) (xa,ya,xb,yb,xc,yc, …) at time t0

\item {} 
\sphinxstyleliteralstrong{polygon0} (\sphinxstyleliteralemphasis{tuple}) \textendash{} the polygon (xa,ya,xb,yb,xc,yc, …) at time t0 
the last point will be auto connected to the start

\item {} 
\sphinxstyleliteralstrong{rectangle0} (\sphinxstyleliteralemphasis{tuple}) \textendash{} the rectangle (xlowerleft,ylowerleft,xupperright,yupperright) at time t0 

\item {} 
\sphinxstyleliteralstrong{image} (\sphinxstyleliteralemphasis{str}\sphinxstyleliteralemphasis{ or }\sphinxstyleliteralemphasis{PIL image}) \textendash{} the image to be displayed 
This may be either a filename or a PIL image

\item {} 
\sphinxstyleliteralstrong{text} (\sphinxstyleliteralemphasis{str}\sphinxstyleliteralemphasis{, }\sphinxstyleliteralemphasis{tuple}\sphinxstyleliteralemphasis{ or }\sphinxstyleliteralemphasis{list}) \textendash{} the text to be displayed 
if text is str, the text may contain linefeeds, which are shown as individual lines

\item {} 
\sphinxstyleliteralstrong{max\_lines} (\sphinxstyleliteralemphasis{int}) \textendash{} the maximum of lines of text to be displayed 
if positive, it refers to the first max\_lines lines 
if negative, it refers to the first -max\_lines lines 
if zero (default), all lines will be displayed

\item {} 
\sphinxstyleliteralstrong{font} (\sphinxstyleliteralemphasis{str}\sphinxstyleliteralemphasis{ or }\sphinxstyleliteralemphasis{list/tuple}) \textendash{} font to be used for texts 
Either a string or a list/tuple of fontnames.
If not found, uses calibri or arial

\item {} 
\sphinxstyleliteralstrong{anchor} (\sphinxstyleliteralemphasis{str}) \textendash{} anchor position 
specifies where to put images or texts relative to the anchor
point 
possible values are (default: c): 
\sphinxcode{nw    n    ne} 
\sphinxcode{w     c     e} 
\sphinxcode{sw    s    se}

\item {} 
\sphinxstyleliteralstrong{as\_points} (\sphinxstyleliteralemphasis{bool}) \textendash{} if False (default), lines in line, rectangle and polygon are drawn 
if True, only the end points are shown in line, rectangle and polygon

\item {} 
\sphinxstyleliteralstrong{linewidth0} (\sphinxstyleliteralemphasis{float}) \textendash{} linewidth of the contour at time t0 (default 0 for polygon, rectangle and circle, 1 for line) 
if as\_point is True, the default size is 3

\item {} 
\sphinxstyleliteralstrong{fillcolor0} (\sphinxstyleliteralemphasis{colorspec}) \textendash{} color of interior at time t0 (default foreground\_color) 
if as\_points is True, fillcolor0 defaults to transparent

\item {} 
\sphinxstyleliteralstrong{linecolor0} (\sphinxstyleliteralemphasis{colorspec}) \textendash{} color of the contour at time t0 (default foreground\_color)

\item {} 
\sphinxstyleliteralstrong{textcolor0} (\sphinxstyleliteralemphasis{colorspec}) \textendash{} color of the text at time 0 (default foreground\_color)

\item {} 
\sphinxstyleliteralstrong{angle0} (\sphinxstyleliteralemphasis{float}) \textendash{} angle of the polygon at time t0 (in degrees) (default 0)

\item {} 
\sphinxstyleliteralstrong{fontsize0} (\sphinxstyleliteralemphasis{float}) \textendash{} fontsize of text at time t0 (default 20)

\item {} 
\sphinxstyleliteralstrong{width0} (\sphinxstyleliteralemphasis{float}) \textendash{} width of the image to be displayed at time t0 
if omitted or None, no scaling

\item {} 
\sphinxstyleliteralstrong{t1} (\sphinxstyleliteralemphasis{float}) \textendash{} time of end of the animation (default inf) 
if keep=True, the animation will continue (frozen) after t1

\item {} 
\sphinxstyleliteralstrong{x1} (\sphinxstyleliteralemphasis{float}) \textendash{} x-coordinate of the origin at time t1(default x0)

\item {} 
\sphinxstyleliteralstrong{y1} (\sphinxstyleliteralemphasis{float}) \textendash{} y-coordinate of the origin at time t1 (default y0)

\item {} 
\sphinxstyleliteralstrong{offsetx1} (\sphinxstyleliteralemphasis{float}) \textendash{} offsets the x-coordinate of the object at time t1 (default offsetx0)

\item {} 
\sphinxstyleliteralstrong{offsety1} (\sphinxstyleliteralemphasis{float}) \textendash{} offsets the y-coordinate of the object at time t1 (default offsety0)

\item {} 
\sphinxstyleliteralstrong{circle1} (\sphinxstyleliteralemphasis{float}\sphinxstyleliteralemphasis{ or }\sphinxstyleliteralemphasis{tuple/list}) \textendash{} the circle spec of the circle at time t1 (default: circle0) 
- radius 
- one item tuple/list containing the radius 
- five items tuple/list cntaining radius, radius1, arc\_angle0, arc\_angle1 and draw\_arc
(see class AnimateCircle for details)

\item {} 
\sphinxstyleliteralstrong{line1} (\sphinxstyleliteralemphasis{tuple}) \textendash{} the line(s) at time t1 (xa,ya,xb,yb,xc,yc, …) (default: line0) 
should have the same number of elements as line0

\item {} 
\sphinxstyleliteralstrong{polygon1} (\sphinxstyleliteralemphasis{tuple}) \textendash{} the polygon at time t1 (xa,ya,xb,yb,xc,yc, …) (default: polygon0) 
should have the same number of elements as polygon0

\item {} 
\sphinxstyleliteralstrong{rectangle1} (\sphinxstyleliteralemphasis{tuple}) \textendash{} the rectangle (xlowerleft,ylowerleft,xupperright,yupperright) at time t1
(default: rectangle0)

\item {} 
\sphinxstyleliteralstrong{linewidth1} (\sphinxstyleliteralemphasis{float}) \textendash{} linewidth of the contour at time t1 (default linewidth0)

\item {} 
\sphinxstyleliteralstrong{fillcolor1} (\sphinxstyleliteralemphasis{colorspec}) \textendash{} color of interior at time t1 (default fillcolor0)

\item {} 
\sphinxstyleliteralstrong{linecolor1} (\sphinxstyleliteralemphasis{colorspec}) \textendash{} color of the contour at time t1 (default linecolor0)

\item {} 
\sphinxstyleliteralstrong{textcolor1} (\sphinxstyleliteralemphasis{colorspec}) \textendash{} color of text at time t1 (default textcolor0)

\item {} 
\sphinxstyleliteralstrong{angle1} (\sphinxstyleliteralemphasis{float}) \textendash{} angle of the polygon at time t1 (in degrees) (default angle0)

\item {} 
\sphinxstyleliteralstrong{fontsize1} (\sphinxstyleliteralemphasis{float}) \textendash{} fontsize of text at time t1 (default: fontsize0)

\item {} 
\sphinxstyleliteralstrong{width1} (\sphinxstyleliteralemphasis{float}) \textendash{} width of the image to be displayed at time t1 (default: width0) 

\end{itemize}

\end{description}\end{quote}

\begin{sphinxadmonition}{note}{Note:}\begin{description}
\item[{one (and only one) of the following parameters is required:}] \leavevmode\begin{itemize}
\item {} 
circle0

\item {} 
image

\item {} 
line0

\item {} 
polygon0

\item {} 
rectangle0

\item {} 
text

\end{itemize}

\item[{colors may be specified as a}] \leavevmode\begin{itemize}
\item {} 
valid colorname

\item {} 
hexname

\item {} 
tuple (R,G,B) or (R,G,B,A)

\item {} 
‘fg’ or ‘bg’

\end{itemize}

\end{description}

colornames may contain an additional alpha, like \sphinxcode{red\#7f} 
hexnames may be either 3 of 4 bytes long (\sphinxcode{\#rrggbb} or \sphinxcode{\#rrggbbaa}) 
both colornames and hexnames may be given as a tuple with an
additional alpha between 0 and 255,
e.g. \sphinxcode{(255,0,255,128)}, (‘red’,127){}`{}` or \sphinxcode{('\#ff00ff',128)} 
fg is the foreground color 
bg is the background color 

Permitted parameters


\begin{savenotes}\sphinxattablestart
\centering
\begin{tabular}[t]{|*{7}{\X{1}{7}|}}
\hline
\sphinxstylethead{\sphinxstyletheadfamily 
parameter
\unskip}\relax &\sphinxstylethead{\sphinxstyletheadfamily 
circle
\unskip}\relax &\sphinxstylethead{\sphinxstyletheadfamily 
image
\unskip}\relax &\sphinxstylethead{\sphinxstyletheadfamily 
line
\unskip}\relax &\sphinxstylethead{\sphinxstyletheadfamily 
polygon
\unskip}\relax &\sphinxstylethead{\sphinxstyletheadfamily 
rectangle
\unskip}\relax &\sphinxstylethead{\sphinxstyletheadfamily 
text
\unskip}\relax \\
\hline
parent
&\begin{itemize}
\item {} 
\end{itemize}
&\begin{itemize}
\item {} 
\end{itemize}
&\begin{itemize}
\item {} 
\end{itemize}
&\begin{itemize}
\item {} 
\end{itemize}
&\begin{itemize}
\item {} 
\end{itemize}
&\begin{itemize}
\item {} 
\end{itemize}
\\
\hline
layer
&\begin{itemize}
\item {} 
\end{itemize}
&\begin{itemize}
\item {} 
\end{itemize}
&\begin{itemize}
\item {} 
\end{itemize}
&\begin{itemize}
\item {} 
\end{itemize}
&\begin{itemize}
\item {} 
\end{itemize}
&\begin{itemize}
\item {} 
\end{itemize}
\\
\hline
keep
&\begin{itemize}
\item {} 
\end{itemize}
&\begin{itemize}
\item {} 
\end{itemize}
&\begin{itemize}
\item {} 
\end{itemize}
&\begin{itemize}
\item {} 
\end{itemize}
&\begin{itemize}
\item {} 
\end{itemize}
&\begin{itemize}
\item {} 
\end{itemize}
\\
\hline
screen\_coordinates
&\begin{itemize}
\item {} 
\end{itemize}
&\begin{itemize}
\item {} 
\end{itemize}
&\begin{itemize}
\item {} 
\end{itemize}
&\begin{itemize}
\item {} 
\end{itemize}
&\begin{itemize}
\item {} 
\end{itemize}
&\begin{itemize}
\item {} 
\end{itemize}
\\
\hline
xy\_anchor
&\begin{itemize}
\item {} 
\end{itemize}
&\begin{itemize}
\item {} 
\end{itemize}
&\begin{itemize}
\item {} 
\end{itemize}
&\begin{itemize}
\item {} 
\end{itemize}
&\begin{itemize}
\item {} 
\end{itemize}
&\begin{itemize}
\item {} 
\end{itemize}
\\
\hline
t0,t1
&\begin{itemize}
\item {} 
\end{itemize}
&\begin{itemize}
\item {} 
\end{itemize}
&\begin{itemize}
\item {} 
\end{itemize}
&\begin{itemize}
\item {} 
\end{itemize}
&\begin{itemize}
\item {} 
\end{itemize}
&\begin{itemize}
\item {} 
\end{itemize}
\\
\hline
x0,x1
&\begin{itemize}
\item {} 
\end{itemize}
&\begin{itemize}
\item {} 
\end{itemize}
&\begin{itemize}
\item {} 
\end{itemize}
&\begin{itemize}
\item {} 
\end{itemize}
&\begin{itemize}
\item {} 
\end{itemize}
&\begin{itemize}
\item {} 
\end{itemize}
\\
\hline
y0,y1
&\begin{itemize}
\item {} 
\end{itemize}
&\begin{itemize}
\item {} 
\end{itemize}
&\begin{itemize}
\item {} 
\end{itemize}
&\begin{itemize}
\item {} 
\end{itemize}
&\begin{itemize}
\item {} 
\end{itemize}
&\begin{itemize}
\item {} 
\end{itemize}
\\
\hline
offsetx0,offsetx1
&\begin{itemize}
\item {} 
\end{itemize}
&\begin{itemize}
\item {} 
\end{itemize}
&\begin{itemize}
\item {} 
\end{itemize}
&\begin{itemize}
\item {} 
\end{itemize}
&\begin{itemize}
\item {} 
\end{itemize}
&\begin{itemize}
\item {} 
\end{itemize}
\\
\hline
offsety0,offsety1
&\begin{itemize}
\item {} 
\end{itemize}
&\begin{itemize}
\item {} 
\end{itemize}
&\begin{itemize}
\item {} 
\end{itemize}
&\begin{itemize}
\item {} 
\end{itemize}
&\begin{itemize}
\item {} 
\end{itemize}
&\begin{itemize}
\item {} 
\end{itemize}
\\
\hline
circle0,circle1
&\begin{itemize}
\item {} 
\end{itemize}
&&&&&\\
\hline
image
&&\begin{itemize}
\item {} 
\end{itemize}
&&&&\\
\hline
line0,line1
&&&\begin{itemize}
\item {} 
\end{itemize}
&&&\\
\hline
polygon0,polygon1
&&&&\begin{itemize}
\item {} 
\end{itemize}
&&\\
\hline
rectangle0,rectangle1
&&&&&\begin{itemize}
\item {} 
\end{itemize}
&\\
\hline
text
&&&&&&\begin{itemize}
\item {} 
\end{itemize}
\\
\hline
anchor
&&\begin{itemize}
\item {} 
\end{itemize}
&&&&\begin{itemize}
\item {} 
\end{itemize}
\\
\hline
linewidth0,linewidth1
&\begin{itemize}
\item {} 
\end{itemize}
&&\begin{itemize}
\item {} 
\end{itemize}
&\begin{itemize}
\item {} 
\end{itemize}
&\begin{itemize}
\item {} 
\end{itemize}
&\\
\hline
fillcolor0,fillcolor1
&\begin{itemize}
\item {} 
\end{itemize}
&&&\begin{itemize}
\item {} 
\end{itemize}
&\begin{itemize}
\item {} 
\end{itemize}
&\\
\hline
linecolor0,linecolor1
&\begin{itemize}
\item {} 
\end{itemize}
&&\begin{itemize}
\item {} 
\end{itemize}
&\begin{itemize}
\item {} 
\end{itemize}
&\begin{itemize}
\item {} 
\end{itemize}
&\\
\hline
textcolor0,textcolor1
&&&&&&\begin{itemize}
\item {} 
\end{itemize}
\\
\hline
angle0,angle1
&&\begin{itemize}
\item {} 
\end{itemize}
&\begin{itemize}
\item {} 
\end{itemize}
&\begin{itemize}
\item {} 
\end{itemize}
&\begin{itemize}
\item {} 
\end{itemize}
&\begin{itemize}
\item {} 
\end{itemize}
\\
\hline
as\_points
&&&\begin{itemize}
\item {} 
\end{itemize}
&\begin{itemize}
\item {} 
\end{itemize}
&\begin{itemize}
\item {} 
\end{itemize}
&\\
\hline
font
&&&&&&\begin{itemize}
\item {} 
\end{itemize}
\\
\hline
fontsize0,fontsize1
&&&&&&\begin{itemize}
\item {} 
\end{itemize}
\\
\hline
width0,width1
&&\begin{itemize}
\item {} 
\end{itemize}
&&&&\\
\hline
\end{tabular}
\par
\sphinxattableend\end{savenotes}
\end{sphinxadmonition}
\index{anchor() (salabim.Animate method)}

\begin{fulllineitems}
\phantomsection\label{\detokenize{Reference:salabim.Animate.anchor}}\pysiglinewithargsret{\sphinxbfcode{anchor}}{\emph{t=None}}{}
anchor of an animate object. May be overridden.
\begin{quote}\begin{description}
\item[{Parameters}] \leavevmode
\sphinxstyleliteralstrong{t} (\sphinxstyleliteralemphasis{float}) \textendash{} current time

\item[{Returns}] \leavevmode
\sphinxstylestrong{anchor} \textendash{} default behaviour: self.anchor0 (anchor given at creation or update)

\item[{Return type}] \leavevmode
str

\end{description}\end{quote}

\end{fulllineitems}

\index{angle() (salabim.Animate method)}

\begin{fulllineitems}
\phantomsection\label{\detokenize{Reference:salabim.Animate.angle}}\pysiglinewithargsret{\sphinxbfcode{angle}}{\emph{t=None}}{}
angle of an animate object. May be overridden.
\begin{quote}\begin{description}
\item[{Parameters}] \leavevmode
\sphinxstyleliteralstrong{t} (\sphinxstyleliteralemphasis{float}) \textendash{} current time

\item[{Returns}] \leavevmode
\sphinxstylestrong{angle} \textendash{} default behaviour: linear interpolation between self.angle0 and self.angle1

\item[{Return type}] \leavevmode
float

\end{description}\end{quote}

\end{fulllineitems}

\index{as\_points() (salabim.Animate method)}

\begin{fulllineitems}
\phantomsection\label{\detokenize{Reference:salabim.Animate.as_points}}\pysiglinewithargsret{\sphinxbfcode{as\_points}}{\emph{t=None}}{}
as\_points of an animate object. May be overridden.
\begin{quote}\begin{description}
\item[{Parameters}] \leavevmode
\sphinxstyleliteralstrong{t} (\sphinxstyleliteralemphasis{float}) \textendash{} current time

\item[{Returns}] \leavevmode
\sphinxstylestrong{as\_points} \textendash{} default behaviour: self.as\_points (text given at creation or update)

\item[{Return type}] \leavevmode
bool

\end{description}\end{quote}

\end{fulllineitems}

\index{circle() (salabim.Animate method)}

\begin{fulllineitems}
\phantomsection\label{\detokenize{Reference:salabim.Animate.circle}}\pysiglinewithargsret{\sphinxbfcode{circle}}{\emph{t=None}}{}
circle of an animate object. May be overridden.
\begin{quote}\begin{description}
\item[{Parameters}] \leavevmode
\sphinxstyleliteralstrong{t} (\sphinxstyleliteralemphasis{float}) \textendash{} current time

\item[{Returns}] \leavevmode
\sphinxstylestrong{circle} \textendash{} either 
- radius 
- one item tuple/list containing the radius 
- five items tuple/list cntaining radius, radius1, arc\_angle0, arc\_angle1 and draw\_arc 
(see class AnimateCircle for details) 
default behaviour: linear interpolation between self.circle0 and self.circle1

\item[{Return type}] \leavevmode
float or tuple/list

\end{description}\end{quote}

\end{fulllineitems}

\index{fillcolor() (salabim.Animate method)}

\begin{fulllineitems}
\phantomsection\label{\detokenize{Reference:salabim.Animate.fillcolor}}\pysiglinewithargsret{\sphinxbfcode{fillcolor}}{\emph{t=None}}{}
fillcolor of an animate object. May be overridden.
\begin{quote}\begin{description}
\item[{Parameters}] \leavevmode
\sphinxstyleliteralstrong{t} (\sphinxstyleliteralemphasis{float}) \textendash{} current time

\item[{Returns}] \leavevmode
\sphinxstylestrong{fillcolor} \textendash{} default behaviour: linear interpolation between self.fillcolor0 and self.fillcolor1

\item[{Return type}] \leavevmode
colorspec

\end{description}\end{quote}

\end{fulllineitems}

\index{font() (salabim.Animate method)}

\begin{fulllineitems}
\phantomsection\label{\detokenize{Reference:salabim.Animate.font}}\pysiglinewithargsret{\sphinxbfcode{font}}{\emph{t=None}}{}
font of an animate object. May be overridden.
\begin{quote}\begin{description}
\item[{Parameters}] \leavevmode
\sphinxstyleliteralstrong{t} (\sphinxstyleliteralemphasis{float}) \textendash{} current time

\item[{Returns}] \leavevmode
\sphinxstylestrong{font} \textendash{} default behaviour: self.font0 (font given at creation or update)

\item[{Return type}] \leavevmode
str

\end{description}\end{quote}

\end{fulllineitems}

\index{fontsize() (salabim.Animate method)}

\begin{fulllineitems}
\phantomsection\label{\detokenize{Reference:salabim.Animate.fontsize}}\pysiglinewithargsret{\sphinxbfcode{fontsize}}{\emph{t=None}}{}
fontsize of an animate object. May be overridden.
\begin{quote}\begin{description}
\item[{Parameters}] \leavevmode
\sphinxstyleliteralstrong{t} (\sphinxstyleliteralemphasis{float}) \textendash{} current time

\item[{Returns}] \leavevmode
\sphinxstylestrong{fontsize} \textendash{} default behaviour: linear interpolation between self.fontsize0 and self.fontsize1

\item[{Return type}] \leavevmode
float

\end{description}\end{quote}

\end{fulllineitems}

\index{image() (salabim.Animate method)}

\begin{fulllineitems}
\phantomsection\label{\detokenize{Reference:salabim.Animate.image}}\pysiglinewithargsret{\sphinxbfcode{image}}{\emph{t=None}}{}
image of an animate object. May be overridden.
\begin{quote}\begin{description}
\item[{Parameters}] \leavevmode
\sphinxstyleliteralstrong{t} (\sphinxstyleliteralemphasis{float}) \textendash{} current time

\item[{Returns}] \leavevmode
\sphinxstylestrong{image} \textendash{} use function spec\_to\_image to load a file
default behaviour: self.image0 (image given at creation or update)

\item[{Return type}] \leavevmode
PIL.Image.Image

\end{description}\end{quote}

\end{fulllineitems}

\index{layer() (salabim.Animate method)}

\begin{fulllineitems}
\phantomsection\label{\detokenize{Reference:salabim.Animate.layer}}\pysiglinewithargsret{\sphinxbfcode{layer}}{\emph{t=None}}{}
layer of an animate object. May be overridden.
\begin{quote}\begin{description}
\item[{Parameters}] \leavevmode
\sphinxstyleliteralstrong{t} (\sphinxstyleliteralemphasis{float}) \textendash{} current time

\item[{Returns}] \leavevmode
\sphinxstylestrong{layer} \textendash{} default behaviour: self.layer0 (layer given at creation or update)

\item[{Return type}] \leavevmode
int or float

\end{description}\end{quote}

\end{fulllineitems}

\index{line() (salabim.Animate method)}

\begin{fulllineitems}
\phantomsection\label{\detokenize{Reference:salabim.Animate.line}}\pysiglinewithargsret{\sphinxbfcode{line}}{\emph{t=None}}{}
line of an animate object. May be overridden.
\begin{quote}\begin{description}
\item[{Parameters}] \leavevmode
\sphinxstyleliteralstrong{t} (\sphinxstyleliteralemphasis{float}) \textendash{} current time

\item[{Returns}] \leavevmode
\sphinxstylestrong{line} \textendash{} series of x- and y-coordinates (xa,ya,xb,yb,xc,yc, …) 
default behaviour: linear interpolation between self.line0 and self.line1

\item[{Return type}] \leavevmode
tuple

\end{description}\end{quote}

\end{fulllineitems}

\index{linecolor() (salabim.Animate method)}

\begin{fulllineitems}
\phantomsection\label{\detokenize{Reference:salabim.Animate.linecolor}}\pysiglinewithargsret{\sphinxbfcode{linecolor}}{\emph{t=None}}{}
linecolor of an animate object. May be overridden.
\begin{quote}\begin{description}
\item[{Parameters}] \leavevmode
\sphinxstyleliteralstrong{t} (\sphinxstyleliteralemphasis{float}) \textendash{} current time

\item[{Returns}] \leavevmode
\sphinxstylestrong{linecolor} \textendash{} default behaviour: linear interpolation between self.linecolor0 and self.linecolor1

\item[{Return type}] \leavevmode
colorspec

\end{description}\end{quote}

\end{fulllineitems}

\index{linewidth() (salabim.Animate method)}

\begin{fulllineitems}
\phantomsection\label{\detokenize{Reference:salabim.Animate.linewidth}}\pysiglinewithargsret{\sphinxbfcode{linewidth}}{\emph{t=None}}{}
linewidth of an animate object. May be overridden.
\begin{quote}\begin{description}
\item[{Parameters}] \leavevmode
\sphinxstyleliteralstrong{t} (\sphinxstyleliteralemphasis{float}) \textendash{} current time

\item[{Returns}] \leavevmode
\sphinxstylestrong{linewidth} \textendash{} default behaviour: linear interpolation between self.linewidth0 and self.linewidth1

\item[{Return type}] \leavevmode
float

\end{description}\end{quote}

\end{fulllineitems}

\index{max\_lines() (salabim.Animate method)}

\begin{fulllineitems}
\phantomsection\label{\detokenize{Reference:salabim.Animate.max_lines}}\pysiglinewithargsret{\sphinxbfcode{max\_lines}}{\emph{t=None}}{}
maximum number of lines to be displayed of text. May be overridden.
\begin{quote}\begin{description}
\item[{Parameters}] \leavevmode
\sphinxstyleliteralstrong{t} (\sphinxstyleliteralemphasis{float}) \textendash{} current time

\item[{Returns}] \leavevmode
\sphinxstylestrong{max\_lines} \textendash{} default behaviour: self.max\_lines0 (max\_lines given at creation or update)

\item[{Return type}] \leavevmode
int

\end{description}\end{quote}

\end{fulllineitems}

\index{offsetx() (salabim.Animate method)}

\begin{fulllineitems}
\phantomsection\label{\detokenize{Reference:salabim.Animate.offsetx}}\pysiglinewithargsret{\sphinxbfcode{offsetx}}{\emph{t=None}}{}
offsetx of an animate object. May be overridden.
\begin{quote}\begin{description}
\item[{Parameters}] \leavevmode
\sphinxstyleliteralstrong{t} (\sphinxstyleliteralemphasis{float}) \textendash{} current time

\item[{Returns}] \leavevmode
\sphinxstylestrong{offsetx} \textendash{} default behaviour: linear interpolation between self.offsetx0 and self.offsetx1

\item[{Return type}] \leavevmode
float

\end{description}\end{quote}

\end{fulllineitems}

\index{offsety() (salabim.Animate method)}

\begin{fulllineitems}
\phantomsection\label{\detokenize{Reference:salabim.Animate.offsety}}\pysiglinewithargsret{\sphinxbfcode{offsety}}{\emph{t=None}}{}
offsety of an animate object. May be overridden.
\begin{quote}\begin{description}
\item[{Parameters}] \leavevmode
\sphinxstyleliteralstrong{t} (\sphinxstyleliteralemphasis{float}) \textendash{} current time

\item[{Returns}] \leavevmode
\sphinxstylestrong{offsety} \textendash{} default behaviour: linear interpolation between self.offsety0 and self.offsety1

\item[{Return type}] \leavevmode
float

\end{description}\end{quote}

\end{fulllineitems}

\index{points() (salabim.Animate method)}

\begin{fulllineitems}
\phantomsection\label{\detokenize{Reference:salabim.Animate.points}}\pysiglinewithargsret{\sphinxbfcode{points}}{\emph{t=None}}{}
points of an animate object. May be overridden.
\begin{quote}\begin{description}
\item[{Parameters}] \leavevmode
\sphinxstyleliteralstrong{t} (\sphinxstyleliteralemphasis{float}) \textendash{} current time

\item[{Returns}] \leavevmode
\sphinxstylestrong{points} \textendash{} series of x- and y-coordinates (xa,ya,xb,yb,xc,yc, …) 
default behaviour: linear interpolation between self.points0 and self.points1

\item[{Return type}] \leavevmode
tuple

\end{description}\end{quote}

\end{fulllineitems}

\index{polygon() (salabim.Animate method)}

\begin{fulllineitems}
\phantomsection\label{\detokenize{Reference:salabim.Animate.polygon}}\pysiglinewithargsret{\sphinxbfcode{polygon}}{\emph{t=None}}{}
polygon of an animate object. May be overridden.
\begin{quote}\begin{description}
\item[{Parameters}] \leavevmode
\sphinxstyleliteralstrong{t} (\sphinxstyleliteralemphasis{float}) \textendash{} current time

\item[{Returns}] \leavevmode
\sphinxstylestrong{polygon} \textendash{} series of x- and y-coordinates describing the polygon (xa,ya,xb,yb,xc,yc, …) 
default behaviour: linear interpolation between self.polygon0 and self.polygon1

\item[{Return type}] \leavevmode
tuple

\end{description}\end{quote}

\end{fulllineitems}

\index{rectangle() (salabim.Animate method)}

\begin{fulllineitems}
\phantomsection\label{\detokenize{Reference:salabim.Animate.rectangle}}\pysiglinewithargsret{\sphinxbfcode{rectangle}}{\emph{t=None}}{}
rectangle of an animate object. May be overridden.
\begin{quote}\begin{description}
\item[{Parameters}] \leavevmode
\sphinxstyleliteralstrong{t} (\sphinxstyleliteralemphasis{float}) \textendash{} current time

\item[{Returns}] \leavevmode
\sphinxstylestrong{rectangle} \textendash{} (xlowerleft,ylowerlef,xupperright,yupperright) 
default behaviour: linear interpolation between self.rectangle0 and self.rectangle1

\item[{Return type}] \leavevmode
tuple

\end{description}\end{quote}

\end{fulllineitems}

\index{remove() (salabim.Animate method)}

\begin{fulllineitems}
\phantomsection\label{\detokenize{Reference:salabim.Animate.remove}}\pysiglinewithargsret{\sphinxbfcode{remove}}{}{}
removes the animation object from the animation queue,
so effectively ending this animation.

\begin{sphinxadmonition}{note}{Note:}
The animation object might be still updated, if required
\end{sphinxadmonition}

\end{fulllineitems}

\index{text() (salabim.Animate method)}

\begin{fulllineitems}
\phantomsection\label{\detokenize{Reference:salabim.Animate.text}}\pysiglinewithargsret{\sphinxbfcode{text}}{\emph{t=None}}{}
text of an animate object. May be overridden.
\begin{quote}\begin{description}
\item[{Parameters}] \leavevmode
\sphinxstyleliteralstrong{t} (\sphinxstyleliteralemphasis{float}) \textendash{} current time

\item[{Returns}] \leavevmode
\sphinxstylestrong{text} \textendash{} default behaviour: self.text0 (text given at creation or update)

\item[{Return type}] \leavevmode
str

\end{description}\end{quote}

\end{fulllineitems}

\index{text\_anchor() (salabim.Animate method)}

\begin{fulllineitems}
\phantomsection\label{\detokenize{Reference:salabim.Animate.text_anchor}}\pysiglinewithargsret{\sphinxbfcode{text\_anchor}}{\emph{t=None}}{}
text\_anchor of an animate object. May be overridden.
\begin{quote}\begin{description}
\item[{Parameters}] \leavevmode
\sphinxstyleliteralstrong{t} (\sphinxstyleliteralemphasis{float}) \textendash{} current time

\item[{Returns}] \leavevmode
\sphinxstylestrong{text\_anchor} \textendash{} default behaviour: self.text\_anchor0 (text\_anchor given at creation or update)

\item[{Return type}] \leavevmode
str

\end{description}\end{quote}

\end{fulllineitems}

\index{textcolor() (salabim.Animate method)}

\begin{fulllineitems}
\phantomsection\label{\detokenize{Reference:salabim.Animate.textcolor}}\pysiglinewithargsret{\sphinxbfcode{textcolor}}{\emph{t=None}}{}
textcolor of an animate object. May be overridden.
\begin{quote}\begin{description}
\item[{Parameters}] \leavevmode
\sphinxstyleliteralstrong{t} (\sphinxstyleliteralemphasis{float}) \textendash{} current time

\item[{Returns}] \leavevmode
\sphinxstylestrong{textcolor} \textendash{} default behaviour: linear interpolation between self.textcolor0 and self.textcolor1

\item[{Return type}] \leavevmode
colorspec

\end{description}\end{quote}

\end{fulllineitems}

\index{update() (salabim.Animate method)}

\begin{fulllineitems}
\phantomsection\label{\detokenize{Reference:salabim.Animate.update}}\pysiglinewithargsret{\sphinxbfcode{update}}{\emph{layer=None}, \emph{keep=None}, \emph{visible=None}, \emph{t0=None}, \emph{x0=None}, \emph{y0=None}, \emph{offsetx0=None}, \emph{offsety0=None}, \emph{circle0=None}, \emph{line0=None}, \emph{polygon0=None}, \emph{rectangle0=None}, \emph{points0=None}, \emph{image=None}, \emph{text=None}, \emph{font=None}, \emph{anchor=None}, \emph{max\_lines=None}, \emph{text\_anchor=None}, \emph{linewidth0=None}, \emph{fillcolor0=None}, \emph{linecolor0=None}, \emph{textcolor0=None}, \emph{angle0=None}, \emph{fontsize0=None}, \emph{width0=None}, \emph{as\_points=None}, \emph{t1=None}, \emph{x1=None}, \emph{y1=None}, \emph{offsetx1=None}, \emph{offsety1=None}, \emph{circle1=None}, \emph{line1=None}, \emph{polygon1=None}, \emph{rectangle1=None}, \emph{points1=None}, \emph{linewidth1=None}, \emph{fillcolor1=None}, \emph{linecolor1=None}, \emph{textcolor1=None}, \emph{angle1=None}, \emph{fontsize1=None}, \emph{width1=None}, \emph{xy\_anchor=None}}{}
updates an animation object
\begin{quote}\begin{description}
\item[{Parameters}] \leavevmode\begin{itemize}
\item {} 
\sphinxstyleliteralstrong{layer} (\sphinxstyleliteralemphasis{int}) \textendash{} layer value 
lower layer values are on top of higher layer values (default see below)

\item {} 
\sphinxstyleliteralstrong{keep} (\sphinxstyleliteralemphasis{bool}) \textendash{} keep 
if False, animation object is hidden after t1, shown otherwise
(default see below)

\item {} 
\sphinxstyleliteralstrong{visible} (\sphinxstyleliteralemphasis{bool}) \textendash{} visible 
if False, animation object is not shown, shown otherwise
(default see below)

\item {} 
\sphinxstyleliteralstrong{xy\_anchor} (\sphinxstyleliteralemphasis{str}) \textendash{} specifies where x and y (i.e. x0, y0, x1 and y1) are relative to 
possible values are: 
\sphinxcode{nw    n    ne} 
\sphinxcode{w     c     e} 
\sphinxcode{sw    s    se} 
If ‘’, the given coordimates are used untranslated 
default see below

\item {} 
\sphinxstyleliteralstrong{t0} (\sphinxstyleliteralemphasis{float}) \textendash{} time of start of the animation (default: now)

\item {} 
\sphinxstyleliteralstrong{x0} (\sphinxstyleliteralemphasis{float}) \textendash{} x-coordinate of the origin at time t0 (default see below)

\item {} 
\sphinxstyleliteralstrong{y0} (\sphinxstyleliteralemphasis{float}) \textendash{} y-coordinate of the origin at time t0 (default see below)

\item {} 
\sphinxstyleliteralstrong{offsetx0} (\sphinxstyleliteralemphasis{float}) \textendash{} offsets the x-coordinate of the object at time t0 (default see below)

\item {} 
\sphinxstyleliteralstrong{offsety0} (\sphinxstyleliteralemphasis{float}) \textendash{} offsets the y-coordinate of the object at time t0 (default see below)

\item {} 
\sphinxstyleliteralstrong{circle0} (\sphinxstyleliteralemphasis{float}\sphinxstyleliteralemphasis{ or }\sphinxstyleliteralemphasis{tuple/list}) \textendash{} the circle spec of the circle at time t0 
- radius 
- one item tuple/list containing the radius 
- five items tuple/list cntaining radius, radius1, arc\_angle0, arc\_angle1 and draw\_arc
(see class AnimateCircle for details)

\item {} 
\sphinxstyleliteralstrong{line0} (\sphinxstyleliteralemphasis{tuple}) \textendash{} the line(s) at time t0 (xa,ya,xb,yb,xc,yc, …) (default see below)

\item {} 
\sphinxstyleliteralstrong{polygon0} (\sphinxstyleliteralemphasis{tuple}) \textendash{} the polygon at time t0 (xa,ya,xb,yb,xc,yc, …) 
the last point will be auto connected to the start (default see below)

\item {} 
\sphinxstyleliteralstrong{rectangle0} (\sphinxstyleliteralemphasis{tuple}) \textendash{} the rectangle at time t0 
(xlowerleft,ylowerlef,xupperright,yupperright) (default see below)

\item {} 
\sphinxstyleliteralstrong{points0} (\sphinxstyleliteralemphasis{tuple}) \textendash{} the points(s) at time t0 (xa,ya,xb,yb,xc,yc, …) (default see below)

\item {} 
\sphinxstyleliteralstrong{image} (\sphinxstyleliteralemphasis{str}\sphinxstyleliteralemphasis{ or }\sphinxstyleliteralemphasis{PIL image}) \textendash{} the image to be displayed 
This may be either a filename or a PIL image (default see below)

\item {} 
\sphinxstyleliteralstrong{text} (\sphinxstyleliteralemphasis{str}) \textendash{} the text to be displayed (default see below)

\item {} 
\sphinxstyleliteralstrong{font} (\sphinxstyleliteralemphasis{str}\sphinxstyleliteralemphasis{ or }\sphinxstyleliteralemphasis{list/tuple}) \textendash{} font to be used for texts 
Either a string or a list/tuple of fontnames. (default see below)
If not found, uses calibri or arial

\item {} 
\sphinxstyleliteralstrong{max\_lines} (\sphinxstyleliteralemphasis{int}) \textendash{} the maximum of lines of text to be displayed 
if positive, it refers to the first max\_lines lines 
if negative, it refers to the first -max\_lines lines 
if zero (default), all lines will be displayed

\item {} 
\sphinxstyleliteralstrong{anchor} (\sphinxstyleliteralemphasis{str}) \textendash{} anchor position 
specifies where to put images or texts relative to the anchor
point (default see below) 
possible values are (default: c): 
\sphinxcode{nw    n    ne} 
\sphinxcode{w     c     e} 
\sphinxcode{sw    s    se}

\item {} 
\sphinxstyleliteralstrong{linewidth0} (\sphinxstyleliteralemphasis{float}) \textendash{} linewidth of the contour at time t0 (default see below)

\item {} 
\sphinxstyleliteralstrong{fillcolor0} (\sphinxstyleliteralemphasis{colorspec}) \textendash{} color of interior/text at time t0 (default see below)

\item {} 
\sphinxstyleliteralstrong{linecolor0} (\sphinxstyleliteralemphasis{colorspec}) \textendash{} color of the contour at time t0 (default see below)

\item {} 
\sphinxstyleliteralstrong{angle0} (\sphinxstyleliteralemphasis{float}) \textendash{} angle of the polygon at time t0 (in degrees) (default see below)

\item {} 
\sphinxstyleliteralstrong{fontsize0} (\sphinxstyleliteralemphasis{float}) \textendash{} fontsize of text at time t0 (default see below)

\item {} 
\sphinxstyleliteralstrong{width0} (\sphinxstyleliteralemphasis{float}) \textendash{} width of the image to be displayed at time t0 (default see below) 
if None, the original width of the image will be used

\item {} 
\sphinxstyleliteralstrong{t1} (\sphinxstyleliteralemphasis{float}) \textendash{} time of end of the animation (default: inf) 
if keep=True, the animation will continue (frozen) after t1

\item {} 
\sphinxstyleliteralstrong{x1} (\sphinxstyleliteralemphasis{float}) \textendash{} x-coordinate of the origin at time t1 (default x0)

\item {} 
\sphinxstyleliteralstrong{y1} (\sphinxstyleliteralemphasis{float}) \textendash{} y-coordinate of the origin at time t1 (default y0)

\item {} 
\sphinxstyleliteralstrong{offsetx1} (\sphinxstyleliteralemphasis{float}) \textendash{} offsets the x-coordinate of the object at time t1 (default offsetx0)

\item {} 
\sphinxstyleliteralstrong{offsety1} (\sphinxstyleliteralemphasis{float}) \textendash{} offsets the y-coordinate of the object at time t1 (default offset0)

\item {} 
\sphinxstyleliteralstrong{circle1} (\sphinxstyleliteralemphasis{float}\sphinxstyleliteralemphasis{ or }\sphinxstyleliteralemphasis{tuple/ist}) \textendash{} the circle spec of the circle at time t1 
- radius 
- one item tuple/list containing the radius 
- five items tuple/list cntaining radius, radius1, arc\_angle0, arc\_angle1 and draw\_arc
(see class AnimateCircle for details)

\item {} 
\sphinxstyleliteralstrong{line1} (\sphinxstyleliteralemphasis{tuple}) \textendash{} the line(s) at time t1 (xa,ya,xb,yb,xc,yc, …) (default: line0) 
should have the same number of elements as line0

\item {} 
\sphinxstyleliteralstrong{polygon1} (\sphinxstyleliteralemphasis{tuple}) \textendash{} the polygon at time t1 (xa,ya,xb,yb,xc,yc, …) (default: polygon0) 
should have the same number of elements as polygon0

\item {} 
\sphinxstyleliteralstrong{rectangle1} (\sphinxstyleliteralemphasis{tuple}) \textendash{} the rectangle at time t (xlowerleft,ylowerleft,xupperright,yupperright)
(default: rectangle0) 

\item {} 
\sphinxstyleliteralstrong{points1} (\sphinxstyleliteralemphasis{tuple}) \textendash{} the points(s) at time t1 (xa,ya,xb,yb,xc,yc, …) (default: points0) 
should have the same number of elements as points1

\item {} 
\sphinxstyleliteralstrong{linewidth1} (\sphinxstyleliteralemphasis{float}) \textendash{} linewidth of the contour at time t1 (default linewidth0)

\item {} 
\sphinxstyleliteralstrong{fillcolor1} (\sphinxstyleliteralemphasis{colorspec}) \textendash{} color of interior/text at time t1 (default fillcolor0)

\item {} 
\sphinxstyleliteralstrong{linecolor1} (\sphinxstyleliteralemphasis{colorspec}) \textendash{} color of the contour at time t1 (default linecolor0)

\item {} 
\sphinxstyleliteralstrong{angle1} (\sphinxstyleliteralemphasis{float}) \textendash{} angle of the polygon at time t1 (in degrees) (default angle0)

\item {} 
\sphinxstyleliteralstrong{fontsize1} (\sphinxstyleliteralemphasis{float}) \textendash{} fontsize of text at time t1 (default: fontsize0)

\item {} 
\sphinxstyleliteralstrong{width1} (\sphinxstyleliteralemphasis{float}) \textendash{} width of the image to be displayed at time t1 (default: width0) 

\end{itemize}

\end{description}\end{quote}

\begin{sphinxadmonition}{note}{Note:}
The type of the animation cannot be changed with this method. 
The default value of most of the parameters is the current value (at time now)
\end{sphinxadmonition}

\end{fulllineitems}

\index{visible() (salabim.Animate method)}

\begin{fulllineitems}
\phantomsection\label{\detokenize{Reference:salabim.Animate.visible}}\pysiglinewithargsret{\sphinxbfcode{visible}}{\emph{t=None}}{}
visible attribute of an animate object. May be overridden.
\begin{quote}\begin{description}
\item[{Parameters}] \leavevmode
\sphinxstyleliteralstrong{t} (\sphinxstyleliteralemphasis{float}) \textendash{} current time

\item[{Returns}] \leavevmode
\sphinxstylestrong{visible} \textendash{} default behaviour: self.visible0 (visible given at creation or update)

\item[{Return type}] \leavevmode
bool

\end{description}\end{quote}

\end{fulllineitems}

\index{width() (salabim.Animate method)}

\begin{fulllineitems}
\phantomsection\label{\detokenize{Reference:salabim.Animate.width}}\pysiglinewithargsret{\sphinxbfcode{width}}{\emph{t=None}}{}
width position of an animated image object. May be overridden.
\begin{quote}\begin{description}
\item[{Parameters}] \leavevmode
\sphinxstyleliteralstrong{t} (\sphinxstyleliteralemphasis{float}) \textendash{} current time

\item[{Returns}] \leavevmode
\sphinxstylestrong{width} \textendash{} default behaviour: linear interpolation between self.width0 and self.width1 
if None, the original width of the image will be used

\item[{Return type}] \leavevmode
float

\end{description}\end{quote}

\end{fulllineitems}

\index{x() (salabim.Animate method)}

\begin{fulllineitems}
\phantomsection\label{\detokenize{Reference:salabim.Animate.x}}\pysiglinewithargsret{\sphinxbfcode{x}}{\emph{t=None}}{}
x-position of an animate object. May be overridden.
\begin{quote}\begin{description}
\item[{Parameters}] \leavevmode
\sphinxstyleliteralstrong{t} (\sphinxstyleliteralemphasis{float}) \textendash{} current time

\item[{Returns}] \leavevmode
\sphinxstylestrong{x} \textendash{} default behaviour: linear interpolation between self.x0 and self.x1

\item[{Return type}] \leavevmode
float

\end{description}\end{quote}

\end{fulllineitems}

\index{xy\_anchor() (salabim.Animate method)}

\begin{fulllineitems}
\phantomsection\label{\detokenize{Reference:salabim.Animate.xy_anchor}}\pysiglinewithargsret{\sphinxbfcode{xy\_anchor}}{\emph{t=None}}{}
xy\_anchor attribute of an animate object. May be overridden.
\begin{quote}\begin{description}
\item[{Parameters}] \leavevmode
\sphinxstyleliteralstrong{t} (\sphinxstyleliteralemphasis{float}) \textendash{} current time

\item[{Returns}] \leavevmode
\sphinxstylestrong{xy\_anchor} \textendash{} default behaviour: self.xy\_anchor0 (xy\_anchor given at creation or update)

\item[{Return type}] \leavevmode
str

\end{description}\end{quote}

\end{fulllineitems}

\index{y() (salabim.Animate method)}

\begin{fulllineitems}
\phantomsection\label{\detokenize{Reference:salabim.Animate.y}}\pysiglinewithargsret{\sphinxbfcode{y}}{\emph{t=None}}{}
y-position of an animate object. May be overridden.
\begin{quote}\begin{description}
\item[{Parameters}] \leavevmode
\sphinxstyleliteralstrong{t} (\sphinxstyleliteralemphasis{float}) \textendash{} current time

\item[{Returns}] \leavevmode
\sphinxstylestrong{y} \textendash{} default behaviour: linear interpolation between self.y0 and self.y1

\item[{Return type}] \leavevmode
float

\end{description}\end{quote}

\end{fulllineitems}


\end{fulllineitems}

\index{AnimateButton (class in salabim)}

\begin{fulllineitems}
\phantomsection\label{\detokenize{Reference:salabim.AnimateButton}}\pysiglinewithargsret{\sphinxbfcode{class }\sphinxcode{salabim.}\sphinxbfcode{AnimateButton}}{\emph{x=0}, \emph{y=0}, \emph{width=80}, \emph{height=30}, \emph{linewidth=0}, \emph{fillcolor='fg'}, \emph{linecolor='fg'}, \emph{color='bg'}, \emph{text=''}, \emph{font=''}, \emph{fontsize=15}, \emph{action=None}, \emph{env=None}, \emph{xy\_anchor='sw'}}{}
defines a button
\begin{quote}\begin{description}
\item[{Parameters}] \leavevmode\begin{itemize}
\item {} 
\sphinxstyleliteralstrong{x} (\sphinxstyleliteralemphasis{int}) \textendash{} x-coordinate of centre of the button in screen coordinates (default 0)

\item {} 
\sphinxstyleliteralstrong{y} (\sphinxstyleliteralemphasis{int}) \textendash{} y-coordinate of centre of the button in screen coordinates (default 0)

\item {} 
\sphinxstyleliteralstrong{width} (\sphinxstyleliteralemphasis{int}) \textendash{} width of button in screen coordinates (default 80)

\item {} 
\sphinxstyleliteralstrong{height} (\sphinxstyleliteralemphasis{int}) \textendash{} height of button in screen coordinates (default 30)

\item {} 
\sphinxstyleliteralstrong{linewidth} (\sphinxstyleliteralemphasis{int}) \textendash{} width of contour in screen coordinates (default 0=no contour)

\item {} 
\sphinxstyleliteralstrong{fillcolor} (\sphinxstyleliteralemphasis{colorspec}) \textendash{} color of the interior (foreground\_color)

\item {} 
\sphinxstyleliteralstrong{linecolor} (\sphinxstyleliteralemphasis{colorspec}) \textendash{} color of contour (default foreground\_color)

\item {} 
\sphinxstyleliteralstrong{color} (\sphinxstyleliteralemphasis{colorspec}) \textendash{} color of the text (default background\_color)

\item {} 
\sphinxstyleliteralstrong{text} (\sphinxstyleliteralemphasis{str}\sphinxstyleliteralemphasis{ or }\sphinxstyleliteralemphasis{function}) \textendash{} text of the button (default null string) 
if text is an argumentless function, this will be called each time;
the button is shown/updated

\item {} 
\sphinxstyleliteralstrong{font} (\sphinxstyleliteralemphasis{str}) \textendash{} font of the text (default Helvetica)

\item {} 
\sphinxstyleliteralstrong{fontsize} (\sphinxstyleliteralemphasis{int}) \textendash{} fontsize of the text (default 15)

\item {} 
\sphinxstyleliteralstrong{action} (\sphinxstyleliteralemphasis{function}) \textendash{} action to take when button is pressed 
executed when the button is pressed (default None)
the function should have no arguments 

\item {} 
\sphinxstyleliteralstrong{xy\_anchor} (\sphinxstyleliteralemphasis{str}) \textendash{} specifies where x and y are relative to 
possible values are (default: sw): 
\sphinxcode{nw    n    ne} 
\sphinxcode{w     c     e} 
\sphinxcode{sw    s    se}

\item {} 
\sphinxstyleliteralstrong{env} ({\hyperref[\detokenize{Reference:salabim.Environment}]{\sphinxcrossref{\sphinxstyleliteralemphasis{Environment}}}}) \textendash{} environment where the component is defined 
if omitted, default\_env will be used

\end{itemize}

\end{description}\end{quote}

\begin{sphinxadmonition}{note}{Note:}
All measures are in screen coordinates 
On Pythonista, this functionality is emulated by salabim
On other platforms, the tkinter functionality is used.
\end{sphinxadmonition}
\index{remove() (salabim.AnimateButton method)}

\begin{fulllineitems}
\phantomsection\label{\detokenize{Reference:salabim.AnimateButton.remove}}\pysiglinewithargsret{\sphinxbfcode{remove}}{}{}
removes the button object. 
the ui object is removed from the ui queue,
so effectively ending this ui

\end{fulllineitems}


\end{fulllineitems}

\index{AnimateCircle (class in salabim)}

\begin{fulllineitems}
\phantomsection\label{\detokenize{Reference:salabim.AnimateCircle}}\pysiglinewithargsret{\sphinxbfcode{class }\sphinxcode{salabim.}\sphinxbfcode{AnimateCircle}}{\emph{radius=100}, \emph{radius1=None}, \emph{arc\_angle0=0}, \emph{arc\_angle1=360}, \emph{draw\_arc=False}, \emph{x=0}, \emph{y=0}, \emph{fillcolor='fg'}, \emph{linecolor=''}, \emph{linewidth=1}, \emph{text=''}, \emph{fontsize=15}, \emph{textcolor='bg'}, \emph{font=''}, \emph{angle=0}, \emph{xy\_anchor=''}, \emph{layer=0}, \emph{max\_lines=0}, \emph{offsetx=0}, \emph{offsety=0}, \emph{text\_anchor='c'}, \emph{text\_offsetx=0}, \emph{text\_offsety=0}, \emph{arg=None}, \emph{parent=None}, \emph{visible=True}, \emph{env=None}, \emph{screen\_coordinates=False}}{}
Displays a (partial) circle or (partial) ellipse , optionally with a text
\begin{quote}\begin{description}
\item[{Parameters}] \leavevmode\begin{itemize}
\item {} 
\sphinxstyleliteralstrong{radius} (\sphinxstyleliteralemphasis{float}) \textendash{} radius of the circle

\item {} 
\sphinxstyleliteralstrong{radius1} (\sphinxstyleliteralemphasis{float}) \textendash{} the ‘height of the ellipse. If None (default), a circle will be drawn

\item {} 
\sphinxstyleliteralstrong{arc\_angle0} (\sphinxstyleliteralemphasis{float}) \textendash{} start angle of the circle (default 0)

\item {} 
\sphinxstyleliteralstrong{arc\_angle1} (\sphinxstyleliteralemphasis{float}) \textendash{} end angle of the circle (default 360) 
when arc\_angle1 \textgreater{} arc\_angle0 + 360, only 360 degrees will be shown

\item {} 
\sphinxstyleliteralstrong{draw\_arc} (\sphinxstyleliteralemphasis{bool}) \textendash{} if False (default), no arcs will be drawn
if True, the arcs from and to the center will be drawn

\item {} 
\sphinxstyleliteralstrong{x} (\sphinxstyleliteralemphasis{float}) \textendash{} position of anchor point (default 0)

\item {} 
\sphinxstyleliteralstrong{y} (\sphinxstyleliteralemphasis{float}) \textendash{} position of anchor point (default 0)

\item {} 
\sphinxstyleliteralstrong{xy\_anchor} (\sphinxstyleliteralemphasis{str}) \textendash{} specifies where x and y are relative to 
possible values are (default: sw) : 
\sphinxcode{nw    n    ne} 
\sphinxcode{w     c     e} 
\sphinxcode{sw    s    se} 
If ‘’, the given coordimates are used untranslated 
The positions corresponds to a full circle even if arc\_angle0 and/or arc\_angle1 are specified.

\item {} 
\sphinxstyleliteralstrong{offsetx} (\sphinxstyleliteralemphasis{float}) \textendash{} offsets the x-coordinate of the circle (default 0)

\item {} 
\sphinxstyleliteralstrong{offsety} (\sphinxstyleliteralemphasis{float}) \textendash{} offsets the y-coordinate of the circle (default 0)

\item {} 
\sphinxstyleliteralstrong{linewidth} (\sphinxstyleliteralemphasis{float}) \textendash{} linewidth of the contour 
default 1

\item {} 
\sphinxstyleliteralstrong{fillcolor} (\sphinxstyleliteralemphasis{colorspec}) \textendash{} color of interior (default foreground\_color) 
default transparent

\item {} 
\sphinxstyleliteralstrong{linecolor} (\sphinxstyleliteralemphasis{colorspec}) \textendash{} color of the contour (default transparent)

\item {} 
\sphinxstyleliteralstrong{angle} (\sphinxstyleliteralemphasis{float}) \textendash{} angle of the circle/ellipse and/or text (in degrees) 
default: 0

\item {} 
\sphinxstyleliteralstrong{text} (\sphinxstyleliteralemphasis{str}\sphinxstyleliteralemphasis{, }\sphinxstyleliteralemphasis{tuple}\sphinxstyleliteralemphasis{ or }\sphinxstyleliteralemphasis{list}) \textendash{} the text to be displayed 
if text is str, the text may contain linefeeds, which are shown as individual lines

\item {} 
\sphinxstyleliteralstrong{max\_lines} (\sphinxstyleliteralemphasis{int}) \textendash{} the maximum of lines of text to be displayed 
if positive, it refers to the first max\_lines lines 
if negative, it refers to the last -max\_lines lines 
if zero (default), all lines will be displayed

\item {} 
\sphinxstyleliteralstrong{font} (\sphinxstyleliteralemphasis{str}\sphinxstyleliteralemphasis{ or }\sphinxstyleliteralemphasis{list/tuple}) \textendash{} font to be used for texts 
Either a string or a list/tuple of fontnames.
If not found, uses calibri or arial

\item {} 
\sphinxstyleliteralstrong{text\_anchor} (\sphinxstyleliteralemphasis{str}) \textendash{} anchor position of text\textbar{}n\textbar{}
specifies where to texts relative to the polygon
point 
possible values are (default: c): 
\sphinxcode{nw    n    ne} 
\sphinxcode{w     c     e} 
\sphinxcode{sw    s    se}

\item {} 
\sphinxstyleliteralstrong{textcolor} (\sphinxstyleliteralemphasis{colorspec}) \textendash{} color of the text (default foreground\_color)

\item {} 
\sphinxstyleliteralstrong{textoffsetx} (\sphinxstyleliteralemphasis{float}) \textendash{} extra x offset to the text\_anchor point

\item {} 
\sphinxstyleliteralstrong{textoffsety} (\sphinxstyleliteralemphasis{float}) \textendash{} extra y offset to the text\_anchor point

\item {} 
\sphinxstyleliteralstrong{fontsize} (\sphinxstyleliteralemphasis{float}) \textendash{} fontsize of text (default 15)

\item {} 
\sphinxstyleliteralstrong{arg} (\sphinxstyleliteralemphasis{any}) \textendash{} this is used when a parameter is a function with two parameters, as the first argument or
if a parameter is a method as the instance 
default: self (instance itself)

\item {} 
\sphinxstyleliteralstrong{parent} ({\hyperref[\detokenize{Reference:salabim.Component}]{\sphinxcrossref{\sphinxstyleliteralemphasis{Component}}}}) \textendash{} component where this animation object belongs to (default None) 
if given, the animation object will be removed
automatically upon termination of the parent

\end{itemize}

\end{description}\end{quote}

\begin{sphinxadmonition}{note}{Note:}
All measures are in screen coordinates 

All parameters, apart from queue and arg can be specified as: 
- a scalar, like 10 
- a function with zero arguments, like lambda: title 
- a function with one argument, being the time t, like lambda t: t + 10 
- a function with two parameters, being arg (as given) and the time, like lambda comp, t: comp.state 
- a method instance arg for time t, like self.state, actually leading to arg.state(t) to be called
\end{sphinxadmonition}

\end{fulllineitems}

\index{AnimateEntry (class in salabim)}

\begin{fulllineitems}
\phantomsection\label{\detokenize{Reference:salabim.AnimateEntry}}\pysiglinewithargsret{\sphinxbfcode{class }\sphinxcode{salabim.}\sphinxbfcode{AnimateEntry}}{\emph{x=0}, \emph{y=0}, \emph{number\_of\_chars=20}, \emph{value=''}, \emph{fillcolor='fg'}, \emph{color='bg'}, \emph{text=''}, \emph{action=None}, \emph{env=None}, \emph{xy\_anchor='sw'}}{}
defines a button
\begin{quote}\begin{description}
\item[{Parameters}] \leavevmode\begin{itemize}
\item {} 
\sphinxstyleliteralstrong{x} (\sphinxstyleliteralemphasis{int}) \textendash{} x-coordinate of centre of the button in screen coordinates (default 0)

\item {} 
\sphinxstyleliteralstrong{y} (\sphinxstyleliteralemphasis{int}) \textendash{} y-coordinate of centre of the button in screen coordinates (default 0)

\item {} 
\sphinxstyleliteralstrong{number\_of\_chars} (\sphinxstyleliteralemphasis{int}) \textendash{} number of characters displayed in the entry field (default 20)

\item {} 
\sphinxstyleliteralstrong{fillcolor} (\sphinxstyleliteralemphasis{colorspec}) \textendash{} color of the entry background (default foreground\_color)

\item {} 
\sphinxstyleliteralstrong{color} (\sphinxstyleliteralemphasis{colorspec}) \textendash{} color of the text (default background\_color)

\item {} 
\sphinxstyleliteralstrong{value} (\sphinxstyleliteralemphasis{str}) \textendash{} initial value of the text of the entry (default null string) 

\item {} 
\sphinxstyleliteralstrong{action} (\sphinxstyleliteralemphasis{function}) \textendash{} action to take when the Enter-key is pressed 
the function should have no arguments 

\item {} 
\sphinxstyleliteralstrong{xy\_anchor} (\sphinxstyleliteralemphasis{str}) \textendash{} specifies where x and y are relative to 
possible values are (default: sw): 
\sphinxcode{nw    n    ne} 
\sphinxcode{w     c     e} 
\sphinxcode{sw    s    se}

\item {} 
\sphinxstyleliteralstrong{env} ({\hyperref[\detokenize{Reference:salabim.Environment}]{\sphinxcrossref{\sphinxstyleliteralemphasis{Environment}}}}) \textendash{} environment where the component is defined 
if omitted, default\_env will be used

\end{itemize}

\end{description}\end{quote}

\begin{sphinxadmonition}{note}{Note:}
All measures are in screen coordinates 
This class is not available under Pythonista.
\end{sphinxadmonition}
\index{get() (salabim.AnimateEntry method)}

\begin{fulllineitems}
\phantomsection\label{\detokenize{Reference:salabim.AnimateEntry.get}}\pysiglinewithargsret{\sphinxbfcode{get}}{}{}
get the current value of the entry
\begin{quote}\begin{description}
\item[{Returns}] \leavevmode
\sphinxstylestrong{Current value of the entry}

\item[{Return type}] \leavevmode
str

\end{description}\end{quote}

\end{fulllineitems}

\index{remove() (salabim.AnimateEntry method)}

\begin{fulllineitems}
\phantomsection\label{\detokenize{Reference:salabim.AnimateEntry.remove}}\pysiglinewithargsret{\sphinxbfcode{remove}}{}{}
removes the entry object. 
the ui object is removed from the ui queue,
so effectively ending this ui

\end{fulllineitems}


\end{fulllineitems}

\index{AnimateImage (class in salabim)}

\begin{fulllineitems}
\phantomsection\label{\detokenize{Reference:salabim.AnimateImage}}\pysiglinewithargsret{\sphinxbfcode{class }\sphinxcode{salabim.}\sphinxbfcode{AnimateImage}}{\emph{spec=''}, \emph{x=0}, \emph{y=0}, \emph{width=None}, \emph{text=''}, \emph{fontsize=15}, \emph{textcolor='bg'}, \emph{font=''}, \emph{angle=0}, \emph{xy\_anchor=''}, \emph{layer=0}, \emph{max\_lines=0}, \emph{offsetx=0}, \emph{offsety=0}, \emph{text\_anchor='c'}, \emph{text\_offsetx=0}, \emph{text\_offsety=0}, \emph{arg=None}, \emph{parent=None}, \emph{anchor='sw'}, \emph{visible=True}, \emph{env=None}, \emph{screen\_coordinates=False}}{}
Displays an image, optionally with a text
\begin{quote}\begin{description}
\item[{Parameters}] \leavevmode\begin{itemize}
\item {} 
\sphinxstyleliteralstrong{image} (\sphinxstyleliteralemphasis{str}) \textendash{} image to be displayed 
if used as function or method or in direct assigmnent,
the image should be a PIL image (most likely via spec\_to\_image)

\item {} 
\sphinxstyleliteralstrong{x} (\sphinxstyleliteralemphasis{float}) \textendash{} position of anchor point (default 0)

\item {} 
\sphinxstyleliteralstrong{y} (\sphinxstyleliteralemphasis{float}) \textendash{} position of anchor point (default 0)

\item {} 
\sphinxstyleliteralstrong{xy\_anchor} (\sphinxstyleliteralemphasis{str}) \textendash{} specifies where x and y are relative to 
possible values are (default: sw) : 
\sphinxcode{nw    n    ne} 
\sphinxcode{w     c     e} 
\sphinxcode{sw    s    se} 
If ‘’, the given coordimates are used untranslated

\item {} 
\sphinxstyleliteralstrong{anchor} (\sphinxstyleliteralemphasis{str}) \textendash{} specifies where the x and refer to 
possible values are (default: sw) : 
\sphinxcode{nw    n    ne} 
\sphinxcode{w     c     e} 
\sphinxcode{sw    s    se} 

\item {} 
\sphinxstyleliteralstrong{offsetx} (\sphinxstyleliteralemphasis{float}) \textendash{} offsets the x-coordinate of the circle (default 0)

\item {} 
\sphinxstyleliteralstrong{offsety} (\sphinxstyleliteralemphasis{float}) \textendash{} offsets the y-coordinate of the circle (default 0)

\item {} 
\sphinxstyleliteralstrong{angle} (\sphinxstyleliteralemphasis{float}) \textendash{} angle of the text (in degrees) 
default: 0

\item {} 
\sphinxstyleliteralstrong{text} (\sphinxstyleliteralemphasis{str}\sphinxstyleliteralemphasis{, }\sphinxstyleliteralemphasis{tuple}\sphinxstyleliteralemphasis{ or }\sphinxstyleliteralemphasis{list}) \textendash{} the text to be displayed 
if text is str, the text may contain linefeeds, which are shown as individual lines

\item {} 
\sphinxstyleliteralstrong{max\_lines} (\sphinxstyleliteralemphasis{int}) \textendash{} the maximum of lines of text to be displayed 
if positive, it refers to the first max\_lines lines 
if negative, it refers to the last -max\_lines lines 
if zero (default), all lines will be displayed

\item {} 
\sphinxstyleliteralstrong{font} (\sphinxstyleliteralemphasis{str}\sphinxstyleliteralemphasis{ or }\sphinxstyleliteralemphasis{list/tuple}) \textendash{} font to be used for texts 
Either a string or a list/tuple of fontnames.
If not found, uses calibri or arial

\item {} 
\sphinxstyleliteralstrong{text\_anchor} (\sphinxstyleliteralemphasis{str}) \textendash{} anchor position of text\textbar{}n\textbar{}
specifies where to texts relative to the polygon
point 
possible values are (default: c): 
\sphinxcode{nw    n    ne} 
\sphinxcode{w     c     e} 
\sphinxcode{sw    s    se}

\item {} 
\sphinxstyleliteralstrong{textcolor} (\sphinxstyleliteralemphasis{colorspec}) \textendash{} color of the text (default foreground\_color)

\item {} 
\sphinxstyleliteralstrong{textoffsetx} (\sphinxstyleliteralemphasis{float}) \textendash{} extra x offset to the text\_anchor point

\item {} 
\sphinxstyleliteralstrong{textoffsety} (\sphinxstyleliteralemphasis{float}) \textendash{} extra y offset to the text\_anchor point

\item {} 
\sphinxstyleliteralstrong{fontsize} (\sphinxstyleliteralemphasis{float}) \textendash{} fontsize of text (default 15)

\item {} 
\sphinxstyleliteralstrong{arg} (\sphinxstyleliteralemphasis{any}) \textendash{} this is used when a parameter is a function with two parameters, as the first argument or
if a parameter is a method as the instance 
default: self (instance itself)

\item {} 
\sphinxstyleliteralstrong{parent} ({\hyperref[\detokenize{Reference:salabim.Component}]{\sphinxcrossref{\sphinxstyleliteralemphasis{Component}}}}) \textendash{} component where this animation object belongs to (default None) 
if given, the animation object will be removed
automatically upon termination of the parent

\end{itemize}

\end{description}\end{quote}

\begin{sphinxadmonition}{note}{Note:}
All measures are in screen coordinates 

All parameters, apart from queue and arg can be specified as: 
- a scalar, like 10 
- a function with zero arguments, like lambda: title 
- a function with one argument, being the time t, like lambda t: t + 10 
- a function with two parameters, being arg (as given) and the time, like lambda comp, t: comp.state 
- a method instance arg for time t, like self.state, actually leading to arg.state(t) to be called
\end{sphinxadmonition}

\end{fulllineitems}

\index{AnimateLine (class in salabim)}

\begin{fulllineitems}
\phantomsection\label{\detokenize{Reference:salabim.AnimateLine}}\pysiglinewithargsret{\sphinxbfcode{class }\sphinxcode{salabim.}\sphinxbfcode{AnimateLine}}{\emph{spec=()}, \emph{x=0}, \emph{y=0}, \emph{linecolor='fg'}, \emph{linewidth=1}, \emph{text=''}, \emph{fontsize=15}, \emph{textcolor='fg'}, \emph{font=''}, \emph{angle=0}, \emph{xy\_anchor=''}, \emph{layer=0}, \emph{max\_lines=0}, \emph{offsetx=0}, \emph{offsety=0}, \emph{as\_points=False}, \emph{text\_anchor='c'}, \emph{text\_offsetx=0}, \emph{text\_offsety=0}, \emph{arg=None}, \emph{parent=None}, \emph{visible=True}, \emph{env=None}, \emph{screen\_coordinates=False}}{}
Displays a line, optionally with a text
\begin{quote}\begin{description}
\item[{Parameters}] \leavevmode\begin{itemize}
\item {} 
\sphinxstyleliteralstrong{spec} (\sphinxstyleliteralemphasis{tuple}\sphinxstyleliteralemphasis{ or }\sphinxstyleliteralemphasis{list}) \textendash{} should specify x0, y0, x1, y1, …

\item {} 
\sphinxstyleliteralstrong{x} (\sphinxstyleliteralemphasis{float}) \textendash{} position of anchor point (default 0)

\item {} 
\sphinxstyleliteralstrong{y} (\sphinxstyleliteralemphasis{float}) \textendash{} position of anchor point (default 0)

\item {} 
\sphinxstyleliteralstrong{xy\_anchor} (\sphinxstyleliteralemphasis{str}) \textendash{} specifies where x and y are relative to 
possible values are (default: sw) : 
\sphinxcode{nw    n    ne} 
\sphinxcode{w     c     e} 
\sphinxcode{sw    s    se} 
If ‘’, the given coordimates are used untranslated

\item {} 
\sphinxstyleliteralstrong{offsetx} (\sphinxstyleliteralemphasis{float}) \textendash{} offsets the x-coordinate of the line (default 0)

\item {} 
\sphinxstyleliteralstrong{offsety} (\sphinxstyleliteralemphasis{float}) \textendash{} offsets the y-coordinate of the line (default 0)

\item {} 
\sphinxstyleliteralstrong{linewidth} (\sphinxstyleliteralemphasis{float}) \textendash{} linewidth of the contour 
default 1

\item {} 
\sphinxstyleliteralstrong{linecolor} (\sphinxstyleliteralemphasis{colorspec}) \textendash{} color of the contour (default foreground\_color)

\item {} 
\sphinxstyleliteralstrong{angle} (\sphinxstyleliteralemphasis{float}) \textendash{} angle of the line (in degrees) 
default: 0

\item {} 
\sphinxstyleliteralstrong{as\_points} (\sphinxstyleliteralemphasis{bool}) \textendash{} if False (default), the contour lines are drawn 
if True, only the corner points are shown

\item {} 
\sphinxstyleliteralstrong{text} (\sphinxstyleliteralemphasis{str}\sphinxstyleliteralemphasis{, }\sphinxstyleliteralemphasis{tuple}\sphinxstyleliteralemphasis{ or }\sphinxstyleliteralemphasis{list}) \textendash{} the text to be displayed 
if text is str, the text may contain linefeeds, which are shown as individual lines

\item {} 
\sphinxstyleliteralstrong{max\_lines} (\sphinxstyleliteralemphasis{int}) \textendash{} the maximum of lines of text to be displayed 
if positive, it refers to the first max\_lines lines 
if negative, it refers to the last -max\_lines lines 
if zero (default), all lines will be displayed

\item {} 
\sphinxstyleliteralstrong{font} (\sphinxstyleliteralemphasis{str}\sphinxstyleliteralemphasis{ or }\sphinxstyleliteralemphasis{list/tuple}) \textendash{} font to be used for texts 
Either a string or a list/tuple of fontnames.
If not found, uses calibri or arial

\item {} 
\sphinxstyleliteralstrong{text\_anchor} (\sphinxstyleliteralemphasis{str}) \textendash{} anchor position of text\textbar{}n\textbar{}
specifies where to texts relative to the polygon
point 
possible values are (default: c): 
\sphinxcode{nw    n    ne} 
\sphinxcode{w     c     e} 
\sphinxcode{sw    s    se}

\item {} 
\sphinxstyleliteralstrong{textcolor} (\sphinxstyleliteralemphasis{colorspec}) \textendash{} color of the text (default foreground\_color)

\item {} 
\sphinxstyleliteralstrong{textoffsetx} (\sphinxstyleliteralemphasis{float}) \textendash{} extra x offset to the text\_anchor point

\item {} 
\sphinxstyleliteralstrong{textoffsety} (\sphinxstyleliteralemphasis{float}) \textendash{} extra y offset to the text\_anchor point

\item {} 
\sphinxstyleliteralstrong{fontsize} (\sphinxstyleliteralemphasis{float}) \textendash{} fontsize of text (default 15)

\item {} 
\sphinxstyleliteralstrong{arg} (\sphinxstyleliteralemphasis{any}) \textendash{} this is used when a parameter is a function with two parameters, as the first argument or
if a parameter is a method as the instance 
default: self (instance itself)

\item {} 
\sphinxstyleliteralstrong{parent} ({\hyperref[\detokenize{Reference:salabim.Component}]{\sphinxcrossref{\sphinxstyleliteralemphasis{Component}}}}) \textendash{} component where this animation object belongs to (default None) 
if given, the animation object will be removed
automatically upon termination of the parent

\end{itemize}

\end{description}\end{quote}

\begin{sphinxadmonition}{note}{Note:}
All measures are in screen coordinates 

All parameters, apart from queue and arg can be specified as: 
- a scalar, like 10 
- a function with zero arguments, like lambda: title 
- a function with one argument, being the time t, like lambda t: t + 10 
- a function with two parameters, being arg (as given) and the time, like lambda comp, t: comp.state 
- a method instance arg for time t, like self.state, actually leading to arg.state(t) to be called
\end{sphinxadmonition}

\end{fulllineitems}

\index{AnimateMonitor (class in salabim)}

\begin{fulllineitems}
\phantomsection\label{\detokenize{Reference:salabim.AnimateMonitor}}\pysiglinewithargsret{\sphinxbfcode{class }\sphinxcode{salabim.}\sphinxbfcode{AnimateMonitor}}{\emph{monitor}, \emph{linecolor='fg'}, \emph{linewidth=None}, \emph{fillcolor=''}, \emph{bordercolor='fg'}, \emph{borderlinewidth=1}, \emph{titlecolor='fg'}, \emph{nowcolor='red'}, \emph{titlefont=''}, \emph{titlefontsize=15}, \emph{title=None}, \emph{x=0}, \emph{y=0}, \emph{vertical\_offset=2}, \emph{parent=None}, \emph{vertical\_scale=5}, \emph{horizontal\_scale=None}, \emph{width=200}, \emph{height=75}, \emph{xy\_anchor='sw'}, \emph{layer=0}}{}
animates a monitor in a panel
\begin{quote}\begin{description}
\item[{Parameters}] \leavevmode\begin{itemize}
\item {} 
\sphinxstyleliteralstrong{linecolor} (\sphinxstyleliteralemphasis{colorspec}) \textendash{} color of the line or points (default foreground color)

\item {} 
\sphinxstyleliteralstrong{linewidth} (\sphinxstyleliteralemphasis{int}) \textendash{} width of the line or points (default 1 for line, 3 for points)

\item {} 
\sphinxstyleliteralstrong{fillcolor} (\sphinxstyleliteralemphasis{colorspec}) \textendash{} color of the panel (default transparent)

\item {} 
\sphinxstyleliteralstrong{bordercolor} (\sphinxstyleliteralemphasis{colorspec}) \textendash{} color of the border (default foreground color)

\item {} 
\sphinxstyleliteralstrong{borderlinewidth} (\sphinxstyleliteralemphasis{int}) \textendash{} width of the line around the panel (default 1)

\item {} 
\sphinxstyleliteralstrong{nowcolor} (\sphinxstyleliteralemphasis{colorspec}) \textendash{} color of the line indicating now (default red)

\item {} 
\sphinxstyleliteralstrong{titlecolor} (\sphinxstyleliteralemphasis{colorspec}) \textendash{} color of the title (default foreground color)

\item {} 
\sphinxstyleliteralstrong{titlefont} ({\hyperref[\detokenize{Reference:salabim.Animate.font}]{\sphinxcrossref{\sphinxstyleliteralemphasis{font}}}}) \textendash{} font of the title (default ‘’)

\item {} 
\sphinxstyleliteralstrong{titlefontsize} (\sphinxstyleliteralemphasis{int}) \textendash{} size of the font of the title (default 15)

\item {} 
\sphinxstyleliteralstrong{title} (\sphinxstyleliteralemphasis{str}) \textendash{} title to be shown above panel 
default: name of the monitor

\item {} 
\sphinxstyleliteralstrong{x} (\sphinxstyleliteralemphasis{int}) \textendash{} x-coordinate of panel, relative to xy\_anchor, default 0

\item {} 
\sphinxstyleliteralstrong{y} (\sphinxstyleliteralemphasis{int}) \textendash{} y-coordinate of panel, relative to xy\_anchor. default 0

\item {} 
\sphinxstyleliteralstrong{xy\_anchor} (\sphinxstyleliteralemphasis{str}) \textendash{} specifies where x and y are relative to 
possible values are (default: sw): 
\sphinxcode{nw    n    ne} 
\sphinxcode{w     c     e} 
\sphinxcode{sw    s    se}

\item {} 
\sphinxstyleliteralstrong{vertical\_offset} (\sphinxstyleliteralemphasis{float}) \textendash{} \begin{description}
\item[{the vertical position of x within the panel is}] \leavevmode
vertical\_offset + x * vertical\_scale (default 0)

\end{description}


\item {} 
\sphinxstyleliteralstrong{vertical\_scale} (\sphinxstyleliteralemphasis{float}) \textendash{} the vertical position of x within the panel is
vertical\_offset + x * vertical\_scale (default 5)

\item {} 
\sphinxstyleliteralstrong{horizontal\_scale} (\sphinxstyleliteralemphasis{float}) \textendash{} for timescaled monitors the relative horizontal position of time t within the panel is on
t * horizontal\_scale, possibly shifted (default 1)\textbar{}n\textbar{}
for non timescaled monitors, the relative horizontal position of index i within the panel is on
i * horizontal\_scale, possibly shifted (default 5)\textbar{}n\textbar{}

\item {} 
\sphinxstyleliteralstrong{width} (\sphinxstyleliteralemphasis{int}) \textendash{} width of the panel (default 200)

\item {} 
\sphinxstyleliteralstrong{height} (\sphinxstyleliteralemphasis{int}) \textendash{} height of the panel (default 75)

\item {} 
\sphinxstyleliteralstrong{layer} (\sphinxstyleliteralemphasis{int}) \textendash{} layer (default 0)

\item {} 
\sphinxstyleliteralstrong{parent} ({\hyperref[\detokenize{Reference:salabim.Component}]{\sphinxcrossref{\sphinxstyleliteralemphasis{Component}}}}) \textendash{} component where this animation object belongs to (default None) 
if given, the animation object will be removed
automatically upon termination of the parent

\end{itemize}

\end{description}\end{quote}

\begin{sphinxadmonition}{note}{Note:}
All measures are in screen coordinates 
\end{sphinxadmonition}
\index{remove() (salabim.AnimateMonitor method)}

\begin{fulllineitems}
\phantomsection\label{\detokenize{Reference:salabim.AnimateMonitor.remove}}\pysiglinewithargsret{\sphinxbfcode{remove}}{}{}
removes the animate object and thus closes this animation

\end{fulllineitems}


\end{fulllineitems}

\index{AnimatePoints (class in salabim)}

\begin{fulllineitems}
\phantomsection\label{\detokenize{Reference:salabim.AnimatePoints}}\pysiglinewithargsret{\sphinxbfcode{class }\sphinxcode{salabim.}\sphinxbfcode{AnimatePoints}}{\emph{spec=()}, \emph{x=0}, \emph{y=0}, \emph{linecolor='fg'}, \emph{linewidth=4}, \emph{text=''}, \emph{fontsize=15}, \emph{textcolor='fg'}, \emph{font=''}, \emph{angle=0}, \emph{xy\_anchor=''}, \emph{layer=0}, \emph{max\_lines=0}, \emph{offsetx=0}, \emph{offsety=0}, \emph{text\_anchor='c'}, \emph{text\_offsetx=0}, \emph{text\_offsety=0}, \emph{arg=None}, \emph{parent=None}, \emph{visible=True}, \emph{env=None}, \emph{screen\_coordinates=False}}{}
Displays a series of points, optionally with a text
\begin{quote}\begin{description}
\item[{Parameters}] \leavevmode\begin{itemize}
\item {} 
\sphinxstyleliteralstrong{spec} (\sphinxstyleliteralemphasis{tuple}\sphinxstyleliteralemphasis{ or }\sphinxstyleliteralemphasis{list}) \textendash{} should specify x0, y0, x1, y1, …

\item {} 
\sphinxstyleliteralstrong{x} (\sphinxstyleliteralemphasis{float}) \textendash{} position of anchor point (default 0)

\item {} 
\sphinxstyleliteralstrong{y} (\sphinxstyleliteralemphasis{float}) \textendash{} position of anchor point (default 0)

\item {} 
\sphinxstyleliteralstrong{xy\_anchor} (\sphinxstyleliteralemphasis{str}) \textendash{} specifies where x and y are relative to 
possible values are (default: sw) : 
\sphinxcode{nw    n    ne} 
\sphinxcode{w     c     e} 
\sphinxcode{sw    s    se} 
If ‘’, the given coordimates are used untranslated

\item {} 
\sphinxstyleliteralstrong{offsetx} (\sphinxstyleliteralemphasis{float}) \textendash{} offsets the x-coordinate of the points (default 0)

\item {} 
\sphinxstyleliteralstrong{offsety} (\sphinxstyleliteralemphasis{float}) \textendash{} offsets the y-coordinate of the points (default 0)

\item {} 
\sphinxstyleliteralstrong{linewidth} (\sphinxstyleliteralemphasis{float}) \textendash{} width of the points 
default 1

\item {} 
\sphinxstyleliteralstrong{linecolor} (\sphinxstyleliteralemphasis{colorspec}) \textendash{} color of the points (default foreground\_color)

\item {} 
\sphinxstyleliteralstrong{angle} (\sphinxstyleliteralemphasis{float}) \textendash{} angle of the points (in degrees) 
default: 0

\item {} 
\sphinxstyleliteralstrong{as\_points} (\sphinxstyleliteralemphasis{bool}) \textendash{} if False (default), the contour lines are drawn 
if True, only the corner points are shown

\item {} 
\sphinxstyleliteralstrong{text} (\sphinxstyleliteralemphasis{str}\sphinxstyleliteralemphasis{, }\sphinxstyleliteralemphasis{tuple}\sphinxstyleliteralemphasis{ or }\sphinxstyleliteralemphasis{list}) \textendash{} the text to be displayed 
if text is str, the text may contain linefeeds, which are shown as individual lines

\item {} 
\sphinxstyleliteralstrong{max\_lines} (\sphinxstyleliteralemphasis{int}) \textendash{} the maximum of lines of text to be displayed 
if positive, it refers to the first max\_lines lines 
if negative, it refers to the last -max\_lines lines 
if zero (default), all lines will be displayed

\item {} 
\sphinxstyleliteralstrong{font} (\sphinxstyleliteralemphasis{str}\sphinxstyleliteralemphasis{ or }\sphinxstyleliteralemphasis{list/tuple}) \textendash{} font to be used for texts 
Either a string or a list/tuple of fontnames.
If not found, uses calibri or arial

\item {} 
\sphinxstyleliteralstrong{text\_anchor} (\sphinxstyleliteralemphasis{str}) \textendash{} anchor position of text\textbar{}n\textbar{}
specifies where to texts relative to the polygon
point 
possible values are (default: c): 
\sphinxcode{nw    n    ne} 
\sphinxcode{w     c     e} 
\sphinxcode{sw    s    se}

\item {} 
\sphinxstyleliteralstrong{textcolor} (\sphinxstyleliteralemphasis{colorspec}) \textendash{} color of the text (default foreground\_color)

\item {} 
\sphinxstyleliteralstrong{textoffsetx} (\sphinxstyleliteralemphasis{float}) \textendash{} extra x offset to the text\_anchor point

\item {} 
\sphinxstyleliteralstrong{textoffsety} (\sphinxstyleliteralemphasis{float}) \textendash{} extra y offset to the text\_anchor point

\item {} 
\sphinxstyleliteralstrong{fontsize} (\sphinxstyleliteralemphasis{float}) \textendash{} fontsize of text (default 15)

\item {} 
\sphinxstyleliteralstrong{arg} (\sphinxstyleliteralemphasis{any}) \textendash{} this is used when a parameter is a function with two parameters, as the first argument or
if a parameter is a method as the instance 
default: self (instance itself)

\item {} 
\sphinxstyleliteralstrong{parent} ({\hyperref[\detokenize{Reference:salabim.Component}]{\sphinxcrossref{\sphinxstyleliteralemphasis{Component}}}}) \textendash{} component where this animation object belongs to (default None) 
if given, the animation object will be removed
automatically upon termination of the parent

\end{itemize}

\end{description}\end{quote}

\begin{sphinxadmonition}{note}{Note:}
All measures are in screen coordinates 

All parameters, apart from queue and arg can be specified as: 
- a scalar, like 10 
- a function with zero arguments, like lambda: title 
- a function with one argument, being the time t, like lambda t: t + 10 
- a function with two parameters, being arg (as given) and the time, like lambda comp, t: comp.state 
- a method instance arg for time t, like self.state, actually leading to arg.state(t) to be called
\end{sphinxadmonition}

\end{fulllineitems}

\index{AnimatePolygon (class in salabim)}

\begin{fulllineitems}
\phantomsection\label{\detokenize{Reference:salabim.AnimatePolygon}}\pysiglinewithargsret{\sphinxbfcode{class }\sphinxcode{salabim.}\sphinxbfcode{AnimatePolygon}}{\emph{spec=()}, \emph{x=0}, \emph{y=0}, \emph{fillcolor='fg'}, \emph{linecolor=''}, \emph{linewidth=1}, \emph{text=''}, \emph{fontsize=15}, \emph{textcolor='bg'}, \emph{font=''}, \emph{angle=0}, \emph{xy\_anchor=''}, \emph{layer=0}, \emph{max\_lines=0}, \emph{offsetx=0}, \emph{offsety=0}, \emph{as\_points=False}, \emph{text\_anchor='c'}, \emph{text\_offsetx=0}, \emph{text\_offsety=0}, \emph{arg=None}, \emph{parent=None}, \emph{visible=True}, \emph{env=None}, \emph{screen\_coordinates=False}}{}
Displays a polygon, optionally with a text
\begin{quote}\begin{description}
\item[{Parameters}] \leavevmode\begin{itemize}
\item {} 
\sphinxstyleliteralstrong{spec} (\sphinxstyleliteralemphasis{tuple}\sphinxstyleliteralemphasis{ or }\sphinxstyleliteralemphasis{list}) \textendash{} should specify x0, y0, x1, y1, …

\item {} 
\sphinxstyleliteralstrong{x} (\sphinxstyleliteralemphasis{float}) \textendash{} position of anchor point (default 0)

\item {} 
\sphinxstyleliteralstrong{y} (\sphinxstyleliteralemphasis{float}) \textendash{} position of anchor point (default 0)

\item {} 
\sphinxstyleliteralstrong{xy\_anchor} (\sphinxstyleliteralemphasis{str}) \textendash{} specifies where x and y are relative to 
possible values are (default: sw) : 
\sphinxcode{nw    n    ne} 
\sphinxcode{w     c     e} 
\sphinxcode{sw    s    se} 
If ‘’, the given coordimates are used untranslated

\item {} 
\sphinxstyleliteralstrong{offsetx} (\sphinxstyleliteralemphasis{float}) \textendash{} offsets the x-coordinate of the polygon (default 0)

\item {} 
\sphinxstyleliteralstrong{offsety} (\sphinxstyleliteralemphasis{float}) \textendash{} offsets the y-coordinate of the polygon (default 0)

\item {} 
\sphinxstyleliteralstrong{linewidth} (\sphinxstyleliteralemphasis{float}) \textendash{} linewidth of the contour 
default 1

\item {} 
\sphinxstyleliteralstrong{fillcolor} (\sphinxstyleliteralemphasis{colorspec}) \textendash{} color of interior (default foreground\_color) 
default transparent

\item {} 
\sphinxstyleliteralstrong{linecolor} (\sphinxstyleliteralemphasis{colorspec}) \textendash{} color of the contour (default transparent)

\item {} 
\sphinxstyleliteralstrong{angle} (\sphinxstyleliteralemphasis{float}) \textendash{} angle of the polygon (in degrees) 
default: 0

\item {} 
\sphinxstyleliteralstrong{as\_points} (\sphinxstyleliteralemphasis{bool}) \textendash{} if False (default), the contour lines are drawn 
if True, only the corner points are shown

\item {} 
\sphinxstyleliteralstrong{text} (\sphinxstyleliteralemphasis{str}\sphinxstyleliteralemphasis{, }\sphinxstyleliteralemphasis{tuple}\sphinxstyleliteralemphasis{ or }\sphinxstyleliteralemphasis{list}) \textendash{} the text to be displayed 
if text is str, the text may contain linefeeds, which are shown as individual lines

\item {} 
\sphinxstyleliteralstrong{max\_lines} (\sphinxstyleliteralemphasis{int}) \textendash{} the maximum of lines of text to be displayed 
if positive, it refers to the first max\_lines lines 
if negative, it refers to the last -max\_lines lines 
if zero (default), all lines will be displayed

\item {} 
\sphinxstyleliteralstrong{font} (\sphinxstyleliteralemphasis{str}\sphinxstyleliteralemphasis{ or }\sphinxstyleliteralemphasis{list/tuple}) \textendash{} font to be used for texts 
Either a string or a list/tuple of fontnames.
If not found, uses calibri or arial

\item {} 
\sphinxstyleliteralstrong{text\_anchor} (\sphinxstyleliteralemphasis{str}) \textendash{} anchor position of text\textbar{}n\textbar{}
specifies where to texts relative to the polygon
point 
possible values are (default: c): 
\sphinxcode{nw    n    ne} 
\sphinxcode{w     c     e} 
\sphinxcode{sw    s    se}

\item {} 
\sphinxstyleliteralstrong{textcolor} (\sphinxstyleliteralemphasis{colorspec}) \textendash{} color of the text (default foreground\_color)

\item {} 
\sphinxstyleliteralstrong{textoffsetx} (\sphinxstyleliteralemphasis{float}) \textendash{} extra x offset to the text\_anchor point

\item {} 
\sphinxstyleliteralstrong{textoffsety} (\sphinxstyleliteralemphasis{float}) \textendash{} extra y offset to the text\_anchor point

\item {} 
\sphinxstyleliteralstrong{fontsize} (\sphinxstyleliteralemphasis{float}) \textendash{} fontsize of text (default 15)

\item {} 
\sphinxstyleliteralstrong{arg} (\sphinxstyleliteralemphasis{any}) \textendash{} this is used when a parameter is a function with two parameters, as the first argument or
if a parameter is a method as the instance 
default: self (instance itself)

\item {} 
\sphinxstyleliteralstrong{parent} ({\hyperref[\detokenize{Reference:salabim.Component}]{\sphinxcrossref{\sphinxstyleliteralemphasis{Component}}}}) \textendash{} component where this animation object belongs to (default None) 
if given, the animation object will be removed
automatically upon termination of the parent

\end{itemize}

\end{description}\end{quote}

\begin{sphinxadmonition}{note}{Note:}
All measures are in screen coordinates 

All parameters, apart from queue and arg can be specified as: 
- a scalar, like 10 
- a function with zero arguments, like lambda: title 
- a function with one argument, being the time t, like lambda t: t + 10 
- a function with two parameters, being arg (as given) and the time, like lambda comp, t: comp.state 
- a method instance arg for time t, like self.state, actually leading to arg.state(t) to be called
\end{sphinxadmonition}

\end{fulllineitems}

\index{AnimateQueue (class in salabim)}

\begin{fulllineitems}
\phantomsection\label{\detokenize{Reference:salabim.AnimateQueue}}\pysiglinewithargsret{\sphinxbfcode{class }\sphinxcode{salabim.}\sphinxbfcode{AnimateQueue}}{\emph{queue}, \emph{x=50}, \emph{y=50}, \emph{direction='w'}, \emph{max\_length=None}, \emph{xy\_anchor='sw'}, \emph{reverse=False}, \emph{title=None}, \emph{titlecolor='fg'}, \emph{titlefontsize=15}, \emph{titlefont=''}, \emph{titleoffsetx=None}, \emph{titleoffsety=None}, \emph{layer=0}, \emph{id=None}, \emph{arg=None}, \emph{parent=None}}{}
Animates the component in a queue.
\begin{quote}\begin{description}
\item[{Parameters}] \leavevmode\begin{itemize}
\item {} 
\sphinxstyleliteralstrong{queue} ({\hyperref[\detokenize{Reference:salabim.Queue}]{\sphinxcrossref{\sphinxstyleliteralemphasis{Queue}}}}) \textendash{} 

\item {} 
\sphinxstyleliteralstrong{x} (\sphinxstyleliteralemphasis{float}) \textendash{} x-position of the first component in the queue 
default: 50

\item {} 
\sphinxstyleliteralstrong{y} (\sphinxstyleliteralemphasis{float}) \textendash{} y-position of the first component in the queue 
default: 50

\item {} 
\sphinxstyleliteralstrong{direction} (\sphinxstyleliteralemphasis{str}) \textendash{} if ‘w’, waiting line runs westwards (i.e. from right to left) 
if ‘n’, waiting line runs northeards (i.e. from bottom to top) 
if ‘e’, waiting line runs eastwards (i.e. from left to right) (default) 
if ‘s’, waiting line runs southwards (i.e. from top to bottom)

\item {} 
\sphinxstyleliteralstrong{reverse} (\sphinxstyleliteralemphasis{bool}) \textendash{} if False (default), display in normal order. If True, reversed.

\item {} 
\sphinxstyleliteralstrong{max\_length} (\sphinxstyleliteralemphasis{int}) \textendash{} maximum number of components to be displayed

\item {} 
\sphinxstyleliteralstrong{xy\_anchor} (\sphinxstyleliteralemphasis{str}) \textendash{} specifies where x and y are relative to 
possible values are (default: sw): 
\sphinxcode{nw    n    ne} 
\sphinxcode{w     c     e} 
\sphinxcode{sw    s    se}

\item {} 
\sphinxstyleliteralstrong{titlecolor} (\sphinxstyleliteralemphasis{colorspec}) \textendash{} color of the title (default foreground color)

\item {} 
\sphinxstyleliteralstrong{titlefont} ({\hyperref[\detokenize{Reference:salabim.Animate.font}]{\sphinxcrossref{\sphinxstyleliteralemphasis{font}}}}) \textendash{} font of the title (default ‘’)

\item {} 
\sphinxstyleliteralstrong{titlefontsize} (\sphinxstyleliteralemphasis{int}) \textendash{} size of the font of the title (default 15)

\item {} 
\sphinxstyleliteralstrong{title} (\sphinxstyleliteralemphasis{str}) \textendash{} title to be shown above queue 
default: name of the queue

\item {} 
\sphinxstyleliteralstrong{titleoffsetx} (\sphinxstyleliteralemphasis{float}) \textendash{} x-offset of the title relative to the start of the queue 
default: 25 if direction is w, -25 otherwise

\item {} 
\sphinxstyleliteralstrong{titleoffsety} (\sphinxstyleliteralemphasis{float}) \textendash{} y-offset of the title relative to the start of the queue 
default: -25 if direction is s, -25 otherwise

\item {} 
\sphinxstyleliteralstrong{layer} (\sphinxstyleliteralemphasis{int}) \textendash{} layer (default 0)

\item {} 
\sphinxstyleliteralstrong{id} (\sphinxstyleliteralemphasis{any}) \textendash{} the animation works by calling the animation\_objects method of each component, optionally
with id. By default, this is self, but can be overriden, particularly with the queue

\item {} 
\sphinxstyleliteralstrong{arg} (\sphinxstyleliteralemphasis{any}) \textendash{} this is used when a parameter is a function with two parameters, as the first argument or
if a parameter is a method as the instance 
default: self (instance itself)

\item {} 
\sphinxstyleliteralstrong{parent} ({\hyperref[\detokenize{Reference:salabim.Component}]{\sphinxcrossref{\sphinxstyleliteralemphasis{Component}}}}) \textendash{} component where this animation object belongs to (default None) 
if given, the animation object will be removed
automatically upon termination of the parent

\end{itemize}

\end{description}\end{quote}

\begin{sphinxadmonition}{note}{Note:}
All measures are in screen coordinates 

All parameters, apart from queue, id, arg and parent can be specified as: 
- a scalar, like 10 
- a function with zero arguments, like lambda: title 
- a function with one argument, being the time t, like lambda t: t + 10 
- a function with two parameters, being arg (as given) and the time, like lambda comp, t: comp.state 
- a method instance arg for time t, like self.state, actually leading to arg.state(t) to be called
\end{sphinxadmonition}

\end{fulllineitems}

\index{AnimateRectangle (class in salabim)}

\begin{fulllineitems}
\phantomsection\label{\detokenize{Reference:salabim.AnimateRectangle}}\pysiglinewithargsret{\sphinxbfcode{class }\sphinxcode{salabim.}\sphinxbfcode{AnimateRectangle}}{\emph{spec=()}, \emph{x=0}, \emph{y=0}, \emph{fillcolor='fg'}, \emph{linecolor=''}, \emph{linewidth=1}, \emph{text=''}, \emph{fontsize=15}, \emph{textcolor='bg'}, \emph{font=''}, \emph{angle=0}, \emph{xy\_anchor=''}, \emph{layer=0}, \emph{max\_lines=0}, \emph{offsetx=0}, \emph{offsety=0}, \emph{as\_points=False}, \emph{text\_anchor='c'}, \emph{text\_offsetx=0}, \emph{text\_offsety=0}, \emph{arg=None}, \emph{parent=None}, \emph{visible=True}, \emph{env=None}, \emph{screen\_coordinates=False}}{}
Displays a rectangle, optionally with a text
\begin{quote}\begin{description}
\item[{Parameters}] \leavevmode\begin{itemize}
\item {} 
\sphinxstyleliteralstrong{spec} (\sphinxstyleliteralemphasis{four item tuple}\sphinxstyleliteralemphasis{ or }\sphinxstyleliteralemphasis{list}) \textendash{} should specify xlowerleft, ylowerleft, xupperright, yupperright

\item {} 
\sphinxstyleliteralstrong{x} (\sphinxstyleliteralemphasis{float}) \textendash{} position of anchor point (default 0)

\item {} 
\sphinxstyleliteralstrong{y} (\sphinxstyleliteralemphasis{float}) \textendash{} position of anchor point (default 0)

\item {} 
\sphinxstyleliteralstrong{xy\_anchor} (\sphinxstyleliteralemphasis{str}) \textendash{} specifies where x and y are relative to 
possible values are (default: sw) : 
\sphinxcode{nw    n    ne} 
\sphinxcode{w     c     e} 
\sphinxcode{sw    s    se} 
If ‘’, the given coordimates are used untranslated

\item {} 
\sphinxstyleliteralstrong{offsetx} (\sphinxstyleliteralemphasis{float}) \textendash{} offsets the x-coordinate of the rectangle (default 0)

\item {} 
\sphinxstyleliteralstrong{offsety} (\sphinxstyleliteralemphasis{float}) \textendash{} offsets the y-coordinate of the rectangle (default 0)

\item {} 
\sphinxstyleliteralstrong{linewidth} (\sphinxstyleliteralemphasis{float}) \textendash{} linewidth of the contour 
default 1

\item {} 
\sphinxstyleliteralstrong{fillcolor} (\sphinxstyleliteralemphasis{colorspec}) \textendash{} color of interior (default foreground\_color) 
default transparent

\item {} 
\sphinxstyleliteralstrong{linecolor} (\sphinxstyleliteralemphasis{colorspec}) \textendash{} color of the contour (default transparent)

\item {} 
\sphinxstyleliteralstrong{angle} (\sphinxstyleliteralemphasis{float}) \textendash{} angle of the rectangle (in degrees) 
default: 0

\item {} 
\sphinxstyleliteralstrong{as\_points} (\sphinxstyleliteralemphasis{bool}) \textendash{} if False (default), the contour lines are drawn 
if True, only the corner points are shown

\item {} 
\sphinxstyleliteralstrong{text} (\sphinxstyleliteralemphasis{str}\sphinxstyleliteralemphasis{, }\sphinxstyleliteralemphasis{tuple}\sphinxstyleliteralemphasis{ or }\sphinxstyleliteralemphasis{list}) \textendash{} the text to be displayed 
if text is str, the text may contain linefeeds, which are shown as individual lines

\item {} 
\sphinxstyleliteralstrong{max\_lines} (\sphinxstyleliteralemphasis{int}) \textendash{} the maximum of lines of text to be displayed 
if positive, it refers to the first max\_lines lines 
if negative, it refers to the last -max\_lines lines 
if zero (default), all lines will be displayed

\item {} 
\sphinxstyleliteralstrong{font} (\sphinxstyleliteralemphasis{str}\sphinxstyleliteralemphasis{ or }\sphinxstyleliteralemphasis{list/tuple}) \textendash{} font to be used for texts 
Either a string or a list/tuple of fontnames.
If not found, uses calibri or arial

\item {} 
\sphinxstyleliteralstrong{text\_anchor} (\sphinxstyleliteralemphasis{str}) \textendash{} anchor position of text\textbar{}n\textbar{}
specifies where to texts relative to the rectangle
point 
possible values are (default: c): 
\sphinxcode{nw    n    ne} 
\sphinxcode{w     c     e} 
\sphinxcode{sw    s    se}

\item {} 
\sphinxstyleliteralstrong{textcolor} (\sphinxstyleliteralemphasis{colorspec}) \textendash{} color of the text (default foreground\_color)

\item {} 
\sphinxstyleliteralstrong{textoffsetx} (\sphinxstyleliteralemphasis{float}) \textendash{} extra x offset to the text\_anchor point

\item {} 
\sphinxstyleliteralstrong{textoffsety} (\sphinxstyleliteralemphasis{float}) \textendash{} extra y offset to the text\_anchor point

\item {} 
\sphinxstyleliteralstrong{fontsize} (\sphinxstyleliteralemphasis{float}) \textendash{} fontsize of text (default 15)

\item {} 
\sphinxstyleliteralstrong{arg} (\sphinxstyleliteralemphasis{any}) \textendash{} this is used when a parameter is a function with two parameters, as the first argument or
if a parameter is a method as the instance 
default: self (instance itself)

\item {} 
\sphinxstyleliteralstrong{parent} ({\hyperref[\detokenize{Reference:salabim.Component}]{\sphinxcrossref{\sphinxstyleliteralemphasis{Component}}}}) \textendash{} component where this animation object belongs to (default None) 
if given, the animation object will be removed
automatically upon termination of the parent

\end{itemize}

\end{description}\end{quote}

\begin{sphinxadmonition}{note}{Note:}
All measures are in screen coordinates 

All parameters, apart from queue and arg can be specified as: 
- a scalar, like 10 
- a function with zero arguments, like lambda: title 
- a function with one argument, being the time t, like lambda t: t + 10 
- a function with two parameters, being arg (as given) and the time, like lambda comp, t: comp.state 
- a method instance arg for time t, like self.state, actually leading to arg.state(t) to be called
\end{sphinxadmonition}

\end{fulllineitems}

\index{AnimateSlider (class in salabim)}

\begin{fulllineitems}
\phantomsection\label{\detokenize{Reference:salabim.AnimateSlider}}\pysiglinewithargsret{\sphinxbfcode{class }\sphinxcode{salabim.}\sphinxbfcode{AnimateSlider}}{\emph{layer=0}, \emph{x=0}, \emph{y=0}, \emph{width=100}, \emph{height=20}, \emph{vmin=0}, \emph{vmax=10}, \emph{v=None}, \emph{resolution=1}, \emph{linecolor='fg'}, \emph{labelcolor='fg'}, \emph{label=''}, \emph{font=''}, \emph{fontsize=12}, \emph{action=None}, \emph{xy\_anchor='sw'}, \emph{env=None}}{}
defines a slider
\begin{quote}\begin{description}
\item[{Parameters}] \leavevmode\begin{itemize}
\item {} 
\sphinxstyleliteralstrong{x} (\sphinxstyleliteralemphasis{int}) \textendash{} x-coordinate of centre of the slider in screen coordinates (default 0)

\item {} 
\sphinxstyleliteralstrong{y} (\sphinxstyleliteralemphasis{int}) \textendash{} y-coordinate of centre of the slider in screen coordinates (default 0)

\item {} 
\sphinxstyleliteralstrong{vmin} (\sphinxstyleliteralemphasis{float}) \textendash{} minimum value of the slider (default 0)

\item {} 
\sphinxstyleliteralstrong{vmax} (\sphinxstyleliteralemphasis{float}) \textendash{} maximum value of the slider (default 0)

\item {} 
\sphinxstyleliteralstrong{v} (\sphinxstyleliteralemphasis{float}) \textendash{} initial value of the slider (default 0) 
should be between vmin and vmax

\item {} 
\sphinxstyleliteralstrong{resolution} (\sphinxstyleliteralemphasis{float}) \textendash{} step size of value (default 1)

\item {} 
\sphinxstyleliteralstrong{width} (\sphinxstyleliteralemphasis{float}) \textendash{} width of slider in screen coordinates (default 100)

\item {} 
\sphinxstyleliteralstrong{height} (\sphinxstyleliteralemphasis{float}) \textendash{} height of slider in screen coordinates (default 20)

\item {} 
\sphinxstyleliteralstrong{linewidth} (\sphinxstyleliteralemphasis{float}) \textendash{} width of contour in screen coordinate (default 0 = no contour)

\item {} 
\sphinxstyleliteralstrong{linecolor} (\sphinxstyleliteralemphasis{colorspec}) \textendash{} color of contour (default foreground\_color)

\item {} 
\sphinxstyleliteralstrong{labelcolor} (\sphinxstyleliteralemphasis{colorspec}) \textendash{} color of the label (default foreground\_color)

\item {} 
\sphinxstyleliteralstrong{label} (\sphinxstyleliteralemphasis{str}) \textendash{} label if the slider (default null string) 
if label is an argumentless function, this function
will be used to display as label, otherwise the
label plus the current value of the slider will be shown

\item {} 
\sphinxstyleliteralstrong{font} (\sphinxstyleliteralemphasis{str}) \textendash{} font of the text (default Helvetica)

\item {} 
\sphinxstyleliteralstrong{fontsize} (\sphinxstyleliteralemphasis{int}) \textendash{} fontsize of the text (default 12)

\item {} 
\sphinxstyleliteralstrong{action} (\sphinxstyleliteralemphasis{function}) \textendash{} function executed when the slider value is changed (default None) 
the function should one arguments, being the new value 
if None (default), no action

\item {} 
\sphinxstyleliteralstrong{xy\_anchor} (\sphinxstyleliteralemphasis{str}) \textendash{} specifies where x and y are relative to 
possible values are (default: sw): 
\sphinxcode{nw    n    ne} 
\sphinxcode{w     c     e} 
\sphinxcode{sw    s    se}

\item {} 
\sphinxstyleliteralstrong{env} ({\hyperref[\detokenize{Reference:salabim.Environment}]{\sphinxcrossref{\sphinxstyleliteralemphasis{Environment}}}}) \textendash{} environment where the component is defined 
if omitted, default\_env will be used

\end{itemize}

\end{description}\end{quote}

\begin{sphinxadmonition}{note}{Note:}
The current value of the slider is the v attibute of the slider. 
All measures are in screen coordinates 
On Pythonista, this functionality is emulated by salabim
On other platforms, the tkinter functionality is used.
\end{sphinxadmonition}
\index{remove() (salabim.AnimateSlider method)}

\begin{fulllineitems}
\phantomsection\label{\detokenize{Reference:salabim.AnimateSlider.remove}}\pysiglinewithargsret{\sphinxbfcode{remove}}{}{}
removes the slider object 
The ui object is removed from the ui queue,
so effectively ending this ui

\end{fulllineitems}

\index{v() (salabim.AnimateSlider method)}

\begin{fulllineitems}
\phantomsection\label{\detokenize{Reference:salabim.AnimateSlider.v}}\pysiglinewithargsret{\sphinxbfcode{v}}{\emph{value=None}}{}
value
\begin{quote}\begin{description}
\item[{Parameters}] \leavevmode
\sphinxstyleliteralstrong{value} (\sphinxstyleliteralemphasis{float}) \textendash{} new value 
if omitted, no change

\item[{Returns}] \leavevmode
\sphinxstylestrong{Current value of the slider}

\item[{Return type}] \leavevmode
float

\end{description}\end{quote}

\end{fulllineitems}


\end{fulllineitems}

\index{AnimateText (class in salabim)}

\begin{fulllineitems}
\phantomsection\label{\detokenize{Reference:salabim.AnimateText}}\pysiglinewithargsret{\sphinxbfcode{class }\sphinxcode{salabim.}\sphinxbfcode{AnimateText}}{\emph{text=''}, \emph{x=0}, \emph{y=0}, \emph{fontsize=15}, \emph{textcolor='fg'}, \emph{font=''}, \emph{text\_anchor='sw'}, \emph{angle=0}, \emph{visible=True}, \emph{xy\_anchor=''}, \emph{layer=0}, \emph{env=None}, \emph{screen\_coordinates=False}, \emph{arg=None}, \emph{parent=None}, \emph{offsetx=0}, \emph{offsety=0}, \emph{max\_lines=0}}{}
Displays a text
\begin{quote}\begin{description}
\item[{Parameters}] \leavevmode\begin{itemize}
\item {} 
\sphinxstyleliteralstrong{text} (\sphinxstyleliteralemphasis{str}\sphinxstyleliteralemphasis{, }\sphinxstyleliteralemphasis{tuple}\sphinxstyleliteralemphasis{ or }\sphinxstyleliteralemphasis{list}) \textendash{} the text to be displayed 
if text is str, the text may contain linefeeds, which are shown as individual lines
if text is tple or list, each item is displayed on a separate line

\item {} 
\sphinxstyleliteralstrong{x} (\sphinxstyleliteralemphasis{float}) \textendash{} position of anchor point (default 0)

\item {} 
\sphinxstyleliteralstrong{y} (\sphinxstyleliteralemphasis{float}) \textendash{} position of anchor point (default 0)

\item {} 
\sphinxstyleliteralstrong{xy\_anchor} (\sphinxstyleliteralemphasis{str}) \textendash{} specifies where x and y are relative to 
possible values are (default: sw) : 
\sphinxcode{nw    n    ne} 
\sphinxcode{w     c     e} 
\sphinxcode{sw    s    se} 
If ‘’, the given coordimates are used untranslated

\item {} 
\sphinxstyleliteralstrong{offsetx} (\sphinxstyleliteralemphasis{float}) \textendash{} offsets the x-coordinate of the rectangle (default 0)

\item {} 
\sphinxstyleliteralstrong{offsety} (\sphinxstyleliteralemphasis{float}) \textendash{} offsets the y-coordinate of the rectangle (default 0)

\item {} 
\sphinxstyleliteralstrong{angle} (\sphinxstyleliteralemphasis{float}) \textendash{} angle of the text (in degrees) 
default: 0

\item {} 
\sphinxstyleliteralstrong{max\_lines} (\sphinxstyleliteralemphasis{int}) \textendash{} the maximum of lines of text to be displayed 
if positive, it refers to the first max\_lines lines 
if negative, it refers to the last -max\_lines lines 
if zero (default), all lines will be displayed

\item {} 
\sphinxstyleliteralstrong{font} (\sphinxstyleliteralemphasis{str}\sphinxstyleliteralemphasis{ or }\sphinxstyleliteralemphasis{list/tuple}) \textendash{} font to be used for texts 
Either a string or a list/tuple of fontnames.
If not found, uses calibri or arial

\item {} 
\sphinxstyleliteralstrong{text\_anchor} (\sphinxstyleliteralemphasis{str}) \textendash{} anchor position of text\textbar{}n\textbar{}
specifies where to texts relative to the rectangle
point 
possible values are (default: c): 
\sphinxcode{nw    n    ne} 
\sphinxcode{w     c     e} 
\sphinxcode{sw    s    se}

\item {} 
\sphinxstyleliteralstrong{textcolor} (\sphinxstyleliteralemphasis{colorspec}) \textendash{} color of the text (default foreground\_color)

\item {} 
\sphinxstyleliteralstrong{fontsize} (\sphinxstyleliteralemphasis{float}) \textendash{} fontsize of text (default 15)

\item {} 
\sphinxstyleliteralstrong{arg} (\sphinxstyleliteralemphasis{any}) \textendash{} this is used when a parameter is a function with two parameters, as the first argument or
if a parameter is a method as the instance 
default: self (instance itself)

\item {} 
\sphinxstyleliteralstrong{parent} ({\hyperref[\detokenize{Reference:salabim.Component}]{\sphinxcrossref{\sphinxstyleliteralemphasis{Component}}}}) \textendash{} component where this animation object belongs to (default None) 
if given, the animation object will be removed
automatically upon termination of the parent

\end{itemize}

\end{description}\end{quote}

\begin{sphinxadmonition}{note}{Note:}
All measures are in screen coordinates 

All parameters, apart from queue and arg can be specified as: 
- a scalar, like 10 
- a function with zero arguments, like lambda: title 
- a function with one argument, being the time t, like lambda t: t + 10 
- a function with two parameters, being arg (as given) and the time, like lambda comp, t: comp.state 
- a method instance arg for time t, like self.state, actually leading to arg.state(t) to be called
\end{sphinxadmonition}

\end{fulllineitems}



\section{Distributions}
\label{\detokenize{Reference:distributions}}\index{\_Distribution (class in salabim)}

\begin{fulllineitems}
\phantomsection\label{\detokenize{Reference:salabim._Distribution}}\pysigline{\sphinxbfcode{class }\sphinxcode{salabim.}\sphinxbfcode{\_Distribution}}~\index{bounded\_sample() (salabim.\_Distribution method)}

\begin{fulllineitems}
\phantomsection\label{\detokenize{Reference:salabim._Distribution.bounded_sample}}\pysiglinewithargsret{\sphinxbfcode{bounded\_sample}}{\emph{lowerbound=-inf}, \emph{upperbound=inf}, \emph{fail\_value=None}, \emph{number\_of\_retries=100}}{}~\begin{quote}\begin{description}
\item[{Parameters}] \leavevmode\begin{itemize}
\item {} 
\sphinxstyleliteralstrong{lowerbound} (\sphinxstyleliteralemphasis{float}) \textendash{} sample values \textless{} lowerbound will be rejected (at most 100 retries) 
if omitted, no lowerbound check

\item {} 
\sphinxstyleliteralstrong{upperbound} (\sphinxstyleliteralemphasis{float}) \textendash{} sample values \textgreater{} upperbound will be rejected (at most 100 retries) 
if omitted, no upperbound check

\item {} 
\sphinxstyleliteralstrong{fail\_value} (\sphinxstyleliteralemphasis{float}) \textendash{} value to be used if. after number\_of\_tries retries, sample is still not within bounds 
default: lowerbound, if specified, otherwise upperbound

\item {} 
\sphinxstyleliteralstrong{number\_of\_tries} (\sphinxstyleliteralemphasis{int}) \textendash{} number of tries before fail\_value is returned 
default: 100

\end{itemize}

\item[{Returns}] \leavevmode
\sphinxstylestrong{Bounded sample of a distribution}

\item[{Return type}] \leavevmode
depending on distribution type (usually float)

\end{description}\end{quote}

\begin{sphinxadmonition}{note}{Note:}
If, after number\_of\_tries retries, the sampled value is still not within the given bounds,
fail\_value  will be returned 
Samples that cannot be converted (only possible with Pdf and CumPdf) to float
are assumed to be within the bounds.
\end{sphinxadmonition}

\end{fulllineitems}


\end{fulllineitems}

\index{Beta (class in salabim)}

\begin{fulllineitems}
\phantomsection\label{\detokenize{Reference:salabim.Beta}}\pysiglinewithargsret{\sphinxbfcode{class }\sphinxcode{salabim.}\sphinxbfcode{Beta}}{\emph{alpha}, \emph{beta}, \emph{randomstream=None}}{}
beta distribution
\begin{quote}\begin{description}
\item[{Parameters}] \leavevmode\begin{itemize}
\item {} 
\sphinxstyleliteralstrong{alpha} (\sphinxstyleliteralemphasis{float}) \textendash{} alpha shape of the distribution 
should be \textgreater{}0

\item {} 
\sphinxstyleliteralstrong{beta} (\sphinxstyleliteralemphasis{float}) \textendash{} beta shape of the distribution 
should be \textgreater{}0

\item {} 
\sphinxstyleliteralstrong{randomstream} (\sphinxstyleliteralemphasis{randomstream}) \textendash{} randomstream to be used 
if omitted, random will be used 
if used as random.Random(12299)
it assigns a new stream with the specified seed

\end{itemize}

\end{description}\end{quote}
\index{mean() (salabim.Beta method)}

\begin{fulllineitems}
\phantomsection\label{\detokenize{Reference:salabim.Beta.mean}}\pysiglinewithargsret{\sphinxbfcode{mean}}{}{}~\begin{quote}\begin{description}
\item[{Returns}] \leavevmode
\sphinxstylestrong{Mean of the distribution}

\item[{Return type}] \leavevmode
float

\end{description}\end{quote}

\end{fulllineitems}

\index{print\_info() (salabim.Beta method)}

\begin{fulllineitems}
\phantomsection\label{\detokenize{Reference:salabim.Beta.print_info}}\pysiglinewithargsret{\sphinxbfcode{print\_info}}{\emph{as\_str=False}, \emph{file=None}}{}
prints information about the distribution
\begin{quote}\begin{description}
\item[{Parameters}] \leavevmode\begin{itemize}
\item {} 
\sphinxstyleliteralstrong{as\_str} (\sphinxstyleliteralemphasis{bool}) \textendash{} if False (default), print the info
if True, return a string containing the info

\item {} 
\sphinxstyleliteralstrong{file} (\sphinxstyleliteralemphasis{file}) \textendash{} if None(default), all output is directed to stdout 
otherwise, the output is directed to the file

\end{itemize}

\item[{Returns}] \leavevmode
\sphinxstylestrong{info (if as\_str is True)}

\item[{Return type}] \leavevmode
str

\end{description}\end{quote}

\end{fulllineitems}

\index{sample() (salabim.Beta method)}

\begin{fulllineitems}
\phantomsection\label{\detokenize{Reference:salabim.Beta.sample}}\pysiglinewithargsret{\sphinxbfcode{sample}}{}{}~\begin{quote}\begin{description}
\item[{Returns}] \leavevmode
\sphinxstylestrong{Sample of the distribution}

\item[{Return type}] \leavevmode
float

\end{description}\end{quote}

\end{fulllineitems}


\end{fulllineitems}

\index{Cdf (class in salabim)}

\begin{fulllineitems}
\phantomsection\label{\detokenize{Reference:salabim.Cdf}}\pysiglinewithargsret{\sphinxbfcode{class }\sphinxcode{salabim.}\sphinxbfcode{Cdf}}{\emph{spec}, \emph{time\_unit=None}, \emph{randomstream=None}, \emph{env=None}}{}
Cumulative distribution function
\begin{quote}\begin{description}
\item[{Parameters}] \leavevmode\begin{itemize}
\item {} 
\sphinxstyleliteralstrong{spec} (\sphinxstyleliteralemphasis{list}\sphinxstyleliteralemphasis{ or }\sphinxstyleliteralemphasis{tuple}) \textendash{} 
list with x-values and corresponding cumulative density
(x1,c1,x2,c2, …xn,cn) 
Requirements:
\begin{quote}

x1\textless{}=x2\textless{}= …\textless{}=xn 
c1\textless{}=c2\textless{}=cn 
c1=0 
cn\textgreater{}0 
all cumulative densities are auto scaled according to cn,
so no need to set cn to 1 or 100.
\end{quote}


\item {} 
\sphinxstyleliteralstrong{time\_unit} (\sphinxstyleliteralemphasis{str}) \textendash{} specifies the time unit 
must be one of ‘years’, ‘weeks’, ‘days’, ‘hours’, ‘minutes’, ‘seconds’, ‘milliseconds’, ‘microseconds’ 
default : no conversion 

\item {} 
\sphinxstyleliteralstrong{randomstream} (\sphinxstyleliteralemphasis{randomstream}) \textendash{} if omitted, random will be used 
if used as random.Random(12299)
it defines a new stream with the specified seed

\item {} 
\sphinxstyleliteralstrong{env} ({\hyperref[\detokenize{Reference:salabim.Environment}]{\sphinxcrossref{\sphinxstyleliteralemphasis{Environment}}}}) \textendash{} environment where the distribution is defined 
if omitted, default\_env will be used

\end{itemize}

\end{description}\end{quote}
\index{mean() (salabim.Cdf method)}

\begin{fulllineitems}
\phantomsection\label{\detokenize{Reference:salabim.Cdf.mean}}\pysiglinewithargsret{\sphinxbfcode{mean}}{}{}~\begin{quote}\begin{description}
\item[{Returns}] \leavevmode
\sphinxstylestrong{Mean of the distribution}

\item[{Return type}] \leavevmode
float

\end{description}\end{quote}

\end{fulllineitems}

\index{print\_info() (salabim.Cdf method)}

\begin{fulllineitems}
\phantomsection\label{\detokenize{Reference:salabim.Cdf.print_info}}\pysiglinewithargsret{\sphinxbfcode{print\_info}}{\emph{as\_str=False}, \emph{file=None}}{}
prints information about the distribution
\begin{quote}\begin{description}
\item[{Parameters}] \leavevmode\begin{itemize}
\item {} 
\sphinxstyleliteralstrong{as\_str} (\sphinxstyleliteralemphasis{bool}) \textendash{} if False (default), print the info
if True, return a string containing the info

\item {} 
\sphinxstyleliteralstrong{file} (\sphinxstyleliteralemphasis{file}) \textendash{} if None(default), all output is directed to stdout 
otherwise, the output is directed to the file

\end{itemize}

\item[{Returns}] \leavevmode
\sphinxstylestrong{info (if as\_str is True)}

\item[{Return type}] \leavevmode
str

\end{description}\end{quote}

\end{fulllineitems}

\index{sample() (salabim.Cdf method)}

\begin{fulllineitems}
\phantomsection\label{\detokenize{Reference:salabim.Cdf.sample}}\pysiglinewithargsret{\sphinxbfcode{sample}}{}{}~\begin{quote}\begin{description}
\item[{Returns}] \leavevmode
\sphinxstylestrong{Sample of the distribution}

\item[{Return type}] \leavevmode
float

\end{description}\end{quote}

\end{fulllineitems}


\end{fulllineitems}

\index{Constant (class in salabim)}

\begin{fulllineitems}
\phantomsection\label{\detokenize{Reference:salabim.Constant}}\pysiglinewithargsret{\sphinxbfcode{class }\sphinxcode{salabim.}\sphinxbfcode{Constant}}{\emph{value}, \emph{time\_unit=None}, \emph{randomstream=None}, \emph{env=None}}{}
constant distribution
\begin{quote}\begin{description}
\item[{Parameters}] \leavevmode\begin{itemize}
\item {} 
\sphinxstyleliteralstrong{value} (\sphinxstyleliteralemphasis{float}) \textendash{} value to be returned in sample

\item {} 
\sphinxstyleliteralstrong{time\_unit} (\sphinxstyleliteralemphasis{str}) \textendash{} specifies the time unit 
must be one of ‘years’, ‘weeks’, ‘days’, ‘hours’, ‘minutes’, ‘seconds’, ‘milliseconds’, ‘microseconds’ 
default : no conversion 

\item {} 
\sphinxstyleliteralstrong{randomstream} (\sphinxstyleliteralemphasis{randomstream}) \textendash{} randomstream to be used 
if omitted, random will be used 
if used as random.Random(12299)
it assigns a new stream with the specified seed 
Note that this is only for compatibility with other distributions

\item {} 
\sphinxstyleliteralstrong{env} ({\hyperref[\detokenize{Reference:salabim.Environment}]{\sphinxcrossref{\sphinxstyleliteralemphasis{Environment}}}}) \textendash{} environment where the distribution is defined 
if omitted, default\_env will be used

\end{itemize}

\end{description}\end{quote}
\index{mean() (salabim.Constant method)}

\begin{fulllineitems}
\phantomsection\label{\detokenize{Reference:salabim.Constant.mean}}\pysiglinewithargsret{\sphinxbfcode{mean}}{}{}~\begin{quote}\begin{description}
\item[{Returns}] \leavevmode
\sphinxstylestrong{mean of the distribution (= the specified constant)}

\item[{Return type}] \leavevmode
float

\end{description}\end{quote}

\end{fulllineitems}

\index{print\_info() (salabim.Constant method)}

\begin{fulllineitems}
\phantomsection\label{\detokenize{Reference:salabim.Constant.print_info}}\pysiglinewithargsret{\sphinxbfcode{print\_info}}{\emph{as\_str=False}, \emph{file=None}}{}
prints information about the distribution
\begin{quote}\begin{description}
\item[{Parameters}] \leavevmode\begin{itemize}
\item {} 
\sphinxstyleliteralstrong{as\_str} (\sphinxstyleliteralemphasis{bool}) \textendash{} if False (default), print the info
if True, return a string containing the info

\item {} 
\sphinxstyleliteralstrong{file} (\sphinxstyleliteralemphasis{file}) \textendash{} if None(default), all output is directed to stdout 
otherwise, the output is directed to the file

\end{itemize}

\item[{Returns}] \leavevmode
\sphinxstylestrong{info (if as\_str is True)}

\item[{Return type}] \leavevmode
str

\end{description}\end{quote}

\end{fulllineitems}

\index{sample() (salabim.Constant method)}

\begin{fulllineitems}
\phantomsection\label{\detokenize{Reference:salabim.Constant.sample}}\pysiglinewithargsret{\sphinxbfcode{sample}}{}{}~\begin{quote}\begin{description}
\item[{Returns}] \leavevmode
\sphinxstylestrong{sample of the distribution (= the specified constant)}

\item[{Return type}] \leavevmode
float

\end{description}\end{quote}

\end{fulllineitems}


\end{fulllineitems}

\index{Distribution (class in salabim)}

\begin{fulllineitems}
\phantomsection\label{\detokenize{Reference:salabim.Distribution}}\pysiglinewithargsret{\sphinxbfcode{class }\sphinxcode{salabim.}\sphinxbfcode{Distribution}}{\emph{spec}, \emph{randomstream=None}}{}
Generate a distribution from a string
\begin{quote}\begin{description}
\item[{Parameters}] \leavevmode\begin{itemize}
\item {} 
\sphinxstyleliteralstrong{spec} (\sphinxstyleliteralemphasis{str}) \textendash{} \begin{itemize}
\item {} 
string containing a valid salabim distribution, where only the first
letters are relevant and casing is not important. Note that Erlang,
Cdf, CumPdf and Poisson require at least two letters
(Er, Cd, Cu and Po)

\item {} 
string containing one float (c1), resulting in Constant(c1)

\item {} 
string containing two floats seperated by a comma (c1,c2),
resulting in a Uniform(c1,c2)

\item {} 
string containing three floats, separated by commas (c1,c2,c3),
resulting in a Triangular(c1,c2,c3)

\end{itemize}


\item {} 
\sphinxstyleliteralstrong{randomstream} (\sphinxstyleliteralemphasis{randomstream}) \textendash{} if omitted, random will be used 
if used as random.Random(12299)
it assigns a new stream with the specified seed 

\end{itemize}

\end{description}\end{quote}

\begin{sphinxadmonition}{note}{Note:}
The randomstream in the specifying string is ignored. 
It is possible to use expressions in the specification, as long these
are valid within the context of the salabim module, which usually implies
a global variable of the salabim package.
\end{sphinxadmonition}
\paragraph{Examples}

Uniform(13)  ==\textgreater{} Uniform(13) 
Uni(12,15)   ==\textgreater{} Uniform(12,15) 
UNIF(12,15)  ==\textgreater{} Uniform(12,15) 
N(12,3)      ==\textgreater{} Normal(12,3) 
Tri(10,20).  ==\textgreater{} Triangular(10,20,15) 
10.          ==\textgreater{} Constant(10) 
12,15        ==\textgreater{} Uniform(12,15) 
(12,15)      ==\textgreater{} Uniform(12,15) 
Exp(a)       ==\textgreater{} Exponential(100), provided sim.a=100 
E(2)         ==\textgreater{} Exponential(2)
Er(2,3)      ==\textgreater{} Erlang(2,3)
\index{mean() (salabim.Distribution method)}

\begin{fulllineitems}
\phantomsection\label{\detokenize{Reference:salabim.Distribution.mean}}\pysiglinewithargsret{\sphinxbfcode{mean}}{}{}~\begin{quote}\begin{description}
\item[{Returns}] \leavevmode
\sphinxstylestrong{Mean of the distribution}

\item[{Return type}] \leavevmode
float

\end{description}\end{quote}

\end{fulllineitems}

\index{print\_info() (salabim.Distribution method)}

\begin{fulllineitems}
\phantomsection\label{\detokenize{Reference:salabim.Distribution.print_info}}\pysiglinewithargsret{\sphinxbfcode{print\_info}}{\emph{as\_str=False}, \emph{file=None}}{}
prints information about the distribution
\begin{quote}\begin{description}
\item[{Parameters}] \leavevmode\begin{itemize}
\item {} 
\sphinxstyleliteralstrong{as\_str} (\sphinxstyleliteralemphasis{bool}) \textendash{} if False (default), print the info
if True, return a string containing the info

\item {} 
\sphinxstyleliteralstrong{file} (\sphinxstyleliteralemphasis{file}) \textendash{} if None(default), all output is directed to stdout 
otherwise, the output is directed to the file

\end{itemize}

\item[{Returns}] \leavevmode
\sphinxstylestrong{info (if as\_str is True)}

\item[{Return type}] \leavevmode
str

\end{description}\end{quote}

\end{fulllineitems}

\index{sample() (salabim.Distribution method)}

\begin{fulllineitems}
\phantomsection\label{\detokenize{Reference:salabim.Distribution.sample}}\pysiglinewithargsret{\sphinxbfcode{sample}}{}{}~\begin{quote}\begin{description}
\item[{Returns}] \leavevmode
\sphinxstylestrong{Sample of the  distribution}

\item[{Return type}] \leavevmode
any (usually float)

\end{description}\end{quote}

\end{fulllineitems}


\end{fulllineitems}

\index{Erlang (class in salabim)}

\begin{fulllineitems}
\phantomsection\label{\detokenize{Reference:salabim.Erlang}}\pysiglinewithargsret{\sphinxbfcode{class }\sphinxcode{salabim.}\sphinxbfcode{Erlang}}{\emph{shape}, \emph{rate=None}, \emph{time\_unit=None}, \emph{scale=None}, \emph{randomstream=None}, \emph{env=None}}{}
erlang distribution
\begin{quote}\begin{description}
\item[{Parameters}] \leavevmode\begin{itemize}
\item {} 
\sphinxstyleliteralstrong{shape} (\sphinxstyleliteralemphasis{int}) \textendash{} shape of the distribution (k) 
should be \textgreater{}0

\item {} 
\sphinxstyleliteralstrong{rate} (\sphinxstyleliteralemphasis{float}) \textendash{} rate parameter (lambda) 
if omitted, the scale is used 
should be \textgreater{}0

\item {} 
\sphinxstyleliteralstrong{time\_unit} (\sphinxstyleliteralemphasis{str}) \textendash{} specifies the time unit 
must be one of ‘years’, ‘weeks’, ‘days’, ‘hours’, ‘minutes’, ‘seconds’, ‘milliseconds’, ‘microseconds’ 
default : no conversion 

\item {} 
\sphinxstyleliteralstrong{scale} (\sphinxstyleliteralemphasis{float}) \textendash{} scale of the distribution (mu) 
if omitted, the rate is used 
should be \textgreater{}0

\item {} 
\sphinxstyleliteralstrong{randomstream} (\sphinxstyleliteralemphasis{randomstream}) \textendash{} randomstream to be used 
if omitted, random will be used 
if used as random.Random(12299)
it assigns a new stream with the specified seed

\item {} 
\sphinxstyleliteralstrong{env} ({\hyperref[\detokenize{Reference:salabim.Environment}]{\sphinxcrossref{\sphinxstyleliteralemphasis{Environment}}}}) \textendash{} environment where the distribution is defined 
if omitted, default\_env will be used

\end{itemize}

\end{description}\end{quote}

\begin{sphinxadmonition}{note}{Note:}
Either rate or scale has to be specified, not both.
\end{sphinxadmonition}
\index{mean() (salabim.Erlang method)}

\begin{fulllineitems}
\phantomsection\label{\detokenize{Reference:salabim.Erlang.mean}}\pysiglinewithargsret{\sphinxbfcode{mean}}{}{}~\begin{quote}\begin{description}
\item[{Returns}] \leavevmode
\sphinxstylestrong{Mean of the distribution}

\item[{Return type}] \leavevmode
float

\end{description}\end{quote}

\end{fulllineitems}

\index{print\_info() (salabim.Erlang method)}

\begin{fulllineitems}
\phantomsection\label{\detokenize{Reference:salabim.Erlang.print_info}}\pysiglinewithargsret{\sphinxbfcode{print\_info}}{\emph{as\_str=False}, \emph{file=None}}{}
prints information about the distribution
\begin{quote}\begin{description}
\item[{Parameters}] \leavevmode\begin{itemize}
\item {} 
\sphinxstyleliteralstrong{as\_str} (\sphinxstyleliteralemphasis{bool}) \textendash{} if False (default), print the info
if True, return a string containing the info

\item {} 
\sphinxstyleliteralstrong{file} (\sphinxstyleliteralemphasis{file}) \textendash{} if None(default), all output is directed to stdout 
otherwise, the output is directed to the file

\end{itemize}

\item[{Returns}] \leavevmode
\sphinxstylestrong{info (if as\_str is True)}

\item[{Return type}] \leavevmode
str

\end{description}\end{quote}

\end{fulllineitems}

\index{sample() (salabim.Erlang method)}

\begin{fulllineitems}
\phantomsection\label{\detokenize{Reference:salabim.Erlang.sample}}\pysiglinewithargsret{\sphinxbfcode{sample}}{}{}~\begin{quote}\begin{description}
\item[{Returns}] \leavevmode
\sphinxstylestrong{Sample of the distribution}

\item[{Return type}] \leavevmode
float

\end{description}\end{quote}

\end{fulllineitems}


\end{fulllineitems}

\index{Exponential (class in salabim)}

\begin{fulllineitems}
\phantomsection\label{\detokenize{Reference:salabim.Exponential}}\pysiglinewithargsret{\sphinxbfcode{class }\sphinxcode{salabim.}\sphinxbfcode{Exponential}}{\emph{mean=None}, \emph{time\_unit=None}, \emph{rate=None}, \emph{randomstream=None}, \emph{env=None}}{}
exponential distribution
\begin{quote}\begin{description}
\item[{Parameters}] \leavevmode\begin{itemize}
\item {} 
\sphinxstyleliteralstrong{mean} (\sphinxstyleliteralemphasis{float}) \textendash{} mean of the distribtion (beta)\textbar{}n\textbar{}
if omitted, the rate is used 
must be \textgreater{}0

\item {} 
\sphinxstyleliteralstrong{time\_unit} (\sphinxstyleliteralemphasis{str}) \textendash{} specifies the time unit 
must be one of ‘years’, ‘weeks’, ‘days’, ‘hours’, ‘minutes’, ‘seconds’, ‘milliseconds’, ‘microseconds’ 
default : no conversion 

\item {} 
\sphinxstyleliteralstrong{rate} (\sphinxstyleliteralemphasis{float}) \textendash{} rate of the distribution (lambda)\textbar{}n\textbar{}
if omitted, the mean is used 
must be \textgreater{}0

\item {} 
\sphinxstyleliteralstrong{randomstream} (\sphinxstyleliteralemphasis{randomstream}) \textendash{} randomstream to be used 
if omitted, random will be used 
if used as random.Random(12299)
it assigns a new stream with the specified seed

\item {} 
\sphinxstyleliteralstrong{env} ({\hyperref[\detokenize{Reference:salabim.Environment}]{\sphinxcrossref{\sphinxstyleliteralemphasis{Environment}}}}) \textendash{} environment where the distribution is defined 
if omitted, default\_env will be used

\end{itemize}

\end{description}\end{quote}

\begin{sphinxadmonition}{note}{Note:}
Either mean or rate has to be specified, not both
\end{sphinxadmonition}
\index{mean() (salabim.Exponential method)}

\begin{fulllineitems}
\phantomsection\label{\detokenize{Reference:salabim.Exponential.mean}}\pysiglinewithargsret{\sphinxbfcode{mean}}{}{}~\begin{quote}\begin{description}
\item[{Returns}] \leavevmode
\sphinxstylestrong{Mean of the distribution}

\item[{Return type}] \leavevmode
float

\end{description}\end{quote}

\end{fulllineitems}

\index{print\_info() (salabim.Exponential method)}

\begin{fulllineitems}
\phantomsection\label{\detokenize{Reference:salabim.Exponential.print_info}}\pysiglinewithargsret{\sphinxbfcode{print\_info}}{\emph{as\_str=False}, \emph{file=None}}{}
prints information about the distribution
\begin{quote}\begin{description}
\item[{Parameters}] \leavevmode\begin{itemize}
\item {} 
\sphinxstyleliteralstrong{as\_str} (\sphinxstyleliteralemphasis{bool}) \textendash{} if False (default), print the info
if True, return a string containing the info

\item {} 
\sphinxstyleliteralstrong{file} (\sphinxstyleliteralemphasis{file}) \textendash{} if None(default), all output is directed to stdout 
otherwise, the output is directed to the file

\end{itemize}

\item[{Returns}] \leavevmode
\sphinxstylestrong{info (if as\_str is True)}

\item[{Return type}] \leavevmode
str

\end{description}\end{quote}

\end{fulllineitems}

\index{sample() (salabim.Exponential method)}

\begin{fulllineitems}
\phantomsection\label{\detokenize{Reference:salabim.Exponential.sample}}\pysiglinewithargsret{\sphinxbfcode{sample}}{}{}~\begin{quote}\begin{description}
\item[{Returns}] \leavevmode
\sphinxstylestrong{Sample of the distribution}

\item[{Return type}] \leavevmode
float

\end{description}\end{quote}

\end{fulllineitems}


\end{fulllineitems}

\index{Gamma (class in salabim)}

\begin{fulllineitems}
\phantomsection\label{\detokenize{Reference:salabim.Gamma}}\pysiglinewithargsret{\sphinxbfcode{class }\sphinxcode{salabim.}\sphinxbfcode{Gamma}}{\emph{shape}, \emph{scale=None}, \emph{time\_unit=None}, \emph{rate=None}, \emph{randomstream=None}, \emph{env=None}}{}
gamma distribution
\begin{quote}\begin{description}
\item[{Parameters}] \leavevmode\begin{itemize}
\item {} 
\sphinxstyleliteralstrong{shape} (\sphinxstyleliteralemphasis{float}) \textendash{} shape of the distribution (k) 
should be \textgreater{}0

\item {} 
\sphinxstyleliteralstrong{scale} (\sphinxstyleliteralemphasis{float}) \textendash{} scale of the distribution (teta) 
should be \textgreater{}0

\item {} 
\sphinxstyleliteralstrong{time\_unit} (\sphinxstyleliteralemphasis{str}) \textendash{} specifies the time unit 
must be one of ‘years’, ‘weeks’, ‘days’, ‘hours’, ‘minutes’, ‘seconds’, ‘milliseconds’, ‘microseconds’ 
default : no conversion 

\item {} 
\sphinxstyleliteralstrong{rate} (\sphinxstyleliteralemphasis{float}) \textendash{} rate of the distribution (beta) 
should be \textgreater{}0

\item {} 
\sphinxstyleliteralstrong{randomstream} (\sphinxstyleliteralemphasis{randomstream}) \textendash{} randomstream to be used 
if omitted, random will be used 
if used as random.Random(12299)
it assigns a new stream with the specified seed

\end{itemize}

\end{description}\end{quote}
\begin{description}
\item[{env}] \leavevmode{[}Environment{]}
environment where the distribution is defined 
if omitted, default\_env will be used

\end{description}

\begin{sphinxadmonition}{note}{Note:}
Either scale or rate has to be specified, not both.
\end{sphinxadmonition}
\index{mean() (salabim.Gamma method)}

\begin{fulllineitems}
\phantomsection\label{\detokenize{Reference:salabim.Gamma.mean}}\pysiglinewithargsret{\sphinxbfcode{mean}}{}{}~\begin{quote}\begin{description}
\item[{Returns}] \leavevmode
\sphinxstylestrong{Mean of the distribution}

\item[{Return type}] \leavevmode
float

\end{description}\end{quote}

\end{fulllineitems}

\index{print\_info() (salabim.Gamma method)}

\begin{fulllineitems}
\phantomsection\label{\detokenize{Reference:salabim.Gamma.print_info}}\pysiglinewithargsret{\sphinxbfcode{print\_info}}{\emph{as\_str=False}, \emph{file=None}}{}
prints information about the distribution
\begin{quote}\begin{description}
\item[{Parameters}] \leavevmode\begin{itemize}
\item {} 
\sphinxstyleliteralstrong{as\_str} (\sphinxstyleliteralemphasis{bool}) \textendash{} if False (default), print the info
if True, return a string containing the info

\item {} 
\sphinxstyleliteralstrong{file} (\sphinxstyleliteralemphasis{file}) \textendash{} if None(default), all output is directed to stdout 
otherwise, the output is directed to the file

\end{itemize}

\item[{Returns}] \leavevmode
\sphinxstylestrong{info (if as\_str is True)}

\item[{Return type}] \leavevmode
str

\end{description}\end{quote}

\end{fulllineitems}

\index{sample() (salabim.Gamma method)}

\begin{fulllineitems}
\phantomsection\label{\detokenize{Reference:salabim.Gamma.sample}}\pysiglinewithargsret{\sphinxbfcode{sample}}{}{}~\begin{quote}\begin{description}
\item[{Returns}] \leavevmode
\sphinxstylestrong{Sample of the distribution}

\item[{Return type}] \leavevmode
float

\end{description}\end{quote}

\end{fulllineitems}


\end{fulllineitems}

\index{Normal (class in salabim)}

\begin{fulllineitems}
\phantomsection\label{\detokenize{Reference:salabim.Normal}}\pysiglinewithargsret{\sphinxbfcode{class }\sphinxcode{salabim.}\sphinxbfcode{Normal}}{\emph{mean}, \emph{standard\_deviation=None}, \emph{time\_unit=None}, \emph{coefficient\_of\_variation=None}, \emph{use\_gauss=False}, \emph{randomstream=None}, \emph{env=None}}{}
normal distribution
\begin{quote}\begin{description}
\item[{Parameters}] \leavevmode\begin{itemize}
\item {} 
\sphinxstyleliteralstrong{mean} (\sphinxstyleliteralemphasis{float}) \textendash{} mean of the distribution

\item {} 
\sphinxstyleliteralstrong{standard\_deviation} (\sphinxstyleliteralemphasis{float}) \textendash{} standard deviation of the distribution 
if omitted, coefficient\_of\_variation, is used to specify the variation
if neither standard\_devation nor coefficient\_of\_variation is given, 0 is used,
thus effectively a contant distribution 
must be \textgreater{}=0

\item {} 
\sphinxstyleliteralstrong{coefficient\_of\_variation} (\sphinxstyleliteralemphasis{float}) \textendash{} coefficient of variation of the distribution 
if omitted, standard\_deviation is used to specify variation 
the resulting standard\_deviation must be \textgreater{}=0

\item {} 
\sphinxstyleliteralstrong{use\_gauss} (\sphinxstyleliteralemphasis{bool}) \textendash{} if False (default), use the random.normalvariate method 
if True, use the random.gauss method 
the documentation for random states that the gauss method should be slightly faster,
although that statement is doubtful.

\item {} 
\sphinxstyleliteralstrong{randomstream} (\sphinxstyleliteralemphasis{randomstream}) \textendash{} randomstream to be used 
if omitted, random will be used 
if used as random.Random(12299)
it assigns a new stream with the specified seed

\item {} 
\sphinxstyleliteralstrong{env} ({\hyperref[\detokenize{Reference:salabim.Environment}]{\sphinxcrossref{\sphinxstyleliteralemphasis{Environment}}}}) \textendash{} environment where the distribution is defined 
if omitted, default\_env will be used

\end{itemize}

\end{description}\end{quote}
\index{mean() (salabim.Normal method)}

\begin{fulllineitems}
\phantomsection\label{\detokenize{Reference:salabim.Normal.mean}}\pysiglinewithargsret{\sphinxbfcode{mean}}{}{}~\begin{quote}\begin{description}
\item[{Returns}] \leavevmode
\sphinxstylestrong{Mean of the distribution}

\item[{Return type}] \leavevmode
float

\end{description}\end{quote}

\end{fulllineitems}

\index{print\_info() (salabim.Normal method)}

\begin{fulllineitems}
\phantomsection\label{\detokenize{Reference:salabim.Normal.print_info}}\pysiglinewithargsret{\sphinxbfcode{print\_info}}{\emph{as\_str=False}, \emph{file=None}}{}
prints information about the distribution
\begin{quote}\begin{description}
\item[{Parameters}] \leavevmode\begin{itemize}
\item {} 
\sphinxstyleliteralstrong{as\_str} (\sphinxstyleliteralemphasis{bool}) \textendash{} if False (default), print the info
if True, return a string containing the info

\item {} 
\sphinxstyleliteralstrong{file} (\sphinxstyleliteralemphasis{file}) \textendash{} if None(default), all output is directed to stdout 
otherwise, the output is directed to the file

\end{itemize}

\item[{Returns}] \leavevmode
\sphinxstylestrong{info (if as\_str is True)}

\item[{Return type}] \leavevmode
str

\end{description}\end{quote}

\end{fulllineitems}

\index{sample() (salabim.Normal method)}

\begin{fulllineitems}
\phantomsection\label{\detokenize{Reference:salabim.Normal.sample}}\pysiglinewithargsret{\sphinxbfcode{sample}}{}{}~\begin{quote}\begin{description}
\item[{Returns}] \leavevmode
\sphinxstylestrong{Sample of the distribution}

\item[{Return type}] \leavevmode
float

\end{description}\end{quote}

\end{fulllineitems}


\end{fulllineitems}

\index{Pdf (class in salabim)}

\begin{fulllineitems}
\phantomsection\label{\detokenize{Reference:salabim.Pdf}}\pysiglinewithargsret{\sphinxbfcode{class }\sphinxcode{salabim.}\sphinxbfcode{Pdf}}{\emph{spec}, \emph{probabilities=None}, \emph{time\_unit=None}, \emph{randomstream=None}, \emph{env=None}}{}
Probability distribution function
\begin{quote}\begin{description}
\item[{Parameters}] \leavevmode\begin{itemize}
\item {} 
\sphinxstyleliteralstrong{spec} (\sphinxstyleliteralemphasis{list}\sphinxstyleliteralemphasis{ or }\sphinxstyleliteralemphasis{tuple}) \textendash{} 
either
\begin{itemize}
\item {} 
if no probabilities specified: 
list with x-values and corresponding probability
(x0, p0, x1, p1, …xn,pn) 

\item {} 
if probabilities is specified: 
list with x-values

\end{itemize}


\item {} 
\sphinxstyleliteralstrong{probabilities} (\sphinxstyleliteralemphasis{list}\sphinxstyleliteralemphasis{, }\sphinxstyleliteralemphasis{tuple}\sphinxstyleliteralemphasis{ or }\sphinxstyleliteralemphasis{float}) \textendash{} if omitted, spec contains the probabilities 
the list (p0, p1, …pn) contains the probabilities of the corresponding
x-values from spec. 
alternatively, if a float is given (e.g. 1), all x-values
have equal probability. The value is not important.

\item {} 
\sphinxstyleliteralstrong{time\_unit} (\sphinxstyleliteralemphasis{str}) \textendash{} specifies the time unit 
must be one of ‘years’, ‘weeks’, ‘days’, ‘hours’, ‘minutes’, ‘seconds’, ‘milliseconds’, ‘microseconds’ 
default : no conversion 

\item {} 
\sphinxstyleliteralstrong{randomstream} (\sphinxstyleliteralemphasis{randomstream}) \textendash{} if omitted, random will be used 
if used as random.Random(12299)
it assigns a new stream with the specified seed

\item {} 
\sphinxstyleliteralstrong{env} ({\hyperref[\detokenize{Reference:salabim.Environment}]{\sphinxcrossref{\sphinxstyleliteralemphasis{Environment}}}}) \textendash{} environment where the distribution is defined 
if omitted, default\_env will be used

\end{itemize}

\end{description}\end{quote}

\begin{sphinxadmonition}{note}{Note:}
p0+p1=…+pn\textgreater{}0 
all densities are auto scaled according to the sum of p0 to pn,
so no need to have p0 to pn add up to 1 or 100. 
The x-values can be any type. 
If it is a salabim distribution, not the distribution,
but a sample will be returned when calling sample.
\end{sphinxadmonition}
\index{mean() (salabim.Pdf method)}

\begin{fulllineitems}
\phantomsection\label{\detokenize{Reference:salabim.Pdf.mean}}\pysiglinewithargsret{\sphinxbfcode{mean}}{}{}~\begin{quote}\begin{description}
\item[{Returns}] \leavevmode
\sphinxstylestrong{mean of the distribution} \textendash{} if the mean can’t be calculated (if not all x-values are scalars or distributions),
nan will be returned.

\item[{Return type}] \leavevmode
float

\end{description}\end{quote}

\end{fulllineitems}

\index{print\_info() (salabim.Pdf method)}

\begin{fulllineitems}
\phantomsection\label{\detokenize{Reference:salabim.Pdf.print_info}}\pysiglinewithargsret{\sphinxbfcode{print\_info}}{\emph{as\_str=False}, \emph{file=None}}{}
prints information about the distribution
\begin{quote}\begin{description}
\item[{Parameters}] \leavevmode\begin{itemize}
\item {} 
\sphinxstyleliteralstrong{as\_str} (\sphinxstyleliteralemphasis{bool}) \textendash{} if False (default), print the info
if True, return a string containing the info

\item {} 
\sphinxstyleliteralstrong{file} (\sphinxstyleliteralemphasis{file}) \textendash{} if None(default), all output is directed to stdout 
otherwise, the output is directed to the file

\end{itemize}

\item[{Returns}] \leavevmode
\sphinxstylestrong{info (if as\_str is True)}

\item[{Return type}] \leavevmode
str

\end{description}\end{quote}

\end{fulllineitems}

\index{sample() (salabim.Pdf method)}

\begin{fulllineitems}
\phantomsection\label{\detokenize{Reference:salabim.Pdf.sample}}\pysiglinewithargsret{\sphinxbfcode{sample}}{}{}~\begin{quote}\begin{description}
\item[{Returns}] \leavevmode
\sphinxstylestrong{Sample of the distribution}

\item[{Return type}] \leavevmode
any (usually float)

\end{description}\end{quote}

\end{fulllineitems}


\end{fulllineitems}

\index{Poisson (class in salabim)}

\begin{fulllineitems}
\phantomsection\label{\detokenize{Reference:salabim.Poisson}}\pysiglinewithargsret{\sphinxbfcode{class }\sphinxcode{salabim.}\sphinxbfcode{Poisson}}{\emph{mean}, \emph{randomstream=None}}{}
Poisson distribution
\begin{quote}\begin{description}
\item[{Parameters}] \leavevmode\begin{itemize}
\item {} 
\sphinxstyleliteralstrong{mean} (\sphinxstyleliteralemphasis{float}) \textendash{} mean (lambda) of the distribution

\item {} 
\sphinxstyleliteralstrong{randomstream} (\sphinxstyleliteralemphasis{randomstream}) \textendash{} randomstream to be used 
if omitted, random will be used 
if used as random.Random(12299)
it assigns a new stream with the specified seed

\end{itemize}

\end{description}\end{quote}

\begin{sphinxadmonition}{note}{Note:}
The run time of this function increases when mean (lambda) increases. 
It is not recommended to use mean (lambda) \textgreater{} 100
\end{sphinxadmonition}
\index{mean() (salabim.Poisson method)}

\begin{fulllineitems}
\phantomsection\label{\detokenize{Reference:salabim.Poisson.mean}}\pysiglinewithargsret{\sphinxbfcode{mean}}{}{}~\begin{quote}\begin{description}
\item[{Returns}] \leavevmode
\sphinxstylestrong{Mean of the distribution}

\item[{Return type}] \leavevmode
float

\end{description}\end{quote}

\end{fulllineitems}

\index{print\_info() (salabim.Poisson method)}

\begin{fulllineitems}
\phantomsection\label{\detokenize{Reference:salabim.Poisson.print_info}}\pysiglinewithargsret{\sphinxbfcode{print\_info}}{\emph{as\_str=False}, \emph{file=None}}{}
prints information about the distribution
\begin{quote}\begin{description}
\item[{Parameters}] \leavevmode\begin{itemize}
\item {} 
\sphinxstyleliteralstrong{as\_str} (\sphinxstyleliteralemphasis{bool}) \textendash{} if False (default), print the info
if True, return a string containing the info

\item {} 
\sphinxstyleliteralstrong{file} (\sphinxstyleliteralemphasis{file}) \textendash{} if None(default), all output is directed to stdout 
otherwise, the output is directed to the file

\end{itemize}

\item[{Returns}] \leavevmode
\sphinxstylestrong{info (if as\_str is True)}

\item[{Return type}] \leavevmode
str

\end{description}\end{quote}

\end{fulllineitems}

\index{sample() (salabim.Poisson method)}

\begin{fulllineitems}
\phantomsection\label{\detokenize{Reference:salabim.Poisson.sample}}\pysiglinewithargsret{\sphinxbfcode{sample}}{}{}~\begin{quote}\begin{description}
\item[{Returns}] \leavevmode
\sphinxstylestrong{Sample of the distribution}

\item[{Return type}] \leavevmode
int

\end{description}\end{quote}

\end{fulllineitems}


\end{fulllineitems}

\index{Triangular (class in salabim)}

\begin{fulllineitems}
\phantomsection\label{\detokenize{Reference:salabim.Triangular}}\pysiglinewithargsret{\sphinxbfcode{class }\sphinxcode{salabim.}\sphinxbfcode{Triangular}}{\emph{low}, \emph{high=None}, \emph{mode=None}, \emph{time\_unit=None}, \emph{randomstream=None}, \emph{env=None}}{}
triangular distribution
\begin{quote}\begin{description}
\item[{Parameters}] \leavevmode\begin{itemize}
\item {} 
\sphinxstyleliteralstrong{low} (\sphinxstyleliteralemphasis{float}) \textendash{} lowerbound of the distribution

\item {} 
\sphinxstyleliteralstrong{high} (\sphinxstyleliteralemphasis{float}) \textendash{} upperbound of the distribution 
if omitted, low will be used, thus effectively a constant distribution 
high must be \textgreater{}= low

\item {} 
\sphinxstyleliteralstrong{mode} (\sphinxstyleliteralemphasis{float}) \textendash{} mode of the distribution 
if omitted, the average of low and high will be used, thus a symmetric triangular distribution 
mode must be between low and high

\item {} 
\sphinxstyleliteralstrong{time\_unit} (\sphinxstyleliteralemphasis{str}) \textendash{} specifies the time unit 
must be one of ‘years’, ‘weeks’, ‘days’, ‘hours’, ‘minutes’, ‘seconds’, ‘milliseconds’, ‘microseconds’ 
default : no conversion 

\item {} 
\sphinxstyleliteralstrong{randomstream} (\sphinxstyleliteralemphasis{randomstream}) \textendash{} randomstream to be used 
if omitted, random will be used 
if used as random.Random(12299)
it assigns a new stream with the specified seed

\item {} 
\sphinxstyleliteralstrong{env} ({\hyperref[\detokenize{Reference:salabim.Environment}]{\sphinxcrossref{\sphinxstyleliteralemphasis{Environment}}}}) \textendash{} environment where the distribution is defined 
if omitted, default\_env will be used

\end{itemize}

\end{description}\end{quote}
\index{mean() (salabim.Triangular method)}

\begin{fulllineitems}
\phantomsection\label{\detokenize{Reference:salabim.Triangular.mean}}\pysiglinewithargsret{\sphinxbfcode{mean}}{}{}~\begin{quote}\begin{description}
\item[{Returns}] \leavevmode
\sphinxstylestrong{Mean of the distribution}

\item[{Return type}] \leavevmode
float

\end{description}\end{quote}

\end{fulllineitems}

\index{print\_info() (salabim.Triangular method)}

\begin{fulllineitems}
\phantomsection\label{\detokenize{Reference:salabim.Triangular.print_info}}\pysiglinewithargsret{\sphinxbfcode{print\_info}}{\emph{as\_str=False}, \emph{file=None}}{}
prints information about the distribution
\begin{quote}\begin{description}
\item[{Parameters}] \leavevmode\begin{itemize}
\item {} 
\sphinxstyleliteralstrong{as\_str} (\sphinxstyleliteralemphasis{bool}) \textendash{} if False (default), print the info
if True, return a string containing the info

\item {} 
\sphinxstyleliteralstrong{file} (\sphinxstyleliteralemphasis{file}) \textendash{} if None(default), all output is directed to stdout 
otherwise, the output is directed to the file

\end{itemize}

\item[{Returns}] \leavevmode
\sphinxstylestrong{info (if as\_str is True)}

\item[{Return type}] \leavevmode
str

\end{description}\end{quote}

\end{fulllineitems}

\index{sample() (salabim.Triangular method)}

\begin{fulllineitems}
\phantomsection\label{\detokenize{Reference:salabim.Triangular.sample}}\pysiglinewithargsret{\sphinxbfcode{sample}}{}{}~\begin{quote}\begin{description}
\item[{Returns}] \leavevmode
\sphinxstylestrong{Sample of the distribtion}

\item[{Return type}] \leavevmode
float

\end{description}\end{quote}

\end{fulllineitems}


\end{fulllineitems}

\index{Uniform (class in salabim)}

\begin{fulllineitems}
\phantomsection\label{\detokenize{Reference:salabim.Uniform}}\pysiglinewithargsret{\sphinxbfcode{class }\sphinxcode{salabim.}\sphinxbfcode{Uniform}}{\emph{lowerbound}, \emph{upperbound=None}, \emph{time\_unit=None}, \emph{randomstream=None}, \emph{env=None}}{}
uniform distribution
\begin{quote}\begin{description}
\item[{Parameters}] \leavevmode\begin{itemize}
\item {} 
\sphinxstyleliteralstrong{lowerbound} (\sphinxstyleliteralemphasis{float}) \textendash{} lowerbound of the distribution

\item {} 
\sphinxstyleliteralstrong{upperbound} (\sphinxstyleliteralemphasis{float}) \textendash{} upperbound of the distribution 
if omitted, lowerbound will be used 
must be \textgreater{}= lowerbound

\item {} 
\sphinxstyleliteralstrong{time\_unit} (\sphinxstyleliteralemphasis{str}) \textendash{} specifies the time unit 
must be one of ‘years’, ‘weeks’, ‘days’, ‘hours’, ‘minutes’, ‘seconds’, ‘milliseconds’, ‘microseconds’ 
default : no conversion 

\item {} 
\sphinxstyleliteralstrong{randomstream} (\sphinxstyleliteralemphasis{randomstream}) \textendash{} randomstream to be used 
if omitted, random will be used 
if used as random.Random(12299)
it assigns a new stream with the specified seed

\item {} 
\sphinxstyleliteralstrong{env} ({\hyperref[\detokenize{Reference:salabim.Environment}]{\sphinxcrossref{\sphinxstyleliteralemphasis{Environment}}}}) \textendash{} environment where the distribution is defined 
if omitted, default\_env will be used

\end{itemize}

\end{description}\end{quote}
\index{mean() (salabim.Uniform method)}

\begin{fulllineitems}
\phantomsection\label{\detokenize{Reference:salabim.Uniform.mean}}\pysiglinewithargsret{\sphinxbfcode{mean}}{}{}~\begin{quote}\begin{description}
\item[{Returns}] \leavevmode
\sphinxstylestrong{Mean of the distribution}

\item[{Return type}] \leavevmode
float

\end{description}\end{quote}

\end{fulllineitems}

\index{print\_info() (salabim.Uniform method)}

\begin{fulllineitems}
\phantomsection\label{\detokenize{Reference:salabim.Uniform.print_info}}\pysiglinewithargsret{\sphinxbfcode{print\_info}}{\emph{as\_str=False}, \emph{file=None}}{}
prints information about the distribution
\begin{quote}\begin{description}
\item[{Parameters}] \leavevmode\begin{itemize}
\item {} 
\sphinxstyleliteralstrong{as\_str} (\sphinxstyleliteralemphasis{bool}) \textendash{} if False (default), print the info
if True, return a string containing the info

\item {} 
\sphinxstyleliteralstrong{file} (\sphinxstyleliteralemphasis{file}) \textendash{} if None(default), all output is directed to stdout 
otherwise, the output is directed to the file

\end{itemize}

\item[{Returns}] \leavevmode
\sphinxstylestrong{info (if as\_str is True)}

\item[{Return type}] \leavevmode
str

\end{description}\end{quote}

\end{fulllineitems}

\index{sample() (salabim.Uniform method)}

\begin{fulllineitems}
\phantomsection\label{\detokenize{Reference:salabim.Uniform.sample}}\pysiglinewithargsret{\sphinxbfcode{sample}}{}{}~\begin{quote}\begin{description}
\item[{Returns}] \leavevmode
\sphinxstylestrong{Sample of the distribution}

\item[{Return type}] \leavevmode
float

\end{description}\end{quote}

\end{fulllineitems}


\end{fulllineitems}

\index{Weibull (class in salabim)}

\begin{fulllineitems}
\phantomsection\label{\detokenize{Reference:salabim.Weibull}}\pysiglinewithargsret{\sphinxbfcode{class }\sphinxcode{salabim.}\sphinxbfcode{Weibull}}{\emph{scale}, \emph{shape}, \emph{time\_unit=None}, \emph{randomstream=None}, \emph{env=None}}{}
weibull distribution
\begin{quote}\begin{description}
\item[{Parameters}] \leavevmode\begin{itemize}
\item {} 
\sphinxstyleliteralstrong{scale} (\sphinxstyleliteralemphasis{float}) \textendash{} scale of the distribution (alpha or k)

\item {} 
\sphinxstyleliteralstrong{shape} (\sphinxstyleliteralemphasis{float}) \textendash{} shape of the distribution (beta or lambda)\textbar{}n\textbar{}
should be \textgreater{}0

\item {} 
\sphinxstyleliteralstrong{time\_unit} (\sphinxstyleliteralemphasis{str}) \textendash{} specifies the time unit 
must be one of ‘years’, ‘weeks’, ‘days’, ‘hours’, ‘minutes’, ‘seconds’, ‘milliseconds’, ‘microseconds’ 
default : no conversion 

\item {} 
\sphinxstyleliteralstrong{randomstream} (\sphinxstyleliteralemphasis{randomstream}) \textendash{} randomstream to be used 
if omitted, random will be used 
if used as random.Random(12299)
it assigns a new stream with the specified seed

\item {} 
\sphinxstyleliteralstrong{env} ({\hyperref[\detokenize{Reference:salabim.Environment}]{\sphinxcrossref{\sphinxstyleliteralemphasis{Environment}}}}) \textendash{} environment where the distribution is defined 
if omitted, default\_env will be used

\end{itemize}

\end{description}\end{quote}
\index{mean() (salabim.Weibull method)}

\begin{fulllineitems}
\phantomsection\label{\detokenize{Reference:salabim.Weibull.mean}}\pysiglinewithargsret{\sphinxbfcode{mean}}{}{}~\begin{quote}\begin{description}
\item[{Returns}] \leavevmode
\sphinxstylestrong{Mean of the distribution}

\item[{Return type}] \leavevmode
float

\end{description}\end{quote}

\end{fulllineitems}

\index{print\_info() (salabim.Weibull method)}

\begin{fulllineitems}
\phantomsection\label{\detokenize{Reference:salabim.Weibull.print_info}}\pysiglinewithargsret{\sphinxbfcode{print\_info}}{\emph{as\_str=False}, \emph{file=None}}{}
prints information about the distribution
\begin{quote}\begin{description}
\item[{Parameters}] \leavevmode\begin{itemize}
\item {} 
\sphinxstyleliteralstrong{as\_str} (\sphinxstyleliteralemphasis{bool}) \textendash{} if False (default), print the info
if True, return a string containing the info

\item {} 
\sphinxstyleliteralstrong{file} (\sphinxstyleliteralemphasis{file}) \textendash{} if None(default), all output is directed to stdout 
otherwise, the output is directed to the file

\end{itemize}

\item[{Returns}] \leavevmode
\sphinxstylestrong{info (if as\_str is True)}

\item[{Return type}] \leavevmode
str

\end{description}\end{quote}

\end{fulllineitems}

\index{sample() (salabim.Weibull method)}

\begin{fulllineitems}
\phantomsection\label{\detokenize{Reference:salabim.Weibull.sample}}\pysiglinewithargsret{\sphinxbfcode{sample}}{}{}~\begin{quote}\begin{description}
\item[{Returns}] \leavevmode
\sphinxstylestrong{Sample of the distribution}

\item[{Return type}] \leavevmode
float

\end{description}\end{quote}

\end{fulllineitems}


\end{fulllineitems}



\section{Component}
\label{\detokenize{Reference:component}}\index{Component (class in salabim)}

\begin{fulllineitems}
\phantomsection\label{\detokenize{Reference:salabim.Component}}\pysiglinewithargsret{\sphinxbfcode{class }\sphinxcode{salabim.}\sphinxbfcode{Component}}{\emph{name=None}, \emph{at=None}, \emph{delay=None}, \emph{urgent=None}, \emph{process=None}, \emph{suppress\_trace=False}, \emph{suppress\_pause\_at\_step=False}, \emph{skip\_standby=False}, \emph{mode=None}, \emph{env=None}, \emph{**kwargs}}{}
Component object

A salabim component is used as component (primarily for queueing)
or as a component with a process 
Usually, a component will be defined as a subclass of Component.
\begin{quote}\begin{description}
\item[{Parameters}] \leavevmode\begin{itemize}
\item {} 
\sphinxstyleliteralstrong{name} (\sphinxstyleliteralemphasis{str}) \textendash{} name of the component. 
if the name ends with a period (.),
auto serializing will be applied 
if the name end with a comma,
auto serializing starting at 1 will be applied 
if omitted, the name will be derived from the class
it is defined in (lowercased)

\item {} 
\sphinxstyleliteralstrong{at} (\sphinxstyleliteralemphasis{float}) \textendash{} schedule time 
if omitted, now is used

\item {} 
\sphinxstyleliteralstrong{delay} (\sphinxstyleliteralemphasis{float}) \textendash{} schedule with a delay 
if omitted, no delay

\item {} 
\sphinxstyleliteralstrong{urgent} (\sphinxstyleliteralemphasis{bool}) \textendash{} urgency indicator 
if False (default), the component will be scheduled
behind all other components scheduled
for the same time 
if True, the component will be scheduled
in front of all components scheduled
for the same time

\item {} 
\sphinxstyleliteralstrong{process} (\sphinxstyleliteralemphasis{str}) \textendash{} name of process to be started. 
if None (default), it will try to start self.process() 
if ‘’, no process will be started even if self.process() exists,
i.e. become a data component. 
note that the function \sphinxstyleemphasis{must} be a generator,
i.e. contains at least one yield.

\item {} 
\sphinxstyleliteralstrong{suppress\_trace} (\sphinxstyleliteralemphasis{bool}) \textendash{} suppress\_trace indicator 
if True, this component will be excluded from the trace 
If False (default), the component will be traced 
Can be queried or set later with the suppress\_trace method.

\item {} 
\sphinxstyleliteralstrong{suppress\_pause\_at\_step} (\sphinxstyleliteralemphasis{bool}) \textendash{} suppress\_pause\_at\_step indicator 
if True, if this component becomes current, do not pause when stepping 
If False (default), the component will be paused when stepping 
Can be queried or set later with the suppress\_pause\_at\_step method.

\item {} 
\sphinxstyleliteralstrong{skip\_standby} (\sphinxstyleliteralemphasis{bool}) \textendash{} skip\_standby indicator 
if True, after this component became current, do not activate standby components 
If False (default), after the component became current  activate standby components 
Can be queried or set later with the skip\_standby method.

\item {} 
\sphinxstyleliteralstrong{mode} (\sphinxstyleliteralemphasis{str preferred}) \textendash{} mode 
will be used in trace and can be used in animations 
if omitted, the mode will be None. 
also mode\_time will be set to now.

\item {} 
\sphinxstyleliteralstrong{env} ({\hyperref[\detokenize{Reference:salabim.Environment}]{\sphinxcrossref{\sphinxstyleliteralemphasis{Environment}}}}) \textendash{} environment where the component is defined 
if omitted, default\_env will be used

\end{itemize}

\end{description}\end{quote}
\index{activate() (salabim.Component method)}

\begin{fulllineitems}
\phantomsection\label{\detokenize{Reference:salabim.Component.activate}}\pysiglinewithargsret{\sphinxbfcode{activate}}{\emph{at=None}, \emph{delay=0}, \emph{urgent=False}, \emph{process=None}, \emph{keep\_request=False}, \emph{keep\_wait=False}, \emph{mode=None}, \emph{**kwargs}}{}
activate component
\begin{quote}\begin{description}
\item[{Parameters}] \leavevmode\begin{itemize}
\item {} 
\sphinxstyleliteralstrong{at} (\sphinxstyleliteralemphasis{float}) \textendash{} schedule time 
if omitted, now is used 
inf is allowed

\item {} 
\sphinxstyleliteralstrong{delay} (\sphinxstyleliteralemphasis{float}) \textendash{} schedule with a delay 
if omitted, no delay

\item {} 
\sphinxstyleliteralstrong{urgent} (\sphinxstyleliteralemphasis{bool}) \textendash{} urgency indicator 
if False (default), the component will be scheduled
behind all other components scheduled
for the same time 
if True, the component will be scheduled
in front of all components scheduled
for the same time

\item {} 
\sphinxstyleliteralstrong{process} (\sphinxstyleliteralemphasis{str}) \textendash{} name of process to be started. 
if None (default), process will not be changed 
if the component is a data component, the
generator function process will be used as the default process. 
note that the function \sphinxstyleemphasis{must} be a generator,
i.e. contains at least one yield.

\item {} 
\sphinxstyleliteralstrong{keep\_request} (\sphinxstyleliteralemphasis{bool}) \textendash{} this affects only components that are requesting. 
if True, the requests will be kept and thus the status will remain requesting 
if False (the default), the request(s) will be canceled and the status will become scheduled

\item {} 
\sphinxstyleliteralstrong{keep\_wait} (\sphinxstyleliteralemphasis{bool}) \textendash{} this affects only components that are waiting. 
if True, the waits will be kept and thus the status will remain waiting 
if False (the default), the wait(s) will be canceled and the status will become scheduled

\item {} 
\sphinxstyleliteralstrong{mode} (\sphinxstyleliteralemphasis{str preferred}) \textendash{} mode 
will be used in the trace and can be used in animations 
if nothing specified, the mode will be unchanged. 
also mode\_time will be set to now, if mode is set.

\end{itemize}

\end{description}\end{quote}

\begin{sphinxadmonition}{note}{Note:}
if to be applied to the current component, use \sphinxcode{yield self.activate()}. 
if both at and delay are specified, the component becomes current at the sum
of the two values.
\end{sphinxadmonition}

\end{fulllineitems}

\index{animation\_objects() (salabim.Component method)}

\begin{fulllineitems}
\phantomsection\label{\detokenize{Reference:salabim.Component.animation_objects}}\pysiglinewithargsret{\sphinxbfcode{animation\_objects}}{\emph{id}}{}
defines how to display a component in AnimateQueue
\begin{quote}\begin{description}
\item[{Parameters}] \leavevmode
\sphinxstyleliteralstrong{id} (\sphinxstyleliteralemphasis{any}) \textendash{} id as given by AnimateQueue. Note that by default this the reference to the AnimateQueue object.

\item[{Returns}] \leavevmode
size\_x : how much to displace the next component in x-direction, if applicable 
size\_y : how much to displace the next component in y-direction, if applicable 
animation objects : instances of Animate class 
default behaviour: 
square of size 40 (displacements 50), with the sequence number centered.

\item[{Return type}] \leavevmode
List or tuple containg 

\end{description}\end{quote}

\begin{sphinxadmonition}{note}{Note:}
If you override this method, be sure to use the same header, either with or without the id parameter. 
\end{sphinxadmonition}

\end{fulllineitems}

\index{base\_name() (salabim.Component method)}

\begin{fulllineitems}
\phantomsection\label{\detokenize{Reference:salabim.Component.base_name}}\pysiglinewithargsret{\sphinxbfcode{base\_name}}{}{}~\begin{quote}\begin{description}
\item[{Returns}] \leavevmode
\sphinxstylestrong{base name of the component (the name used at initialization)}

\item[{Return type}] \leavevmode
str

\end{description}\end{quote}

\end{fulllineitems}

\index{cancel() (salabim.Component method)}

\begin{fulllineitems}
\phantomsection\label{\detokenize{Reference:salabim.Component.cancel}}\pysiglinewithargsret{\sphinxbfcode{cancel}}{\emph{mode=None}}{}
cancel component (makes the component data)
\begin{quote}\begin{description}
\item[{Parameters}] \leavevmode
\sphinxstyleliteralstrong{mode} (\sphinxstyleliteralemphasis{str preferred}) \textendash{} mode 
will be used in trace and can be used in animations 
if nothing specified, the mode will be unchanged. 
also mode\_time will be set to now, if mode is set.

\end{description}\end{quote}

\begin{sphinxadmonition}{note}{Note:}
if to be used for the current component, use \sphinxcode{yield self.cancel()}.
\end{sphinxadmonition}

\end{fulllineitems}

\index{claimed\_quantity() (salabim.Component method)}

\begin{fulllineitems}
\phantomsection\label{\detokenize{Reference:salabim.Component.claimed_quantity}}\pysiglinewithargsret{\sphinxbfcode{claimed\_quantity}}{\emph{resource}}{}~\begin{quote}\begin{description}
\item[{Parameters}] \leavevmode
\sphinxstyleliteralstrong{resource} (\sphinxstyleliteralemphasis{Resoure}) \textendash{} resource to be queried

\item[{Returns}] \leavevmode
\sphinxstylestrong{the claimed quantity from a resource} \textendash{} if the resource is not claimed, 0 will be returned

\item[{Return type}] \leavevmode
float or int

\end{description}\end{quote}

\end{fulllineitems}

\index{claimed\_resources() (salabim.Component method)}

\begin{fulllineitems}
\phantomsection\label{\detokenize{Reference:salabim.Component.claimed_resources}}\pysiglinewithargsret{\sphinxbfcode{claimed\_resources}}{}{}~\begin{quote}\begin{description}
\item[{Returns}] \leavevmode
\sphinxstylestrong{list of claimed resources}

\item[{Return type}] \leavevmode
list

\end{description}\end{quote}

\end{fulllineitems}

\index{count() (salabim.Component method)}

\begin{fulllineitems}
\phantomsection\label{\detokenize{Reference:salabim.Component.count}}\pysiglinewithargsret{\sphinxbfcode{count}}{\emph{q=None}}{}
queue count
\begin{quote}\begin{description}
\item[{Parameters}] \leavevmode
\sphinxstyleliteralstrong{q} ({\hyperref[\detokenize{Reference:salabim.Queue}]{\sphinxcrossref{\sphinxstyleliteralemphasis{Queue}}}}) \textendash{} queue to check or 
if omitted, the number of queues where the component is in

\item[{Returns}] \leavevmode
\sphinxstylestrong{1 if component is in q, 0 otherwise} \textendash{} 
if q is omitted, the number of queues where the component is in

\item[{Return type}] \leavevmode
int

\end{description}\end{quote}

\end{fulllineitems}

\index{creation\_time() (salabim.Component method)}

\begin{fulllineitems}
\phantomsection\label{\detokenize{Reference:salabim.Component.creation_time}}\pysiglinewithargsret{\sphinxbfcode{creation\_time}}{}{}~\begin{quote}\begin{description}
\item[{Returns}] \leavevmode
\sphinxstylestrong{time the component was created}

\item[{Return type}] \leavevmode
float

\end{description}\end{quote}

\end{fulllineitems}

\index{deregister() (salabim.Component method)}

\begin{fulllineitems}
\phantomsection\label{\detokenize{Reference:salabim.Component.deregister}}\pysiglinewithargsret{\sphinxbfcode{deregister}}{\emph{registry}}{}
deregisters the component in the registry
\begin{quote}\begin{description}
\item[{Parameters}] \leavevmode
\sphinxstyleliteralstrong{registry} (\sphinxstyleliteralemphasis{list}) \textendash{} list of registered components

\item[{Returns}] \leavevmode
\sphinxstylestrong{component (self)}

\item[{Return type}] \leavevmode
{\hyperref[\detokenize{Reference:salabim.Component}]{\sphinxcrossref{Component}}}

\end{description}\end{quote}

\end{fulllineitems}

\index{enter() (salabim.Component method)}

\begin{fulllineitems}
\phantomsection\label{\detokenize{Reference:salabim.Component.enter}}\pysiglinewithargsret{\sphinxbfcode{enter}}{\emph{q}}{}
enters a queue at the tail
\begin{quote}\begin{description}
\item[{Parameters}] \leavevmode
\sphinxstyleliteralstrong{q} ({\hyperref[\detokenize{Reference:salabim.Queue}]{\sphinxcrossref{\sphinxstyleliteralemphasis{Queue}}}}) \textendash{} queue to enter

\end{description}\end{quote}

\begin{sphinxadmonition}{note}{Note:}
the priority will be set to
the priority of the tail component of the queue, if any
or 0 if queue is empty
\end{sphinxadmonition}

\end{fulllineitems}

\index{enter\_at\_head() (salabim.Component method)}

\begin{fulllineitems}
\phantomsection\label{\detokenize{Reference:salabim.Component.enter_at_head}}\pysiglinewithargsret{\sphinxbfcode{enter\_at\_head}}{\emph{q}}{}
enters a queue at the head
\begin{quote}\begin{description}
\item[{Parameters}] \leavevmode
\sphinxstyleliteralstrong{q} ({\hyperref[\detokenize{Reference:salabim.Queue}]{\sphinxcrossref{\sphinxstyleliteralemphasis{Queue}}}}) \textendash{} queue to enter

\end{description}\end{quote}

\begin{sphinxadmonition}{note}{Note:}
the priority will be set to
the priority of the head component of the queue, if any
or 0 if queue is empty
\end{sphinxadmonition}

\end{fulllineitems}

\index{enter\_behind() (salabim.Component method)}

\begin{fulllineitems}
\phantomsection\label{\detokenize{Reference:salabim.Component.enter_behind}}\pysiglinewithargsret{\sphinxbfcode{enter\_behind}}{\emph{q}, \emph{poscomponent}}{}
enters a queue behind a component
\begin{quote}\begin{description}
\item[{Parameters}] \leavevmode\begin{itemize}
\item {} 
\sphinxstyleliteralstrong{q} ({\hyperref[\detokenize{Reference:salabim.Queue}]{\sphinxcrossref{\sphinxstyleliteralemphasis{Queue}}}}) \textendash{} queue to enter

\item {} 
\sphinxstyleliteralstrong{poscomponent} ({\hyperref[\detokenize{Reference:salabim.Component}]{\sphinxcrossref{\sphinxstyleliteralemphasis{Component}}}}) \textendash{} component to be entered behind

\end{itemize}

\end{description}\end{quote}

\begin{sphinxadmonition}{note}{Note:}
the priority will be set to the priority of poscomponent
\end{sphinxadmonition}

\end{fulllineitems}

\index{enter\_in\_front\_of() (salabim.Component method)}

\begin{fulllineitems}
\phantomsection\label{\detokenize{Reference:salabim.Component.enter_in_front_of}}\pysiglinewithargsret{\sphinxbfcode{enter\_in\_front\_of}}{\emph{q}, \emph{poscomponent}}{}
enters a queue in front of a component
\begin{quote}\begin{description}
\item[{Parameters}] \leavevmode\begin{itemize}
\item {} 
\sphinxstyleliteralstrong{q} ({\hyperref[\detokenize{Reference:salabim.Queue}]{\sphinxcrossref{\sphinxstyleliteralemphasis{Queue}}}}) \textendash{} queue to enter

\item {} 
\sphinxstyleliteralstrong{poscomponent} ({\hyperref[\detokenize{Reference:salabim.Component}]{\sphinxcrossref{\sphinxstyleliteralemphasis{Component}}}}) \textendash{} component to be entered in front of

\end{itemize}

\end{description}\end{quote}

\begin{sphinxadmonition}{note}{Note:}
the priority will be set to the priority of poscomponent
\end{sphinxadmonition}

\end{fulllineitems}

\index{enter\_sorted() (salabim.Component method)}

\begin{fulllineitems}
\phantomsection\label{\detokenize{Reference:salabim.Component.enter_sorted}}\pysiglinewithargsret{\sphinxbfcode{enter\_sorted}}{\emph{q}, \emph{priority}}{}
enters a queue, according to the priority
\begin{quote}\begin{description}
\item[{Parameters}] \leavevmode\begin{itemize}
\item {} 
\sphinxstyleliteralstrong{q} ({\hyperref[\detokenize{Reference:salabim.Queue}]{\sphinxcrossref{\sphinxstyleliteralemphasis{Queue}}}}) \textendash{} queue to enter

\item {} 
\sphinxstyleliteralstrong{priority} (\sphinxstyleliteralemphasis{type that can be compared with other priorities in the queue}) \textendash{} priority in the queue

\end{itemize}

\end{description}\end{quote}

\begin{sphinxadmonition}{note}{Note:}
The component is placed just before the first component with a priority \textgreater{} given priority
\end{sphinxadmonition}

\end{fulllineitems}

\index{enter\_time() (salabim.Component method)}

\begin{fulllineitems}
\phantomsection\label{\detokenize{Reference:salabim.Component.enter_time}}\pysiglinewithargsret{\sphinxbfcode{enter\_time}}{\emph{q}}{}~\begin{quote}\begin{description}
\item[{Parameters}] \leavevmode
\sphinxstyleliteralstrong{q} ({\hyperref[\detokenize{Reference:salabim.Queue}]{\sphinxcrossref{\sphinxstyleliteralemphasis{Queue}}}}) \textendash{} queue where component belongs to

\item[{Returns}] \leavevmode
\sphinxstylestrong{time the component entered the queue}

\item[{Return type}] \leavevmode
float

\end{description}\end{quote}

\end{fulllineitems}

\index{failed() (salabim.Component method)}

\begin{fulllineitems}
\phantomsection\label{\detokenize{Reference:salabim.Component.failed}}\pysiglinewithargsret{\sphinxbfcode{failed}}{}{}~\begin{quote}\begin{description}
\item[{Returns}] \leavevmode
\begin{itemize}
\item {} 
\sphinxstylestrong{True, if the latest request/wait has failed (either by timeout or external)} (\sphinxstyleemphasis{bool})

\item {} 
\sphinxstyleemphasis{False, otherwise}

\end{itemize}


\end{description}\end{quote}

\end{fulllineitems}

\index{hold() (salabim.Component method)}

\begin{fulllineitems}
\phantomsection\label{\detokenize{Reference:salabim.Component.hold}}\pysiglinewithargsret{\sphinxbfcode{hold}}{\emph{duration=None}, \emph{till=None}, \emph{urgent=False}, \emph{mode=None}}{}
hold the component
\begin{quote}\begin{description}
\item[{Parameters}] \leavevmode\begin{itemize}
\item {} 
\sphinxstyleliteralstrong{duration} (\sphinxstyleliteralemphasis{float}) \textendash{} specifies the duration 
if omitted, 0 is used 
inf is allowed

\item {} 
\sphinxstyleliteralstrong{till} (\sphinxstyleliteralemphasis{float}) \textendash{} specifies at what time the component will become current 
if omitted, now is used 
inf is allowed

\item {} 
\sphinxstyleliteralstrong{urgent} (\sphinxstyleliteralemphasis{bool}) \textendash{} urgency indicator 
if False (default), the component will be scheduled
behind all other components scheduled
for the same time 
if True, the component will be scheduled
in front of all components scheduled
for the same time

\item {} 
\sphinxstyleliteralstrong{mode} (\sphinxstyleliteralemphasis{str preferred}) \textendash{} mode 
will be used in trace and can be used in animations 
if nothing specified, the mode will be unchanged. 
also mode\_time will be set to now, if mode is set.

\end{itemize}

\end{description}\end{quote}

\begin{sphinxadmonition}{note}{Note:}
if to be used for the current component, use \sphinxcode{yield self.hold(...)}. 

if both duration and till are specified, the component will become current at the sum of
these two.
\end{sphinxadmonition}

\end{fulllineitems}

\index{index() (salabim.Component method)}

\begin{fulllineitems}
\phantomsection\label{\detokenize{Reference:salabim.Component.index}}\pysiglinewithargsret{\sphinxbfcode{index}}{\emph{q}}{}~\begin{quote}\begin{description}
\item[{Parameters}] \leavevmode
\sphinxstyleliteralstrong{q} ({\hyperref[\detokenize{Reference:salabim.Queue}]{\sphinxcrossref{\sphinxstyleliteralemphasis{Queue}}}}) \textendash{} queue to be queried

\item[{Returns}] \leavevmode
\sphinxstylestrong{index of component in q} \textendash{} if component belongs to q 
-1 if component does not belong to q

\item[{Return type}] \leavevmode
int

\end{description}\end{quote}

\end{fulllineitems}

\index{interrupt() (salabim.Component method)}

\begin{fulllineitems}
\phantomsection\label{\detokenize{Reference:salabim.Component.interrupt}}\pysiglinewithargsret{\sphinxbfcode{interrupt}}{\emph{mode=None}}{}
interrupt the component
\begin{quote}\begin{description}
\item[{Parameters}] \leavevmode
\sphinxstyleliteralstrong{mode} (\sphinxstyleliteralemphasis{str preferred}) \textendash{} mode 
will be used in trace and can be used in animations 
if nothing is specified, the mode will be unchanged. 
also mode\_time will be set to now, if mode is set.

\end{description}\end{quote}

\begin{sphinxadmonition}{note}{Note:}
Cannot be applied on the current component. 
Use resume() to resume
\end{sphinxadmonition}

\end{fulllineitems}

\index{interrupt\_level() (salabim.Component method)}

\begin{fulllineitems}
\phantomsection\label{\detokenize{Reference:salabim.Component.interrupt_level}}\pysiglinewithargsret{\sphinxbfcode{interrupt\_level}}{}{}
returns interrupt level of an interrupted component 
non interrupted components return 0

\end{fulllineitems}

\index{interrupted\_status() (salabim.Component method)}

\begin{fulllineitems}
\phantomsection\label{\detokenize{Reference:salabim.Component.interrupted_status}}\pysiglinewithargsret{\sphinxbfcode{interrupted\_status}}{}{}
returns the original status of an interrupted component
\begin{description}
\item[{possible values are}] \leavevmode\begin{itemize}
\item {} 
passive

\item {} 
scheduled

\item {} 
requesting

\item {} 
waiting

\item {} 
standby

\end{itemize}

\end{description}

\end{fulllineitems}

\index{iscurrent() (salabim.Component method)}

\begin{fulllineitems}
\phantomsection\label{\detokenize{Reference:salabim.Component.iscurrent}}\pysiglinewithargsret{\sphinxbfcode{iscurrent}}{}{}~\begin{quote}\begin{description}
\item[{Returns}] \leavevmode
\sphinxstylestrong{True if status is current, False otherwise}

\item[{Return type}] \leavevmode
bool

\end{description}\end{quote}

\begin{sphinxadmonition}{note}{Note:}
Be sure to always include the parentheses, otherwise the result will be always True!
\end{sphinxadmonition}

\end{fulllineitems}

\index{isdata() (salabim.Component method)}

\begin{fulllineitems}
\phantomsection\label{\detokenize{Reference:salabim.Component.isdata}}\pysiglinewithargsret{\sphinxbfcode{isdata}}{}{}~\begin{quote}\begin{description}
\item[{Returns}] \leavevmode
\sphinxstylestrong{True if status is data, False otherwise}

\item[{Return type}] \leavevmode
bool

\end{description}\end{quote}

\begin{sphinxadmonition}{note}{Note:}
Be sure to always include the parentheses, otherwise the result will be always True!
\end{sphinxadmonition}

\end{fulllineitems}

\index{isinterrupted() (salabim.Component method)}

\begin{fulllineitems}
\phantomsection\label{\detokenize{Reference:salabim.Component.isinterrupted}}\pysiglinewithargsret{\sphinxbfcode{isinterrupted}}{}{}~\begin{quote}\begin{description}
\item[{Returns}] \leavevmode
\sphinxstylestrong{True if status is interrupted, False otherwise}

\item[{Return type}] \leavevmode
bool

\end{description}\end{quote}

\begin{sphinxadmonition}{note}{Note:}
Be sure to always include the parentheses, otherwise the result will be always True
\end{sphinxadmonition}

\end{fulllineitems}

\index{ispassive() (salabim.Component method)}

\begin{fulllineitems}
\phantomsection\label{\detokenize{Reference:salabim.Component.ispassive}}\pysiglinewithargsret{\sphinxbfcode{ispassive}}{}{}~\begin{quote}\begin{description}
\item[{Returns}] \leavevmode
\sphinxstylestrong{True if status is passive, False otherwise}

\item[{Return type}] \leavevmode
bool

\end{description}\end{quote}

\begin{sphinxadmonition}{note}{Note:}
Be sure to always include the parentheses, otherwise the result will be always True!
\end{sphinxadmonition}

\end{fulllineitems}

\index{isrequesting() (salabim.Component method)}

\begin{fulllineitems}
\phantomsection\label{\detokenize{Reference:salabim.Component.isrequesting}}\pysiglinewithargsret{\sphinxbfcode{isrequesting}}{}{}~\begin{quote}\begin{description}
\item[{Returns}] \leavevmode
\sphinxstylestrong{True if status is requesting, False otherwise}

\item[{Return type}] \leavevmode
bool

\end{description}\end{quote}

\begin{sphinxadmonition}{note}{Note:}
Be sure to always include the parentheses, otherwise the result will be always True!
\end{sphinxadmonition}

\end{fulllineitems}

\index{isscheduled() (salabim.Component method)}

\begin{fulllineitems}
\phantomsection\label{\detokenize{Reference:salabim.Component.isscheduled}}\pysiglinewithargsret{\sphinxbfcode{isscheduled}}{}{}~\begin{quote}\begin{description}
\item[{Returns}] \leavevmode
\sphinxstylestrong{True if status is scheduled, False otherwise}

\item[{Return type}] \leavevmode
bool

\end{description}\end{quote}

\begin{sphinxadmonition}{note}{Note:}
Be sure to always include the parentheses, otherwise the result will be always True!
\end{sphinxadmonition}

\end{fulllineitems}

\index{isstandby() (salabim.Component method)}

\begin{fulllineitems}
\phantomsection\label{\detokenize{Reference:salabim.Component.isstandby}}\pysiglinewithargsret{\sphinxbfcode{isstandby}}{}{}~\begin{quote}\begin{description}
\item[{Returns}] \leavevmode
\sphinxstylestrong{True if status is standby, False otherwise}

\item[{Return type}] \leavevmode
bool

\end{description}\end{quote}

\begin{sphinxadmonition}{note}{Note:}
Be sure to always include the parentheses, otherwise the result will be always True
\end{sphinxadmonition}

\end{fulllineitems}

\index{iswaiting() (salabim.Component method)}

\begin{fulllineitems}
\phantomsection\label{\detokenize{Reference:salabim.Component.iswaiting}}\pysiglinewithargsret{\sphinxbfcode{iswaiting}}{}{}~\begin{quote}\begin{description}
\item[{Returns}] \leavevmode
\sphinxstylestrong{True if status is waiting, False otherwise}

\item[{Return type}] \leavevmode
bool

\end{description}\end{quote}

\begin{sphinxadmonition}{note}{Note:}
Be sure to always include the parentheses, otherwise the result will be always True!
\end{sphinxadmonition}

\end{fulllineitems}

\index{leave() (salabim.Component method)}

\begin{fulllineitems}
\phantomsection\label{\detokenize{Reference:salabim.Component.leave}}\pysiglinewithargsret{\sphinxbfcode{leave}}{\emph{q=None}}{}
leave queue
\begin{quote}\begin{description}
\item[{Parameters}] \leavevmode
\sphinxstyleliteralstrong{q} ({\hyperref[\detokenize{Reference:salabim.Queue}]{\sphinxcrossref{\sphinxstyleliteralemphasis{Queue}}}}) \textendash{} queue to leave

\end{description}\end{quote}

\begin{sphinxadmonition}{note}{Note:}
statistics are updated accordingly
\end{sphinxadmonition}

\end{fulllineitems}

\index{mode() (salabim.Component method)}

\begin{fulllineitems}
\phantomsection\label{\detokenize{Reference:salabim.Component.mode}}\pysiglinewithargsret{\sphinxbfcode{mode}}{\emph{value=None}}{}~\begin{quote}\begin{description}
\item[{Parameters}] \leavevmode
\sphinxstyleliteralstrong{value} (\sphinxstyleliteralemphasis{any}\sphinxstyleliteralemphasis{, }\sphinxstyleliteralemphasis{str recommended}) \textendash{} new mode 
if omitted, no change 
mode\_time will be set if a new mode is specified

\item[{Returns}] \leavevmode
\sphinxstylestrong{mode of the component} \textendash{} the mode is useful for tracing and animations. 
Usually the mode will be set in a call to passivate, hold, activate, request or standby.

\item[{Return type}] \leavevmode
any, usually str

\end{description}\end{quote}

\end{fulllineitems}

\index{mode\_time() (salabim.Component method)}

\begin{fulllineitems}
\phantomsection\label{\detokenize{Reference:salabim.Component.mode_time}}\pysiglinewithargsret{\sphinxbfcode{mode\_time}}{}{}~\begin{quote}\begin{description}
\item[{Returns}] \leavevmode
\sphinxstylestrong{time the component got it’s latest mode} \textendash{} For a new component this is
the time the component was created. 
this function is particularly useful for animations.

\item[{Return type}] \leavevmode
float

\end{description}\end{quote}

\end{fulllineitems}

\index{name() (salabim.Component method)}

\begin{fulllineitems}
\phantomsection\label{\detokenize{Reference:salabim.Component.name}}\pysiglinewithargsret{\sphinxbfcode{name}}{\emph{value=None}}{}~\begin{quote}\begin{description}
\item[{Parameters}] \leavevmode
\sphinxstyleliteralstrong{value} (\sphinxstyleliteralemphasis{str}) \textendash{} new name of the component
if omitted, no change

\item[{Returns}] \leavevmode
\sphinxstylestrong{Name of the component}

\item[{Return type}] \leavevmode
str

\end{description}\end{quote}

\begin{sphinxadmonition}{note}{Note:}
base\_name and sequence\_number are not affected if the name is changed
\end{sphinxadmonition}

\end{fulllineitems}

\index{passivate() (salabim.Component method)}

\begin{fulllineitems}
\phantomsection\label{\detokenize{Reference:salabim.Component.passivate}}\pysiglinewithargsret{\sphinxbfcode{passivate}}{\emph{mode=None}}{}
passivate the component
\begin{quote}\begin{description}
\item[{Parameters}] \leavevmode
\sphinxstyleliteralstrong{mode} (\sphinxstyleliteralemphasis{str preferred}) \textendash{} mode 
will be used in trace and can be used in animations 
if nothing is specified, the mode will be unchanged. 
also mode\_time will be set to now, if mode is set.

\end{description}\end{quote}

\begin{sphinxadmonition}{note}{Note:}
if to be used for the current component (nearly always the case), use \sphinxcode{yield self.passivate()}.
\end{sphinxadmonition}

\end{fulllineitems}

\index{predecessor() (salabim.Component method)}

\begin{fulllineitems}
\phantomsection\label{\detokenize{Reference:salabim.Component.predecessor}}\pysiglinewithargsret{\sphinxbfcode{predecessor}}{\emph{q}}{}~\begin{quote}\begin{description}
\item[{Parameters}] \leavevmode\begin{itemize}
\item {} 
\sphinxstyleliteralstrong{q} ({\hyperref[\detokenize{Reference:salabim.Queue}]{\sphinxcrossref{\sphinxstyleliteralemphasis{Queue}}}}) \textendash{} queue where the component belongs to

\item {} 
\sphinxstyleliteralstrong{Returns} ({\hyperref[\detokenize{Reference:salabim.Component}]{\sphinxcrossref{\sphinxstyleliteralemphasis{Component}}}}) \textendash{} predecessor of the component in the queue
if component is not at the head. 
returns None if component is at the head.

\end{itemize}

\end{description}\end{quote}

\end{fulllineitems}

\index{print\_info() (salabim.Component method)}

\begin{fulllineitems}
\phantomsection\label{\detokenize{Reference:salabim.Component.print_info}}\pysiglinewithargsret{\sphinxbfcode{print\_info}}{\emph{as\_str=False}, \emph{file=None}}{}
prints information about the component
\begin{quote}\begin{description}
\item[{Parameters}] \leavevmode\begin{itemize}
\item {} 
\sphinxstyleliteralstrong{as\_str} (\sphinxstyleliteralemphasis{bool}) \textendash{} if False (default), print the info
if True, return a string containing the info

\item {} 
\sphinxstyleliteralstrong{file} (\sphinxstyleliteralemphasis{file}) \textendash{} if None(default), all output is directed to stdout 
otherwise, the output is directed to the file

\end{itemize}

\item[{Returns}] \leavevmode
\sphinxstylestrong{info (if as\_str is True)}

\item[{Return type}] \leavevmode
str

\end{description}\end{quote}

\end{fulllineitems}

\index{priority() (salabim.Component method)}

\begin{fulllineitems}
\phantomsection\label{\detokenize{Reference:salabim.Component.priority}}\pysiglinewithargsret{\sphinxbfcode{priority}}{\emph{q}, \emph{priority=None}}{}
gets/sets the priority of a component in a queue
\begin{quote}\begin{description}
\item[{Parameters}] \leavevmode\begin{itemize}
\item {} 
\sphinxstyleliteralstrong{q} ({\hyperref[\detokenize{Reference:salabim.Queue}]{\sphinxcrossref{\sphinxstyleliteralemphasis{Queue}}}}) \textendash{} queue where the component belongs to

\item {} 
\sphinxstyleliteralstrong{priority} (\sphinxstyleliteralemphasis{type that can be compared with other priorities in the queue}) \textendash{} priority in queue 
if omitted, no change

\end{itemize}

\item[{Returns}] \leavevmode
\sphinxstylestrong{the priority of the component in the queue}

\item[{Return type}] \leavevmode
float

\end{description}\end{quote}

\begin{sphinxadmonition}{note}{Note:}
if you change the priority, the order of the queue may change
\end{sphinxadmonition}

\end{fulllineitems}

\index{queues() (salabim.Component method)}

\begin{fulllineitems}
\phantomsection\label{\detokenize{Reference:salabim.Component.queues}}\pysiglinewithargsret{\sphinxbfcode{queues}}{}{}~\begin{quote}\begin{description}
\item[{Returns}] \leavevmode
\sphinxstylestrong{set of queues where the component belongs to}

\item[{Return type}] \leavevmode
{\hyperref[\detokenize{Reference:salabim.State.set}]{\sphinxcrossref{set}}}

\end{description}\end{quote}

\end{fulllineitems}

\index{register() (salabim.Component method)}

\begin{fulllineitems}
\phantomsection\label{\detokenize{Reference:salabim.Component.register}}\pysiglinewithargsret{\sphinxbfcode{register}}{\emph{registry}}{}
registers the component in the registry
\begin{quote}\begin{description}
\item[{Parameters}] \leavevmode
\sphinxstyleliteralstrong{registry} (\sphinxstyleliteralemphasis{list}) \textendash{} list of (to be) registered objects

\item[{Returns}] \leavevmode
\sphinxstylestrong{component (self)}

\item[{Return type}] \leavevmode
{\hyperref[\detokenize{Reference:salabim.Component}]{\sphinxcrossref{Component}}}

\end{description}\end{quote}

\begin{sphinxadmonition}{note}{Note:}
Use Component.deregister if component does not longer need to be registered.
\end{sphinxadmonition}

\end{fulllineitems}

\index{release() (salabim.Component method)}

\begin{fulllineitems}
\phantomsection\label{\detokenize{Reference:salabim.Component.release}}\pysiglinewithargsret{\sphinxbfcode{release}}{\emph{*args}}{}
release a quantity from a resource or resources
\begin{quote}\begin{description}
\item[{Parameters}] \leavevmode
\sphinxstyleliteralstrong{args} (\sphinxstyleliteralemphasis{sequence of items}\sphinxstyleliteralemphasis{, }\sphinxstyleliteralemphasis{where each items can be}) \textendash{} \begin{itemize}
\item {} 
a resources, where quantity=current claimed quantity

\item {} 
a tuple/list containing a resource and the quantity to be released

\end{itemize}


\end{description}\end{quote}

\begin{sphinxadmonition}{note}{Note:}
It is not possible to release from an anonymous resource, this way.
Use Resource.release() in that case.
\end{sphinxadmonition}
\paragraph{Example}

yield self.request(r1,(r2,2),(r3,3,100)) 
\textendash{}\textgreater{} requests 1 from r1, 2 from r2 and 3 from r3 with priority 100 

c1.release 
\textendash{}\textgreater{} releases 1 from r1, 2 from r2 and 3 from r3 

yield self.request(r1,(r2,2),(r3,3,100)) 
c1.release((r2,1)) 
\textendash{}\textgreater{} releases 1 from r2 

yield self.request(r1,(r2,2),(r3,3,100)) 
c1.release((r2,1),r3) 
\textendash{}\textgreater{} releases 2 from r2,and 3 from r3

\end{fulllineitems}

\index{remaining\_duration() (salabim.Component method)}

\begin{fulllineitems}
\phantomsection\label{\detokenize{Reference:salabim.Component.remaining_duration}}\pysiglinewithargsret{\sphinxbfcode{remaining\_duration}}{\emph{value=None}, \emph{urgent=False}}{}~\begin{quote}\begin{description}
\item[{Parameters}] \leavevmode\begin{itemize}
\item {} 
\sphinxstyleliteralstrong{value} (\sphinxstyleliteralemphasis{float}) \textendash{} set the remaining\_duration 
The action depends on the status where the component is in: 
- passive: the remaining duration is update according to the given value 
- standby and current: not allowed 
- scheduled: the component is rescheduled according to the given value 
- waiting or requesting: the fail\_at is set according to the given value 
- interrupted: the remaining\_duration is updated according to the given value 

\item {} 
\sphinxstyleliteralstrong{urgent} (\sphinxstyleliteralemphasis{bool}) \textendash{} urgency indicator 
if False (default), the component will be scheduled
behind all other components scheduled
for the same time 
if True, the component will be scheduled
in front of all components scheduled
for the same time

\end{itemize}

\item[{Returns}] \leavevmode
\sphinxstylestrong{remaining duration} \textendash{} if passive, remaining time at time of passivate 
if scheduled, remaing time till scheduled time 
if requesting or waiting, time till fail\_at time 
else: 0

\item[{Return type}] \leavevmode
float

\end{description}\end{quote}

\begin{sphinxadmonition}{note}{Note:}
This method is usefu for interrupting a process and then resuming it,
after some (breakdown) time
\end{sphinxadmonition}

\end{fulllineitems}

\index{request() (salabim.Component method)}

\begin{fulllineitems}
\phantomsection\label{\detokenize{Reference:salabim.Component.request}}\pysiglinewithargsret{\sphinxbfcode{request}}{\emph{*args}, \emph{**kwargs}}{}
request from a resource or resources
\begin{quote}\begin{description}
\item[{Parameters}] \leavevmode\begin{itemize}
\item {} 
\sphinxstyleliteralstrong{args} (\sphinxstyleliteralemphasis{sequence of items where each item can be:}) \textendash{} \begin{itemize}
\item {} 
resource, where quantity=1, priority=tail of requesters queue

\item {} \begin{description}
\item[{tuples/list containing a resource, a quantity and optionally a priority.}] \leavevmode
if the priority is not specified, the request
for the resource be added to the tail of
the requesters queue 

\end{description}

\end{itemize}


\item {} 
\sphinxstyleliteralstrong{fail\_at} (\sphinxstyleliteralemphasis{float}) \textendash{} time out 
if the request is not honored before fail\_at,
the request will be cancelled and the
parameter failed will be set. 
if not specified, the request will not time out.

\item {} 
\sphinxstyleliteralstrong{fail\_delay} (\sphinxstyleliteralemphasis{float}) \textendash{} time out 
if the request is not honored before now+fail\_delay,
the request will be cancelled and the
parameter failed will be set. 
if not specified, the request will not time out.

\item {} 
\sphinxstyleliteralstrong{mode} (\sphinxstyleliteralemphasis{str preferred}) \textendash{} mode 
will be used in trace and can be used in animations 
if nothing specified, the mode will be unchanged. 
also mode\_time will be set to now, if mode is set.

\end{itemize}

\end{description}\end{quote}

\begin{sphinxadmonition}{note}{Note:}
Not allowed for data components or main.

If to be used for the current component
(which will be nearly always the case),
use \sphinxcode{yield self.request(...)}.

If the same resource is specified more that once, the quantities are summed 

The requested quantity may exceed the current capacity of a resource 

The parameter failed will be reset by a calling request or wait
\end{sphinxadmonition}
\paragraph{Example}

\sphinxcode{yield self.request(r1)} 
\textendash{}\textgreater{} requests 1 from r1 
\sphinxcode{yield self.request(r1,r2)} 
\textendash{}\textgreater{} requests 1 from r1 and 1 from r2 
\sphinxcode{yield self.request(r1,(r2,2),(r3,3,100))} 
\textendash{}\textgreater{} requests 1 from r1, 2 from r2 and 3 from r3 with priority 100 
\sphinxcode{yield self.request((r1,1),(r2,2))} 
\textendash{}\textgreater{} requests 1 from r1, 2 from r2 

\end{fulllineitems}

\index{requested\_quantity() (salabim.Component method)}

\begin{fulllineitems}
\phantomsection\label{\detokenize{Reference:salabim.Component.requested_quantity}}\pysiglinewithargsret{\sphinxbfcode{requested\_quantity}}{\emph{resource}}{}~\begin{quote}\begin{description}
\item[{Parameters}] \leavevmode
\sphinxstyleliteralstrong{resource} (\sphinxstyleliteralemphasis{Resoure}) \textendash{} resource to be queried

\item[{Returns}] \leavevmode
\sphinxstylestrong{the requested (not yet honored) quantity from a resource} \textendash{} if there is no request for the resource, 0 will be returned

\item[{Return type}] \leavevmode
float or int

\end{description}\end{quote}

\end{fulllineitems}

\index{requested\_resources() (salabim.Component method)}

\begin{fulllineitems}
\phantomsection\label{\detokenize{Reference:salabim.Component.requested_resources}}\pysiglinewithargsret{\sphinxbfcode{requested\_resources}}{}{}~\begin{quote}\begin{description}
\item[{Returns}] \leavevmode
\sphinxstylestrong{list of requested resources}

\item[{Return type}] \leavevmode
list

\end{description}\end{quote}

\end{fulllineitems}

\index{resume() (salabim.Component method)}

\begin{fulllineitems}
\phantomsection\label{\detokenize{Reference:salabim.Component.resume}}\pysiglinewithargsret{\sphinxbfcode{resume}}{\emph{all=False}, \emph{mode=None}, \emph{urgent=False}}{}
resumes an interrupted component
\begin{quote}\begin{description}
\item[{Parameters}] \leavevmode\begin{itemize}
\item {} 
\sphinxstyleliteralstrong{all} (\sphinxstyleliteralemphasis{bool}) \textendash{} if True, the component returns to the original status, regardless of the number of interrupt levels 
if False (default), the interrupt level will be decremented and if the level reaches 0,
the component will return to the original status.

\item {} 
\sphinxstyleliteralstrong{mode} (\sphinxstyleliteralemphasis{str preferred}) \textendash{} mode 
will be used in trace and can be used in animations 
if nothing is specified, the mode will be unchanged. 
also mode\_time will be set to now, if mode is set.

\item {} 
\sphinxstyleliteralstrong{urgent} (\sphinxstyleliteralemphasis{bool}) \textendash{} urgency indicator 
if False (default), the component will be scheduled
behind all other components scheduled
for the same time 
if True, the component will be scheduled
in front of all components scheduled
for the same time

\end{itemize}

\end{description}\end{quote}

\begin{sphinxadmonition}{note}{Note:}
Can be only applied to interrupted components. 
\end{sphinxadmonition}

\end{fulllineitems}

\index{running\_process() (salabim.Component method)}

\begin{fulllineitems}
\phantomsection\label{\detokenize{Reference:salabim.Component.running_process}}\pysiglinewithargsret{\sphinxbfcode{running\_process}}{}{}~\begin{quote}\begin{description}
\item[{Returns}] \leavevmode
\sphinxstylestrong{name of the running process} \textendash{} if data component, None

\item[{Return type}] \leavevmode
str

\end{description}\end{quote}

\end{fulllineitems}

\index{scheduled\_time() (salabim.Component method)}

\begin{fulllineitems}
\phantomsection\label{\detokenize{Reference:salabim.Component.scheduled_time}}\pysiglinewithargsret{\sphinxbfcode{scheduled\_time}}{}{}~\begin{quote}\begin{description}
\item[{Returns}] \leavevmode
\sphinxstylestrong{time the component scheduled for, if it is scheduled} \textendash{} returns inf otherwise

\item[{Return type}] \leavevmode
float

\end{description}\end{quote}

\end{fulllineitems}

\index{sequence\_number() (salabim.Component method)}

\begin{fulllineitems}
\phantomsection\label{\detokenize{Reference:salabim.Component.sequence_number}}\pysiglinewithargsret{\sphinxbfcode{sequence\_number}}{}{}~\begin{quote}\begin{description}
\item[{Returns}] \leavevmode
\sphinxstylestrong{sequence\_number of the component} \textendash{} (the sequence number at initialization) 
normally this will be the integer value of a serialized name,
but also non serialized names (without a dotcomma at the end)
will be numbered)

\item[{Return type}] \leavevmode
int

\end{description}\end{quote}

\end{fulllineitems}

\index{setup() (salabim.Component method)}

\begin{fulllineitems}
\phantomsection\label{\detokenize{Reference:salabim.Component.setup}}\pysiglinewithargsret{\sphinxbfcode{setup}}{}{}
called immediately after initialization of a component.

by default this is a dummy method, but it can be overridden.

only keyword arguments will be passed
\paragraph{Example}
\begin{description}
\item[{class Car(sim.Component):}] \leavevmode\begin{description}
\item[{def setup(self, color):}] \leavevmode
self.color = color

\item[{def process(self):}] \leavevmode
…

\end{description}

\end{description}

redcar=Car(color=’red’) 
bluecar=Car(color=’blue’)

\end{fulllineitems}

\index{skip\_standby() (salabim.Component method)}

\begin{fulllineitems}
\phantomsection\label{\detokenize{Reference:salabim.Component.skip_standby}}\pysiglinewithargsret{\sphinxbfcode{skip\_standby}}{\emph{value=None}}{}~\begin{quote}\begin{description}
\item[{Parameters}] \leavevmode
\sphinxstyleliteralstrong{value} (\sphinxstyleliteralemphasis{bool}) \textendash{} new skip\_standby value 
if omitted, no change

\item[{Returns}] \leavevmode
\sphinxstylestrong{skip\_standby indicator} \textendash{} components with the skip\_standby indicator of True, will not activate standby components after
the component became current.

\item[{Return type}] \leavevmode
bool

\end{description}\end{quote}

\end{fulllineitems}

\index{standby() (salabim.Component method)}

\begin{fulllineitems}
\phantomsection\label{\detokenize{Reference:salabim.Component.standby}}\pysiglinewithargsret{\sphinxbfcode{standby}}{\emph{mode=None}}{}
puts the component in standby mode
\begin{quote}\begin{description}
\item[{Parameters}] \leavevmode
\sphinxstyleliteralstrong{mode} (\sphinxstyleliteralemphasis{str preferred}) \textendash{} mode 
will be used in trace and can be used in animations 
if nothing specified, the mode will be unchanged. 
also mode\_time will be set to now, if mode is set.

\end{description}\end{quote}

\begin{sphinxadmonition}{note}{Note:}
Not allowed for data components or main.

if to be used for the current component
(which will be nearly always the case),
use \sphinxcode{yield self.standby()}.
\end{sphinxadmonition}

\end{fulllineitems}

\index{status() (salabim.Component method)}

\begin{fulllineitems}
\phantomsection\label{\detokenize{Reference:salabim.Component.status}}\pysiglinewithargsret{\sphinxbfcode{status}}{}{}
returns the status of a component
\begin{description}
\item[{possible values are}] \leavevmode\begin{itemize}
\item {} 
data

\item {} 
passive

\item {} 
scheduled

\item {} 
requesting

\item {} 
waiting

\item {} 
current

\item {} 
standby

\item {} 
interrupted

\end{itemize}

\end{description}

\end{fulllineitems}

\index{successor() (salabim.Component method)}

\begin{fulllineitems}
\phantomsection\label{\detokenize{Reference:salabim.Component.successor}}\pysiglinewithargsret{\sphinxbfcode{successor}}{\emph{q}}{}~\begin{quote}\begin{description}
\item[{Parameters}] \leavevmode
\sphinxstyleliteralstrong{q} ({\hyperref[\detokenize{Reference:salabim.Queue}]{\sphinxcrossref{\sphinxstyleliteralemphasis{Queue}}}}) \textendash{} queue where the component belongs to

\item[{Returns}] \leavevmode
\sphinxstylestrong{the successor of the component in the queue} \textendash{} if component is not at the tail. 
returns None if component is at the tail.

\item[{Return type}] \leavevmode
{\hyperref[\detokenize{Reference:salabim.Component}]{\sphinxcrossref{Component}}}

\end{description}\end{quote}

\end{fulllineitems}

\index{suppress\_pause\_at\_step() (salabim.Component method)}

\begin{fulllineitems}
\phantomsection\label{\detokenize{Reference:salabim.Component.suppress_pause_at_step}}\pysiglinewithargsret{\sphinxbfcode{suppress\_pause\_at\_step}}{\emph{value=None}}{}~\begin{quote}\begin{description}
\item[{Parameters}] \leavevmode
\sphinxstyleliteralstrong{value} (\sphinxstyleliteralemphasis{bool}) \textendash{} new suppress\_trace value 
if omitted, no change

\item[{Returns}] \leavevmode
\sphinxstylestrong{suppress\_pause\_at\_step} \textendash{} components with the suppress\_pause\_at\_step of True, will be ignored in a step

\item[{Return type}] \leavevmode
bool

\end{description}\end{quote}

\end{fulllineitems}

\index{suppress\_trace() (salabim.Component method)}

\begin{fulllineitems}
\phantomsection\label{\detokenize{Reference:salabim.Component.suppress_trace}}\pysiglinewithargsret{\sphinxbfcode{suppress\_trace}}{\emph{value=None}}{}~\begin{quote}\begin{description}
\item[{Parameters}] \leavevmode
\sphinxstyleliteralstrong{value} (\sphinxstyleliteralemphasis{bool}) \textendash{} new suppress\_trace value 
if omitted, no change

\item[{Returns}] \leavevmode
\sphinxstylestrong{suppress\_trace} \textendash{} components with the suppress\_status of True, will be ignored in the trace

\item[{Return type}] \leavevmode
bool

\end{description}\end{quote}

\end{fulllineitems}

\index{wait() (salabim.Component method)}

\begin{fulllineitems}
\phantomsection\label{\detokenize{Reference:salabim.Component.wait}}\pysiglinewithargsret{\sphinxbfcode{wait}}{\emph{*args}, \emph{**kwargs}}{}
wait for any or all of the given state values are met
\begin{quote}\begin{description}
\item[{Parameters}] \leavevmode\begin{itemize}
\item {} 
\sphinxstyleliteralstrong{args} (\sphinxstyleliteralemphasis{sequence of items}\sphinxstyleliteralemphasis{, }\sphinxstyleliteralemphasis{where each item can be}) \textendash{} \begin{itemize}
\item {} 
a state, where value=True, priority=tail of waiters queue)

\item {} \begin{description}
\item[{a tuple/list containing }] \leavevmode
state, a value and optionally a priority. 
if the priority is not specified, this component will
be added to the tail of
the waiters queue 

\end{description}

\end{itemize}


\item {} 
\sphinxstyleliteralstrong{fail\_at} (\sphinxstyleliteralemphasis{float}) \textendash{} time out 
if the wait is not honored before fail\_at,
the wait will be cancelled and the
parameter failed will be set. 
if not specified, the wait will not time out.

\item {} 
\sphinxstyleliteralstrong{fail\_delay} (\sphinxstyleliteralemphasis{float}) \textendash{} time out 
if the wait is not honored before now+fail\_delay,
the request will be cancelled and the
parameter failed will be set. 
if not specified, the wait will not time out.

\item {} 
\sphinxstyleliteralstrong{all} (\sphinxstyleliteralemphasis{bool}) \textendash{} if False (default), continue, if any of the given state/values is met 
if True, continue if all of the given state/values are met

\item {} 
\sphinxstyleliteralstrong{mode} (\sphinxstyleliteralemphasis{str preferred}) \textendash{} mode 
will be used in trace and can be used in animations 
if nothing specified, the mode will be unchanged. 
also mode\_time will be set to now, if mode is set.

\end{itemize}

\end{description}\end{quote}

\begin{sphinxadmonition}{note}{Note:}
Not allowed for data components or main.

If to be used for the current component
(which will be nearly always the case),
use \sphinxcode{yield self.wait(...)}.

It is allowed to wait for more than one value of a state 
the parameter failed will be reset by a calling wait

If you want to check for all components to meet a value (and clause),
use Component.wait(…, all=True)

The value may be specified in three different ways:
\begin{itemize}
\item {} 
constant, that value is just compared to state.value() 
yield self.wait((light,’red’))

\item {} 
an expression, containg one or more \$-signs
the \$ is replaced by state.value(), each time the condition is tested. 
self refers to the component under test, state refers to the state
under test. 
yield self.wait((light,’\$ in (“red”,”yellow”)’)) 
yield self.wait((level,’\$\textless{}30’)) 

\item {} 
a function. In that case the parameter should function that
should accept three arguments: the value, the component under test and the
state under test. 
usually the function will be a lambda function, but that’s not
a requirement. 
yield self.wait((light,lambda t, comp, state: t in (‘red’,’yellow’))) 
yield self.wait((level,lambda t, comp, state: t \textless{} 30)) 

\end{itemize}
\end{sphinxadmonition}
\paragraph{Example}

\sphinxcode{yield self.wait(s1)} 
\textendash{}\textgreater{} waits for s1.value()==True 
\sphinxcode{yield self.wait(s1,s2)} 
\textendash{}\textgreater{} waits for s1.value()==True or s2.value==True 
\sphinxcode{yield self.wait((s1,False,100),(s2,'on'),s3)} 
\textendash{}\textgreater{} waits for s1.value()==False or s2.value==’on’ or s3.value()==True 
s1 is at the tail of waiters, because of the set priority 
\sphinxcode{yield self.wait(s1,s2,all=True)} 
\textendash{}\textgreater{} waits for s1.value()==True and s2.value==True 

\end{fulllineitems}


\end{fulllineitems}



\section{Environment}
\label{\detokenize{Reference:environment}}\index{Environment (class in salabim)}

\begin{fulllineitems}
\phantomsection\label{\detokenize{Reference:salabim.Environment}}\pysiglinewithargsret{\sphinxbfcode{class }\sphinxcode{salabim.}\sphinxbfcode{Environment}}{\emph{trace=False}, \emph{random\_seed=None}, \emph{time\_unit='n/a'}, \emph{name=None}, \emph{print\_trace\_header=True}, \emph{isdefault\_env=True}, \emph{*args}, \emph{**kwargs}}{}
environment object
\begin{quote}\begin{description}
\item[{Parameters}] \leavevmode\begin{itemize}
\item {} 
\sphinxstyleliteralstrong{trace} (\sphinxstyleliteralemphasis{bool}) \textendash{} defines whether to trace or not 
if omitted, False

\item {} 
\sphinxstyleliteralstrong{random\_seed} (\sphinxstyleliteralemphasis{hashable object}\sphinxstyleliteralemphasis{, }\sphinxstyleliteralemphasis{usually int}) \textendash{} the seed for random, equivalent to random.seed() 
if ‘*’, a purely random value (based on the current time) will be used
(not reproducable) 
if the null string (‘’), no action on random is taken 
if None (the default), 1234567 will be used.

\item {} 
\sphinxstyleliteralstrong{time\_unit} (\sphinxstyleliteralemphasis{str}) \textendash{} Supported time\_units: 
‘years’, ‘weeks’, ‘days’, ‘hours’, ‘minutes’, ‘seconds’, ‘milliseconds’, ‘microseconds’, ‘n/a’

\item {} 
\sphinxstyleliteralstrong{name} (\sphinxstyleliteralemphasis{str}) \textendash{} name of the environment 
if the name ends with a period (.),
auto serializing will be applied 
if the name end with a comma,
auto serializing starting at 1 will be applied 
if omitted, the name will be derived from the class (lowercased)
or ‘default environment’ if isdefault\_env is True.

\item {} 
\sphinxstyleliteralstrong{print\_trace\_header} (\sphinxstyleliteralemphasis{bool}) \textendash{} if True (default) print a (two line) header line as a legend 
if False, do not print a header 
note that the header is only printed if trace=True

\item {} 
\sphinxstyleliteralstrong{isdefault\_env} (\sphinxstyleliteralemphasis{bool}) \textendash{} if True (default), this environment becomes the default environment 
if False, this environment will not be the default environment 
if omitted, this environment becomes the default environment 

\end{itemize}

\end{description}\end{quote}

\begin{sphinxadmonition}{note}{Note:}
The trace may be switched on/off later with trace 
The seed may be later set with random\_seed() 
Initially, the random stream will be seeded with the value 1234567.
If required to be purely, not not reproducable, values, use
random\_seed=’*’.
\end{sphinxadmonition}
\index{an\_clocktext() (salabim.Environment method)}

\begin{fulllineitems}
\phantomsection\label{\detokenize{Reference:salabim.Environment.an_clocktext}}\pysiglinewithargsret{\sphinxbfcode{an\_clocktext}}{}{}
function to initialize the system clocktext 
called by run(), if animation is True. 
may be overridden to change the standard behaviour.

\end{fulllineitems}

\index{an\_menu\_buttons() (salabim.Environment method)}

\begin{fulllineitems}
\phantomsection\label{\detokenize{Reference:salabim.Environment.an_menu_buttons}}\pysiglinewithargsret{\sphinxbfcode{an\_menu\_buttons}}{}{}
function to initialize the menu buttons 
may be overridden to change the standard behaviour.

\end{fulllineitems}

\index{an\_modelname() (salabim.Environment method)}

\begin{fulllineitems}
\phantomsection\label{\detokenize{Reference:salabim.Environment.an_modelname}}\pysiglinewithargsret{\sphinxbfcode{an\_modelname}}{}{}
function to show the modelname 
called by run(), if animation is True. 
may be overridden to change the standard behaviour.

\end{fulllineitems}

\index{an\_synced\_buttons() (salabim.Environment method)}

\begin{fulllineitems}
\phantomsection\label{\detokenize{Reference:salabim.Environment.an_synced_buttons}}\pysiglinewithargsret{\sphinxbfcode{an\_synced\_buttons}}{}{}
function to initialize the synced buttons 
may be overridden to change the standard behaviour.

\end{fulllineitems}

\index{an\_unsynced\_buttons() (salabim.Environment method)}

\begin{fulllineitems}
\phantomsection\label{\detokenize{Reference:salabim.Environment.an_unsynced_buttons}}\pysiglinewithargsret{\sphinxbfcode{an\_unsynced\_buttons}}{}{}
function to initialize the unsynced buttons 
may be overridden to change the standard behaviour.

\end{fulllineitems}

\index{animate() (salabim.Environment method)}

\begin{fulllineitems}
\phantomsection\label{\detokenize{Reference:salabim.Environment.animate}}\pysiglinewithargsret{\sphinxbfcode{animate}}{\emph{value=None}}{}
animate indicator
\begin{quote}\begin{description}
\item[{Parameters}] \leavevmode
\sphinxstyleliteralstrong{value} (\sphinxstyleliteralemphasis{bool}) \textendash{} new animate indicator 
if not specified, no change

\item[{Returns}] \leavevmode
\sphinxstylestrong{animate status}

\item[{Return type}] \leavevmode
bool

\end{description}\end{quote}

\begin{sphinxadmonition}{note}{Note:}
When the run is not issued, no acction will be taken.
\end{sphinxadmonition}

\end{fulllineitems}

\index{animation\_parameters() (salabim.Environment method)}

\begin{fulllineitems}
\phantomsection\label{\detokenize{Reference:salabim.Environment.animation_parameters}}\pysiglinewithargsret{\sphinxbfcode{animation\_parameters}}{\emph{animate=True}, \emph{synced=None}, \emph{speed=None}, \emph{width=None}, \emph{height=None}, \emph{x0=None}, \emph{y0=None}, \emph{x1=None}, \emph{background\_color=None}, \emph{foreground\_color=None}, \emph{fps=None}, \emph{modelname=None}, \emph{use\_toplevel=None}, \emph{show\_fps=None}, \emph{show\_time=None}, \emph{video=None}, \emph{video\_repeat=None}, \emph{video\_pingpong=None}}{}
set animation parameters
\begin{quote}\begin{description}
\item[{Parameters}] \leavevmode\begin{itemize}
\item {} 
\sphinxstyleliteralstrong{animate} (\sphinxstyleliteralemphasis{bool}) \textendash{} animate indicator 
if not specified, set animate on 

\item {} 
\sphinxstyleliteralstrong{synced} (\sphinxstyleliteralemphasis{bool}) \textendash{} specifies whether animation is synced 
if omitted, no change. At init of the environment synced will be set to True

\item {} 
\sphinxstyleliteralstrong{speed} (\sphinxstyleliteralemphasis{float}) \textendash{} speed 
specifies how much faster or slower than real time the animation will run.
e.g. if 2, 2 simulation time units will be displayed per second.

\item {} 
\sphinxstyleliteralstrong{width} (\sphinxstyleliteralemphasis{int}) \textendash{} width of the animation in screen coordinates 
if omitted, no change. At init of the environment, the width will be
set to 1024 for non Pythonista and the current screen width for Pythonista.

\item {} 
\sphinxstyleliteralstrong{height} (\sphinxstyleliteralemphasis{int}) \textendash{} height of the animation in screen coordinates 
if omitted, no change. At init of the environment, the height will be
set to 768 for non Pythonista and the current screen height for Pythonista.

\item {} 
\sphinxstyleliteralstrong{x0} (\sphinxstyleliteralemphasis{float}) \textendash{} user x-coordinate of the lower left corner 
if omitted, no change. At init of the environment, x0 will be set to 0.

\item {} 
\sphinxstyleliteralstrong{y0} (\sphinxstyleliteralemphasis{float}) \textendash{} user y\_coordinate of the lower left corner 
if omitted, no change. At init of the environment, y0 will be set to 0.

\item {} 
\sphinxstyleliteralstrong{x1} (\sphinxstyleliteralemphasis{float}) \textendash{} user x-coordinate of the lower right corner 
if omitted, no change. At init of the environment, x1 will be set to 1024
for non Pythonista and the current screen width for Pythonista.

\item {} 
\sphinxstyleliteralstrong{background\_color} (\sphinxstyleliteralemphasis{colorspec}) \textendash{} color of the background 
if omitted, no change. At init of the environment, this will be set to white.

\item {} 
\sphinxstyleliteralstrong{foreground\_color} (\sphinxstyleliteralemphasis{colorspec}) \textendash{} color of foreground (texts) 
if omitted and background\_color is specified, either white of black will be used,
in order to get a good contrast with the background color. 
if omitted and background\_color is also omitted, no change. At init of the
environment, this will be set to black.

\item {} 
\sphinxstyleliteralstrong{fps} (\sphinxstyleliteralemphasis{float}) \textendash{} number of frames per second

\item {} 
\sphinxstyleliteralstrong{modelname} (\sphinxstyleliteralemphasis{str}) \textendash{} name of model to be shown in upper left corner,
along with text ‘a salabim model’ 
if omitted, no change. At init of the environment, this will be set
to the null string, which implies suppression of this feature.

\item {} 
\sphinxstyleliteralstrong{use\_toplevel} (\sphinxstyleliteralemphasis{bool}) \textendash{} if salabim animation is used in parallel with
other modules using tkinter, it might be necessary to
initialize the root with tkinter.TopLevel().
In that case, set this parameter to True. 
if False (default), the root will be initialized with tkinter.Tk()

\item {} 
\sphinxstyleliteralstrong{show\_fps} (\sphinxstyleliteralemphasis{bool}) \textendash{} if True, show the number of frames per second 
if False, do not show the number of frames per second (default)

\item {} 
\sphinxstyleliteralstrong{show\_time} (\sphinxstyleliteralemphasis{bool}) \textendash{} if True, show the time (default)  
if False, do not show the time

\item {} 
\sphinxstyleliteralstrong{video} (\sphinxstyleliteralemphasis{str}) \textendash{} if video is not omitted, a video with the name video
will be created. 
Normally, use .mp4 as extension. 
If the video extension is not .gif, a codec may be added
by appending a plus sign and the four letter code name,
like ‘myvideo.avi+DIVX’. 
If no codec is given, MP4V will be used as codec.

\item {} 
\sphinxstyleliteralstrong{video\_repeat} (\sphinxstyleliteralemphasis{int}) \textendash{} number of times gif should be repeated 
0 means inifinite 
at init of the environment video\_repeat is 1 
this only applies to gif files production.

\item {} 
\sphinxstyleliteralstrong{video\_pingpong} (\sphinxstyleliteralemphasis{bool}) \textendash{} if True, all frames will be added reversed at the end of the video (useful for smooth loops)
at init of the environment video\_pingpong is False 
this only applies to gif files production.

\end{itemize}

\end{description}\end{quote}

\begin{sphinxadmonition}{note}{Note:}
The y-coordinate of the upper right corner is determined automatically
in such a way that the x and scaling are the same. 
\end{sphinxadmonition}

\end{fulllineitems}

\index{animation\_post\_tick() (salabim.Environment method)}

\begin{fulllineitems}
\phantomsection\label{\detokenize{Reference:salabim.Environment.animation_post_tick}}\pysiglinewithargsret{\sphinxbfcode{animation\_post\_tick}}{\emph{t}}{}
called just after the animation object loop. 
Default behaviour: just return
\begin{quote}\begin{description}
\item[{Parameters}] \leavevmode
\sphinxstyleliteralstrong{t} (\sphinxstyleliteralemphasis{float}) \textendash{} Current (animation) time.

\end{description}\end{quote}

\end{fulllineitems}

\index{animation\_pre\_tick() (salabim.Environment method)}

\begin{fulllineitems}
\phantomsection\label{\detokenize{Reference:salabim.Environment.animation_pre_tick}}\pysiglinewithargsret{\sphinxbfcode{animation\_pre\_tick}}{\emph{t}}{}
called just before the animation object loop. 
Default behaviour: just return
\begin{quote}\begin{description}
\item[{Parameters}] \leavevmode
\sphinxstyleliteralstrong{t} (\sphinxstyleliteralemphasis{float}) \textendash{} Current (animation) time.

\end{description}\end{quote}

\end{fulllineitems}

\index{background\_color() (salabim.Environment method)}

\begin{fulllineitems}
\phantomsection\label{\detokenize{Reference:salabim.Environment.background_color}}\pysiglinewithargsret{\sphinxbfcode{background\_color}}{\emph{value=None}}{}
background\_color of the animation
\begin{quote}\begin{description}
\item[{Parameters}] \leavevmode
\sphinxstyleliteralstrong{value} (\sphinxstyleliteralemphasis{colorspec}) \textendash{} new background\_color 
if not specified, no change

\item[{Returns}] \leavevmode
\sphinxstylestrong{background\_color of animation}

\item[{Return type}] \leavevmode
colorspec

\end{description}\end{quote}

\end{fulllineitems}

\index{base\_name() (salabim.Environment method)}

\begin{fulllineitems}
\phantomsection\label{\detokenize{Reference:salabim.Environment.base_name}}\pysiglinewithargsret{\sphinxbfcode{base\_name}}{}{}
returns the base name of the environment (the name used at initialization)

\end{fulllineitems}

\index{beep() (salabim.Environment method)}

\begin{fulllineitems}
\phantomsection\label{\detokenize{Reference:salabim.Environment.beep}}\pysiglinewithargsret{\sphinxbfcode{beep}}{}{}
Beeps

Works only on Windows and iOS (Pythonista). For other platforms this is just a dummy method.

\end{fulllineitems}

\index{colorinterpolate() (salabim.Environment method)}

\begin{fulllineitems}
\phantomsection\label{\detokenize{Reference:salabim.Environment.colorinterpolate}}\pysiglinewithargsret{\sphinxbfcode{colorinterpolate}}{\emph{t}, \emph{t0}, \emph{t1}, \emph{v0}, \emph{v1}}{}
does linear interpolation of colorspecs
\begin{quote}\begin{description}
\item[{Parameters}] \leavevmode\begin{itemize}
\item {} 
\sphinxstyleliteralstrong{t} (\sphinxstyleliteralemphasis{float}) \textendash{} value to be interpolated from

\item {} 
\sphinxstyleliteralstrong{t0} (\sphinxstyleliteralemphasis{float}) \textendash{} f(t0)=v0

\item {} 
\sphinxstyleliteralstrong{t1} (\sphinxstyleliteralemphasis{float}) \textendash{} f(t1)=v1

\item {} 
\sphinxstyleliteralstrong{v0} (\sphinxstyleliteralemphasis{colorspec}) \textendash{} f(t0)=v0

\item {} 
\sphinxstyleliteralstrong{v1} (\sphinxstyleliteralemphasis{colorspec}) \textendash{} f(t1)=v1

\end{itemize}

\item[{Returns}] \leavevmode
\sphinxstylestrong{linear interpolation between v0 and v1 based on t between t0 and t}

\item[{Return type}] \leavevmode
colorspec

\end{description}\end{quote}

\begin{sphinxadmonition}{note}{Note:}
Note that no extrapolation is done, so if t\textless{}t0 ==\textgreater{} v0  and t\textgreater{}t1 ==\textgreater{} v1 
This function is heavily used during animation
\end{sphinxadmonition}

\end{fulllineitems}

\index{colorspec\_to\_tuple() (salabim.Environment method)}

\begin{fulllineitems}
\phantomsection\label{\detokenize{Reference:salabim.Environment.colorspec_to_tuple}}\pysiglinewithargsret{\sphinxbfcode{colorspec\_to\_tuple}}{\emph{colorspec}}{}
translates a colorspec to a tuple
\begin{quote}\begin{description}
\item[{Parameters}] \leavevmode
\sphinxstyleliteralstrong{colorspec} (\sphinxstyleliteralemphasis{tuple}\sphinxstyleliteralemphasis{, }\sphinxstyleliteralemphasis{list}\sphinxstyleliteralemphasis{ or }\sphinxstyleliteralemphasis{str}) \textendash{} \sphinxcode{\#rrggbb} ==\textgreater{} alpha = 255 (rr, gg, bb in hex) 
\sphinxcode{\#rrggbbaa} ==\textgreater{} alpha = aa (rr, gg, bb, aa in hex) 
\sphinxcode{colorname} ==\textgreater{} alpha = 255 
\sphinxcode{(colorname, alpha)} 
\sphinxcode{(r, g, b)} ==\textgreater{} alpha = 255 
\sphinxcode{(r, g, b, alpha)} 
\sphinxcode{'fg'} ==\textgreater{} foreground\_color 
\sphinxcode{'bg'} ==\textgreater{} background\_color

\item[{Returns}] \leavevmode


\item[{Return type}] \leavevmode
(r, g, b, a)

\end{description}\end{quote}

\end{fulllineitems}

\index{current\_component() (salabim.Environment method)}

\begin{fulllineitems}
\phantomsection\label{\detokenize{Reference:salabim.Environment.current_component}}\pysiglinewithargsret{\sphinxbfcode{current\_component}}{}{}~\begin{quote}\begin{description}
\item[{Returns}] \leavevmode
\sphinxstylestrong{the current\_component}

\item[{Return type}] \leavevmode
{\hyperref[\detokenize{Reference:salabim.Component}]{\sphinxcrossref{Component}}}

\end{description}\end{quote}

\end{fulllineitems}

\index{days() (salabim.Environment method)}

\begin{fulllineitems}
\phantomsection\label{\detokenize{Reference:salabim.Environment.days}}\pysiglinewithargsret{\sphinxbfcode{days}}{\emph{t}}{}
convert the given time in days to the current time unit
\begin{quote}\begin{description}
\item[{Parameters}] \leavevmode
\sphinxstyleliteralstrong{t} (\sphinxstyleliteralemphasis{float}) \textendash{} time in days

\item[{Returns}] \leavevmode
\sphinxstylestrong{time in days, converted to the current time\_unit}

\item[{Return type}] \leavevmode
float

\end{description}\end{quote}

\end{fulllineitems}

\index{foreground\_color() (salabim.Environment method)}

\begin{fulllineitems}
\phantomsection\label{\detokenize{Reference:salabim.Environment.foreground_color}}\pysiglinewithargsret{\sphinxbfcode{foreground\_color}}{\emph{value=None}}{}
foreground\_color of the animation
\begin{quote}\begin{description}
\item[{Parameters}] \leavevmode
\sphinxstyleliteralstrong{value} (\sphinxstyleliteralemphasis{colorspec}) \textendash{} new foreground\_color 
if not specified, no change

\item[{Returns}] \leavevmode
\sphinxstylestrong{foreground\_color of animation}

\item[{Return type}] \leavevmode
colorspec

\end{description}\end{quote}

\end{fulllineitems}

\index{fps() (salabim.Environment method)}

\begin{fulllineitems}
\phantomsection\label{\detokenize{Reference:salabim.Environment.fps}}\pysiglinewithargsret{\sphinxbfcode{fps}}{\emph{value=None}}{}~\begin{quote}\begin{description}
\item[{Parameters}] \leavevmode
\sphinxstyleliteralstrong{value} (\sphinxstyleliteralemphasis{int}) \textendash{} new fps 
if not specified, no change

\item[{Returns}] \leavevmode
\sphinxstylestrong{fps}

\item[{Return type}] \leavevmode
bool

\end{description}\end{quote}

\end{fulllineitems}

\index{get\_time\_unit() (salabim.Environment method)}

\begin{fulllineitems}
\phantomsection\label{\detokenize{Reference:salabim.Environment.get_time_unit}}\pysiglinewithargsret{\sphinxbfcode{get\_time\_unit}}{}{}
gets time unit
\begin{quote}\begin{description}
\item[{Returns}] \leavevmode
\sphinxstylestrong{Current time unit dimension (default ‘n/a’)}

\item[{Return type}] \leavevmode
str

\end{description}\end{quote}

\end{fulllineitems}

\index{height() (salabim.Environment method)}

\begin{fulllineitems}
\phantomsection\label{\detokenize{Reference:salabim.Environment.height}}\pysiglinewithargsret{\sphinxbfcode{height}}{\emph{value=None}}{}
height of the animation in screen coordinates
\begin{quote}\begin{description}
\item[{Parameters}] \leavevmode
\sphinxstyleliteralstrong{value} (\sphinxstyleliteralemphasis{int}) \textendash{} new height 
if not specified, no change

\item[{Returns}] \leavevmode
\sphinxstylestrong{height of animation}

\item[{Return type}] \leavevmode
int

\end{description}\end{quote}

\end{fulllineitems}

\index{hours() (salabim.Environment method)}

\begin{fulllineitems}
\phantomsection\label{\detokenize{Reference:salabim.Environment.hours}}\pysiglinewithargsret{\sphinxbfcode{hours}}{\emph{t}}{}
convert the given time in hours to the current time unit
\begin{quote}\begin{description}
\item[{Parameters}] \leavevmode
\sphinxstyleliteralstrong{t} (\sphinxstyleliteralemphasis{float}) \textendash{} time in hours

\item[{Returns}] \leavevmode
\sphinxstylestrong{time in hours, converted to the current time\_unit}

\item[{Return type}] \leavevmode
float

\end{description}\end{quote}

\end{fulllineitems}

\index{is\_dark() (salabim.Environment method)}

\begin{fulllineitems}
\phantomsection\label{\detokenize{Reference:salabim.Environment.is_dark}}\pysiglinewithargsret{\sphinxbfcode{is\_dark}}{\emph{colorspec}}{}~\begin{quote}\begin{description}
\item[{Parameters}] \leavevmode
\sphinxstyleliteralstrong{colorspec} (\sphinxstyleliteralemphasis{colorspec}) \textendash{} color to check

\item[{Returns}] \leavevmode
True, if the colorspec is dark (rather black than white) 
False, if the colorspec is light (rather white than black 
if colorspec has alpha=0 (total transparent), the background\_color will be tested

\item[{Return type}] \leavevmode
bool

\end{description}\end{quote}

\end{fulllineitems}

\index{main() (salabim.Environment method)}

\begin{fulllineitems}
\phantomsection\label{\detokenize{Reference:salabim.Environment.main}}\pysiglinewithargsret{\sphinxbfcode{main}}{}{}~\begin{quote}\begin{description}
\item[{Returns}] \leavevmode
\sphinxstylestrong{the main component}

\item[{Return type}] \leavevmode
{\hyperref[\detokenize{Reference:salabim.Component}]{\sphinxcrossref{Component}}}

\end{description}\end{quote}

\end{fulllineitems}

\index{microseconds() (salabim.Environment method)}

\begin{fulllineitems}
\phantomsection\label{\detokenize{Reference:salabim.Environment.microseconds}}\pysiglinewithargsret{\sphinxbfcode{microseconds}}{\emph{t}}{}
convert the given time in microseconds to the current time unit
\begin{quote}\begin{description}
\item[{Parameters}] \leavevmode
\sphinxstyleliteralstrong{t} (\sphinxstyleliteralemphasis{float}) \textendash{} time in microseconds

\item[{Returns}] \leavevmode
\sphinxstylestrong{time in microseconds, converted to the current time\_unit}

\item[{Return type}] \leavevmode
float

\end{description}\end{quote}

\end{fulllineitems}

\index{milliseconds() (salabim.Environment method)}

\begin{fulllineitems}
\phantomsection\label{\detokenize{Reference:salabim.Environment.milliseconds}}\pysiglinewithargsret{\sphinxbfcode{milliseconds}}{\emph{t}}{}
convert the given time in milliseconds to the current time unit
\begin{quote}\begin{description}
\item[{Parameters}] \leavevmode
\sphinxstyleliteralstrong{t} (\sphinxstyleliteralemphasis{float}) \textendash{} time in milliseconds

\item[{Returns}] \leavevmode
\sphinxstylestrong{time in milliseconds, converted to the current time\_unit}

\item[{Return type}] \leavevmode
float

\end{description}\end{quote}

\end{fulllineitems}

\index{minutes() (salabim.Environment method)}

\begin{fulllineitems}
\phantomsection\label{\detokenize{Reference:salabim.Environment.minutes}}\pysiglinewithargsret{\sphinxbfcode{minutes}}{\emph{t}}{}
convert the given time in minutes to the current time unit
\begin{quote}\begin{description}
\item[{Parameters}] \leavevmode
\sphinxstyleliteralstrong{t} (\sphinxstyleliteralemphasis{float}) \textendash{} time in minutes

\item[{Returns}] \leavevmode
\sphinxstylestrong{time in minutes, converted to the current time\_unit}

\item[{Return type}] \leavevmode
float

\end{description}\end{quote}

\end{fulllineitems}

\index{modelname() (salabim.Environment method)}

\begin{fulllineitems}
\phantomsection\label{\detokenize{Reference:salabim.Environment.modelname}}\pysiglinewithargsret{\sphinxbfcode{modelname}}{\emph{value=None}}{}~\begin{quote}\begin{description}
\item[{Parameters}] \leavevmode
\sphinxstyleliteralstrong{value} (\sphinxstyleliteralemphasis{str}) \textendash{} new modelname 
if not specified, no change

\item[{Returns}] \leavevmode
\sphinxstylestrong{modelname}

\item[{Return type}] \leavevmode
str

\end{description}\end{quote}

\begin{sphinxadmonition}{note}{Note:}
If modelname is the null string, nothing will be displayed.
\end{sphinxadmonition}

\end{fulllineitems}

\index{name() (salabim.Environment method)}

\begin{fulllineitems}
\phantomsection\label{\detokenize{Reference:salabim.Environment.name}}\pysiglinewithargsret{\sphinxbfcode{name}}{\emph{value=None}}{}~\begin{quote}\begin{description}
\item[{Parameters}] \leavevmode
\sphinxstyleliteralstrong{value} (\sphinxstyleliteralemphasis{str}) \textendash{} new name of the environment
if omitted, no change

\item[{Returns}] \leavevmode
\sphinxstylestrong{Name of the environment}

\item[{Return type}] \leavevmode
str

\end{description}\end{quote}

\begin{sphinxadmonition}{note}{Note:}
base\_name and sequence\_number are not affected if the name is changed
\end{sphinxadmonition}

\end{fulllineitems}

\index{now() (salabim.Environment method)}

\begin{fulllineitems}
\phantomsection\label{\detokenize{Reference:salabim.Environment.now}}\pysiglinewithargsret{\sphinxbfcode{now}}{}{}~\begin{quote}\begin{description}
\item[{Returns}] \leavevmode
\sphinxstylestrong{the current simulation time}

\item[{Return type}] \leavevmode
float

\end{description}\end{quote}

\end{fulllineitems}

\index{peek() (salabim.Environment method)}

\begin{fulllineitems}
\phantomsection\label{\detokenize{Reference:salabim.Environment.peek}}\pysiglinewithargsret{\sphinxbfcode{peek}}{}{}
returns the time of the next component to become current 
if there are no more events, peek will return inf 
Only for advance use with animation / GUI event loops

\end{fulllineitems}

\index{print\_info() (salabim.Environment method)}

\begin{fulllineitems}
\phantomsection\label{\detokenize{Reference:salabim.Environment.print_info}}\pysiglinewithargsret{\sphinxbfcode{print\_info}}{\emph{as\_str=False}, \emph{file=None}}{}
prints information about the environment
\begin{quote}\begin{description}
\item[{Parameters}] \leavevmode\begin{itemize}
\item {} 
\sphinxstyleliteralstrong{as\_str} (\sphinxstyleliteralemphasis{bool}) \textendash{} if False (default), print the info
if True, return a string containing the info

\item {} 
\sphinxstyleliteralstrong{file} (\sphinxstyleliteralemphasis{file}) \textendash{} if None(default), all output is directed to stdout 
otherwise, the output is directed to the file

\end{itemize}

\item[{Returns}] \leavevmode
\sphinxstylestrong{info (if as\_str is True)}

\item[{Return type}] \leavevmode
str

\end{description}\end{quote}

\end{fulllineitems}

\index{print\_trace() (salabim.Environment method)}

\begin{fulllineitems}
\phantomsection\label{\detokenize{Reference:salabim.Environment.print_trace}}\pysiglinewithargsret{\sphinxbfcode{print\_trace}}{\emph{s1=''}, \emph{s2=''}, \emph{s3=''}, \emph{s4=''}, \emph{s0=None}, \emph{\_optional=False}}{}
prints a trace line
\begin{quote}\begin{description}
\item[{Parameters}] \leavevmode\begin{itemize}
\item {} 
\sphinxstyleliteralstrong{s1} (\sphinxstyleliteralemphasis{str}) \textendash{} part 1 (usually formatted  now), padded to 10 characters

\item {} 
\sphinxstyleliteralstrong{s2} (\sphinxstyleliteralemphasis{str}) \textendash{} part 2 (usually only used for the compoent that gets current), padded to 20 characters

\item {} 
\sphinxstyleliteralstrong{s3} (\sphinxstyleliteralemphasis{str}) \textendash{} part 3, padded to 35 characters

\item {} 
\sphinxstyleliteralstrong{s4} (\sphinxstyleliteralemphasis{str}) \textendash{} part 4

\item {} 
\sphinxstyleliteralstrong{s0} (\sphinxstyleliteralemphasis{str}) \textendash{} part 0. if omitted, the line number from where the call was given will be used at
the start of the line. Otherwise s0, left padded to 7 characters will be used at
the start of the line.

\item {} 
\sphinxstyleliteralstrong{\_optional} (\sphinxstyleliteralemphasis{bool}) \textendash{} for internal use only. Do not set this flag!

\end{itemize}

\end{description}\end{quote}

\begin{sphinxadmonition}{note}{Note:}
if self.trace is False, nothing is printed 
if the current component’s suppress\_trace is True, nothing is printed 
\end{sphinxadmonition}

\end{fulllineitems}

\index{print\_trace\_header() (salabim.Environment method)}

\begin{fulllineitems}
\phantomsection\label{\detokenize{Reference:salabim.Environment.print_trace_header}}\pysiglinewithargsret{\sphinxbfcode{print\_trace\_header}}{}{}
print a (two line) header line as a legend 
also the legend for line numbers will be printed 
not that the header is only printed if trace=True

\end{fulllineitems}

\index{reset\_now() (salabim.Environment method)}

\begin{fulllineitems}
\phantomsection\label{\detokenize{Reference:salabim.Environment.reset_now}}\pysiglinewithargsret{\sphinxbfcode{reset\_now}}{\emph{new\_now=0}}{}
reset the current time
\begin{quote}\begin{description}
\item[{Parameters}] \leavevmode
\sphinxstyleliteralstrong{new\_now} (\sphinxstyleliteralemphasis{float}) \textendash{} now will be set to new\_now 
default: 0

\end{description}\end{quote}

\begin{sphinxadmonition}{note}{Note:}
Internally, salabim still works with the ‘old’ time. Only in the interface
from and to the user program, a correction will be applied.

The registered time in monitors will be always is the ‘old’ time.
This is only relevant when using the time value in Monitor.xt() or Monitor.tx().
\end{sphinxadmonition}

\end{fulllineitems}

\index{run() (salabim.Environment method)}

\begin{fulllineitems}
\phantomsection\label{\detokenize{Reference:salabim.Environment.run}}\pysiglinewithargsret{\sphinxbfcode{run}}{\emph{duration=None}, \emph{till=None}, \emph{urgent=False}}{}
start execution of the simulation
\begin{quote}\begin{description}
\item[{Parameters}] \leavevmode\begin{itemize}
\item {} 
\sphinxstyleliteralstrong{duration} (\sphinxstyleliteralemphasis{float}) \textendash{} schedule with a delay of duration 
if 0, now is used

\item {} 
\sphinxstyleliteralstrong{till} (\sphinxstyleliteralemphasis{float}) \textendash{} schedule time 
if omitted, inf is assumed

\item {} 
\sphinxstyleliteralstrong{urgent} (\sphinxstyleliteralemphasis{bool}) \textendash{} urgency indicator 
if False (default), main will be scheduled
behind all other components scheduled
for the same time 
if True, main will be scheduled
in front of all components scheduled
for the same time

\end{itemize}

\end{description}\end{quote}

\begin{sphinxadmonition}{note}{Note:}
if neither till nor duration is specified, the main component will be reactivated at
the time there are no more events on the eventlist, i.e. usualy not at inf. 
if you want to run till inf, issue run(sim.inf) 
only issue run() from the main level
\end{sphinxadmonition}

\end{fulllineitems}

\index{scale() (salabim.Environment method)}

\begin{fulllineitems}
\phantomsection\label{\detokenize{Reference:salabim.Environment.scale}}\pysiglinewithargsret{\sphinxbfcode{scale}}{}{}
scale of the animation, i.e. width / (x1 - x0)
\begin{quote}\begin{description}
\item[{Returns}] \leavevmode
\sphinxstylestrong{scale}

\item[{Return type}] \leavevmode
float

\end{description}\end{quote}

\begin{sphinxadmonition}{note}{Note:}
It is not possible to set this value explicitely.
\end{sphinxadmonition}

\end{fulllineitems}

\index{screen\_to\_usercoordinates\_size() (salabim.Environment method)}

\begin{fulllineitems}
\phantomsection\label{\detokenize{Reference:salabim.Environment.screen_to_usercoordinates_size}}\pysiglinewithargsret{\sphinxbfcode{screen\_to\_usercoordinates\_size}}{\emph{screensize}}{}
converts a screen size to a value to be used with user coordinates
\begin{quote}\begin{description}
\item[{Parameters}] \leavevmode
\sphinxstyleliteralstrong{screensize} (\sphinxstyleliteralemphasis{float}) \textendash{} screen size to be converted

\item[{Returns}] \leavevmode
\sphinxstylestrong{value corresponding with screensize in user coordinates}

\item[{Return type}] \leavevmode
float

\end{description}\end{quote}

\end{fulllineitems}

\index{screen\_to\_usercoordinates\_x() (salabim.Environment method)}

\begin{fulllineitems}
\phantomsection\label{\detokenize{Reference:salabim.Environment.screen_to_usercoordinates_x}}\pysiglinewithargsret{\sphinxbfcode{screen\_to\_usercoordinates\_x}}{\emph{screenx}}{}
converts a screen x coordinate to a user x coordinate
\begin{quote}\begin{description}
\item[{Parameters}] \leavevmode
\sphinxstyleliteralstrong{screenx} (\sphinxstyleliteralemphasis{float}) \textendash{} screen x coordinate to be converted

\item[{Returns}] \leavevmode
\sphinxstylestrong{user x coordinate}

\item[{Return type}] \leavevmode
float

\end{description}\end{quote}

\end{fulllineitems}

\index{screen\_to\_usercoordinates\_y() (salabim.Environment method)}

\begin{fulllineitems}
\phantomsection\label{\detokenize{Reference:salabim.Environment.screen_to_usercoordinates_y}}\pysiglinewithargsret{\sphinxbfcode{screen\_to\_usercoordinates\_y}}{\emph{screeny}}{}
converts a screen x coordinate to a user x coordinate
\begin{quote}\begin{description}
\item[{Parameters}] \leavevmode
\sphinxstyleliteralstrong{screeny} (\sphinxstyleliteralemphasis{float}) \textendash{} screen y coordinate to be converted

\item[{Returns}] \leavevmode
\sphinxstylestrong{user y coordinate}

\item[{Return type}] \leavevmode
float

\end{description}\end{quote}

\end{fulllineitems}

\index{seconds() (salabim.Environment method)}

\begin{fulllineitems}
\phantomsection\label{\detokenize{Reference:salabim.Environment.seconds}}\pysiglinewithargsret{\sphinxbfcode{seconds}}{\emph{t}}{}
convert the given time in seconds to the current time unit
\begin{quote}\begin{description}
\item[{Parameters}] \leavevmode
\sphinxstyleliteralstrong{t} (\sphinxstyleliteralemphasis{float}) \textendash{} time in seconds

\item[{Returns}] \leavevmode
\sphinxstylestrong{time in secoonds, converted to the current time\_unit}

\item[{Return type}] \leavevmode
float

\end{description}\end{quote}

\end{fulllineitems}

\index{sequence\_number() (salabim.Environment method)}

\begin{fulllineitems}
\phantomsection\label{\detokenize{Reference:salabim.Environment.sequence_number}}\pysiglinewithargsret{\sphinxbfcode{sequence\_number}}{}{}~\begin{quote}\begin{description}
\item[{Returns}] \leavevmode
\sphinxstylestrong{sequence\_number of the environment} \textendash{} (the sequence number at initialization) 
normally this will be the integer value of a serialized name,
but also non serialized names (without a dot or a comma at the end)
will be numbered)

\item[{Return type}] \leavevmode
int

\end{description}\end{quote}

\end{fulllineitems}

\index{setup() (salabim.Environment method)}

\begin{fulllineitems}
\phantomsection\label{\detokenize{Reference:salabim.Environment.setup}}\pysiglinewithargsret{\sphinxbfcode{setup}}{}{}
called immediately after initialization of an environment.

by default this is a dummy method, but it can be overridden.

only keyword arguments are passed

\end{fulllineitems}

\index{show\_fps() (salabim.Environment method)}

\begin{fulllineitems}
\phantomsection\label{\detokenize{Reference:salabim.Environment.show_fps}}\pysiglinewithargsret{\sphinxbfcode{show\_fps}}{\emph{value=None}}{}~\begin{quote}\begin{description}
\item[{Parameters}] \leavevmode
\sphinxstyleliteralstrong{value} (\sphinxstyleliteralemphasis{bool}) \textendash{} new show\_fps 
if not specified, no change

\item[{Returns}] \leavevmode
\sphinxstylestrong{show\_fps}

\item[{Return type}] \leavevmode
bool

\end{description}\end{quote}

\end{fulllineitems}

\index{show\_time() (salabim.Environment method)}

\begin{fulllineitems}
\phantomsection\label{\detokenize{Reference:salabim.Environment.show_time}}\pysiglinewithargsret{\sphinxbfcode{show\_time}}{\emph{value=None}}{}~\begin{quote}\begin{description}
\item[{Parameters}] \leavevmode
\sphinxstyleliteralstrong{value} (\sphinxstyleliteralemphasis{bool}) \textendash{} new show\_time 
if not specified, no change

\item[{Returns}] \leavevmode
\sphinxstylestrong{show\_time}

\item[{Return type}] \leavevmode
bool

\end{description}\end{quote}

\end{fulllineitems}

\index{snapshot() (salabim.Environment method)}

\begin{fulllineitems}
\phantomsection\label{\detokenize{Reference:salabim.Environment.snapshot}}\pysiglinewithargsret{\sphinxbfcode{snapshot}}{\emph{filename}}{}
Takes a snapshot of the current animated frame (at time = now()) and saves it to a file
\begin{quote}\begin{description}
\item[{Parameters}] \leavevmode
\sphinxstyleliteralstrong{filename} (\sphinxstyleliteralemphasis{str}) \textendash{} file to save the current animated frame to. 
The following formats are accepted: .PNG, .JPG, .BMP, .GIF and .TIFF are supported.
Other formats are not possible.
Note that, apart from .JPG files. the background may be semi transparent by setting
the alpha value to something else than 255.

\end{description}\end{quote}

\end{fulllineitems}

\index{speed() (salabim.Environment method)}

\begin{fulllineitems}
\phantomsection\label{\detokenize{Reference:salabim.Environment.speed}}\pysiglinewithargsret{\sphinxbfcode{speed}}{\emph{value=None}}{}~\begin{quote}\begin{description}
\item[{Parameters}] \leavevmode
\sphinxstyleliteralstrong{value} (\sphinxstyleliteralemphasis{float}) \textendash{} new speed 
if not specified, no change

\item[{Returns}] \leavevmode
\sphinxstylestrong{speed}

\item[{Return type}] \leavevmode
float

\end{description}\end{quote}

\end{fulllineitems}

\index{step() (salabim.Environment method)}

\begin{fulllineitems}
\phantomsection\label{\detokenize{Reference:salabim.Environment.step}}\pysiglinewithargsret{\sphinxbfcode{step}}{}{}
executes the next step of the future event list

for advanced use with animation / GUI loops

\end{fulllineitems}

\index{suppress\_trace\_standby() (salabim.Environment method)}

\begin{fulllineitems}
\phantomsection\label{\detokenize{Reference:salabim.Environment.suppress_trace_standby}}\pysiglinewithargsret{\sphinxbfcode{suppress\_trace\_standby}}{\emph{value=None}}{}
suppress\_trace\_standby status
\begin{quote}\begin{description}
\item[{Parameters}] \leavevmode
\sphinxstyleliteralstrong{value} (\sphinxstyleliteralemphasis{bool}) \textendash{} new suppress\_trace\_standby status 
if omitted, no change

\item[{Returns}] \leavevmode
\sphinxstylestrong{suppress trace status}

\item[{Return type}] \leavevmode
bool

\end{description}\end{quote}

\begin{sphinxadmonition}{note}{Note:}
By default, suppress\_trace\_standby is True, meaning that standby components are
(apart from when they become non standby) suppressed from the trace. 
If you set suppress\_trace\_standby to False, standby components are fully traced.
\end{sphinxadmonition}

\end{fulllineitems}

\index{synced() (salabim.Environment method)}

\begin{fulllineitems}
\phantomsection\label{\detokenize{Reference:salabim.Environment.synced}}\pysiglinewithargsret{\sphinxbfcode{synced}}{\emph{value=None}}{}~\begin{quote}\begin{description}
\item[{Parameters}] \leavevmode
\sphinxstyleliteralstrong{value} (\sphinxstyleliteralemphasis{bool}) \textendash{} new synced 
if not specified, no change

\item[{Returns}] \leavevmode
\sphinxstylestrong{synced}

\item[{Return type}] \leavevmode
bool

\end{description}\end{quote}

\end{fulllineitems}

\index{time\_to\_str\_format() (salabim.Environment method)}

\begin{fulllineitems}
\phantomsection\label{\detokenize{Reference:salabim.Environment.time_to_str_format}}\pysiglinewithargsret{\sphinxbfcode{time\_to\_str\_format}}{\emph{format=None}}{}
sets / gets the the format to display times in trace, animation, etc.
\begin{quote}\begin{description}
\item[{Parameters}] \leavevmode
\sphinxstyleliteralstrong{format} (\sphinxstyleliteralemphasis{str}) \textendash{} specifies how the time should be displayed in trace, animation, etc. 
the format specifier should result in 10 characters. Examples: 
‘\{:10.3f\}’, ‘\{:10.4f\}’, ‘\{:10.0f\}’ and ‘\{:8.1f\} h’ 
Make sure that the returned length is exactly 10 characters.

\item[{Returns}] \leavevmode
\sphinxstylestrong{current specifier (initialized to ‘\{}

\item[{Return type}] \leavevmode
10.3f\}’)

\end{description}\end{quote}

\end{fulllineitems}

\index{to\_days() (salabim.Environment method)}

\begin{fulllineitems}
\phantomsection\label{\detokenize{Reference:salabim.Environment.to_days}}\pysiglinewithargsret{\sphinxbfcode{to\_days}}{\emph{t}}{}
convert time t to days
\begin{quote}\begin{description}
\item[{Parameters}] \leavevmode
\sphinxstyleliteralstrong{t} (\sphinxstyleliteralemphasis{time}) \textendash{} 

\item[{Returns}] \leavevmode
\sphinxstylestrong{Time t converted to days}

\item[{Return type}] \leavevmode
float

\end{description}\end{quote}

\end{fulllineitems}

\index{to\_hours() (salabim.Environment method)}

\begin{fulllineitems}
\phantomsection\label{\detokenize{Reference:salabim.Environment.to_hours}}\pysiglinewithargsret{\sphinxbfcode{to\_hours}}{\emph{t}}{}
convert time t to hours
\begin{quote}\begin{description}
\item[{Parameters}] \leavevmode
\sphinxstyleliteralstrong{t} (\sphinxstyleliteralemphasis{time}) \textendash{} 

\item[{Returns}] \leavevmode
\sphinxstylestrong{Time t converted to hours}

\item[{Return type}] \leavevmode
float

\end{description}\end{quote}

\end{fulllineitems}

\index{to\_microseconds() (salabim.Environment method)}

\begin{fulllineitems}
\phantomsection\label{\detokenize{Reference:salabim.Environment.to_microseconds}}\pysiglinewithargsret{\sphinxbfcode{to\_microseconds}}{\emph{t}}{}
convert time t to microseconds
\begin{quote}\begin{description}
\item[{Parameters}] \leavevmode
\sphinxstyleliteralstrong{t} (\sphinxstyleliteralemphasis{time in microseconds}) \textendash{} 

\item[{Returns}] \leavevmode
\sphinxstylestrong{Time t converted to microseconds}

\item[{Return type}] \leavevmode
float

\end{description}\end{quote}

\end{fulllineitems}

\index{to\_milliseconds() (salabim.Environment method)}

\begin{fulllineitems}
\phantomsection\label{\detokenize{Reference:salabim.Environment.to_milliseconds}}\pysiglinewithargsret{\sphinxbfcode{to\_milliseconds}}{\emph{t}}{}
convert time t to milliseconds
\begin{quote}\begin{description}
\item[{Parameters}] \leavevmode
\sphinxstyleliteralstrong{t} (\sphinxstyleliteralemphasis{time in milliseconds}) \textendash{} 

\item[{Returns}] \leavevmode
\sphinxstylestrong{Time t converted to milliseconds}

\item[{Return type}] \leavevmode
float

\end{description}\end{quote}

\end{fulllineitems}

\index{to\_minutes() (salabim.Environment method)}

\begin{fulllineitems}
\phantomsection\label{\detokenize{Reference:salabim.Environment.to_minutes}}\pysiglinewithargsret{\sphinxbfcode{to\_minutes}}{\emph{t}}{}
convert time t to minutes
\begin{quote}\begin{description}
\item[{Parameters}] \leavevmode
\sphinxstyleliteralstrong{t} (\sphinxstyleliteralemphasis{time}) \textendash{} 

\item[{Returns}] \leavevmode
\sphinxstylestrong{Time t converted to minutes}

\item[{Return type}] \leavevmode
float

\end{description}\end{quote}

\end{fulllineitems}

\index{to\_seconds() (salabim.Environment method)}

\begin{fulllineitems}
\phantomsection\label{\detokenize{Reference:salabim.Environment.to_seconds}}\pysiglinewithargsret{\sphinxbfcode{to\_seconds}}{\emph{t}}{}
convert time t to seconds
\begin{quote}\begin{description}
\item[{Parameters}] \leavevmode
\sphinxstyleliteralstrong{t} (\sphinxstyleliteralemphasis{time}) \textendash{} 

\item[{Returns}] \leavevmode
\sphinxstylestrong{Time t converted to seconds}

\item[{Return type}] \leavevmode
float

\end{description}\end{quote}

\end{fulllineitems}

\index{to\_weeks() (salabim.Environment method)}

\begin{fulllineitems}
\phantomsection\label{\detokenize{Reference:salabim.Environment.to_weeks}}\pysiglinewithargsret{\sphinxbfcode{to\_weeks}}{\emph{t}}{}
convert time t to weeks
\begin{quote}\begin{description}
\item[{Parameters}] \leavevmode
\sphinxstyleliteralstrong{t} (\sphinxstyleliteralemphasis{time}) \textendash{} 

\item[{Returns}] \leavevmode
\sphinxstylestrong{Time t converted to weeks}

\item[{Return type}] \leavevmode
float

\end{description}\end{quote}

\end{fulllineitems}

\index{to\_years() (salabim.Environment method)}

\begin{fulllineitems}
\phantomsection\label{\detokenize{Reference:salabim.Environment.to_years}}\pysiglinewithargsret{\sphinxbfcode{to\_years}}{\emph{t}}{}
convert time t to years
\begin{quote}\begin{description}
\item[{Parameters}] \leavevmode
\sphinxstyleliteralstrong{t} (\sphinxstyleliteralemphasis{time}) \textendash{} 

\item[{Returns}] \leavevmode
\sphinxstylestrong{Time t converted to years}

\item[{Return type}] \leavevmode
float

\end{description}\end{quote}

\end{fulllineitems}

\index{trace() (salabim.Environment method)}

\begin{fulllineitems}
\phantomsection\label{\detokenize{Reference:salabim.Environment.trace}}\pysiglinewithargsret{\sphinxbfcode{trace}}{\emph{value=None}}{}
trace status
\begin{quote}\begin{description}
\item[{Parameters}] \leavevmode
\sphinxstyleliteralstrong{value} (\sphinxstyleliteralemphasis{bool}) \textendash{} new trace status 
if omitted, no change

\item[{Returns}] \leavevmode
\sphinxstylestrong{trace status}

\item[{Return type}] \leavevmode
bool

\end{description}\end{quote}

\begin{sphinxadmonition}{note}{Note:}
If you want to test the status, always include
parentheses, like
\begin{quote}

\sphinxcode{if env.trace():}
\end{quote}
\end{sphinxadmonition}

\end{fulllineitems}

\index{user\_to\_screencoordinates\_size() (salabim.Environment method)}

\begin{fulllineitems}
\phantomsection\label{\detokenize{Reference:salabim.Environment.user_to_screencoordinates_size}}\pysiglinewithargsret{\sphinxbfcode{user\_to\_screencoordinates\_size}}{\emph{usersize}}{}
converts a user size to a value to be used with screen coordinates
\begin{quote}\begin{description}
\item[{Parameters}] \leavevmode
\sphinxstyleliteralstrong{usersize} (\sphinxstyleliteralemphasis{float}) \textendash{} user size to be converted

\item[{Returns}] \leavevmode
\sphinxstylestrong{value corresponding with usersize in screen coordinates}

\item[{Return type}] \leavevmode
float

\end{description}\end{quote}

\end{fulllineitems}

\index{user\_to\_screencoordinates\_x() (salabim.Environment method)}

\begin{fulllineitems}
\phantomsection\label{\detokenize{Reference:salabim.Environment.user_to_screencoordinates_x}}\pysiglinewithargsret{\sphinxbfcode{user\_to\_screencoordinates\_x}}{\emph{userx}}{}
converts a user x coordinate to a screen x coordinate
\begin{quote}\begin{description}
\item[{Parameters}] \leavevmode
\sphinxstyleliteralstrong{userx} (\sphinxstyleliteralemphasis{float}) \textendash{} user x coordinate to be converted

\item[{Returns}] \leavevmode
\sphinxstylestrong{screen x coordinate}

\item[{Return type}] \leavevmode
float

\end{description}\end{quote}

\end{fulllineitems}

\index{user\_to\_screencoordinates\_y() (salabim.Environment method)}

\begin{fulllineitems}
\phantomsection\label{\detokenize{Reference:salabim.Environment.user_to_screencoordinates_y}}\pysiglinewithargsret{\sphinxbfcode{user\_to\_screencoordinates\_y}}{\emph{usery}}{}
converts a user x coordinate to a screen x coordinate
\begin{quote}\begin{description}
\item[{Parameters}] \leavevmode
\sphinxstyleliteralstrong{usery} (\sphinxstyleliteralemphasis{float}) \textendash{} user y coordinate to be converted

\item[{Returns}] \leavevmode
\sphinxstylestrong{screen y coordinate}

\item[{Return type}] \leavevmode
float

\end{description}\end{quote}

\end{fulllineitems}

\index{video() (salabim.Environment method)}

\begin{fulllineitems}
\phantomsection\label{\detokenize{Reference:salabim.Environment.video}}\pysiglinewithargsret{\sphinxbfcode{video}}{\emph{value=None}}{}
video name
\begin{quote}\begin{description}
\item[{Parameters}] \leavevmode
\sphinxstyleliteralstrong{value} (\sphinxstyleliteralemphasis{str}\sphinxstyleliteralemphasis{, }\sphinxstyleliteralemphasis{list}\sphinxstyleliteralemphasis{ or }\sphinxstyleliteralemphasis{tuple}) \textendash{} new video name 
if not specified, no change 
for explanation see animation\_parameters()

\item[{Returns}] \leavevmode
\sphinxstylestrong{video}

\item[{Return type}] \leavevmode
str, list or tuple

\end{description}\end{quote}

\begin{sphinxadmonition}{note}{Note:}
If video is the null string, the video (if any) will be closed.
\end{sphinxadmonition}

\end{fulllineitems}

\index{video\_close() (salabim.Environment method)}

\begin{fulllineitems}
\phantomsection\label{\detokenize{Reference:salabim.Environment.video_close}}\pysiglinewithargsret{\sphinxbfcode{video\_close}}{}{}
closes the current animation video recording, if any.

\end{fulllineitems}

\index{video\_pingpong() (salabim.Environment method)}

\begin{fulllineitems}
\phantomsection\label{\detokenize{Reference:salabim.Environment.video_pingpong}}\pysiglinewithargsret{\sphinxbfcode{video\_pingpong}}{\emph{value=None}}{}
video pingponf
\begin{quote}\begin{description}
\item[{Parameters}] \leavevmode
\sphinxstyleliteralstrong{value} (\sphinxstyleliteralemphasis{bool}) \textendash{} new video pingpong 
if not specified, no change

\item[{Returns}] \leavevmode
\sphinxstylestrong{video pingpong}

\item[{Return type}] \leavevmode
bool

\end{description}\end{quote}

\begin{sphinxadmonition}{note}{Note:}
Applies only to gif animation.
\end{sphinxadmonition}

\end{fulllineitems}

\index{video\_repeat() (salabim.Environment method)}

\begin{fulllineitems}
\phantomsection\label{\detokenize{Reference:salabim.Environment.video_repeat}}\pysiglinewithargsret{\sphinxbfcode{video\_repeat}}{\emph{value=None}}{}
video repeat
\begin{quote}\begin{description}
\item[{Parameters}] \leavevmode
\sphinxstyleliteralstrong{value} (\sphinxstyleliteralemphasis{int}) \textendash{} new video repeat 
if not specified, no change

\item[{Returns}] \leavevmode
\sphinxstylestrong{video repeat}

\item[{Return type}] \leavevmode
int

\end{description}\end{quote}

\begin{sphinxadmonition}{note}{Note:}
Applies only to gif animation.
\end{sphinxadmonition}

\end{fulllineitems}

\index{weeks() (salabim.Environment method)}

\begin{fulllineitems}
\phantomsection\label{\detokenize{Reference:salabim.Environment.weeks}}\pysiglinewithargsret{\sphinxbfcode{weeks}}{\emph{t}}{}
convert the given time in weeks to the current time unit
\begin{quote}\begin{description}
\item[{Parameters}] \leavevmode
\sphinxstyleliteralstrong{t} (\sphinxstyleliteralemphasis{float}) \textendash{} time in weeks

\item[{Returns}] \leavevmode
\sphinxstylestrong{time in weeks, converted to the current time\_unit}

\item[{Return type}] \leavevmode
float

\end{description}\end{quote}

\end{fulllineitems}

\index{width() (salabim.Environment method)}

\begin{fulllineitems}
\phantomsection\label{\detokenize{Reference:salabim.Environment.width}}\pysiglinewithargsret{\sphinxbfcode{width}}{\emph{value=None}}{}
width of the animation in screen coordinates
\begin{quote}\begin{description}
\item[{Parameters}] \leavevmode
\sphinxstyleliteralstrong{value} (\sphinxstyleliteralemphasis{int}) \textendash{} new width 
if not specified, no change

\item[{Returns}] \leavevmode
\sphinxstylestrong{width of animation}

\item[{Return type}] \leavevmode
int

\end{description}\end{quote}

\end{fulllineitems}

\index{x0() (salabim.Environment method)}

\begin{fulllineitems}
\phantomsection\label{\detokenize{Reference:salabim.Environment.x0}}\pysiglinewithargsret{\sphinxbfcode{x0}}{\emph{value=None}}{}
x coordinate of lower left corner of animation
\begin{quote}\begin{description}
\item[{Parameters}] \leavevmode
\sphinxstyleliteralstrong{value} (\sphinxstyleliteralemphasis{float}) \textendash{} new x coordinate

\item[{Returns}] \leavevmode
\sphinxstylestrong{x coordinate of lower left corner of animation}

\item[{Return type}] \leavevmode
float

\end{description}\end{quote}

\end{fulllineitems}

\index{x1() (salabim.Environment method)}

\begin{fulllineitems}
\phantomsection\label{\detokenize{Reference:salabim.Environment.x1}}\pysiglinewithargsret{\sphinxbfcode{x1}}{\emph{value=None}}{}
x coordinate of upper right corner of animation : float
\begin{quote}\begin{description}
\item[{Parameters}] \leavevmode
\sphinxstyleliteralstrong{value} (\sphinxstyleliteralemphasis{float}) \textendash{} new x coordinate 
if not specified, no change

\item[{Returns}] \leavevmode
\sphinxstylestrong{x coordinate of upper right corner of animation}

\item[{Return type}] \leavevmode
float

\end{description}\end{quote}

\end{fulllineitems}

\index{y0() (salabim.Environment method)}

\begin{fulllineitems}
\phantomsection\label{\detokenize{Reference:salabim.Environment.y0}}\pysiglinewithargsret{\sphinxbfcode{y0}}{\emph{value=None}}{}
y coordinate of lower left corner of animation
\begin{quote}\begin{description}
\item[{Parameters}] \leavevmode
\sphinxstyleliteralstrong{value} (\sphinxstyleliteralemphasis{float}) \textendash{} new y coordinate 
if not specified, no change

\item[{Returns}] \leavevmode
\sphinxstylestrong{y coordinate of lower left corner of animation}

\item[{Return type}] \leavevmode
float

\end{description}\end{quote}

\end{fulllineitems}

\index{y1() (salabim.Environment method)}

\begin{fulllineitems}
\phantomsection\label{\detokenize{Reference:salabim.Environment.y1}}\pysiglinewithargsret{\sphinxbfcode{y1}}{}{}
y coordinate of upper right corner of animation
\begin{quote}\begin{description}
\item[{Returns}] \leavevmode
\sphinxstylestrong{y coordinate of upper right corner of animation}

\item[{Return type}] \leavevmode
float

\end{description}\end{quote}

\begin{sphinxadmonition}{note}{Note:}
It is not possible to set this value explicitely.
\end{sphinxadmonition}

\end{fulllineitems}

\index{years() (salabim.Environment method)}

\begin{fulllineitems}
\phantomsection\label{\detokenize{Reference:salabim.Environment.years}}\pysiglinewithargsret{\sphinxbfcode{years}}{\emph{t}}{}
convert the given time in years to the current time unit
\begin{quote}\begin{description}
\item[{Parameters}] \leavevmode
\sphinxstyleliteralstrong{t} (\sphinxstyleliteralemphasis{float}) \textendash{} time in years

\item[{Returns}] \leavevmode
\sphinxstylestrong{time in years, converted to the current time\_unit}

\item[{Return type}] \leavevmode
float

\end{description}\end{quote}

\end{fulllineitems}


\end{fulllineitems}



\section{ItemFile}
\label{\detokenize{Reference:itemfile}}\index{ItemFile (class in salabim)}

\begin{fulllineitems}
\phantomsection\label{\detokenize{Reference:salabim.ItemFile}}\pysiglinewithargsret{\sphinxbfcode{class }\sphinxcode{salabim.}\sphinxbfcode{ItemFile}}{\emph{filename}}{}
define an item file to be used with read\_item, read\_item\_int, read\_item\_float and read\_item\_bool
\begin{quote}\begin{description}
\item[{Parameters}] \leavevmode
\sphinxstyleliteralstrong{filename} (\sphinxstyleliteralemphasis{str}) \textendash{} file to be used for subsequent read\_item, read\_item\_int, read\_item\_float and read\_item\_bool calls 
or 
content to be interpreted used in subsequent read\_item calls. The content should have at least one linefeed
character and will be usually  triple quoted.

\end{description}\end{quote}

\begin{sphinxadmonition}{note}{Note:}
It is advised to use ItemFile with a context manager, like

\begin{sphinxVerbatim}[commandchars=\\\{\}]
\PYG{k}{with} \PYG{n}{sim}\PYG{o}{.}\PYG{n}{ItemFile}\PYG{p}{(}\PYG{l+s+s1}{\PYGZsq{}}\PYG{l+s+s1}{experiment0.txt}\PYG{l+s+s1}{\PYGZsq{}}\PYG{p}{)} \PYG{k}{as} \PYG{n}{f}\PYG{p}{:}
    \PYG{n}{run\PYGZus{}length} \PYG{o}{=} \PYG{n}{f}\PYG{o}{.}\PYG{n}{read\PYGZus{}item\PYGZus{}float}\PYG{p}{(}\PYG{p}{)} \PYG{o}{\textbar{}}\PYG{n}{n}\PYG{o}{\textbar{}}
    \PYG{n}{run\PYGZus{}name} \PYG{o}{=} \PYG{n}{f}\PYG{o}{.}\PYG{n}{read\PYGZus{}item}\PYG{p}{(}\PYG{p}{)} \PYG{o}{\textbar{}}\PYG{n}{n}\PYG{o}{\textbar{}}
\end{sphinxVerbatim}

Alternatively, the file can be opened and closed explicitely, like

\begin{sphinxVerbatim}[commandchars=\\\{\}]
\PYG{n}{f} \PYG{o}{=} \PYG{n}{sim}\PYG{o}{.}\PYG{n}{ItemFile}\PYG{p}{(}\PYG{l+s+s1}{\PYGZsq{}}\PYG{l+s+s1}{experiment0.txt}\PYG{l+s+s1}{\PYGZsq{}}\PYG{p}{)}
\PYG{n}{run\PYGZus{}length} \PYG{o}{=} \PYG{n}{f}\PYG{o}{.}\PYG{n}{read\PYGZus{}item\PYGZus{}float}\PYG{p}{(}\PYG{p}{)}
\PYG{n}{run\PYGZus{}name} \PYG{o}{=} \PYG{n}{f}\PYG{o}{.}\PYG{n}{read\PYGZus{}item}\PYG{p}{(}\PYG{p}{)}
\PYG{n}{f}\PYG{o}{.}\PYG{n}{close}\PYG{p}{(}\PYG{p}{)}
\end{sphinxVerbatim}

Item files consist of individual items separated by whitespace (blank or tab)\textbar{}n\textbar{}
If a blank or tab is required in an item, use single or double quotes 
All text following \# on a line is ignored 
All texts on a line within curly brackets \{\} is ignored and considered white space. 
Curly braces cannot spawn multiple lines and cannot be nested.

Example

\begin{sphinxVerbatim}[commandchars=\\\{\}]
\PYG{n}{Item1}
\PYG{l+s+s1}{\PYGZsq{}}\PYG{l+s+s1}{Item 2}\PYG{l+s+s1}{\PYGZsq{}}
    \PYG{n}{Item3} \PYG{n}{Item4} \PYG{c+c1}{\PYGZsh{} comment}
\PYG{n}{Item5} \PYG{p}{\PYGZob{}}\PYG{n}{five}\PYG{p}{\PYGZcb{}} \PYG{n}{Item6} \PYG{p}{\PYGZob{}}\PYG{n}{six}\PYG{p}{\PYGZcb{}}
\PYG{l+s+s1}{\PYGZsq{}}\PYG{l+s+s1}{Double quote}\PYG{l+s+s1}{\PYGZdq{}}\PYG{l+s+s1}{ in item}\PYG{l+s+s1}{\PYGZsq{}}
\PYG{l+s+s2}{\PYGZdq{}}\PYG{l+s+s2}{Single quote}\PYG{l+s+s2}{\PYGZsq{}}\PYG{l+s+s2}{ in item}\PYG{l+s+s2}{\PYGZdq{}}
\PYG{k+kc}{True}
\end{sphinxVerbatim}
\end{sphinxadmonition}
\index{read\_item() (salabim.ItemFile method)}

\begin{fulllineitems}
\phantomsection\label{\detokenize{Reference:salabim.ItemFile.read_item}}\pysiglinewithargsret{\sphinxbfcode{read\_item}}{}{}
read next item from the ItemFile

if the end of file is reached, EOFError is raised

\end{fulllineitems}

\index{read\_item\_bool() (salabim.ItemFile method)}

\begin{fulllineitems}
\phantomsection\label{\detokenize{Reference:salabim.ItemFile.read_item_bool}}\pysiglinewithargsret{\sphinxbfcode{read\_item\_bool}}{}{}
read next item from the ItemFile as bool

A value of False (not case sensitive) will return False 
A value of 0 will return False 
The null string will return False 
Any other value will return True

if the end of file is reached, EOFError is raised

\end{fulllineitems}

\index{read\_item\_float() (salabim.ItemFile method)}

\begin{fulllineitems}
\phantomsection\label{\detokenize{Reference:salabim.ItemFile.read_item_float}}\pysiglinewithargsret{\sphinxbfcode{read\_item\_float}}{}{}
read next item from the ItemFile as float

if the end of file is reached, EOFError is raised

\end{fulllineitems}

\index{read\_item\_int() (salabim.ItemFile method)}

\begin{fulllineitems}
\phantomsection\label{\detokenize{Reference:salabim.ItemFile.read_item_int}}\pysiglinewithargsret{\sphinxbfcode{read\_item\_int}}{}{}
read next field from the ItemFile as int.

if the end of file is reached, EOFError is raised

\end{fulllineitems}


\end{fulllineitems}



\section{Monitor}
\label{\detokenize{Reference:monitor}}\index{Monitor (class in salabim)}

\begin{fulllineitems}
\phantomsection\label{\detokenize{Reference:salabim.Monitor}}\pysiglinewithargsret{\sphinxbfcode{class }\sphinxcode{salabim.}\sphinxbfcode{Monitor}}{\emph{name=None}, \emph{monitor=True}, \emph{level=False}, \emph{initial\_tally=None}, \emph{type=None}, \emph{weight\_legend=None}, \emph{env=None}, \emph{*args}, \emph{**kwargs}}{}
Monitor object
\begin{quote}\begin{description}
\item[{Parameters}] \leavevmode\begin{itemize}
\item {} 
\sphinxstyleliteralstrong{name} (\sphinxstyleliteralemphasis{str}) \textendash{} name of the monitor 
if the name ends with a period (.),
auto serializing will be applied 
if the name end with a comma,
auto serializing starting at 1 will be applied 
if omitted, the name will be derived from the class
it is defined in (lowercased)

\item {} 
\sphinxstyleliteralstrong{monitor} (\sphinxstyleliteralemphasis{bool}) \textendash{} if True (default), monitoring will be on. 
if False, monitoring is disabled 
it is possible to control monitoring later,
with the monitor method

\item {} 
\sphinxstyleliteralstrong{level} (\sphinxstyleliteralemphasis{bool}) \textendash{} if False (default), individual values are tallied, optionally with weight 
if True, the tallied vslues are interpreted as levels

\item {} 
\sphinxstyleliteralstrong{initial\_tally} (\sphinxstyleliteralemphasis{any}\sphinxstyleliteralemphasis{, }\sphinxstyleliteralemphasis{preferably int}\sphinxstyleliteralemphasis{, }\sphinxstyleliteralemphasis{float}\sphinxstyleliteralemphasis{ or }\sphinxstyleliteralemphasis{translatable into int}\sphinxstyleliteralemphasis{ or }\sphinxstyleliteralemphasis{float}) \textendash{} initial value for the a level monitor 
it is important to set the value correctly.
default: 0 
not available for non level monitors

\item {} 
\sphinxstyleliteralstrong{type} (\sphinxstyleliteralemphasis{str}) \textendash{} \begin{description}
\item[{specifies how tallied values are to be stored}] \leavevmode\begin{itemize}
\item {} \begin{description}
\item[{’any’ (default) stores values in a list. This allows}] \leavevmode
non numeric values. In calculations the values are
forced to a numeric value (0 if not possible)

\end{description}

\item {} 
’bool’ (True, False) Actually integer \textgreater{}= 0 \textless{}= 255 1 byte

\item {} 
’int8’ integer \textgreater{}= -128 \textless{}= 127 1 byte

\item {} 
’uint8’ integer \textgreater{}= 0 \textless{}= 255 1 byte

\item {} 
’int16’ integer \textgreater{}= -32768 \textless{}= 32767 2 bytes

\item {} 
’uint16’ integer \textgreater{}= 0 \textless{}= 65535 2 bytes

\item {} 
’int32’ integer \textgreater{}= -2147483648\textless{}= 2147483647 4 bytes

\item {} 
’uint32’ integer \textgreater{}= 0 \textless{}= 4294967295 4 bytes

\item {} 
’int64’ integer \textgreater{}= -9223372036854775808 \textless{}= 9223372036854775807 8 bytes

\item {} 
’uint64’ integer \textgreater{}= 0 \textless{}= 18446744073709551615 8 bytes

\item {} 
’float’ float 8 bytes

\end{itemize}

\end{description}


\item {} 
\sphinxstyleliteralstrong{weight\_legend} (\sphinxstyleliteralemphasis{str}) \textendash{} used in print\_statistics and print\_histogram to indicate the dimension of weight or duration (for
level monitors, e.g. minutes. Default: weight for non level monitors, duration for level monitors.

\item {} 
\sphinxstyleliteralstrong{env} ({\hyperref[\detokenize{Reference:salabim.Environment}]{\sphinxcrossref{\sphinxstyleliteralemphasis{Environment}}}}) \textendash{} environment where the monitor is defined 
if omitted, default\_env will be used

\end{itemize}

\end{description}\end{quote}
\index{animate() (salabim.Monitor method)}

\begin{fulllineitems}
\phantomsection\label{\detokenize{Reference:salabim.Monitor.animate}}\pysiglinewithargsret{\sphinxbfcode{animate}}{\emph{*args}, \emph{**kwargs}}{}
animates the monitor in a panel
\begin{quote}\begin{description}
\item[{Parameters}] \leavevmode\begin{itemize}
\item {} 
\sphinxstyleliteralstrong{linecolor} (\sphinxstyleliteralemphasis{colorspec}) \textendash{} color of the line or points (default foreground color)

\item {} 
\sphinxstyleliteralstrong{linewidth} (\sphinxstyleliteralemphasis{int}) \textendash{} width of the line or points (default 1 for line, 3 for points)

\item {} 
\sphinxstyleliteralstrong{fillcolor} (\sphinxstyleliteralemphasis{colorspec}) \textendash{} color of the panel (default transparent)

\item {} 
\sphinxstyleliteralstrong{bordercolor} (\sphinxstyleliteralemphasis{colorspec}) \textendash{} color of the border (default foreground color)

\item {} 
\sphinxstyleliteralstrong{borderlinewidth} (\sphinxstyleliteralemphasis{int}) \textendash{} width of the line around the panel (default 1)

\item {} 
\sphinxstyleliteralstrong{nowcolor} (\sphinxstyleliteralemphasis{colorspec}) \textendash{} color of the line indicating now (default red)

\item {} 
\sphinxstyleliteralstrong{titlecolor} (\sphinxstyleliteralemphasis{colorspec}) \textendash{} color of the title (default foreground color)

\item {} 
\sphinxstyleliteralstrong{titlefont} ({\hyperref[\detokenize{Reference:salabim.Animate.font}]{\sphinxcrossref{\sphinxstyleliteralemphasis{font}}}}) \textendash{} font of the title (default ‘’)

\item {} 
\sphinxstyleliteralstrong{titlefontsize} (\sphinxstyleliteralemphasis{int}) \textendash{} size of the font of the title (default 15)

\item {} 
\sphinxstyleliteralstrong{title} (\sphinxstyleliteralemphasis{str}) \textendash{} title to be shown above panel 
default: name of the monitor

\item {} 
\sphinxstyleliteralstrong{x} (\sphinxstyleliteralemphasis{int}) \textendash{} x-coordinate of panel, relative to xy\_anchor, default 0

\item {} 
\sphinxstyleliteralstrong{y} (\sphinxstyleliteralemphasis{int}) \textendash{} y-coordinate of panel, relative to xy\_anchor. default 0

\item {} 
\sphinxstyleliteralstrong{xy\_anchor} (\sphinxstyleliteralemphasis{str}) \textendash{} specifies where x and y are relative to 
possible values are (default: sw): 
\sphinxcode{nw    n    ne} 
\sphinxcode{w     c     e} 
\sphinxcode{sw    s    se}

\item {} 
\sphinxstyleliteralstrong{vertical\_offset} (\sphinxstyleliteralemphasis{float}) \textendash{} \begin{description}
\item[{the vertical position of x within the panel is}] \leavevmode
vertical\_offset + x * vertical\_scale (default 0)

\end{description}


\item {} 
\sphinxstyleliteralstrong{vertical\_scale} (\sphinxstyleliteralemphasis{float}) \textendash{} the vertical position of x within the panel is
vertical\_offset + x * vertical\_scale (default 5)

\item {} 
\sphinxstyleliteralstrong{horizontal\_scale} (\sphinxstyleliteralemphasis{float}) \textendash{} for timescaled monitors the relative horizontal position of time t within the panel is on
t * horizontal\_scale, possibly shifted (default 1)\textbar{}n\textbar{}
for non timescaled monitors, the relative horizontal position of index i within the panel is on
i * horizontal\_scale, possibly shifted (default 5)\textbar{}n\textbar{}

\item {} 
\sphinxstyleliteralstrong{width} (\sphinxstyleliteralemphasis{int}) \textendash{} width of the panel (default 200)

\item {} 
\sphinxstyleliteralstrong{height} (\sphinxstyleliteralemphasis{int}) \textendash{} height of the panel (default 75)

\item {} 
\sphinxstyleliteralstrong{layer} (\sphinxstyleliteralemphasis{int}) \textendash{} layer (default 0)

\end{itemize}

\item[{Returns}] \leavevmode
\sphinxstylestrong{reference to AnimateMonitor object}

\item[{Return type}] \leavevmode
{\hyperref[\detokenize{Reference:salabim.AnimateMonitor}]{\sphinxcrossref{AnimateMonitor}}}

\end{description}\end{quote}

\begin{sphinxadmonition}{note}{Note:}
It is recommended to use sim.AnimateMonitor instead 

All measures are in screen coordinates 
\end{sphinxadmonition}

\end{fulllineitems}

\index{base\_name() (salabim.Monitor method)}

\begin{fulllineitems}
\phantomsection\label{\detokenize{Reference:salabim.Monitor.base_name}}\pysiglinewithargsret{\sphinxbfcode{base\_name}}{}{}~\begin{quote}\begin{description}
\item[{Returns}] \leavevmode
\sphinxstylestrong{base name of the monitor (the name used at initialization)}

\item[{Return type}] \leavevmode
str

\end{description}\end{quote}

\end{fulllineitems}

\index{bin\_duration() (salabim.Monitor method)}

\begin{fulllineitems}
\phantomsection\label{\detokenize{Reference:salabim.Monitor.bin_duration}}\pysiglinewithargsret{\sphinxbfcode{bin\_duration}}{\emph{lowerbound}, \emph{upperbound}}{}
total duration of tallied values in range (lowerbound,upperbound{]}
\begin{quote}\begin{description}
\item[{Parameters}] \leavevmode\begin{itemize}
\item {} 
\sphinxstyleliteralstrong{lowerbound} (\sphinxstyleliteralemphasis{float}) \textendash{} non inclusive lowerbound

\item {} 
\sphinxstyleliteralstrong{upperbound} (\sphinxstyleliteralemphasis{float}) \textendash{} inclusive upperbound

\item {} 
\sphinxstyleliteralstrong{ex0} (\sphinxstyleliteralemphasis{bool}) \textendash{} if False (default), include zeroes. if True, exclude zeroes

\end{itemize}

\item[{Returns}] \leavevmode
\sphinxstylestrong{total duration of values \textgreater{}lowerbound and \textless{}=upperbound}

\item[{Return type}] \leavevmode
int

\end{description}\end{quote}

\begin{sphinxadmonition}{note}{Note:}
Not available for level monitors
\end{sphinxadmonition}

\end{fulllineitems}

\index{bin\_number\_of\_entries() (salabim.Monitor method)}

\begin{fulllineitems}
\phantomsection\label{\detokenize{Reference:salabim.Monitor.bin_number_of_entries}}\pysiglinewithargsret{\sphinxbfcode{bin\_number\_of\_entries}}{\emph{lowerbound}, \emph{upperbound}, \emph{ex0=False}}{}
count of the number of tallied values in range (lowerbound,upperbound{]}
\begin{quote}\begin{description}
\item[{Parameters}] \leavevmode\begin{itemize}
\item {} 
\sphinxstyleliteralstrong{lowerbound} (\sphinxstyleliteralemphasis{float}) \textendash{} non inclusive lowerbound

\item {} 
\sphinxstyleliteralstrong{upperbound} (\sphinxstyleliteralemphasis{float}) \textendash{} inclusive upperbound

\item {} 
\sphinxstyleliteralstrong{ex0} (\sphinxstyleliteralemphasis{bool}) \textendash{} if False (default), include zeroes. if True, exclude zeroes

\end{itemize}

\item[{Returns}] \leavevmode
\sphinxstylestrong{number of values \textgreater{}lowerbound and \textless{}=upperbound}

\item[{Return type}] \leavevmode
int

\end{description}\end{quote}

\begin{sphinxadmonition}{note}{Note:}
Not available for level monitors
\end{sphinxadmonition}

\end{fulllineitems}

\index{bin\_weight() (salabim.Monitor method)}

\begin{fulllineitems}
\phantomsection\label{\detokenize{Reference:salabim.Monitor.bin_weight}}\pysiglinewithargsret{\sphinxbfcode{bin\_weight}}{\emph{lowerbound}, \emph{upperbound}}{}
total weight of tallied values in range (lowerbound,upperbound{]}
\begin{quote}\begin{description}
\item[{Parameters}] \leavevmode\begin{itemize}
\item {} 
\sphinxstyleliteralstrong{lowerbound} (\sphinxstyleliteralemphasis{float}) \textendash{} non inclusive lowerbound

\item {} 
\sphinxstyleliteralstrong{upperbound} (\sphinxstyleliteralemphasis{float}) \textendash{} inclusive upperbound

\item {} 
\sphinxstyleliteralstrong{ex0} (\sphinxstyleliteralemphasis{bool}) \textendash{} if False (default), include zeroes. if True, exclude zeroes

\end{itemize}

\item[{Returns}] \leavevmode
\sphinxstylestrong{total weight of values \textgreater{}lowerbound and \textless{}=upperbound}

\item[{Return type}] \leavevmode
int

\end{description}\end{quote}

\begin{sphinxadmonition}{note}{Note:}
Not available for level monitors
\end{sphinxadmonition}

\end{fulllineitems}

\index{deregister() (salabim.Monitor method)}

\begin{fulllineitems}
\phantomsection\label{\detokenize{Reference:salabim.Monitor.deregister}}\pysiglinewithargsret{\sphinxbfcode{deregister}}{\emph{registry}}{}
deregisters the monitor in the registry
\begin{quote}\begin{description}
\item[{Parameters}] \leavevmode
\sphinxstyleliteralstrong{registry} (\sphinxstyleliteralemphasis{list}) \textendash{} list of registered objects

\item[{Returns}] \leavevmode
\sphinxstylestrong{monitor (self)}

\item[{Return type}] \leavevmode
{\hyperref[\detokenize{Reference:salabim.Monitor}]{\sphinxcrossref{Monitor}}}

\end{description}\end{quote}

\end{fulllineitems}

\index{duration() (salabim.Monitor method)}

\begin{fulllineitems}
\phantomsection\label{\detokenize{Reference:salabim.Monitor.duration}}\pysiglinewithargsret{\sphinxbfcode{duration}}{\emph{ex0=False}}{}
total duration
\begin{quote}\begin{description}
\item[{Parameters}] \leavevmode
\sphinxstyleliteralstrong{ex0} (\sphinxstyleliteralemphasis{bool}) \textendash{} if False (default), include zeroes. if True, exclude zeroes

\item[{Returns}] \leavevmode
\sphinxstylestrong{total duration}

\item[{Return type}] \leavevmode
float

\end{description}\end{quote}

\begin{sphinxadmonition}{note}{Note:}
Not available for non level monitors
\end{sphinxadmonition}

\end{fulllineitems}

\index{duration\_zero() (salabim.Monitor method)}

\begin{fulllineitems}
\phantomsection\label{\detokenize{Reference:salabim.Monitor.duration_zero}}\pysiglinewithargsret{\sphinxbfcode{duration\_zero}}{}{}
total duratiom of zero entries
\begin{quote}\begin{description}
\item[{Returns}] \leavevmode
\sphinxstylestrong{total duration of zero entries}

\item[{Return type}] \leavevmode
float

\end{description}\end{quote}

\begin{sphinxadmonition}{note}{Note:}
Not available for non level monitors
\end{sphinxadmonition}

\end{fulllineitems}

\index{get() (salabim.Monitor method)}

\begin{fulllineitems}
\phantomsection\label{\detokenize{Reference:salabim.Monitor.get}}\pysiglinewithargsret{\sphinxbfcode{get}}{\emph{t=None}}{}~\begin{quote}\begin{description}
\item[{Parameters}] \leavevmode
\sphinxstyleliteralstrong{t} (\sphinxstyleliteralemphasis{float}) \textendash{} time at which the value of the level is to be returned 
default: now

\item[{Returns}] \leavevmode

\sphinxstylestrong{last tallied value} \textendash{} Instead of this method, the level monitor can also be called directly, like 

level = sim.Monitor(‘level’, level=True) 
… 
print(level()) 
print(level.get())  \# identical 


\item[{Return type}] \leavevmode
any, usually float

\end{description}\end{quote}

\begin{sphinxadmonition}{note}{Note:}
If the value is not available, self.off will be returned.
\end{sphinxadmonition}

\end{fulllineitems}

\index{histogram\_autoscale() (salabim.Monitor method)}

\begin{fulllineitems}
\phantomsection\label{\detokenize{Reference:salabim.Monitor.histogram_autoscale}}\pysiglinewithargsret{\sphinxbfcode{histogram\_autoscale}}{\emph{ex0=False}}{}
used by histogram\_print to autoscale 
may be overridden.
\begin{quote}\begin{description}
\item[{Parameters}] \leavevmode
\sphinxstyleliteralstrong{ex0} (\sphinxstyleliteralemphasis{bool}) \textendash{} if False (default), include zeroes. if True, exclude zeroes

\item[{Returns}] \leavevmode
\sphinxstylestrong{bin\_width, lowerbound, number\_of\_bins}

\item[{Return type}] \leavevmode
tuple

\end{description}\end{quote}

\end{fulllineitems}

\index{maximum() (salabim.Monitor method)}

\begin{fulllineitems}
\phantomsection\label{\detokenize{Reference:salabim.Monitor.maximum}}\pysiglinewithargsret{\sphinxbfcode{maximum}}{\emph{ex0=False}}{}
maximum of tallied values
\begin{quote}\begin{description}
\item[{Parameters}] \leavevmode
\sphinxstyleliteralstrong{ex0} (\sphinxstyleliteralemphasis{bool}) \textendash{} if False (default), include zeroes. if True, exclude zeroes

\item[{Returns}] \leavevmode
\sphinxstylestrong{maximum}

\item[{Return type}] \leavevmode
float

\end{description}\end{quote}

\end{fulllineitems}

\index{mean() (salabim.Monitor method)}

\begin{fulllineitems}
\phantomsection\label{\detokenize{Reference:salabim.Monitor.mean}}\pysiglinewithargsret{\sphinxbfcode{mean}}{\emph{ex0=False}}{}
mean of tallied values
\begin{quote}\begin{description}
\item[{Parameters}] \leavevmode
\sphinxstyleliteralstrong{ex0} (\sphinxstyleliteralemphasis{bool}) \textendash{} if False (default), include zeroes. if True, exclude zeroes

\item[{Returns}] \leavevmode
\sphinxstylestrong{mean}

\item[{Return type}] \leavevmode
float

\end{description}\end{quote}

\begin{sphinxadmonition}{note}{Note:}
For weighs are applied , the weighted mean is returned
\end{sphinxadmonition}

\end{fulllineitems}

\index{median() (salabim.Monitor method)}

\begin{fulllineitems}
\phantomsection\label{\detokenize{Reference:salabim.Monitor.median}}\pysiglinewithargsret{\sphinxbfcode{median}}{\emph{ex0=False}}{}
median of tallied values
\begin{quote}\begin{description}
\item[{Parameters}] \leavevmode
\sphinxstyleliteralstrong{ex0} (\sphinxstyleliteralemphasis{bool}) \textendash{} if False (default), include zeroes. if True, exclude zeroes

\item[{Returns}] \leavevmode
\sphinxstylestrong{median}

\item[{Return type}] \leavevmode
float

\end{description}\end{quote}

\begin{sphinxadmonition}{note}{Note:}
If weight are applied, the weighted median is returned
\end{sphinxadmonition}

\end{fulllineitems}

\index{merge() (salabim.Monitor method)}

\begin{fulllineitems}
\phantomsection\label{\detokenize{Reference:salabim.Monitor.merge}}\pysiglinewithargsret{\sphinxbfcode{merge}}{\emph{*monitors}, \emph{**kwargs}}{}
merges this monitor with other monitors
\begin{quote}\begin{description}
\item[{Parameters}] \leavevmode\begin{itemize}
\item {} 
\sphinxstyleliteralstrong{monitors} (\sphinxstyleliteralemphasis{sequence}) \textendash{} zero of more monitors to be merged to this monitor

\item {} 
\sphinxstyleliteralstrong{name} (\sphinxstyleliteralemphasis{str}) \textendash{} name of the merged monitor 
default: name of this monitor + ‘.merged’

\end{itemize}

\item[{Returns}] \leavevmode
\sphinxstylestrong{merged monitor}

\item[{Return type}] \leavevmode
{\hyperref[\detokenize{Reference:salabim.Monitor}]{\sphinxcrossref{Monitor}}}

\end{description}\end{quote}

\begin{sphinxadmonition}{note}{Note:}
Level monitors can only be merged with level monitors 
Non level monitors can only be merged with non level monitors 
Only monitors with the same type can be merged 
If no monitors are specified, a copy is created. 
For level monitors, merging means summing the available x-values\textbar{}n\textbar{}
\end{sphinxadmonition}

\end{fulllineitems}

\index{minimum() (salabim.Monitor method)}

\begin{fulllineitems}
\phantomsection\label{\detokenize{Reference:salabim.Monitor.minimum}}\pysiglinewithargsret{\sphinxbfcode{minimum}}{\emph{ex0=False}}{}
minimum of tallied values
\begin{quote}\begin{description}
\item[{Parameters}] \leavevmode
\sphinxstyleliteralstrong{ex0} (\sphinxstyleliteralemphasis{bool}) \textendash{} if False (default), include zeroes. if True, exclude zeroes

\item[{Returns}] \leavevmode
\sphinxstylestrong{minimum}

\item[{Return type}] \leavevmode
float

\end{description}\end{quote}

\end{fulllineitems}

\index{monitor() (salabim.Monitor method)}

\begin{fulllineitems}
\phantomsection\label{\detokenize{Reference:salabim.Monitor.monitor}}\pysiglinewithargsret{\sphinxbfcode{monitor}}{\emph{value=None}}{}
enables/disables monitor
\begin{quote}\begin{description}
\item[{Parameters}] \leavevmode
\sphinxstyleliteralstrong{value} (\sphinxstyleliteralemphasis{bool}) \textendash{} if True, monitoring will be on. 
if False, monitoring is disabled 
if omitted, no change

\item[{Returns}] \leavevmode
\sphinxstylestrong{True, if monitoring enabled. False, if not}

\item[{Return type}] \leavevmode
bool

\end{description}\end{quote}

\end{fulllineitems}

\index{name() (salabim.Monitor method)}

\begin{fulllineitems}
\phantomsection\label{\detokenize{Reference:salabim.Monitor.name}}\pysiglinewithargsret{\sphinxbfcode{name}}{\emph{value=None}}{}~\begin{quote}\begin{description}
\item[{Parameters}] \leavevmode
\sphinxstyleliteralstrong{value} (\sphinxstyleliteralemphasis{str}) \textendash{} new name of the monitor
if omitted, no change

\item[{Returns}] \leavevmode
\sphinxstylestrong{Name of the monitor}

\item[{Return type}] \leavevmode
str

\end{description}\end{quote}

\begin{sphinxadmonition}{note}{Note:}
base\_name and sequence\_number are not affected if the name is changed
\end{sphinxadmonition}

\end{fulllineitems}

\index{number\_of\_entries() (salabim.Monitor method)}

\begin{fulllineitems}
\phantomsection\label{\detokenize{Reference:salabim.Monitor.number_of_entries}}\pysiglinewithargsret{\sphinxbfcode{number\_of\_entries}}{\emph{ex0=False}}{}
count of the number of entries
\begin{quote}\begin{description}
\item[{Parameters}] \leavevmode
\sphinxstyleliteralstrong{ex0} (\sphinxstyleliteralemphasis{bool}) \textendash{} if False (default), include zeroes. if True, exclude zeroes

\item[{Returns}] \leavevmode
\sphinxstylestrong{number of entries}

\item[{Return type}] \leavevmode
int

\end{description}\end{quote}

\begin{sphinxadmonition}{note}{Note:}
Not available for level monitors
\end{sphinxadmonition}

\end{fulllineitems}

\index{number\_of\_entries\_zero() (salabim.Monitor method)}

\begin{fulllineitems}
\phantomsection\label{\detokenize{Reference:salabim.Monitor.number_of_entries_zero}}\pysiglinewithargsret{\sphinxbfcode{number\_of\_entries\_zero}}{}{}
count of the number of zero entries
\begin{quote}\begin{description}
\item[{Returns}] \leavevmode
\sphinxstylestrong{number of zero entries}

\item[{Return type}] \leavevmode
int

\end{description}\end{quote}

\begin{sphinxadmonition}{note}{Note:}
Not available for level monitors
\end{sphinxadmonition}

\end{fulllineitems}

\index{percentile() (salabim.Monitor method)}

\begin{fulllineitems}
\phantomsection\label{\detokenize{Reference:salabim.Monitor.percentile}}\pysiglinewithargsret{\sphinxbfcode{percentile}}{\emph{q}, \emph{ex0=False}}{}
q-th percentile of tallied values
\begin{quote}\begin{description}
\item[{Parameters}] \leavevmode\begin{itemize}
\item {} 
\sphinxstyleliteralstrong{q} (\sphinxstyleliteralemphasis{float}) \textendash{} percentage of the distribution 
values \textless{}0 are treated a 0 
values \textgreater{}100 are treated as 100

\item {} 
\sphinxstyleliteralstrong{ex0} (\sphinxstyleliteralemphasis{bool}) \textendash{} if False (default), include zeroes. if True, exclude zeroes

\end{itemize}

\item[{Returns}] \leavevmode
q-th percentile 
0 returns the minimum, 50 the median and 100 the maximum

\item[{Return type}] \leavevmode
float

\end{description}\end{quote}

\begin{sphinxadmonition}{note}{Note:}
If weights are applied, the weighted percentile is returned
\end{sphinxadmonition}

\end{fulllineitems}

\index{print\_histogram() (salabim.Monitor method)}

\begin{fulllineitems}
\phantomsection\label{\detokenize{Reference:salabim.Monitor.print_histogram}}\pysiglinewithargsret{\sphinxbfcode{print\_histogram}}{\emph{number\_of\_bins=None}, \emph{lowerbound=None}, \emph{bin\_width=None}, \emph{values=False}, \emph{ex0=False}, \emph{as\_str=False}, \emph{file=None}}{}
print monitor statistics and histogram
\begin{quote}\begin{description}
\item[{Parameters}] \leavevmode\begin{itemize}
\item {} 
\sphinxstyleliteralstrong{number\_of\_bins} (\sphinxstyleliteralemphasis{int}) \textendash{} number of bins 
default: 30 
if \textless{}0, also the header of the histogram will be surpressed

\item {} 
\sphinxstyleliteralstrong{lowerbound} (\sphinxstyleliteralemphasis{float}) \textendash{} first bin 
default: 0

\item {} 
\sphinxstyleliteralstrong{bin\_width} (\sphinxstyleliteralemphasis{float}) \textendash{} width of the bins 
default: 1

\item {} 
\sphinxstyleliteralstrong{values} (\sphinxstyleliteralemphasis{bool}) \textendash{} if False (default), bins will be used 
if True, the individual values will be shown (in the right order).
in that case, no cumulative values will be given 

\item {} 
\sphinxstyleliteralstrong{ex0} (\sphinxstyleliteralemphasis{bool}) \textendash{} if False (default), include zeroes. if True, exclude zeroes

\end{itemize}

\end{description}\end{quote}
\begin{description}
\item[{as\_str: bool}] \leavevmode
if False (default), print the histogram
if True, return a string containing the histogram

\item[{file: file}] \leavevmode
if None(default), all output is directed to stdout 
otherwise, the output is directed to the file

\end{description}
\begin{quote}\begin{description}
\item[{Returns}] \leavevmode
\sphinxstylestrong{histogram (if as\_str is True)}

\item[{Return type}] \leavevmode
str

\end{description}\end{quote}

\begin{sphinxadmonition}{note}{Note:}
If number\_of\_bins, lowerbound and bin\_width are omitted, the histogram will be autoscaled,
with a maximum of 30 classes.
\end{sphinxadmonition}

\end{fulllineitems}

\index{print\_histograms() (salabim.Monitor method)}

\begin{fulllineitems}
\phantomsection\label{\detokenize{Reference:salabim.Monitor.print_histograms}}\pysiglinewithargsret{\sphinxbfcode{print\_histograms}}{\emph{number\_of\_bins=None}, \emph{lowerbound=None}, \emph{bin\_width=None}, \emph{values=False}, \emph{ex0=False}, \emph{as\_str=False}, \emph{file=None}}{}
print monitor statistics and histogram
\begin{quote}\begin{description}
\item[{Parameters}] \leavevmode\begin{itemize}
\item {} 
\sphinxstyleliteralstrong{number\_of\_bins} (\sphinxstyleliteralemphasis{int}) \textendash{} number of bins 
default: 30 
if \textless{}0, also the header of the histogram will be surpressed

\item {} 
\sphinxstyleliteralstrong{lowerbound} (\sphinxstyleliteralemphasis{float}) \textendash{} first bin 
default: 0

\item {} 
\sphinxstyleliteralstrong{bin\_width} (\sphinxstyleliteralemphasis{float}) \textendash{} width of the bins 
default: 1

\item {} 
\sphinxstyleliteralstrong{values} (\sphinxstyleliteralemphasis{bool}) \textendash{} if False (default), bins will be used 
if True, the individual values will be shown (sorted on the value).
in that case, no cumulative values will be given 

\item {} 
\sphinxstyleliteralstrong{ex0} (\sphinxstyleliteralemphasis{bool}) \textendash{} if False (default), include zeroes. if True, exclude zeroes

\item {} 
\sphinxstyleliteralstrong{as\_str} (\sphinxstyleliteralemphasis{bool}) \textendash{} if False (default), print the histogram
if True, return a string containing the histogram

\item {} 
\sphinxstyleliteralstrong{file} (\sphinxstyleliteralemphasis{file}) \textendash{} if None(default), all output is directed to stdout 
otherwise, the output is directed to the file

\end{itemize}

\item[{Returns}] \leavevmode
\sphinxstylestrong{histogram (if as\_str is True)}

\item[{Return type}] \leavevmode
str

\end{description}\end{quote}

\begin{sphinxadmonition}{note}{Note:}
If number\_of\_bins, lowerbound and bin\_width are omitted, the histogram will be autoscaled,
with a maximum of 30 classes. 
Exactly same functionality as Monitor.print\_histogram()
\end{sphinxadmonition}

\end{fulllineitems}

\index{print\_statistics() (salabim.Monitor method)}

\begin{fulllineitems}
\phantomsection\label{\detokenize{Reference:salabim.Monitor.print_statistics}}\pysiglinewithargsret{\sphinxbfcode{print\_statistics}}{\emph{show\_header=True}, \emph{show\_legend=True}, \emph{do\_indent=False}, \emph{as\_str=False}, \emph{file=None}}{}
print monitor statistics
\begin{quote}\begin{description}
\item[{Parameters}] \leavevmode\begin{itemize}
\item {} 
\sphinxstyleliteralstrong{show\_header} (\sphinxstyleliteralemphasis{bool}) \textendash{} primarily for internal use

\item {} 
\sphinxstyleliteralstrong{show\_legend} (\sphinxstyleliteralemphasis{bool}) \textendash{} primarily for internal use

\item {} 
\sphinxstyleliteralstrong{do\_indent} (\sphinxstyleliteralemphasis{bool}) \textendash{} primarily for internal use

\item {} 
\sphinxstyleliteralstrong{as\_str} (\sphinxstyleliteralemphasis{bool}) \textendash{} if False (default), print the statistics
if True, return a string containing the statistics

\item {} 
\sphinxstyleliteralstrong{file} (\sphinxstyleliteralemphasis{file}) \textendash{} if Noneb(default), all output is directed to stdout 
otherwise, the output is directed to the file

\end{itemize}

\item[{Returns}] \leavevmode
\sphinxstylestrong{statistics (if as\_str is True)}

\item[{Return type}] \leavevmode
str

\end{description}\end{quote}

\end{fulllineitems}

\index{register() (salabim.Monitor method)}

\begin{fulllineitems}
\phantomsection\label{\detokenize{Reference:salabim.Monitor.register}}\pysiglinewithargsret{\sphinxbfcode{register}}{\emph{registry}}{}
registers the monitor in the registry
\begin{quote}\begin{description}
\item[{Parameters}] \leavevmode
\sphinxstyleliteralstrong{registry} (\sphinxstyleliteralemphasis{list}) \textendash{} list of (to be) registered objects

\item[{Returns}] \leavevmode
\sphinxstylestrong{monitor (self)}

\item[{Return type}] \leavevmode
{\hyperref[\detokenize{Reference:salabim.Monitor}]{\sphinxcrossref{Monitor}}}

\end{description}\end{quote}

\begin{sphinxadmonition}{note}{Note:}
Use Monitor.deregister if monitor does not longer need to be registered.
\end{sphinxadmonition}

\end{fulllineitems}

\index{reset() (salabim.Monitor method)}

\begin{fulllineitems}
\phantomsection\label{\detokenize{Reference:salabim.Monitor.reset}}\pysiglinewithargsret{\sphinxbfcode{reset}}{\emph{monitor=None}}{}
resets monitor
\begin{quote}\begin{description}
\item[{Parameters}] \leavevmode
\sphinxstyleliteralstrong{monitor} (\sphinxstyleliteralemphasis{bool}) \textendash{} if True, monitoring will be on. 
if False, monitoring is disabled
if omitted, no change of monitoring state

\end{description}\end{quote}

\end{fulllineitems}

\index{reset\_monitors() (salabim.Monitor method)}

\begin{fulllineitems}
\phantomsection\label{\detokenize{Reference:salabim.Monitor.reset_monitors}}\pysiglinewithargsret{\sphinxbfcode{reset\_monitors}}{\emph{monitor=None}}{}
resets monitor
\begin{quote}\begin{description}
\item[{Parameters}] \leavevmode
\sphinxstyleliteralstrong{monitor} (\sphinxstyleliteralemphasis{bool}) \textendash{} if True (default), monitoring will be on. 
if False, monitoring is disabled 
if omitted, the monitor state remains unchanged

\end{description}\end{quote}

\begin{sphinxadmonition}{note}{Note:}
Exactly same functionality as Monitor.reset()
\end{sphinxadmonition}

\end{fulllineitems}

\index{sequence\_number() (salabim.Monitor method)}

\begin{fulllineitems}
\phantomsection\label{\detokenize{Reference:salabim.Monitor.sequence_number}}\pysiglinewithargsret{\sphinxbfcode{sequence\_number}}{}{}~\begin{quote}\begin{description}
\item[{Returns}] \leavevmode
\sphinxstylestrong{sequence\_number of the monitor} \textendash{} (the sequence number at initialization) 
normally this will be the integer value of a serialized name,
but also non serialized names (without a dot or a comma at the end)
will be numbered)

\item[{Return type}] \leavevmode
int

\end{description}\end{quote}

\end{fulllineitems}

\index{setup() (salabim.Monitor method)}

\begin{fulllineitems}
\phantomsection\label{\detokenize{Reference:salabim.Monitor.setup}}\pysiglinewithargsret{\sphinxbfcode{setup}}{}{}
called immediately after initialization of a monitor.

by default this is a dummy method, but it can be overridden.

only keyword arguments are passed

\end{fulllineitems}

\index{slice() (salabim.Monitor method)}

\begin{fulllineitems}
\phantomsection\label{\detokenize{Reference:salabim.Monitor.slice}}\pysiglinewithargsret{\sphinxbfcode{slice}}{\emph{start=None}, \emph{stop=None}, \emph{modulo=None}, \emph{name=None}}{}
slices this monitor (creates a subset)
\begin{quote}\begin{description}
\item[{Parameters}] \leavevmode\begin{itemize}
\item {} 
\sphinxstyleliteralstrong{start} (\sphinxstyleliteralemphasis{float}) \textendash{} if modulo is not given, the start of the slice 
if modulo is given, this is indicates the slice period start (modulo modulo)

\item {} 
\sphinxstyleliteralstrong{stop} (\sphinxstyleliteralemphasis{float}) \textendash{} if modulo is not given, the end of the slice 
if modulo is given, this is indicates the slice period end (modulo modulo) 
note that stop is excluded from the slice (open at right hand side)

\item {} 
\sphinxstyleliteralstrong{modulo} (\sphinxstyleliteralemphasis{float}) \textendash{} specifies the distance between slice periods 
if not specified, just one slice subset is used.

\item {} 
\sphinxstyleliteralstrong{name} (\sphinxstyleliteralemphasis{str}) \textendash{} name of the sliced monitor 
default: name of this monitor + ‘.sliced’

\end{itemize}

\item[{Returns}] \leavevmode
\sphinxstylestrong{sliced monitor}

\item[{Return type}] \leavevmode
{\hyperref[\detokenize{Reference:salabim.Monitor}]{\sphinxcrossref{Monitor}}}

\end{description}\end{quote}

\end{fulllineitems}

\index{std() (salabim.Monitor method)}

\begin{fulllineitems}
\phantomsection\label{\detokenize{Reference:salabim.Monitor.std}}\pysiglinewithargsret{\sphinxbfcode{std}}{\emph{ex0=False}}{}
standard deviation of tallied values
\begin{quote}\begin{description}
\item[{Parameters}] \leavevmode
\sphinxstyleliteralstrong{ex0} (\sphinxstyleliteralemphasis{bool}) \textendash{} if False (default), include zeroes. if True, exclude zeroes

\item[{Returns}] \leavevmode
\sphinxstylestrong{standard deviation}

\item[{Return type}] \leavevmode
float

\end{description}\end{quote}

\begin{sphinxadmonition}{note}{Note:}
For weights are applied, the weighted standard deviation is returned
\end{sphinxadmonition}

\end{fulllineitems}

\index{tally() (salabim.Monitor method)}

\begin{fulllineitems}
\phantomsection\label{\detokenize{Reference:salabim.Monitor.tally}}\pysiglinewithargsret{\sphinxbfcode{tally}}{\emph{value}, \emph{weight=1}}{}~\begin{quote}\begin{description}
\item[{Parameters}] \leavevmode\begin{itemize}
\item {} 
\sphinxstyleliteralstrong{x} (\sphinxstyleliteralemphasis{any}\sphinxstyleliteralemphasis{, }\sphinxstyleliteralemphasis{preferably int}\sphinxstyleliteralemphasis{, }\sphinxstyleliteralemphasis{float}\sphinxstyleliteralemphasis{ or }\sphinxstyleliteralemphasis{translatable into int}\sphinxstyleliteralemphasis{ or }\sphinxstyleliteralemphasis{float}) \textendash{} value to be tallied

\item {} 
\sphinxstyleliteralstrong{weight} (\sphinxstyleliteralemphasis{float}) \textendash{} weight to be tallied 
default : 1 

\end{itemize}

\end{description}\end{quote}

\end{fulllineitems}

\index{tx() (salabim.Monitor method)}

\begin{fulllineitems}
\phantomsection\label{\detokenize{Reference:salabim.Monitor.tx}}\pysiglinewithargsret{\sphinxbfcode{tx}}{\emph{ex0=False}, \emph{exoff=False}, \emph{force\_numeric=False}, \emph{add\_now=True}}{}
tuple of array with timestamps and array/list with x-values
\begin{quote}\begin{description}
\item[{Parameters}] \leavevmode\begin{itemize}
\item {} 
\sphinxstyleliteralstrong{ex0} (\sphinxstyleliteralemphasis{bool}) \textendash{} if False (default), include zeroes. if True, exclude zeroes

\item {} 
\sphinxstyleliteralstrong{exoff} (\sphinxstyleliteralemphasis{bool}) \textendash{} if False (default), include self.off. if True, exclude self.off’s 
non level monitors will return all values, regardless of exoff

\item {} 
\sphinxstyleliteralstrong{force\_numeric} (\sphinxstyleliteralemphasis{bool}) \textendash{} if True (default), convert non numeric tallied values numeric if possible, otherwise assume 0 
if False, do not interpret x-values, return as list if type is list

\item {} 
\sphinxstyleliteralstrong{add\_now} (\sphinxstyleliteralemphasis{bool}) \textendash{} if True (default), the last tallied x-value and the current time is added to the result 
if False, the result ends with the last tallied value and the time that was tallied 
non level monitors will never add now

\end{itemize}

\item[{Returns}] \leavevmode
\sphinxstylestrong{array with timestamps and array/list with x-values}

\item[{Return type}] \leavevmode
tuple

\end{description}\end{quote}

\begin{sphinxadmonition}{note}{Note:}
The value self.off is stored when monitoring is turned off 
The timestamps are not corrected for any reset\_now() adjustment.
\end{sphinxadmonition}

\end{fulllineitems}

\index{value\_duration() (salabim.Monitor method)}

\begin{fulllineitems}
\phantomsection\label{\detokenize{Reference:salabim.Monitor.value_duration}}\pysiglinewithargsret{\sphinxbfcode{value\_duration}}{\emph{value}}{}
total duration of tallied values equal to value or in value
\begin{quote}\begin{description}
\item[{Parameters}] \leavevmode
\sphinxstyleliteralstrong{value} (\sphinxstyleliteralemphasis{any}) \textendash{} if list, tuple or set, check whether the tallied value is in value 
otherwise, check whether the tallied value equals the given value

\item[{Returns}] \leavevmode
\sphinxstylestrong{total of duration of tallied values in value or equal to value}

\item[{Return type}] \leavevmode
int

\end{description}\end{quote}

\begin{sphinxadmonition}{note}{Note:}
Not available for non level monitors
\end{sphinxadmonition}

\end{fulllineitems}

\index{value\_number\_of\_entries() (salabim.Monitor method)}

\begin{fulllineitems}
\phantomsection\label{\detokenize{Reference:salabim.Monitor.value_number_of_entries}}\pysiglinewithargsret{\sphinxbfcode{value\_number\_of\_entries}}{\emph{value}}{}
count of the number of tallied values equal to value or in value
\begin{quote}\begin{description}
\item[{Parameters}] \leavevmode
\sphinxstyleliteralstrong{value} (\sphinxstyleliteralemphasis{any}) \textendash{} if list, tuple or set, check whether the tallied value is in value 
otherwise, check whether the tallied value equals the given value

\item[{Returns}] \leavevmode
\sphinxstylestrong{number of tallied values in value or equal to value}

\item[{Return type}] \leavevmode
int

\end{description}\end{quote}

\begin{sphinxadmonition}{note}{Note:}
Not available for level monitors
\end{sphinxadmonition}

\end{fulllineitems}

\index{value\_weight() (salabim.Monitor method)}

\begin{fulllineitems}
\phantomsection\label{\detokenize{Reference:salabim.Monitor.value_weight}}\pysiglinewithargsret{\sphinxbfcode{value\_weight}}{\emph{value}}{}
total weight of tallied values equal to value or in value
\begin{quote}\begin{description}
\item[{Parameters}] \leavevmode
\sphinxstyleliteralstrong{value} (\sphinxstyleliteralemphasis{any}) \textendash{} if list, tuple or set, check whether the tallied value is in value 
otherwise, check whether the tallied value equals the given value

\item[{Returns}] \leavevmode
\sphinxstylestrong{total of weights of tallied values in value or equal to value}

\item[{Return type}] \leavevmode
int

\end{description}\end{quote}

\begin{sphinxadmonition}{note}{Note:}
Not available for level monitors
\end{sphinxadmonition}

\end{fulllineitems}

\index{weight() (salabim.Monitor method)}

\begin{fulllineitems}
\phantomsection\label{\detokenize{Reference:salabim.Monitor.weight}}\pysiglinewithargsret{\sphinxbfcode{weight}}{\emph{ex0=False}}{}
sum of weights
\begin{quote}\begin{description}
\item[{Parameters}] \leavevmode
\sphinxstyleliteralstrong{ex0} (\sphinxstyleliteralemphasis{bool}) \textendash{} if False (default), include zeroes. if True, exclude zeroes

\item[{Returns}] \leavevmode
\sphinxstylestrong{sum of weights}

\item[{Return type}] \leavevmode
float

\end{description}\end{quote}

\begin{sphinxadmonition}{note}{Note:}
Not available for level monitors
\end{sphinxadmonition}

\end{fulllineitems}

\index{weight\_zero() (salabim.Monitor method)}

\begin{fulllineitems}
\phantomsection\label{\detokenize{Reference:salabim.Monitor.weight_zero}}\pysiglinewithargsret{\sphinxbfcode{weight\_zero}}{}{}
sum of weights of zero entries
\begin{quote}\begin{description}
\item[{Returns}] \leavevmode
\sphinxstylestrong{sum of weights of zero entries}

\item[{Return type}] \leavevmode
float

\end{description}\end{quote}

\begin{sphinxadmonition}{note}{Note:}
Not available for level monitors
\end{sphinxadmonition}

\end{fulllineitems}

\index{x() (salabim.Monitor method)}

\begin{fulllineitems}
\phantomsection\label{\detokenize{Reference:salabim.Monitor.x}}\pysiglinewithargsret{\sphinxbfcode{x}}{\emph{ex0=False}, \emph{force\_numeric=True}}{}
array/list of tallied values
\begin{quote}\begin{description}
\item[{Parameters}] \leavevmode\begin{itemize}
\item {} 
\sphinxstyleliteralstrong{ex0} (\sphinxstyleliteralemphasis{bool}) \textendash{} if False (default), include zeroes. if True, exclude zeroes

\item {} 
\sphinxstyleliteralstrong{force\_numeric} (\sphinxstyleliteralemphasis{bool}) \textendash{} if True (default), convert non numeric tallied values numeric if possible, otherwise assume 0 
if False, do not interpret x-values, return as list if type is any (list)

\end{itemize}

\item[{Returns}] \leavevmode
\sphinxstylestrong{all tallied values}

\item[{Return type}] \leavevmode
array/list

\end{description}\end{quote}

\begin{sphinxadmonition}{note}{Note:}
Not available for level monitors. Use xduration(), xt() or tx() instead.
\end{sphinxadmonition}

\end{fulllineitems}

\index{xduration() (salabim.Monitor method)}

\begin{fulllineitems}
\phantomsection\label{\detokenize{Reference:salabim.Monitor.xduration}}\pysiglinewithargsret{\sphinxbfcode{xduration}}{\emph{ex0=False}, \emph{force\_numeric=True}}{}
array/list of tallied values
\begin{quote}\begin{description}
\item[{Parameters}] \leavevmode\begin{itemize}
\item {} 
\sphinxstyleliteralstrong{ex0} (\sphinxstyleliteralemphasis{bool}) \textendash{} if False (default), include zeroes. if True, exclude zeroes

\item {} 
\sphinxstyleliteralstrong{force\_numeric} (\sphinxstyleliteralemphasis{bool}) \textendash{} if True (default), convert non numeric tallied values numeric if possible, otherwise assume 0 
if False, do not interpret x-values, return as list if type is list

\end{itemize}

\item[{Returns}] \leavevmode
\sphinxstylestrong{all tallied values}

\item[{Return type}] \leavevmode
array/list

\end{description}\end{quote}

\begin{sphinxadmonition}{note}{Note:}
not available for non level monitors
\end{sphinxadmonition}

\end{fulllineitems}

\index{xt() (salabim.Monitor method)}

\begin{fulllineitems}
\phantomsection\label{\detokenize{Reference:salabim.Monitor.xt}}\pysiglinewithargsret{\sphinxbfcode{xt}}{\emph{ex0=False}, \emph{exoff=False}, \emph{force\_numeric=True}, \emph{add\_now=True}}{}
tuple of array/list with x-values and array with timestamp
\begin{quote}\begin{description}
\item[{Parameters}] \leavevmode\begin{itemize}
\item {} 
\sphinxstyleliteralstrong{ex0} (\sphinxstyleliteralemphasis{bool}) \textendash{} if False (default), include zeroes. if True, exclude zeroes

\item {} 
\sphinxstyleliteralstrong{exoff} (\sphinxstyleliteralemphasis{bool}) \textendash{} if False (default), include self.off. if True, exclude self.off’s 
non level monitors will return all values, regardless of exoff

\item {} 
\sphinxstyleliteralstrong{force\_numeric} (\sphinxstyleliteralemphasis{bool}) \textendash{} if True (default), convert non numeric tallied values numeric if possible, otherwise assume 0 
if False, do not interpret x-values, return as list if type is list

\item {} 
\sphinxstyleliteralstrong{add\_now} (\sphinxstyleliteralemphasis{bool}) \textendash{} if True (default), the last tallied x-value and the current time is added to the result 
if False, the result ends with the last tallied value and the time that was tallied 
non level monitors will never add now

\end{itemize}

\item[{Returns}] \leavevmode
\sphinxstylestrong{array/list with x-values and array with timestamps}

\item[{Return type}] \leavevmode
tuple

\end{description}\end{quote}

\begin{sphinxadmonition}{note}{Note:}
The value self.off is stored when monitoring is turned off 
The timestamps are not corrected for any reset\_now() adjustment.
\end{sphinxadmonition}

\end{fulllineitems}

\index{xweight() (salabim.Monitor method)}

\begin{fulllineitems}
\phantomsection\label{\detokenize{Reference:salabim.Monitor.xweight}}\pysiglinewithargsret{\sphinxbfcode{xweight}}{\emph{ex0=False}, \emph{force\_numeric=True}}{}
array/list of tallied values
\begin{quote}\begin{description}
\item[{Parameters}] \leavevmode\begin{itemize}
\item {} 
\sphinxstyleliteralstrong{ex0} (\sphinxstyleliteralemphasis{bool}) \textendash{} if False (default), include zeroes. if True, exclude zeroes

\item {} 
\sphinxstyleliteralstrong{force\_numeric} (\sphinxstyleliteralemphasis{bool}) \textendash{} if True (default), convert non numeric tallied values numeric if possible, otherwise assume 0 
if False, do not interpret x-values, return as list if type is list

\end{itemize}

\item[{Returns}] \leavevmode
\sphinxstylestrong{all tallied values}

\item[{Return type}] \leavevmode
array/list

\end{description}\end{quote}

\begin{sphinxadmonition}{note}{Note:}
not available for level monitors
\end{sphinxadmonition}

\end{fulllineitems}


\end{fulllineitems}



\section{Queue}
\label{\detokenize{Reference:queue}}\index{Queue (class in salabim)}

\begin{fulllineitems}
\phantomsection\label{\detokenize{Reference:salabim.Queue}}\pysiglinewithargsret{\sphinxbfcode{class }\sphinxcode{salabim.}\sphinxbfcode{Queue}}{\emph{name=None}, \emph{monitor=True}, \emph{fill=None}, \emph{env=None}, \emph{*args}, \emph{**kwargs}}{}
Queue object
\begin{quote}\begin{description}
\item[{Parameters}] \leavevmode\begin{itemize}
\item {} 
\sphinxstyleliteralstrong{fill} ({\hyperref[\detokenize{Reference:salabim.Queue}]{\sphinxcrossref{\sphinxstyleliteralemphasis{Queue}}}}\sphinxstyleliteralemphasis{, }\sphinxstyleliteralemphasis{list}\sphinxstyleliteralemphasis{ or }\sphinxstyleliteralemphasis{tuple}) \textendash{} fill the queue with the components in fill 
if omitted, the queue will be empty at initialization

\item {} 
\sphinxstyleliteralstrong{name} (\sphinxstyleliteralemphasis{str}) \textendash{} name of the queue 
if the name ends with a period (.),
auto serializing will be applied 
if the name end with a comma,
auto serializing starting at 1 will be applied 
if omitted, the name will be derived from the class
it is defined in (lowercased)

\item {} 
\sphinxstyleliteralstrong{monitor} (\sphinxstyleliteralemphasis{bool}) \textendash{} if True (default) , both length and length\_of\_stay are monitored 
if False, monitoring is disabled.

\item {} 
\sphinxstyleliteralstrong{env} ({\hyperref[\detokenize{Reference:salabim.Environment}]{\sphinxcrossref{\sphinxstyleliteralemphasis{Environment}}}}) \textendash{} environment where the queue is defined 
if omitted, default\_env will be used

\end{itemize}

\end{description}\end{quote}
\index{add() (salabim.Queue method)}

\begin{fulllineitems}
\phantomsection\label{\detokenize{Reference:salabim.Queue.add}}\pysiglinewithargsret{\sphinxbfcode{add}}{\emph{component}}{}
adds a component to the tail of a queue
\begin{quote}\begin{description}
\item[{Parameters}] \leavevmode
\sphinxstyleliteralstrong{component} ({\hyperref[\detokenize{Reference:salabim.Component}]{\sphinxcrossref{\sphinxstyleliteralemphasis{Component}}}}) \textendash{} component to be added to the tail of the queue 
may not be member of the queue yet

\end{description}\end{quote}

\begin{sphinxadmonition}{note}{Note:}
the priority will be set to
the priority of the tail of the queue, if any
or 0 if queue is empty 
This method is equivalent to append()
\end{sphinxadmonition}

\end{fulllineitems}

\index{add\_at\_head() (salabim.Queue method)}

\begin{fulllineitems}
\phantomsection\label{\detokenize{Reference:salabim.Queue.add_at_head}}\pysiglinewithargsret{\sphinxbfcode{add\_at\_head}}{\emph{component}}{}
adds a component to the head of a queue
\begin{quote}\begin{description}
\item[{Parameters}] \leavevmode
\sphinxstyleliteralstrong{component} ({\hyperref[\detokenize{Reference:salabim.Component}]{\sphinxcrossref{\sphinxstyleliteralemphasis{Component}}}}) \textendash{} component to be added to the head of the queue 
may not be member of the queue yet

\end{description}\end{quote}

\begin{sphinxadmonition}{note}{Note:}
the priority will be set to
the priority of the head of the queue, if any
or 0 if queue is empty
\end{sphinxadmonition}

\end{fulllineitems}

\index{add\_behind() (salabim.Queue method)}

\begin{fulllineitems}
\phantomsection\label{\detokenize{Reference:salabim.Queue.add_behind}}\pysiglinewithargsret{\sphinxbfcode{add\_behind}}{\emph{component}, \emph{poscomponent}}{}
adds a component to a queue, just behind a component
\begin{quote}\begin{description}
\item[{Parameters}] \leavevmode\begin{itemize}
\item {} 
\sphinxstyleliteralstrong{component} ({\hyperref[\detokenize{Reference:salabim.Component}]{\sphinxcrossref{\sphinxstyleliteralemphasis{Component}}}}) \textendash{} component to be added to the queue 
may not be member of the queue yet

\item {} 
\sphinxstyleliteralstrong{poscomponent} ({\hyperref[\detokenize{Reference:salabim.Component}]{\sphinxcrossref{\sphinxstyleliteralemphasis{Component}}}}) \textendash{} component behind which component will be inserted 
must be member of the queue

\end{itemize}

\end{description}\end{quote}

\begin{sphinxadmonition}{note}{Note:}
the priority of component will be set to the priority of poscomponent
\end{sphinxadmonition}

\end{fulllineitems}

\index{add\_in\_front\_of() (salabim.Queue method)}

\begin{fulllineitems}
\phantomsection\label{\detokenize{Reference:salabim.Queue.add_in_front_of}}\pysiglinewithargsret{\sphinxbfcode{add\_in\_front\_of}}{\emph{component}, \emph{poscomponent}}{}
adds a component to a queue, just in front of a component
\begin{quote}\begin{description}
\item[{Parameters}] \leavevmode\begin{itemize}
\item {} 
\sphinxstyleliteralstrong{component} ({\hyperref[\detokenize{Reference:salabim.Component}]{\sphinxcrossref{\sphinxstyleliteralemphasis{Component}}}}) \textendash{} component to be added to the queue 
may not be member of the queue yet

\item {} 
\sphinxstyleliteralstrong{poscomponent} ({\hyperref[\detokenize{Reference:salabim.Component}]{\sphinxcrossref{\sphinxstyleliteralemphasis{Component}}}}) \textendash{} component in front of which component will be inserted 
must be member of the queue

\end{itemize}

\end{description}\end{quote}

\begin{sphinxadmonition}{note}{Note:}
the priority of component will be set to the priority of poscomponent
\end{sphinxadmonition}

\end{fulllineitems}

\index{add\_sorted() (salabim.Queue method)}

\begin{fulllineitems}
\phantomsection\label{\detokenize{Reference:salabim.Queue.add_sorted}}\pysiglinewithargsret{\sphinxbfcode{add\_sorted}}{\emph{component}, \emph{priority}}{}
adds a component to a queue, according to the priority
\begin{quote}\begin{description}
\item[{Parameters}] \leavevmode\begin{itemize}
\item {} 
\sphinxstyleliteralstrong{component} ({\hyperref[\detokenize{Reference:salabim.Component}]{\sphinxcrossref{\sphinxstyleliteralemphasis{Component}}}}) \textendash{} component to be added to the queue 
may not be member of the queue yet

\item {} 
\sphinxstyleliteralstrong{priority} (\sphinxstyleliteralemphasis{type that can be compared with other priorities in the queue}) \textendash{} priority in the queue

\end{itemize}

\end{description}\end{quote}

\begin{sphinxadmonition}{note}{Note:}
The component is placed just before the first component with a priority \textgreater{} given priority
\end{sphinxadmonition}

\end{fulllineitems}

\index{animate() (salabim.Queue method)}

\begin{fulllineitems}
\phantomsection\label{\detokenize{Reference:salabim.Queue.animate}}\pysiglinewithargsret{\sphinxbfcode{animate}}{\emph{*args}, \emph{**kwargs}}{}
Animates the components in the queue.
\begin{quote}\begin{description}
\item[{Parameters}] \leavevmode\begin{itemize}
\item {} 
\sphinxstyleliteralstrong{x} (\sphinxstyleliteralemphasis{float}) \textendash{} x-position of the first component in the queue 
default: 50

\item {} 
\sphinxstyleliteralstrong{y} (\sphinxstyleliteralemphasis{float}) \textendash{} y-position of the first component in the queue 
default: 50

\item {} 
\sphinxstyleliteralstrong{direction} (\sphinxstyleliteralemphasis{str}) \textendash{} if ‘w’, waiting line runs westwards (i.e. from right to left) 
if ‘n’, waiting line runs northeards (i.e. from bottom to top) 
if ‘e’, waiting line runs eastwards (i.e. from left to right) (default) 
if ‘s’, waiting line runs southwards (i.e. from top to bottom)

\item {} 
\sphinxstyleliteralstrong{reverse} (\sphinxstyleliteralemphasis{bool}) \textendash{} if False (default), display in normal order. If True, reversed.

\item {} 
\sphinxstyleliteralstrong{max\_length} (\sphinxstyleliteralemphasis{int}) \textendash{} maximum number of components to be displayed

\item {} 
\sphinxstyleliteralstrong{xy\_anchor} (\sphinxstyleliteralemphasis{str}) \textendash{} specifies where x and y are relative to 
possible values are (default: sw): 
\sphinxcode{nw    n    ne} 
\sphinxcode{w     c     e} 
\sphinxcode{sw    s    se}

\item {} 
\sphinxstyleliteralstrong{id} (\sphinxstyleliteralemphasis{any}) \textendash{} the animation works by calling the animation\_objects method of each component, optionally
with id. By default, this is self, but can be overriden, particularly with the queue

\item {} 
\sphinxstyleliteralstrong{arg} (\sphinxstyleliteralemphasis{any}) \textendash{} this is used when a parameter is a function with two parameters, as the first argument or
if a parameter is a method as the instance 
default: self (instance itself)

\end{itemize}

\item[{Returns}] \leavevmode
\sphinxstylestrong{reference to AnimationQueue object}

\item[{Return type}] \leavevmode
AnimationQueue

\end{description}\end{quote}

\begin{sphinxadmonition}{note}{Note:}
It is recommended to use sim.AnimateQueue instead 

All measures are in screen coordinates 

All parameters, apart from queue and arg can be specified as: 
- a scalar, like 10 
- a function with zero arguments, like lambda: title 
- a function with one argument, being the time t, like lambda t: t + 10 
- a function with two parameters, being arg (as given) and the time, like lambda comp, t: comp.state 
- a method instance arg for time t, like self.state, actually leading to arg.state(t) to be called
\end{sphinxadmonition}

\end{fulllineitems}

\index{append() (salabim.Queue method)}

\begin{fulllineitems}
\phantomsection\label{\detokenize{Reference:salabim.Queue.append}}\pysiglinewithargsret{\sphinxbfcode{append}}{\emph{component}}{}
appends a component to the tail of a queue
\begin{quote}\begin{description}
\item[{Parameters}] \leavevmode
\sphinxstyleliteralstrong{component} ({\hyperref[\detokenize{Reference:salabim.Component}]{\sphinxcrossref{\sphinxstyleliteralemphasis{Component}}}}) \textendash{} component to be appened to the tail of the queue 
may not be member of the queue yet

\end{description}\end{quote}

\begin{sphinxadmonition}{note}{Note:}
the priority will be set to
the priority of the tail of the queue, if any
or 0 if queue is empty 
This method is equivalent to add()
\end{sphinxadmonition}

\end{fulllineitems}

\index{base\_name() (salabim.Queue method)}

\begin{fulllineitems}
\phantomsection\label{\detokenize{Reference:salabim.Queue.base_name}}\pysiglinewithargsret{\sphinxbfcode{base\_name}}{}{}~\begin{quote}\begin{description}
\item[{Returns}] \leavevmode
\sphinxstylestrong{base name of the queue (the name used at initialization)}

\item[{Return type}] \leavevmode
str

\end{description}\end{quote}

\end{fulllineitems}

\index{clear() (salabim.Queue method)}

\begin{fulllineitems}
\phantomsection\label{\detokenize{Reference:salabim.Queue.clear}}\pysiglinewithargsret{\sphinxbfcode{clear}}{}{}
empties a queue

removes all components from a queue

\end{fulllineitems}

\index{component\_with\_name() (salabim.Queue method)}

\begin{fulllineitems}
\phantomsection\label{\detokenize{Reference:salabim.Queue.component_with_name}}\pysiglinewithargsret{\sphinxbfcode{component\_with\_name}}{\emph{txt}}{}
returns a component in the queue according to its name
\begin{quote}\begin{description}
\item[{Parameters}] \leavevmode
\sphinxstyleliteralstrong{txt} (\sphinxstyleliteralemphasis{str}) \textendash{} name of component to be retrieved

\item[{Returns}] \leavevmode
\sphinxstylestrong{the first component in the queue with name txt} \textendash{} returns None if not found

\item[{Return type}] \leavevmode
Component 

\end{description}\end{quote}

\end{fulllineitems}

\index{copy() (salabim.Queue method)}

\begin{fulllineitems}
\phantomsection\label{\detokenize{Reference:salabim.Queue.copy}}\pysiglinewithargsret{\sphinxbfcode{copy}}{\emph{name=None}, \emph{monitor=\textless{}function Queue.monitor\textgreater{}}}{}
returns a copy of two queues
\begin{quote}\begin{description}
\item[{Parameters}] \leavevmode\begin{itemize}
\item {} 
\sphinxstyleliteralstrong{name} (\sphinxstyleliteralemphasis{str}) \textendash{} name of the new queue 
if omitted, ‘copy of ‘ + self.name()

\item {} 
\sphinxstyleliteralstrong{monitor} (\sphinxstyleliteralemphasis{bool}) \textendash{} if True, monitor the queue 
if False (default), do not monitor the queue

\end{itemize}

\item[{Returns}] \leavevmode
\sphinxstylestrong{queue with all elements of self}

\item[{Return type}] \leavevmode
{\hyperref[\detokenize{Reference:salabim.Queue}]{\sphinxcrossref{Queue}}}

\end{description}\end{quote}

\begin{sphinxadmonition}{note}{Note:}
The priority will be copied from original queue.
Also, the order will be maintained.
\end{sphinxadmonition}

\end{fulllineitems}

\index{count() (salabim.Queue method)}

\begin{fulllineitems}
\phantomsection\label{\detokenize{Reference:salabim.Queue.count}}\pysiglinewithargsret{\sphinxbfcode{count}}{\emph{component}}{}
component count
\begin{quote}\begin{description}
\item[{Parameters}] \leavevmode
\sphinxstyleliteralstrong{component} ({\hyperref[\detokenize{Reference:salabim.Component}]{\sphinxcrossref{\sphinxstyleliteralemphasis{Component}}}}) \textendash{} component to count

\item[{Returns}] \leavevmode


\item[{Return type}] \leavevmode
number of occurences of component in the queue

\end{description}\end{quote}

\begin{sphinxadmonition}{note}{Note:}
The result can only be 0 or 1
\end{sphinxadmonition}

\end{fulllineitems}

\index{deregister() (salabim.Queue method)}

\begin{fulllineitems}
\phantomsection\label{\detokenize{Reference:salabim.Queue.deregister}}\pysiglinewithargsret{\sphinxbfcode{deregister}}{\emph{registry}}{}
deregisters the queue in the registry
\begin{quote}\begin{description}
\item[{Parameters}] \leavevmode
\sphinxstyleliteralstrong{registry} (\sphinxstyleliteralemphasis{list}) \textendash{} list of registered queues

\item[{Returns}] \leavevmode
\sphinxstylestrong{queue (self)}

\item[{Return type}] \leavevmode
{\hyperref[\detokenize{Reference:salabim.Queue}]{\sphinxcrossref{Queue}}}

\end{description}\end{quote}

\end{fulllineitems}

\index{difference() (salabim.Queue method)}

\begin{fulllineitems}
\phantomsection\label{\detokenize{Reference:salabim.Queue.difference}}\pysiglinewithargsret{\sphinxbfcode{difference}}{\emph{q}, \emph{name=None}, \emph{monitor=\textless{}function Queue.monitor\textgreater{}}}{}
returns the difference of two queues
\begin{quote}\begin{description}
\item[{Parameters}] \leavevmode\begin{itemize}
\item {} 
\sphinxstyleliteralstrong{q} ({\hyperref[\detokenize{Reference:salabim.Queue}]{\sphinxcrossref{\sphinxstyleliteralemphasis{Queue}}}}) \textendash{} queue to be ‘subtracted’ from self

\item {} 
\sphinxstyleliteralstrong{name} (\sphinxstyleliteralemphasis{str}) \textendash{} name of the  new queue 
if omitted, self.name() - q.name()

\item {} 
\sphinxstyleliteralstrong{monitor} (\sphinxstyleliteralemphasis{bool}) \textendash{} if True, monitor the queue 
if False (default), do not monitor the queue

\end{itemize}

\item[{Returns}] \leavevmode


\item[{Return type}] \leavevmode
queue containing all elements of self that are not in q

\end{description}\end{quote}

\begin{sphinxadmonition}{note}{Note:}
the priority will be copied from the original queue.
Also, the order will be maintained. 
Alternatively, the more pythonic - operator is also supported, e.g. q1 - q2
\end{sphinxadmonition}

\end{fulllineitems}

\index{extend() (salabim.Queue method)}

\begin{fulllineitems}
\phantomsection\label{\detokenize{Reference:salabim.Queue.extend}}\pysiglinewithargsret{\sphinxbfcode{extend}}{\emph{q}}{}
extends the queue with components of q that are not already in self
\begin{quote}\begin{description}
\item[{Parameters}] \leavevmode
\sphinxstyleliteralstrong{q} (\sphinxstyleliteralemphasis{queue}\sphinxstyleliteralemphasis{, }\sphinxstyleliteralemphasis{list}\sphinxstyleliteralemphasis{ or }\sphinxstyleliteralemphasis{tuple}) \textendash{} 

\end{description}\end{quote}

\begin{sphinxadmonition}{note}{Note:}
The components added to the queue will get the priority of the tail of self.
\end{sphinxadmonition}

\end{fulllineitems}

\index{head() (salabim.Queue method)}

\begin{fulllineitems}
\phantomsection\label{\detokenize{Reference:salabim.Queue.head}}\pysiglinewithargsret{\sphinxbfcode{head}}{}{}~\begin{quote}\begin{description}
\item[{Returns}] \leavevmode
\sphinxstylestrong{the head component of the queue, if any. None otherwise}

\item[{Return type}] \leavevmode
{\hyperref[\detokenize{Reference:salabim.Component}]{\sphinxcrossref{Component}}}

\end{description}\end{quote}

\begin{sphinxadmonition}{note}{Note:}
q{[}0{]} is a more Pythonic way to access the head of the queue
\end{sphinxadmonition}

\end{fulllineitems}

\index{index() (salabim.Queue method)}

\begin{fulllineitems}
\phantomsection\label{\detokenize{Reference:salabim.Queue.index}}\pysiglinewithargsret{\sphinxbfcode{index}}{\emph{component}}{}
get the index of a component in the queue
\begin{quote}\begin{description}
\item[{Parameters}] \leavevmode
\sphinxstyleliteralstrong{component} ({\hyperref[\detokenize{Reference:salabim.Component}]{\sphinxcrossref{\sphinxstyleliteralemphasis{Component}}}}) \textendash{} component to be queried 
does not need to be in the queue

\item[{Returns}] \leavevmode
\sphinxstylestrong{index of component in the queue} \textendash{} 0 denotes the head, 
returns -1 if component is not in the queue

\item[{Return type}] \leavevmode
int

\end{description}\end{quote}

\end{fulllineitems}

\index{insert() (salabim.Queue method)}

\begin{fulllineitems}
\phantomsection\label{\detokenize{Reference:salabim.Queue.insert}}\pysiglinewithargsret{\sphinxbfcode{insert}}{\emph{index}, \emph{component}}{}
Insert component before index-th element of the queue
\begin{quote}\begin{description}
\item[{Parameters}] \leavevmode\begin{itemize}
\item {} 
\sphinxstyleliteralstrong{index} (\sphinxstyleliteralemphasis{int}) \textendash{} component to be added just before index’th element 
should be \textgreater{}=0 and \textless{}=len(self)

\item {} 
\sphinxstyleliteralstrong{component} ({\hyperref[\detokenize{Reference:salabim.Component}]{\sphinxcrossref{\sphinxstyleliteralemphasis{Component}}}}) \textendash{} component to be added to the queue

\end{itemize}

\end{description}\end{quote}

\begin{sphinxadmonition}{note}{Note:}
the priority of component will be set to the priority of the index’th component,
or 0 if the queue is empty
\end{sphinxadmonition}

\end{fulllineitems}

\index{intersection() (salabim.Queue method)}

\begin{fulllineitems}
\phantomsection\label{\detokenize{Reference:salabim.Queue.intersection}}\pysiglinewithargsret{\sphinxbfcode{intersection}}{\emph{q}, \emph{name=None}, \emph{monitor=False}}{}
returns the intersect of two queues
\begin{quote}\begin{description}
\item[{Parameters}] \leavevmode\begin{itemize}
\item {} 
\sphinxstyleliteralstrong{q} ({\hyperref[\detokenize{Reference:salabim.Queue}]{\sphinxcrossref{\sphinxstyleliteralemphasis{Queue}}}}) \textendash{} queue to be intersected with self

\item {} 
\sphinxstyleliteralstrong{name} (\sphinxstyleliteralemphasis{str}) \textendash{} name of the  new queue 
if omitted, self.name() + q.name()

\item {} 
\sphinxstyleliteralstrong{monitor} (\sphinxstyleliteralemphasis{bool}) \textendash{} if True, monitor the queue 
if False (default), do not monitor the queue

\end{itemize}

\item[{Returns}] \leavevmode
\sphinxstylestrong{queue with all elements that are in self and q}

\item[{Return type}] \leavevmode
{\hyperref[\detokenize{Reference:salabim.Queue}]{\sphinxcrossref{Queue}}}

\end{description}\end{quote}

\begin{sphinxadmonition}{note}{Note:}
the priority will be set to 0 for all components in the
resulting  queue 
the order of the resulting queue is as follows: 
in the same order as in self. 
Alternatively, the more pythonic \& operator is also supported, e.g. q1 \& q2
\end{sphinxadmonition}

\end{fulllineitems}

\index{monitor() (salabim.Queue method)}

\begin{fulllineitems}
\phantomsection\label{\detokenize{Reference:salabim.Queue.monitor}}\pysiglinewithargsret{\sphinxbfcode{monitor}}{\emph{value}}{}
enables/disables monitoring of length\_of\_stay and length
\begin{quote}\begin{description}
\item[{Parameters}] \leavevmode
\sphinxstyleliteralstrong{value} (\sphinxstyleliteralemphasis{bool}) \textendash{} if True, monitoring will be on. 
if False, monitoring is disabled 

\end{description}\end{quote}

\begin{sphinxadmonition}{note}{Note:}
it is possible to individually control monitoring with length\_of\_stay.monitor() and length.monitor()
\end{sphinxadmonition}

\end{fulllineitems}

\index{move() (salabim.Queue method)}

\begin{fulllineitems}
\phantomsection\label{\detokenize{Reference:salabim.Queue.move}}\pysiglinewithargsret{\sphinxbfcode{move}}{\emph{name=None}, \emph{monitor=\textless{}function Queue.monitor\textgreater{}}}{}
makes a copy of a queue and empties the original
\begin{quote}\begin{description}
\item[{Parameters}] \leavevmode\begin{itemize}
\item {} 
\sphinxstyleliteralstrong{name} (\sphinxstyleliteralemphasis{str}) \textendash{} name of the new queue

\item {} 
\sphinxstyleliteralstrong{monitor} (\sphinxstyleliteralemphasis{bool}) \textendash{} if True, monitor the queue 
if False (default), do not monitor the yqueue

\end{itemize}

\item[{Returns}] \leavevmode
\sphinxstylestrong{queue containing all elements of self}

\item[{Return type}] \leavevmode
{\hyperref[\detokenize{Reference:salabim.Queue}]{\sphinxcrossref{Queue}}}

\end{description}\end{quote}

\begin{sphinxadmonition}{note}{Note:}
Priorities will be kept 
self will be emptied
\end{sphinxadmonition}

\end{fulllineitems}

\index{name() (salabim.Queue method)}

\begin{fulllineitems}
\phantomsection\label{\detokenize{Reference:salabim.Queue.name}}\pysiglinewithargsret{\sphinxbfcode{name}}{\emph{value=None}}{}~\begin{quote}\begin{description}
\item[{Parameters}] \leavevmode
\sphinxstyleliteralstrong{value} (\sphinxstyleliteralemphasis{str}) \textendash{} new name of the queue
if omitted, no change

\item[{Returns}] \leavevmode
\sphinxstylestrong{Name of the queue}

\item[{Return type}] \leavevmode
str

\end{description}\end{quote}

\begin{sphinxadmonition}{note}{Note:}
base\_name and sequence\_number are not affected if the name is changed 
All derived named are updated as well.
\end{sphinxadmonition}

\end{fulllineitems}

\index{pop() (salabim.Queue method)}

\begin{fulllineitems}
\phantomsection\label{\detokenize{Reference:salabim.Queue.pop}}\pysiglinewithargsret{\sphinxbfcode{pop}}{\emph{index=None}}{}
removes a component by its position (or head)
\begin{quote}\begin{description}
\item[{Parameters}] \leavevmode
\sphinxstyleliteralstrong{index} (\sphinxstyleliteralemphasis{int}) \textendash{} index-th element to remove, if any 
if omitted, return the head of the queue, if any

\item[{Returns}] \leavevmode
\sphinxstylestrong{The i-th component or head} \textendash{} None if not existing

\item[{Return type}] \leavevmode
{\hyperref[\detokenize{Reference:salabim.Component}]{\sphinxcrossref{Component}}}

\end{description}\end{quote}

\end{fulllineitems}

\index{predecessor() (salabim.Queue method)}

\begin{fulllineitems}
\phantomsection\label{\detokenize{Reference:salabim.Queue.predecessor}}\pysiglinewithargsret{\sphinxbfcode{predecessor}}{\emph{component}}{}
predecessor in queue
\begin{quote}\begin{description}
\item[{Parameters}] \leavevmode
\sphinxstyleliteralstrong{component} ({\hyperref[\detokenize{Reference:salabim.Component}]{\sphinxcrossref{\sphinxstyleliteralemphasis{Component}}}}) \textendash{} component whose predecessor to return 
must be member of the queue

\item[{Returns}] \leavevmode
\sphinxstylestrong{predecessor of component, if any} \textendash{} None otherwise.

\item[{Return type}] \leavevmode
Component 

\end{description}\end{quote}

\end{fulllineitems}

\index{print\_histograms() (salabim.Queue method)}

\begin{fulllineitems}
\phantomsection\label{\detokenize{Reference:salabim.Queue.print_histograms}}\pysiglinewithargsret{\sphinxbfcode{print\_histograms}}{\emph{exclude=()}, \emph{as\_str=False}, \emph{file=None}}{}
prints the histograms of the length and length\_of\_stay monitor of the queue
\begin{quote}\begin{description}
\item[{Parameters}] \leavevmode\begin{itemize}
\item {} 
\sphinxstyleliteralstrong{exclude} (\sphinxstyleliteralemphasis{tuple}\sphinxstyleliteralemphasis{ or }\sphinxstyleliteralemphasis{list}) \textendash{} specifies which monitors to exclude 
default: () 

\item {} 
\sphinxstyleliteralstrong{as\_str} (\sphinxstyleliteralemphasis{bool}) \textendash{} if False (default), print the histograms
if True, return a string containing the histograms

\item {} 
\sphinxstyleliteralstrong{file} (\sphinxstyleliteralemphasis{file}) \textendash{} if None(default), all output is directed to stdout 
otherwise, the output is directed to the file

\end{itemize}

\item[{Returns}] \leavevmode
\sphinxstylestrong{histograms (if as\_str is True)}

\item[{Return type}] \leavevmode
str

\end{description}\end{quote}

\end{fulllineitems}

\index{print\_info() (salabim.Queue method)}

\begin{fulllineitems}
\phantomsection\label{\detokenize{Reference:salabim.Queue.print_info}}\pysiglinewithargsret{\sphinxbfcode{print\_info}}{\emph{as\_str=False}, \emph{file=None}}{}
prints information about the queue
\begin{quote}\begin{description}
\item[{Parameters}] \leavevmode\begin{itemize}
\item {} 
\sphinxstyleliteralstrong{as\_str} (\sphinxstyleliteralemphasis{bool}) \textendash{} if False (default), print the info
if True, return a string containing the info

\item {} 
\sphinxstyleliteralstrong{file} (\sphinxstyleliteralemphasis{file}) \textendash{} if None(default), all output is directed to stdout 
otherwise, the output is directed to the file

\end{itemize}

\item[{Returns}] \leavevmode
\sphinxstylestrong{info (if as\_str is True)}

\item[{Return type}] \leavevmode
str

\end{description}\end{quote}

\end{fulllineitems}

\index{print\_statistics() (salabim.Queue method)}

\begin{fulllineitems}
\phantomsection\label{\detokenize{Reference:salabim.Queue.print_statistics}}\pysiglinewithargsret{\sphinxbfcode{print\_statistics}}{\emph{as\_str=False}, \emph{file=None}}{}
prints a summary of statistics of a queue
\begin{quote}\begin{description}
\item[{Parameters}] \leavevmode\begin{itemize}
\item {} 
\sphinxstyleliteralstrong{as\_str} (\sphinxstyleliteralemphasis{bool}) \textendash{} if False (default), print the statistics
if True, return a string containing the statistics

\item {} 
\sphinxstyleliteralstrong{file} (\sphinxstyleliteralemphasis{file}) \textendash{} if None(default), all output is directed to stdout 
otherwise, the output is directed to the file

\end{itemize}

\item[{Returns}] \leavevmode
\sphinxstylestrong{statistics (if as\_str is True)}

\item[{Return type}] \leavevmode
str

\end{description}\end{quote}

\end{fulllineitems}

\index{register() (salabim.Queue method)}

\begin{fulllineitems}
\phantomsection\label{\detokenize{Reference:salabim.Queue.register}}\pysiglinewithargsret{\sphinxbfcode{register}}{\emph{registry}}{}
registers the queue in the registry
\begin{quote}\begin{description}
\item[{Parameters}] \leavevmode
\sphinxstyleliteralstrong{registry} (\sphinxstyleliteralemphasis{list}) \textendash{} list of (to be) registered objects

\item[{Returns}] \leavevmode
\sphinxstylestrong{queue (self)}

\item[{Return type}] \leavevmode
{\hyperref[\detokenize{Reference:salabim.Queue}]{\sphinxcrossref{Queue}}}

\end{description}\end{quote}

\begin{sphinxadmonition}{note}{Note:}
Use Queue.deregister if queue does not longer need to be registered.
\end{sphinxadmonition}

\end{fulllineitems}

\index{remove() (salabim.Queue method)}

\begin{fulllineitems}
\phantomsection\label{\detokenize{Reference:salabim.Queue.remove}}\pysiglinewithargsret{\sphinxbfcode{remove}}{\emph{component=None}}{}
removes component from the queue
\begin{quote}\begin{description}
\item[{Parameters}] \leavevmode
\sphinxstyleliteralstrong{component} ({\hyperref[\detokenize{Reference:salabim.Component}]{\sphinxcrossref{\sphinxstyleliteralemphasis{Component}}}}) \textendash{} component to be removed 
if omitted, all components will be removed.

\end{description}\end{quote}

\begin{sphinxadmonition}{note}{Note:}
component must be member of the queue
\end{sphinxadmonition}

\end{fulllineitems}

\index{reset\_monitors() (salabim.Queue method)}

\begin{fulllineitems}
\phantomsection\label{\detokenize{Reference:salabim.Queue.reset_monitors}}\pysiglinewithargsret{\sphinxbfcode{reset\_monitors}}{\emph{monitor=None}}{}
resets queue monitor length\_of\_stay and length
\begin{quote}\begin{description}
\item[{Parameters}] \leavevmode
\sphinxstyleliteralstrong{monitor} (\sphinxstyleliteralemphasis{bool}) \textendash{} if True, monitoring will be on. 
if False, monitoring is disabled 
if omitted, no change of monitoring state

\end{description}\end{quote}

\begin{sphinxadmonition}{note}{Note:}
it is possible to reset individual monitoring with length\_of\_stay.reset() and length.reset()
\end{sphinxadmonition}

\end{fulllineitems}

\index{sequence\_number() (salabim.Queue method)}

\begin{fulllineitems}
\phantomsection\label{\detokenize{Reference:salabim.Queue.sequence_number}}\pysiglinewithargsret{\sphinxbfcode{sequence\_number}}{}{}~\begin{quote}\begin{description}
\item[{Returns}] \leavevmode
\sphinxstylestrong{sequence\_number of the queue} \textendash{} (the sequence number at initialization) 
normally this will be the integer value of a serialized name,
but also non serialized names (without a dot or a comma at the end)
will be numbered)

\item[{Return type}] \leavevmode
int

\end{description}\end{quote}

\end{fulllineitems}

\index{setup() (salabim.Queue method)}

\begin{fulllineitems}
\phantomsection\label{\detokenize{Reference:salabim.Queue.setup}}\pysiglinewithargsret{\sphinxbfcode{setup}}{}{}
called immediately after initialization of a queue.

by default this is a dummy method, but it can be overridden.

only keyword arguments are passed

\end{fulllineitems}

\index{successor() (salabim.Queue method)}

\begin{fulllineitems}
\phantomsection\label{\detokenize{Reference:salabim.Queue.successor}}\pysiglinewithargsret{\sphinxbfcode{successor}}{\emph{component}}{}
successor in queue
\begin{quote}\begin{description}
\item[{Parameters}] \leavevmode
\sphinxstyleliteralstrong{component} ({\hyperref[\detokenize{Reference:salabim.Component}]{\sphinxcrossref{\sphinxstyleliteralemphasis{Component}}}}) \textendash{} component whose successor to return 
must be member of the queue

\item[{Returns}] \leavevmode
\sphinxstylestrong{successor of component, if any} \textendash{} None otherwise

\item[{Return type}] \leavevmode
{\hyperref[\detokenize{Reference:salabim.Component}]{\sphinxcrossref{Component}}}

\end{description}\end{quote}

\end{fulllineitems}

\index{symmetric\_difference() (salabim.Queue method)}

\begin{fulllineitems}
\phantomsection\label{\detokenize{Reference:salabim.Queue.symmetric_difference}}\pysiglinewithargsret{\sphinxbfcode{symmetric\_difference}}{\emph{q}, \emph{name=None}, \emph{monitor=\textless{}function Queue.monitor\textgreater{}}}{}
returns the symmetric difference of two queues
\begin{quote}\begin{description}
\item[{Parameters}] \leavevmode\begin{itemize}
\item {} 
\sphinxstyleliteralstrong{q} ({\hyperref[\detokenize{Reference:salabim.Queue}]{\sphinxcrossref{\sphinxstyleliteralemphasis{Queue}}}}) \textendash{} queue to be ‘subtracted’ from self

\item {} 
\sphinxstyleliteralstrong{name} (\sphinxstyleliteralemphasis{str}) \textendash{} name of the  new queue 
if omitted, self.name() - q.name()

\item {} 
\sphinxstyleliteralstrong{monitor} (\sphinxstyleliteralemphasis{bool}) \textendash{} if True, monitor the queue 
if False (default), do not monitor the queue

\end{itemize}

\item[{Returns}] \leavevmode


\item[{Return type}] \leavevmode
queue containing all elements that are either in self or q, but not in both

\end{description}\end{quote}

\begin{sphinxadmonition}{note}{Note:}
the priority of all elements will be set to 0 for all components in the new queue.
Order: First, elelements in self (in that order), then elements in q (in that order)
Alternatively, the more pythonic \textasciicircum{} operator is also supported, e.g. q1 \textasciicircum{} q2
\end{sphinxadmonition}

\end{fulllineitems}

\index{tail() (salabim.Queue method)}

\begin{fulllineitems}
\phantomsection\label{\detokenize{Reference:salabim.Queue.tail}}\pysiglinewithargsret{\sphinxbfcode{tail}}{}{}~\begin{quote}\begin{description}
\item[{Returns}] \leavevmode
\sphinxstylestrong{the tail component of the queue, if any. None otherwise}

\item[{Return type}] \leavevmode
{\hyperref[\detokenize{Reference:salabim.Component}]{\sphinxcrossref{Component}}}

\end{description}\end{quote}

\begin{sphinxadmonition}{note}{Note:}
q{[}-1{]} is a more Pythonic way to access the tail of the queue
\end{sphinxadmonition}

\end{fulllineitems}

\index{union() (salabim.Queue method)}

\begin{fulllineitems}
\phantomsection\label{\detokenize{Reference:salabim.Queue.union}}\pysiglinewithargsret{\sphinxbfcode{union}}{\emph{q}, \emph{name=None}, \emph{monitor=False}}{}~\begin{quote}\begin{description}
\item[{Parameters}] \leavevmode\begin{itemize}
\item {} 
\sphinxstyleliteralstrong{q} ({\hyperref[\detokenize{Reference:salabim.Queue}]{\sphinxcrossref{\sphinxstyleliteralemphasis{Queue}}}}) \textendash{} queue to be unioned with self

\item {} 
\sphinxstyleliteralstrong{name} (\sphinxstyleliteralemphasis{str}) \textendash{} name of the  new queue 
if omitted, self.name() + q.name()

\item {} 
\sphinxstyleliteralstrong{monitor} (\sphinxstyleliteralemphasis{bool}) \textendash{} if True, monitor the queue 
if False (default), do not monitor the queue

\end{itemize}

\item[{Returns}] \leavevmode
\sphinxstylestrong{queue containing all elements of self and q}

\item[{Return type}] \leavevmode
{\hyperref[\detokenize{Reference:salabim.Queue}]{\sphinxcrossref{Queue}}}

\end{description}\end{quote}

\begin{sphinxadmonition}{note}{Note:}
the priority will be set to 0 for all components in the
resulting  queue 
the order of the resulting queue is as follows: 
first all components of self, in that order,
followed by all components in q that are not in self,
in that order. 
Alternatively, the more pythonic \textbar{} operator is also supported, e.g. q1 \textbar{} q2
\end{sphinxadmonition}

\end{fulllineitems}


\end{fulllineitems}



\section{Resource}
\label{\detokenize{Reference:resource}}\index{Resource (class in salabim)}

\begin{fulllineitems}
\phantomsection\label{\detokenize{Reference:salabim.Resource}}\pysiglinewithargsret{\sphinxbfcode{class }\sphinxcode{salabim.}\sphinxbfcode{Resource}}{\emph{name=None}, \emph{capacity=1}, \emph{anonymous=False}, \emph{monitor=True}, \emph{env=None}, \emph{*args}, \emph{**kwargs}}{}~\begin{quote}\begin{description}
\item[{Parameters}] \leavevmode\begin{itemize}
\item {} 
\sphinxstyleliteralstrong{name} (\sphinxstyleliteralemphasis{str}) \textendash{} name of the resource 
if the name ends with a period (.),
auto serializing will be applied 
if the name end with a comma,
auto serializing starting at 1 will be applied 
if omitted, the name will be derived from the class
it is defined in (lowercased)

\item {} 
\sphinxstyleliteralstrong{capacity} (\sphinxstyleliteralemphasis{float}) \textendash{} capacity of the resouce 
if omitted, 1

\item {} 
\sphinxstyleliteralstrong{anonymous} (\sphinxstyleliteralemphasis{bool}) \textendash{} anonymous specifier 
if True, claims are not related to any component. This is useful
if the resource is actually just a level. 
if False, claims belong to a component.

\item {} 
\sphinxstyleliteralstrong{monitor} (\sphinxstyleliteralemphasis{bool}) \textendash{} if True (default) , the requesters queue, the claimers queue,
the capacity, the available\_quantity and the claimed\_quantity are monitored 
if False, monitoring is disabled.

\item {} 
\sphinxstyleliteralstrong{env} ({\hyperref[\detokenize{Reference:salabim.Environment}]{\sphinxcrossref{\sphinxstyleliteralemphasis{Environment}}}}) \textendash{} environment to be used 
if omitted, default\_env is used

\end{itemize}

\end{description}\end{quote}
\index{base\_name() (salabim.Resource method)}

\begin{fulllineitems}
\phantomsection\label{\detokenize{Reference:salabim.Resource.base_name}}\pysiglinewithargsret{\sphinxbfcode{base\_name}}{}{}~\begin{quote}\begin{description}
\item[{Returns}] \leavevmode
\sphinxstylestrong{base name of the resource (the name used at initialization)}

\item[{Return type}] \leavevmode
str

\end{description}\end{quote}

\end{fulllineitems}

\index{claimers() (salabim.Resource method)}

\begin{fulllineitems}
\phantomsection\label{\detokenize{Reference:salabim.Resource.claimers}}\pysiglinewithargsret{\sphinxbfcode{claimers}}{}{}~\begin{quote}\begin{description}
\item[{Returns}] \leavevmode
\sphinxstylestrong{queue with all components claiming from the resource} \textendash{} will be an empty queue for an anonymous resource

\item[{Return type}] \leavevmode
{\hyperref[\detokenize{Reference:salabim.Queue}]{\sphinxcrossref{Queue}}}

\end{description}\end{quote}

\end{fulllineitems}

\index{deregister() (salabim.Resource method)}

\begin{fulllineitems}
\phantomsection\label{\detokenize{Reference:salabim.Resource.deregister}}\pysiglinewithargsret{\sphinxbfcode{deregister}}{\emph{registry}}{}
deregisters the resource in the registry
\begin{quote}\begin{description}
\item[{Parameters}] \leavevmode
\sphinxstyleliteralstrong{registry} (\sphinxstyleliteralemphasis{list}) \textendash{} list of registered components

\item[{Returns}] \leavevmode
\sphinxstylestrong{resource (self)}

\item[{Return type}] \leavevmode
{\hyperref[\detokenize{Reference:salabim.Resource}]{\sphinxcrossref{Resource}}}

\end{description}\end{quote}

\end{fulllineitems}

\index{monitor() (salabim.Resource method)}

\begin{fulllineitems}
\phantomsection\label{\detokenize{Reference:salabim.Resource.monitor}}\pysiglinewithargsret{\sphinxbfcode{monitor}}{\emph{value}}{}
enables/disables the resource monitors
\begin{quote}\begin{description}
\item[{Parameters}] \leavevmode
\sphinxstyleliteralstrong{value} (\sphinxstyleliteralemphasis{bool}) \textendash{} if True, monitoring is enabled 
if False, monitoring is disabled 

\end{description}\end{quote}

\begin{sphinxadmonition}{note}{Note:}
it is possible to individually control monitoring with claimers().monitor()
and requesters().monitor(), capacity.monitor(), available\_quantity.monitor),
claimed\_quantity.monitor() or occupancy.monitor()
\end{sphinxadmonition}

\end{fulllineitems}

\index{name() (salabim.Resource method)}

\begin{fulllineitems}
\phantomsection\label{\detokenize{Reference:salabim.Resource.name}}\pysiglinewithargsret{\sphinxbfcode{name}}{\emph{value=None}}{}~\begin{quote}\begin{description}
\item[{Parameters}] \leavevmode
\sphinxstyleliteralstrong{value} (\sphinxstyleliteralemphasis{str}) \textendash{} new name of the resource
if omitted, no change

\item[{Returns}] \leavevmode
\sphinxstylestrong{Name of the resource}

\item[{Return type}] \leavevmode
str

\end{description}\end{quote}

\begin{sphinxadmonition}{note}{Note:}
base\_name and sequence\_number are not affected if the name is changed 
All derived named are updated as well.
\end{sphinxadmonition}

\end{fulllineitems}

\index{print\_histograms() (salabim.Resource method)}

\begin{fulllineitems}
\phantomsection\label{\detokenize{Reference:salabim.Resource.print_histograms}}\pysiglinewithargsret{\sphinxbfcode{print\_histograms}}{\emph{exclude=()}, \emph{as\_str=False}, \emph{file=None}}{}
prints histograms of the requesters and claimers queue as well as
the capacity, available\_quantity and claimed\_quantity timstamped monitors of the resource
\begin{quote}\begin{description}
\item[{Parameters}] \leavevmode\begin{itemize}
\item {} 
\sphinxstyleliteralstrong{exclude} (\sphinxstyleliteralemphasis{tuple}\sphinxstyleliteralemphasis{ or }\sphinxstyleliteralemphasis{list}) \textendash{} specifies which queues or monitors to exclude 
default: ()

\item {} 
\sphinxstyleliteralstrong{as\_str} (\sphinxstyleliteralemphasis{bool}) \textendash{} if False (default), print the histograms
if True, return a string containing the histograms

\item {} 
\sphinxstyleliteralstrong{file} (\sphinxstyleliteralemphasis{file}) \textendash{} if None(default), all output is directed to stdout 
otherwise, the output is directed to the file

\end{itemize}

\item[{Returns}] \leavevmode
\sphinxstylestrong{histograms (if as\_str is True)}

\item[{Return type}] \leavevmode
str

\end{description}\end{quote}

\end{fulllineitems}

\index{print\_info() (salabim.Resource method)}

\begin{fulllineitems}
\phantomsection\label{\detokenize{Reference:salabim.Resource.print_info}}\pysiglinewithargsret{\sphinxbfcode{print\_info}}{\emph{as\_str=False}, \emph{file=None}}{}
prints info about the resource
\begin{quote}\begin{description}
\item[{Parameters}] \leavevmode\begin{itemize}
\item {} 
\sphinxstyleliteralstrong{as\_str} (\sphinxstyleliteralemphasis{bool}) \textendash{} if False (default), print the info
if True, return a string containing the info

\item {} 
\sphinxstyleliteralstrong{file} (\sphinxstyleliteralemphasis{file}) \textendash{} if None(default), all output is directed to stdout 
otherwise, the output is directed to the file

\end{itemize}

\item[{Returns}] \leavevmode
\sphinxstylestrong{info (if as\_str is True)}

\item[{Return type}] \leavevmode
str

\end{description}\end{quote}

\end{fulllineitems}

\index{print\_statistics() (salabim.Resource method)}

\begin{fulllineitems}
\phantomsection\label{\detokenize{Reference:salabim.Resource.print_statistics}}\pysiglinewithargsret{\sphinxbfcode{print\_statistics}}{\emph{as\_str=False}, \emph{file=None}}{}
prints a summary of statistics of a resource
\begin{quote}\begin{description}
\item[{Parameters}] \leavevmode\begin{itemize}
\item {} 
\sphinxstyleliteralstrong{as\_str} (\sphinxstyleliteralemphasis{bool}) \textendash{} if False (default), print the statistics
if True, return a string containing the statistics

\item {} 
\sphinxstyleliteralstrong{file} (\sphinxstyleliteralemphasis{file}) \textendash{} if None(default), all output is directed to stdout 
otherwise, the output is directed to the file

\end{itemize}

\item[{Returns}] \leavevmode
\sphinxstylestrong{statistics (if as\_str is True)}

\item[{Return type}] \leavevmode
str

\end{description}\end{quote}

\end{fulllineitems}

\index{register() (salabim.Resource method)}

\begin{fulllineitems}
\phantomsection\label{\detokenize{Reference:salabim.Resource.register}}\pysiglinewithargsret{\sphinxbfcode{register}}{\emph{registry}}{}
registers the resource in the registry
\begin{quote}\begin{description}
\item[{Parameters}] \leavevmode
\sphinxstyleliteralstrong{registry} (\sphinxstyleliteralemphasis{list}) \textendash{} list of (to be) registered objects

\item[{Returns}] \leavevmode
\sphinxstylestrong{resource (self)}

\item[{Return type}] \leavevmode
{\hyperref[\detokenize{Reference:salabim.Resource}]{\sphinxcrossref{Resource}}}

\end{description}\end{quote}

\begin{sphinxadmonition}{note}{Note:}
Use Resource.deregister if resource does not longer need to be registered.
\end{sphinxadmonition}

\end{fulllineitems}

\index{release() (salabim.Resource method)}

\begin{fulllineitems}
\phantomsection\label{\detokenize{Reference:salabim.Resource.release}}\pysiglinewithargsret{\sphinxbfcode{release}}{\emph{quantity=None}}{}
releases all claims or a specified quantity
\begin{quote}\begin{description}
\item[{Parameters}] \leavevmode
\sphinxstyleliteralstrong{quantity} (\sphinxstyleliteralemphasis{float}) \textendash{} quantity to be released 
if not specified, the resource will be emptied completely 
for non-anonymous resources, all components claiming from this resource
will be released.

\end{description}\end{quote}

\begin{sphinxadmonition}{note}{Note:}
quantity may not be specified for a non-anomymous resoure
\end{sphinxadmonition}

\end{fulllineitems}

\index{requesters() (salabim.Resource method)}

\begin{fulllineitems}
\phantomsection\label{\detokenize{Reference:salabim.Resource.requesters}}\pysiglinewithargsret{\sphinxbfcode{requesters}}{}{}~\begin{quote}\begin{description}
\item[{Returns}] \leavevmode
\sphinxstylestrong{queue containing all components with not yet honored requests}

\item[{Return type}] \leavevmode
{\hyperref[\detokenize{Reference:salabim.Queue}]{\sphinxcrossref{Queue}}}

\end{description}\end{quote}

\end{fulllineitems}

\index{reset\_monitors() (salabim.Resource method)}

\begin{fulllineitems}
\phantomsection\label{\detokenize{Reference:salabim.Resource.reset_monitors}}\pysiglinewithargsret{\sphinxbfcode{reset\_monitors}}{\emph{monitor=None}}{}
resets the resource monitors
\begin{quote}\begin{description}
\item[{Parameters}] \leavevmode
\sphinxstyleliteralstrong{monitor} (\sphinxstyleliteralemphasis{bool}) \textendash{} if True, monitoring will be on. 
if False, monitoring is disabled 
if omitted, no change of monitoring state

\end{description}\end{quote}

\begin{sphinxadmonition}{note}{Note:}
it is possible to reset individual monitoring with
claimers().reset\_monitors(),
requesters().reset\_monitors,
capacity.reset(),
available\_quantity.reset() or
claimed\_quantity.reset() or
occupancy.reset()
\end{sphinxadmonition}

\end{fulllineitems}

\index{sequence\_number() (salabim.Resource method)}

\begin{fulllineitems}
\phantomsection\label{\detokenize{Reference:salabim.Resource.sequence_number}}\pysiglinewithargsret{\sphinxbfcode{sequence\_number}}{}{}~\begin{quote}\begin{description}
\item[{Returns}] \leavevmode
\sphinxstylestrong{sequence\_number of the resource} \textendash{} (the sequence number at initialization) 
normally this will be the integer value of a serialized name,
but also non serialized names (without a dot or a comma at the end)
will be numbered)

\item[{Return type}] \leavevmode
int

\end{description}\end{quote}

\end{fulllineitems}

\index{set\_capacity() (salabim.Resource method)}

\begin{fulllineitems}
\phantomsection\label{\detokenize{Reference:salabim.Resource.set_capacity}}\pysiglinewithargsret{\sphinxbfcode{set\_capacity}}{\emph{cap}}{}~\begin{quote}\begin{description}
\item[{Parameters}] \leavevmode
\sphinxstyleliteralstrong{cap} (\sphinxstyleliteralemphasis{float}\sphinxstyleliteralemphasis{ or }\sphinxstyleliteralemphasis{int}) \textendash{} capacity of the resource 
this may lead to honoring one or more requests. 
if omitted, no change

\end{description}\end{quote}

\end{fulllineitems}

\index{setup() (salabim.Resource method)}

\begin{fulllineitems}
\phantomsection\label{\detokenize{Reference:salabim.Resource.setup}}\pysiglinewithargsret{\sphinxbfcode{setup}}{}{}
called immediately after initialization of a resource.

by default this is a dummy method, but it can be overridden.

only keyword arguments are passed

\end{fulllineitems}


\end{fulllineitems}



\section{State}
\label{\detokenize{Reference:state}}\index{State (class in salabim)}

\begin{fulllineitems}
\phantomsection\label{\detokenize{Reference:salabim.State}}\pysiglinewithargsret{\sphinxbfcode{class }\sphinxcode{salabim.}\sphinxbfcode{State}}{\emph{name=None}, \emph{value=False}, \emph{type='any'}, \emph{monitor=True}, \emph{animation\_objects=None}, \emph{env=None}, \emph{*args}, \emph{**kwargs}}{}~\begin{quote}\begin{description}
\item[{Parameters}] \leavevmode\begin{itemize}
\item {} 
\sphinxstyleliteralstrong{name} (\sphinxstyleliteralemphasis{str}) \textendash{} name of the state 
if the name ends with a period (.),
auto serializing will be applied 
if the name end with a comma,
auto serializing starting at 1 will be applied 
if omitted, the name will be derived from the class
it is defined in (lowercased)

\item {} 
\sphinxstyleliteralstrong{value} (\sphinxstyleliteralemphasis{any}\sphinxstyleliteralemphasis{, }\sphinxstyleliteralemphasis{preferably printable}) \textendash{} initial value of the state 
if omitted, False

\item {} 
\sphinxstyleliteralstrong{monitor} (\sphinxstyleliteralemphasis{bool}) \textendash{} if True (default) , the waiters queue and the value are monitored 
if False, monitoring is disabled.

\item {} 
\sphinxstyleliteralstrong{type} (\sphinxstyleliteralemphasis{str}) \textendash{} 
specifies how the state values are monitored. Using a
int, uint of float type results in less memory usage and better
performance. Note that you should avoid the number not to use
as this is used to indicate ‘off’
\begin{itemize}
\item {} 
’any’ (default) stores values in a list. This allows for
non numeric values. In calculations the values are
forced to a numeric value (0 if not possible) do not use -inf

\item {} 
’bool’ bool (False, True). Actually integer \textgreater{}= 0 \textless{}= 254 1 byte do not use 255

\item {} 
’int8’ integer \textgreater{}= -127 \textless{}= 127 1 byte do not use -128

\item {} 
’uint8’ integer \textgreater{}= 0 \textless{}= 254 1 byte do not use 255

\item {} 
’int16’ integer \textgreater{}= -32767 \textless{}= 32767 2 bytes do not use -32768

\item {} 
’uint16’ integer \textgreater{}= 0 \textless{}= 65534 2 bytes do not use 65535

\item {} 
’int32’ integer \textgreater{}= -2147483647 \textless{}= 2147483647 4 bytes do not use -2147483648

\item {} 
’uint32’ integer \textgreater{}= 0 \textless{}= 4294967294 4 bytes do not use 4294967295

\item {} 
’int64’ integer \textgreater{}= -9223372036854775807 \textless{}= 9223372036854775807 8 bytes do not use -9223372036854775808

\item {} 
’uint64’ integer \textgreater{}= 0 \textless{}= 18446744073709551614 8 bytes do not use 18446744073709551615

\item {} 
’float’ float 8 bytes do not use -inf

\end{itemize}


\item {} 
\sphinxstyleliteralstrong{animation\_objects} (\sphinxstyleliteralemphasis{list}\sphinxstyleliteralemphasis{ or }\sphinxstyleliteralemphasis{tuple}) \textendash{} overrides the default animation\_object method 
the method should have a header like 
\sphinxcode{def animation\_objects(self, value):} 
and should return a list or tuple of animation objects, which
will be used when the state changes value. 
The default method displays a square of size 40. If the value
is a valid color, that will be the color of the square. Otherwise,
the square will be black with the value displayed in white in
the centre.

\item {} 
\sphinxstyleliteralstrong{env} ({\hyperref[\detokenize{Reference:salabim.Environment}]{\sphinxcrossref{\sphinxstyleliteralemphasis{Environment}}}}) \textendash{} environment to be used 
if omitted, default\_env is used

\end{itemize}

\end{description}\end{quote}
\index{base\_name() (salabim.State method)}

\begin{fulllineitems}
\phantomsection\label{\detokenize{Reference:salabim.State.base_name}}\pysiglinewithargsret{\sphinxbfcode{base\_name}}{}{}~\begin{quote}\begin{description}
\item[{Returns}] \leavevmode
\sphinxstylestrong{base name of the state (the name used at initialization)}

\item[{Return type}] \leavevmode
str

\end{description}\end{quote}

\end{fulllineitems}

\index{deregister() (salabim.State method)}

\begin{fulllineitems}
\phantomsection\label{\detokenize{Reference:salabim.State.deregister}}\pysiglinewithargsret{\sphinxbfcode{deregister}}{\emph{registry}}{}
deregisters the state in the registry
\begin{quote}\begin{description}
\item[{Parameters}] \leavevmode
\sphinxstyleliteralstrong{registry} (\sphinxstyleliteralemphasis{list}) \textendash{} list of registered states

\item[{Returns}] \leavevmode
\sphinxstylestrong{state (self)}

\item[{Return type}] \leavevmode
{\hyperref[\detokenize{Reference:salabim.State}]{\sphinxcrossref{State}}}

\end{description}\end{quote}

\end{fulllineitems}

\index{get() (salabim.State method)}

\begin{fulllineitems}
\phantomsection\label{\detokenize{Reference:salabim.State.get}}\pysiglinewithargsret{\sphinxbfcode{get}}{}{}
get value of the state
\begin{quote}\begin{description}
\item[{Returns}] \leavevmode

\sphinxstylestrong{value of the state} \textendash{} Instead of this method, the state can also be called directly, like 

level = sim.State(‘level’) 
… 
print(level()) 
print(level.get())  \# identical 


\item[{Return type}] \leavevmode
any

\end{description}\end{quote}

\end{fulllineitems}

\index{monitor() (salabim.State method)}

\begin{fulllineitems}
\phantomsection\label{\detokenize{Reference:salabim.State.monitor}}\pysiglinewithargsret{\sphinxbfcode{monitor}}{\emph{value=None}}{}
enables/disables the state monitors and value monitor
\begin{quote}\begin{description}
\item[{Parameters}] \leavevmode
\sphinxstyleliteralstrong{value} (\sphinxstyleliteralemphasis{bool}) \textendash{} if True, monitoring will be on. 
if False, monitoring is disabled 
if not specified, no change

\end{description}\end{quote}

\begin{sphinxadmonition}{note}{Note:}\begin{description}
\item[{it is possible to individually control requesters().monitor(),}] \leavevmode
value.monitor()

\end{description}
\end{sphinxadmonition}

\end{fulllineitems}

\index{name() (salabim.State method)}

\begin{fulllineitems}
\phantomsection\label{\detokenize{Reference:salabim.State.name}}\pysiglinewithargsret{\sphinxbfcode{name}}{\emph{value=None}}{}~\begin{quote}\begin{description}
\item[{Parameters}] \leavevmode
\sphinxstyleliteralstrong{value} (\sphinxstyleliteralemphasis{str}) \textendash{} new name of the state
if omitted, no change

\item[{Returns}] \leavevmode
\sphinxstylestrong{Name of the state}

\item[{Return type}] \leavevmode
str

\end{description}\end{quote}

\begin{sphinxadmonition}{note}{Note:}
base\_name and sequence\_number are not affected if the name is changed 
All derived named are updated as well.
\end{sphinxadmonition}

\end{fulllineitems}

\index{print\_histograms() (salabim.State method)}

\begin{fulllineitems}
\phantomsection\label{\detokenize{Reference:salabim.State.print_histograms}}\pysiglinewithargsret{\sphinxbfcode{print\_histograms}}{\emph{exclude=()}, \emph{as\_str=False}, \emph{file=None}}{}
print histograms of the waiters queue and the value monitor
\begin{quote}\begin{description}
\item[{Parameters}] \leavevmode\begin{itemize}
\item {} 
\sphinxstyleliteralstrong{exclude} (\sphinxstyleliteralemphasis{tuple}\sphinxstyleliteralemphasis{ or }\sphinxstyleliteralemphasis{list}) \textendash{} specifies which queues or monitors to exclude 
default: ()

\item {} 
\sphinxstyleliteralstrong{as\_str} (\sphinxstyleliteralemphasis{bool}) \textendash{} if False (default), print the histograms
if True, return a string containing the histograms

\item {} 
\sphinxstyleliteralstrong{file} (\sphinxstyleliteralemphasis{file}) \textendash{} if None(default), all output is directed to stdout 
otherwise, the output is directed to the file

\end{itemize}

\item[{Returns}] \leavevmode
\sphinxstylestrong{histograms (if as\_str is True)}

\item[{Return type}] \leavevmode
str

\end{description}\end{quote}

\end{fulllineitems}

\index{print\_info() (salabim.State method)}

\begin{fulllineitems}
\phantomsection\label{\detokenize{Reference:salabim.State.print_info}}\pysiglinewithargsret{\sphinxbfcode{print\_info}}{\emph{as\_str=False}, \emph{file=None}}{}
prints info about the state
\begin{quote}\begin{description}
\item[{Parameters}] \leavevmode\begin{itemize}
\item {} 
\sphinxstyleliteralstrong{as\_str} (\sphinxstyleliteralemphasis{bool}) \textendash{} if False (default), print the info
if True, return a string containing the info

\item {} 
\sphinxstyleliteralstrong{file} (\sphinxstyleliteralemphasis{file}) \textendash{} if None(default), all output is directed to stdout 
otherwise, the output is directed to the file

\end{itemize}

\item[{Returns}] \leavevmode
\sphinxstylestrong{info (if as\_str is True)}

\item[{Return type}] \leavevmode
str

\end{description}\end{quote}

\end{fulllineitems}

\index{print\_statistics() (salabim.State method)}

\begin{fulllineitems}
\phantomsection\label{\detokenize{Reference:salabim.State.print_statistics}}\pysiglinewithargsret{\sphinxbfcode{print\_statistics}}{\emph{as\_str=False}, \emph{file=None}}{}
prints a summary of statistics of the state
\begin{quote}\begin{description}
\item[{Parameters}] \leavevmode\begin{itemize}
\item {} 
\sphinxstyleliteralstrong{as\_str} (\sphinxstyleliteralemphasis{bool}) \textendash{} if False (default), print the statistics
if True, return a string containing the statistics

\item {} 
\sphinxstyleliteralstrong{file} (\sphinxstyleliteralemphasis{file}) \textendash{} if None(default), all output is directed to stdout 
otherwise, the output is directed to the file

\end{itemize}

\item[{Returns}] \leavevmode
\sphinxstylestrong{statistics (if as\_str is True)}

\item[{Return type}] \leavevmode
str

\end{description}\end{quote}

\end{fulllineitems}

\index{register() (salabim.State method)}

\begin{fulllineitems}
\phantomsection\label{\detokenize{Reference:salabim.State.register}}\pysiglinewithargsret{\sphinxbfcode{register}}{\emph{registry}}{}
registers the state in the registry
\begin{quote}\begin{description}
\item[{Parameters}] \leavevmode
\sphinxstyleliteralstrong{registry} (\sphinxstyleliteralemphasis{list}) \textendash{} list of (to be) registered objetcs

\item[{Returns}] \leavevmode
\sphinxstylestrong{state (self)}

\item[{Return type}] \leavevmode
{\hyperref[\detokenize{Reference:salabim.State}]{\sphinxcrossref{State}}}

\end{description}\end{quote}

\begin{sphinxadmonition}{note}{Note:}
Use State.deregister if state does not longer need to be registered.
\end{sphinxadmonition}

\end{fulllineitems}

\index{reset() (salabim.State method)}

\begin{fulllineitems}
\phantomsection\label{\detokenize{Reference:salabim.State.reset}}\pysiglinewithargsret{\sphinxbfcode{reset}}{\emph{value=False}}{}
reset the value of the state
\begin{quote}\begin{description}
\item[{Parameters}] \leavevmode
\sphinxstyleliteralstrong{value} (\sphinxstyleliteralemphasis{any}\sphinxstyleliteralemphasis{ (}\sphinxstyleliteralemphasis{preferably printable}\sphinxstyleliteralemphasis{)}) \textendash{} if omitted, False 
if there is a change, the waiters queue will be checked
to see whether there are waiting components to be honored

\end{description}\end{quote}

\begin{sphinxadmonition}{note}{Note:}
This method is identical to set, except the default value is False.
\end{sphinxadmonition}

\end{fulllineitems}

\index{reset\_monitors() (salabim.State method)}

\begin{fulllineitems}
\phantomsection\label{\detokenize{Reference:salabim.State.reset_monitors}}\pysiglinewithargsret{\sphinxbfcode{reset\_monitors}}{\emph{monitor=None}}{}
resets the monitor for the state’s value and the monitors of the waiters queue
\begin{quote}\begin{description}
\item[{Parameters}] \leavevmode
\sphinxstyleliteralstrong{monitor} (\sphinxstyleliteralemphasis{bool}) \textendash{} if True, monitoring will be on. 
if False, monitoring is disabled 
if omitted, no change of monitoring state

\end{description}\end{quote}

\end{fulllineitems}

\index{sequence\_number() (salabim.State method)}

\begin{fulllineitems}
\phantomsection\label{\detokenize{Reference:salabim.State.sequence_number}}\pysiglinewithargsret{\sphinxbfcode{sequence\_number}}{}{}~\begin{quote}\begin{description}
\item[{Returns}] \leavevmode
\sphinxstylestrong{sequence\_number of the state} \textendash{} (the sequence number at initialization) 
normally this will be the integer value of a serialized name,
but also non serialized names (without a dot or a comma at the end)
will be numbered)

\item[{Return type}] \leavevmode
int

\end{description}\end{quote}

\end{fulllineitems}

\index{set() (salabim.State method)}

\begin{fulllineitems}
\phantomsection\label{\detokenize{Reference:salabim.State.set}}\pysiglinewithargsret{\sphinxbfcode{set}}{\emph{value=True}}{}
set the value of the state
\begin{quote}\begin{description}
\item[{Parameters}] \leavevmode
\sphinxstyleliteralstrong{value} (\sphinxstyleliteralemphasis{any}\sphinxstyleliteralemphasis{ (}\sphinxstyleliteralemphasis{preferably printable}\sphinxstyleliteralemphasis{)}) \textendash{} if omitted, True 
if there is a change, the waiters queue will be checked
to see whether there are waiting components to be honored

\end{description}\end{quote}

\begin{sphinxadmonition}{note}{Note:}
This method is identical to reset, except the default value is True.
\end{sphinxadmonition}

\end{fulllineitems}

\index{setup() (salabim.State method)}

\begin{fulllineitems}
\phantomsection\label{\detokenize{Reference:salabim.State.setup}}\pysiglinewithargsret{\sphinxbfcode{setup}}{}{}
called immediately after initialization of a state.

by default this is a dummy method, but it can be overridden.

only keyword arguments will be passed

\end{fulllineitems}

\index{trigger() (salabim.State method)}

\begin{fulllineitems}
\phantomsection\label{\detokenize{Reference:salabim.State.trigger}}\pysiglinewithargsret{\sphinxbfcode{trigger}}{\emph{value=True}, \emph{value\_after=None}, \emph{max=inf}}{}
triggers the value of the state
\begin{quote}\begin{description}
\item[{Parameters}] \leavevmode\begin{itemize}
\item {} 
\sphinxstyleliteralstrong{value} (\sphinxstyleliteralemphasis{any}\sphinxstyleliteralemphasis{ (}\sphinxstyleliteralemphasis{preferably printable}\sphinxstyleliteralemphasis{)}) \textendash{} if omitted, True 

\item {} 
\sphinxstyleliteralstrong{value\_after} (\sphinxstyleliteralemphasis{any}\sphinxstyleliteralemphasis{ (}\sphinxstyleliteralemphasis{preferably printable}\sphinxstyleliteralemphasis{)}) \textendash{} after the trigger, this will be the new value. 
if omitted, return to the the before the trigger.

\item {} 
\sphinxstyleliteralstrong{max} (\sphinxstyleliteralemphasis{int}) \textendash{} maximum number of components to be honored for the trigger value 
default: inf

\end{itemize}

\end{description}\end{quote}

\begin{sphinxadmonition}{note}{Note:}
The value of the state will be set to value, then at most
max waiting components for this state  will be honored and next
the value will be set to value\_after and again checked for possible
honors.
\end{sphinxadmonition}

\end{fulllineitems}

\index{waiters() (salabim.State method)}

\begin{fulllineitems}
\phantomsection\label{\detokenize{Reference:salabim.State.waiters}}\pysiglinewithargsret{\sphinxbfcode{waiters}}{}{}~\begin{quote}\begin{description}
\item[{Returns}] \leavevmode
\sphinxstylestrong{queue containing all components waiting for this state}

\item[{Return type}] \leavevmode
{\hyperref[\detokenize{Reference:salabim.Queue}]{\sphinxcrossref{Queue}}}

\end{description}\end{quote}

\end{fulllineitems}


\end{fulllineitems}



\section{Miscellaneous}
\label{\detokenize{Reference:miscellaneous}}\index{arrow\_polygon() (in module salabim)}

\begin{fulllineitems}
\phantomsection\label{\detokenize{Reference:salabim.arrow_polygon}}\pysiglinewithargsret{\sphinxcode{salabim.}\sphinxbfcode{arrow\_polygon}}{\emph{size}}{}
creates a polygon tuple with a centerd arrow for use with sim.Animate
\begin{quote}\begin{description}
\item[{Parameters}] \leavevmode
\sphinxstyleliteralstrong{size} (\sphinxstyleliteralemphasis{float}) \textendash{} length of the arrow

\end{description}\end{quote}

\end{fulllineitems}

\index{can\_animate() (in module salabim)}

\begin{fulllineitems}
\phantomsection\label{\detokenize{Reference:salabim.can_animate}}\pysiglinewithargsret{\sphinxcode{salabim.}\sphinxbfcode{can\_animate}}{\emph{try\_only=True}}{}
Tests whether animation is supported.
\begin{quote}\begin{description}
\item[{Parameters}] \leavevmode
\sphinxstyleliteralstrong{try\_only} (\sphinxstyleliteralemphasis{bool}) \textendash{} if True (default), the function does not raise an error when the required modules cannot be imported 
if False, the function will only return if the required modules could be imported.

\item[{Returns}] \leavevmode
\sphinxstylestrong{True, if required modules could be imported, False otherwise}

\item[{Return type}] \leavevmode
bool

\end{description}\end{quote}

\end{fulllineitems}

\index{can\_video() (in module salabim)}

\begin{fulllineitems}
\phantomsection\label{\detokenize{Reference:salabim.can_video}}\pysiglinewithargsret{\sphinxcode{salabim.}\sphinxbfcode{can\_video}}{\emph{try\_only=True}}{}
Tests whether video is supported.
\begin{quote}\begin{description}
\item[{Parameters}] \leavevmode
\sphinxstyleliteralstrong{try\_only} (\sphinxstyleliteralemphasis{bool}) \textendash{} if True (default), the function does not raise an error when the required modules cannot be imported 
if False, the function will only return if the required modules could be imported.

\item[{Returns}] \leavevmode
\sphinxstylestrong{True, if required modules could be imported, False otherwise}

\item[{Return type}] \leavevmode
bool

\end{description}\end{quote}

\end{fulllineitems}

\index{centered\_rectangle() (in module salabim)}

\begin{fulllineitems}
\phantomsection\label{\detokenize{Reference:salabim.centered_rectangle}}\pysiglinewithargsret{\sphinxcode{salabim.}\sphinxbfcode{centered\_rectangle}}{\emph{width}, \emph{height}}{}
creates a rectangle tuple with a centered rectangle for use with sim.Animate
\begin{quote}\begin{description}
\item[{Parameters}] \leavevmode\begin{itemize}
\item {} 
\sphinxstyleliteralstrong{width} (\sphinxstyleliteralemphasis{float}) \textendash{} width of the rectangle

\item {} 
\sphinxstyleliteralstrong{height} (\sphinxstyleliteralemphasis{float}) \textendash{} height of the rectangle

\end{itemize}

\end{description}\end{quote}

\end{fulllineitems}

\index{colornames() (in module salabim)}

\begin{fulllineitems}
\phantomsection\label{\detokenize{Reference:salabim.colornames}}\pysiglinewithargsret{\sphinxcode{salabim.}\sphinxbfcode{colornames}}{}{}
available colornames
\begin{quote}\begin{description}
\item[{Returns}] \leavevmode
\sphinxstylestrong{dict with name of color as key, \#rrggbb or \#rrggbbaa as value}

\item[{Return type}] \leavevmode
dict

\end{description}\end{quote}

\end{fulllineitems}

\index{default\_env() (in module salabim)}

\begin{fulllineitems}
\phantomsection\label{\detokenize{Reference:salabim.default_env}}\pysiglinewithargsret{\sphinxcode{salabim.}\sphinxbfcode{default\_env}}{}{}~\begin{quote}\begin{description}
\item[{Returns}] \leavevmode
\sphinxstylestrong{default environment}

\item[{Return type}] \leavevmode
{\hyperref[\detokenize{Reference:salabim.Environment}]{\sphinxcrossref{Environment}}}

\end{description}\end{quote}

\end{fulllineitems}

\index{interpolate() (in module salabim)}

\begin{fulllineitems}
\phantomsection\label{\detokenize{Reference:salabim.interpolate}}\pysiglinewithargsret{\sphinxcode{salabim.}\sphinxbfcode{interpolate}}{\emph{t}, \emph{t0}, \emph{t1}, \emph{v0}, \emph{v1}}{}
does linear interpolation
\begin{quote}\begin{description}
\item[{Parameters}] \leavevmode\begin{itemize}
\item {} 
\sphinxstyleliteralstrong{t} (\sphinxstyleliteralemphasis{float}) \textendash{} value to be interpolated from

\item {} 
\sphinxstyleliteralstrong{t0} (\sphinxstyleliteralemphasis{float}) \textendash{} f(t0)=v0

\item {} 
\sphinxstyleliteralstrong{t1} (\sphinxstyleliteralemphasis{float}) \textendash{} f(t1)=v1

\item {} 
\sphinxstyleliteralstrong{v0} (\sphinxstyleliteralemphasis{float}\sphinxstyleliteralemphasis{, }\sphinxstyleliteralemphasis{list}\sphinxstyleliteralemphasis{ or }\sphinxstyleliteralemphasis{tuple}) \textendash{} f(t0)=v0

\item {} 
\sphinxstyleliteralstrong{v1} (\sphinxstyleliteralemphasis{float}\sphinxstyleliteralemphasis{, }\sphinxstyleliteralemphasis{list}\sphinxstyleliteralemphasis{ or }\sphinxstyleliteralemphasis{tuple}) \textendash{} f(t1)=v1 
if list or tuple, len(v0) should equal len(v1)

\end{itemize}

\item[{Returns}] \leavevmode
\sphinxstylestrong{linear interpolation between v0 and v1 based on t between t0 and t1}

\item[{Return type}] \leavevmode
float or tuple

\end{description}\end{quote}

\begin{sphinxadmonition}{note}{Note:}
Note that no extrapolation is done, so if t\textless{}t0 ==\textgreater{} v0  and t\textgreater{}t1 ==\textgreater{} v1 
This function is heavily used during animation.
\end{sphinxadmonition}

\end{fulllineitems}

\index{random\_seed() (in module salabim)}

\begin{fulllineitems}
\phantomsection\label{\detokenize{Reference:salabim.random_seed}}\pysiglinewithargsret{\sphinxcode{salabim.}\sphinxbfcode{random\_seed}}{\emph{seed}, \emph{randomstream=None}}{}
Reseeds a randomstream
\begin{quote}\begin{description}
\item[{Parameters}] \leavevmode\begin{itemize}
\item {} 
\sphinxstyleliteralstrong{seed} (\sphinxstyleliteralemphasis{hashable object}\sphinxstyleliteralemphasis{, }\sphinxstyleliteralemphasis{usually int}) \textendash{} the seed for random, equivalent to random.seed() 
if None or ‘*’, a purely random value (based on the current time) will be used
(not reproducable) 

\item {} 
\sphinxstyleliteralstrong{randomstream} (\sphinxstyleliteralemphasis{randomstream}) \textendash{} randomstream to be used 
if omitted, random will be used 

\end{itemize}

\end{description}\end{quote}

\end{fulllineitems}

\index{regular\_polygon() (in module salabim)}

\begin{fulllineitems}
\phantomsection\label{\detokenize{Reference:salabim.regular_polygon}}\pysiglinewithargsret{\sphinxcode{salabim.}\sphinxbfcode{regular\_polygon}}{\emph{radius=1}, \emph{number\_of\_sides=3}, \emph{initial\_angle=0}}{}
creates a polygon tuple with a regular polygon (within a circle) for use with sim.Animate
\begin{quote}\begin{description}
\item[{Parameters}] \leavevmode\begin{itemize}
\item {} 
\sphinxstyleliteralstrong{radius} (\sphinxstyleliteralemphasis{float}) \textendash{} radius of the corner points of the polygon 
default : 1

\item {} 
\sphinxstyleliteralstrong{number\_of\_sides} (\sphinxstyleliteralemphasis{int}) \textendash{} number of sides (corners) 
must be \textgreater{}= 3 
default : 3

\item {} 
\sphinxstyleliteralstrong{initial\_angle} (\sphinxstyleliteralemphasis{float}) \textendash{} angle of the first corner point, relative to the origin 
default : 0

\end{itemize}

\end{description}\end{quote}

\end{fulllineitems}

\index{reset() (in module salabim)}

\begin{fulllineitems}
\phantomsection\label{\detokenize{Reference:salabim.reset}}\pysiglinewithargsret{\sphinxcode{salabim.}\sphinxbfcode{reset}}{}{}
resets global variables

used internally at import of salabim

might be useful for REPLs or for Pythonista

\end{fulllineitems}

\index{show\_colornames() (in module salabim)}

\begin{fulllineitems}
\phantomsection\label{\detokenize{Reference:salabim.show_colornames}}\pysiglinewithargsret{\sphinxcode{salabim.}\sphinxbfcode{show\_colornames}}{}{}
show (print) all available color names and their value.

\end{fulllineitems}

\index{show\_fonts() (in module salabim)}

\begin{fulllineitems}
\phantomsection\label{\detokenize{Reference:salabim.show_fonts}}\pysiglinewithargsret{\sphinxcode{salabim.}\sphinxbfcode{show\_fonts}}{}{}
show (print) all available fonts on this machine

\end{fulllineitems}

\index{spec\_to\_image() (in module salabim)}

\begin{fulllineitems}
\phantomsection\label{\detokenize{Reference:salabim.spec_to_image}}\pysiglinewithargsret{\sphinxcode{salabim.}\sphinxbfcode{spec\_to\_image}}{\emph{spec}}{}
convert an image specification to an image
\begin{quote}\begin{description}
\item[{Parameters}] \leavevmode
\sphinxstyleliteralstrong{image} (\sphinxstyleliteralemphasis{str}\sphinxstyleliteralemphasis{ or }\sphinxstyleliteralemphasis{PIL.Image.Image}) \textendash{} if str: filename of file to be loaded 
if ‘’: dummy image will be returned 
if PIL.Image.Image: return this image untranslated

\item[{Returns}] \leavevmode
\sphinxstylestrong{image}

\item[{Return type}] \leavevmode
PIL.Image.Image

\end{description}\end{quote}

\end{fulllineitems}



\chapter{About}
\label{\detokenize{About:about}}\label{\detokenize{About::doc}}

\section{Who is behind salabim?}
\label{\detokenize{About:who-is-behind-salabim}}
I, Ruud van der Ham, am the sole developer of salabim. I have a long history in simulation, both in
applications and tool building.

It all started in the mid 70’s when modeling container terminals in Prosim, a package
in PL/1 that was inspired by Simula and run on big IBM 360/370 mainframes. 
In the eighties, Prosim was ported to smaller computers, but at the same time I
developed a discrete event simulation tool called Must to run on CP/M machines, later
on MSDOS machines, again under PL/1. A bit later, Must was ported to Pascal and was
used in many projects. Must was never ported to Windows. Instead, Hans Veeke (Delft University)
came with Tomas, a package that is still available and runs under Delphi. 
End 2016, I wanted an easy to use and open source package for a project, preferably
in Python. Unfortunately, Simpy (particularly version 3) does not support the essential
process interaction methods activate, hold, passivate and standby. First I tried to
build a wrapper around Simpy 3, but that didn’t work too well.

That was the start of a new package, called salabim.
One of the key features of salabim is the powerful animation engine that is heavily
inspired by some more creative projects where every animation object can change position,
shape, colour, orientation over time. Although rarely used in normal simulation models,
all that functionality is available in salabim. 
Over the year 2017, a lot of functionality was added as well bugs were fixed. During that year
the package became available on PyPI and GitHub and the documentation was made available. 
Large parts of salabim were actually developed on an iPad on the excellent Pythonista platform. The full
functionality is thus available under iOS as well.


\section{Why is the package called salabim?}
\label{\detokenize{About:why-is-the-package-called-salabim}}
\begin{sphinxVerbatim}[commandchars=\\\{\}]
\PYG{n}{s} \PYG{o}{=} \PYG{l+s+s1}{\PYGZsq{}}\PYG{l+s+s1}{sim ulation}\PYG{l+s+s1}{\PYGZsq{}}
\PYG{k}{print}\PYG{p}{(}\PYG{n}{s}\PYG{p}{)}
\PYG{n}{s} \PYG{o}{=} \PYG{n}{s}\PYG{p}{[}\PYG{p}{:}\PYG{l+m+mi}{4}\PYG{p}{]}
\PYG{k}{print}\PYG{p}{(}\PYG{n}{s}\PYG{p}{)}
\PYG{n}{s} \PYG{o}{+}\PYG{o}{=} \PYG{l+s+s1}{\PYGZsq{}}\PYG{l+s+s1}{salabim}\PYG{l+s+s1}{\PYGZsq{}}
\PYG{k}{print}\PYG{p}{(}\PYG{n}{s}\PYG{p}{)}
\PYG{n}{s} \PYG{o}{=} \PYG{l+s+s1}{\PYGZsq{}}\PYG{l+s+s1}{ }\PYG{l+s+s1}{\PYGZsq{}} \PYG{o}{*} \PYG{l+m+mi}{4} \PYG{o}{+} \PYG{n}{s}\PYG{p}{[}\PYG{l+m+mi}{4}\PYG{p}{:}\PYG{p}{]}
\PYG{k}{print}\PYG{p}{(}\PYG{n}{s}\PYG{p}{)}
\end{sphinxVerbatim}

\begin{sphinxVerbatim}[commandchars=\\\{\}]
sim ulation
sim
sim salabim
    salabim
\end{sphinxVerbatim}


\section{Contributing and reporting issues}
\label{\detokenize{About:contributing-and-reporting-issues}}
It is very much appreciated to contribute to the salabim, by issuing a pull request or issue on GitHub.

Also, issues can be reported this way.

Alternatively, the Google group can be used for this.


\section{Support}
\label{\detokenize{About:support}}
Ruud van der Ham is able and willing to help users with issues with the package or modelling in general.

He is also available for consultancy and or trainings.

Contact him or other users via the Google group or \sphinxhref{mailto:info@salabim.org}{info@salabim.org}.


\section{License}
\label{\detokenize{About:license}}
The MIT License (MIT)

Copyright (c) 2016, 2017, 2018 Ruud van der Ham, \sphinxhref{mailto:ruud@salabim.org}{ruud@salabim.org}

Permission is hereby granted, free of charge, to any person obtaining a copy of
this software and associated documentation files (the “Software”), to deal in
the Software without restriction, including without limitation the rights to
use, copy, modify, merge, publish, distribute, sublicense, and/or sell copies
of the Software, and to permit persons to who the Software is furnished to do
so, subject to the following conditions:

The above copyright notice and this permission notice shall be included in all
copies or substantial portions of the Software.

THE SOFTWARE IS PROVIDED “AS IS”, WITHOUT WARRANTY OF ANY KIND, EXPRESS OR
IMPLIED, INCLUDING BUT NOT LIMITED TO THE WARRANTIES OF MERCHANTABILITY,
FITNESS FOR A PARTICULAR PURPOSE AND NONINFRINGEMENT. IN NO EVENT SHALL THE
AUTHORS OR COPYRIGHT HOLDERS BE LIABLE FOR ANY CLAIM, DAMAGES OR OTHER
LIABILITY, WHETHER IN AN ACTION OF CONTRACT, TORT OR OTHERWISE, ARISING FROM,
OUT OF OR IN CONNECTION WITH THE SOFTWARE OR THE USE OR OTHER DEALINGS IN THE
SOFTWARE.


\chapter{Indices and tables}
\label{\detokenize{Indices:indices-and-tables}}\label{\detokenize{Indices::doc}}\begin{itemize}
\item {} 
\DUrole{xref,std,std-ref}{genindex}

\item {} 
\DUrole{xref,std,std-ref}{search}

\end{itemize}



\renewcommand{\indexname}{Index}
\printindex
\end{document}